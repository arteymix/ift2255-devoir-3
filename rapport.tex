\documentclass{article}

\usepackage[utf8]{inputenc}
\usepackage[T1]{fontenc}
\usepackage[french]{babel}

\usepackage{graphics}
\usepackage{fullpage}
\usepackage{rotating}
\usepackage{tikz}

\title{IFT2255 -- Génie logiciel \\Devoir 3}
\author{Vincent Antaki \\ Guillaume Poirier-Morency \\ Émile Trottier}

\begin{document}

  \maketitle

  \abstract
  L'objectif du travail est d'analyser et conçevoir un système confronté à une
  série de changements.

  Le projet a été séparé en trois paquetages Java. Pour minimiser
  l'implantation, les classes non-modifiées sont réutilisés depuis les
  paquetages précédents.

  Le projet est accompagné d'une série de tests JUnit au lieu de spécifier une
  classe \textsf{Main}. C'est plus simple et cela nous permet de soumettre notre
  implantation à des cas d'utilisation plus réalistes.

  \section{Diagramme de classes initial}
  \begin{sidewaysfigure}
    \centering
    \resizebox{\textwidth}{!}{% Graphic for TeX using PGF
% Title: /home/guillaume/Documents/Université de Montréal/Automne 2014/Génie logiciel/Devoir/devoir-3/diagramme-de-classes-initial.dia
% Creator: Dia v0.97.3
% CreationDate: Sun Dec  7 16:07:32 2014
% For: guillaume
% \usepackage{tikz}
% The following commands are not supported in PSTricks at present
% We define them conditionally, so when they are implemented,
% this pgf file will use them.
\ifx\du\undefined
  \newlength{\du}
\fi
\setlength{\du}{15\unitlength}
\begin{tikzpicture}
\pgftransformxscale{1.000000}
\pgftransformyscale{-1.000000}
\definecolor{dialinecolor}{rgb}{0.000000, 0.000000, 0.000000}
\pgfsetstrokecolor{dialinecolor}
\definecolor{dialinecolor}{rgb}{1.000000, 1.000000, 1.000000}
\pgfsetfillcolor{dialinecolor}
\pgfsetlinewidth{0.100000\du}
\pgfsetdash{}{0pt}
\definecolor{dialinecolor}{rgb}{1.000000, 1.000000, 1.000000}
\pgfsetfillcolor{dialinecolor}
\fill (18.557700\du,-6.718780\du)--(18.557700\du,-5.318780\du)--(32.917700\du,-5.318780\du)--(32.917700\du,-6.718780\du)--cycle;
\definecolor{dialinecolor}{rgb}{0.000000, 0.000000, 0.000000}
\pgfsetstrokecolor{dialinecolor}
\draw (18.557700\du,-6.718780\du)--(18.557700\du,-5.318780\du)--(32.917700\du,-5.318780\du)--(32.917700\du,-6.718780\du)--cycle;
% setfont left to latex
\definecolor{dialinecolor}{rgb}{0.000000, 0.000000, 0.000000}
\pgfsetstrokecolor{dialinecolor}
\node at (25.737700\du,-5.768780\du){Element};
\definecolor{dialinecolor}{rgb}{1.000000, 1.000000, 1.000000}
\pgfsetfillcolor{dialinecolor}
\fill (18.557700\du,-5.318780\du)--(18.557700\du,-1.118780\du)--(32.917700\du,-1.118780\du)--(32.917700\du,-5.318780\du)--cycle;
\definecolor{dialinecolor}{rgb}{0.000000, 0.000000, 0.000000}
\pgfsetstrokecolor{dialinecolor}
\draw (18.557700\du,-5.318780\du)--(18.557700\du,-1.118780\du)--(32.917700\du,-1.118780\du)--(32.917700\du,-5.318780\du)--cycle;
% setfont left to latex
\definecolor{dialinecolor}{rgb}{0.000000, 0.000000, 0.000000}
\pgfsetstrokecolor{dialinecolor}
\node[anchor=west] at (18.707700\du,-4.618780\du){\#Nom: String};
% setfont left to latex
\definecolor{dialinecolor}{rgb}{0.000000, 0.000000, 0.000000}
\pgfsetstrokecolor{dialinecolor}
\node[anchor=west] at (18.707700\du,-3.818780\du){\#Date de création: Date};
% setfont left to latex
\definecolor{dialinecolor}{rgb}{0.000000, 0.000000, 0.000000}
\pgfsetstrokecolor{dialinecolor}
\node[anchor=west] at (18.707700\du,-3.018780\du){\#Date de dernière modification: Date};
% setfont left to latex
\definecolor{dialinecolor}{rgb}{0.000000, 0.000000, 0.000000}
\pgfsetstrokecolor{dialinecolor}
\node[anchor=west] at (18.707700\du,-2.218780\du){\#Chemin: String};
% setfont left to latex
\definecolor{dialinecolor}{rgb}{0.000000, 0.000000, 0.000000}
\pgfsetstrokecolor{dialinecolor}
\node[anchor=west] at (18.707700\du,-1.418780\du){\#Ouvert: Boolean = False};
\definecolor{dialinecolor}{rgb}{1.000000, 1.000000, 1.000000}
\pgfsetfillcolor{dialinecolor}
\fill (18.557700\du,-1.118780\du)--(18.557700\du,1.481220\du)--(32.917700\du,1.481220\du)--(32.917700\du,-1.118780\du)--cycle;
\definecolor{dialinecolor}{rgb}{0.000000, 0.000000, 0.000000}
\pgfsetstrokecolor{dialinecolor}
\draw (18.557700\du,-1.118780\du)--(18.557700\du,1.481220\du)--(32.917700\du,1.481220\du)--(32.917700\du,-1.118780\du)--cycle;
% setfont left to latex
\definecolor{dialinecolor}{rgb}{0.000000, 0.000000, 0.000000}
\pgfsetstrokecolor{dialinecolor}
\node[anchor=west] at (18.707700\du,-0.418780\du){+open()};
% setfont left to latex
\definecolor{dialinecolor}{rgb}{0.000000, 0.000000, 0.000000}
\pgfsetstrokecolor{dialinecolor}
\node[anchor=west] at (18.707700\du,0.381220\du){+close()};
% setfont left to latex
\definecolor{dialinecolor}{rgb}{0.000000, 0.000000, 0.000000}
\pgfsetstrokecolor{dialinecolor}
\node[anchor=west] at (18.707700\du,1.181220\du){+delete()};
\pgfsetlinewidth{0.100000\du}
\pgfsetdash{}{0pt}
\definecolor{dialinecolor}{rgb}{1.000000, 1.000000, 1.000000}
\pgfsetfillcolor{dialinecolor}
\fill (27.745400\du,13.848800\du)--(27.745400\du,15.248800\du)--(31.325400\du,15.248800\du)--(31.325400\du,13.848800\du)--cycle;
\definecolor{dialinecolor}{rgb}{0.000000, 0.000000, 0.000000}
\pgfsetstrokecolor{dialinecolor}
\draw (27.745400\du,13.848800\du)--(27.745400\du,15.248800\du)--(31.325400\du,15.248800\du)--(31.325400\du,13.848800\du)--cycle;
% setfont left to latex
\definecolor{dialinecolor}{rgb}{0.000000, 0.000000, 0.000000}
\pgfsetstrokecolor{dialinecolor}
\node at (29.535400\du,14.798800\du){Fichier};
\definecolor{dialinecolor}{rgb}{1.000000, 1.000000, 1.000000}
\pgfsetfillcolor{dialinecolor}
\fill (27.745400\du,15.248800\du)--(27.745400\du,16.248800\du)--(31.325400\du,16.248800\du)--(31.325400\du,15.248800\du)--cycle;
\definecolor{dialinecolor}{rgb}{0.000000, 0.000000, 0.000000}
\pgfsetstrokecolor{dialinecolor}
\draw (27.745400\du,15.248800\du)--(27.745400\du,16.248800\du)--(31.325400\du,16.248800\du)--(31.325400\du,15.248800\du)--cycle;
% setfont left to latex
\definecolor{dialinecolor}{rgb}{0.000000, 0.000000, 0.000000}
\pgfsetstrokecolor{dialinecolor}
\node[anchor=west] at (27.895400\du,15.948800\du){-taille};
\definecolor{dialinecolor}{rgb}{1.000000, 1.000000, 1.000000}
\pgfsetfillcolor{dialinecolor}
\fill (27.745400\du,16.248800\du)--(27.745400\du,16.648800\du)--(31.325400\du,16.648800\du)--(31.325400\du,16.248800\du)--cycle;
\definecolor{dialinecolor}{rgb}{0.000000, 0.000000, 0.000000}
\pgfsetstrokecolor{dialinecolor}
\draw (27.745400\du,16.248800\du)--(27.745400\du,16.648800\du)--(31.325400\du,16.648800\du)--(31.325400\du,16.248800\du)--cycle;
\pgfsetlinewidth{0.100000\du}
\pgfsetdash{}{0pt}
\definecolor{dialinecolor}{rgb}{1.000000, 1.000000, 1.000000}
\pgfsetfillcolor{dialinecolor}
\fill (7.373110\du,12.000000\du)--(7.373110\du,13.400000\du)--(20.963110\du,13.400000\du)--(20.963110\du,12.000000\du)--cycle;
\definecolor{dialinecolor}{rgb}{0.000000, 0.000000, 0.000000}
\pgfsetstrokecolor{dialinecolor}
\draw (7.373110\du,12.000000\du)--(7.373110\du,13.400000\du)--(20.963110\du,13.400000\du)--(20.963110\du,12.000000\du)--cycle;
% setfont left to latex
\definecolor{dialinecolor}{rgb}{0.000000, 0.000000, 0.000000}
\pgfsetstrokecolor{dialinecolor}
\node at (14.168110\du,12.950000\du){Dossier};
\definecolor{dialinecolor}{rgb}{1.000000, 1.000000, 1.000000}
\pgfsetfillcolor{dialinecolor}
\fill (7.373110\du,13.400000\du)--(7.373110\du,13.800000\du)--(20.963110\du,13.800000\du)--(20.963110\du,13.400000\du)--cycle;
\definecolor{dialinecolor}{rgb}{0.000000, 0.000000, 0.000000}
\pgfsetstrokecolor{dialinecolor}
\draw (7.373110\du,13.400000\du)--(7.373110\du,13.800000\du)--(20.963110\du,13.800000\du)--(20.963110\du,13.400000\du)--cycle;
\definecolor{dialinecolor}{rgb}{1.000000, 1.000000, 1.000000}
\pgfsetfillcolor{dialinecolor}
\fill (7.373110\du,13.800000\du)--(7.373110\du,17.200000\du)--(20.963110\du,17.200000\du)--(20.963110\du,13.800000\du)--cycle;
\definecolor{dialinecolor}{rgb}{0.000000, 0.000000, 0.000000}
\pgfsetstrokecolor{dialinecolor}
\draw (7.373110\du,13.800000\du)--(7.373110\du,17.200000\du)--(20.963110\du,17.200000\du)--(20.963110\du,13.800000\du)--cycle;
% setfont left to latex
\definecolor{dialinecolor}{rgb}{0.000000, 0.000000, 0.000000}
\pgfsetstrokecolor{dialinecolor}
\node[anchor=west] at (7.523110\du,14.500000\du){+activate()};
% setfont left to latex
\definecolor{dialinecolor}{rgb}{0.000000, 0.000000, 0.000000}
\pgfsetstrokecolor{dialinecolor}
\node[anchor=west] at (7.523110\du,15.300000\du){+attach(observer:ActivateObserver)};
% setfont left to latex
\definecolor{dialinecolor}{rgb}{0.000000, 0.000000, 0.000000}
\pgfsetstrokecolor{dialinecolor}
\node[anchor=west] at (7.523110\du,16.100000\du){+detach(observer:ActivateObserver)};
% setfont left to latex
\definecolor{dialinecolor}{rgb}{0.000000, 0.000000, 0.000000}
\pgfsetstrokecolor{dialinecolor}
\node[anchor=west] at (7.523110\du,16.900000\du){+notifyActivate()};
\pgfsetlinewidth{0.100000\du}
\pgfsetdash{}{0pt}
\definecolor{dialinecolor}{rgb}{1.000000, 1.000000, 1.000000}
\pgfsetfillcolor{dialinecolor}
\fill (-29.000000\du,24.000000\du)--(-29.000000\du,25.400000\du)--(-18.490000\du,25.400000\du)--(-18.490000\du,24.000000\du)--cycle;
\definecolor{dialinecolor}{rgb}{0.000000, 0.000000, 0.000000}
\pgfsetstrokecolor{dialinecolor}
\draw (-29.000000\du,24.000000\du)--(-29.000000\du,25.400000\du)--(-18.490000\du,25.400000\du)--(-18.490000\du,24.000000\du)--cycle;
% setfont left to latex
\definecolor{dialinecolor}{rgb}{0.000000, 0.000000, 0.000000}
\pgfsetstrokecolor{dialinecolor}
\node at (-23.745000\du,24.950000\du){Navigateur};
\definecolor{dialinecolor}{rgb}{1.000000, 1.000000, 1.000000}
\pgfsetfillcolor{dialinecolor}
\fill (-29.000000\du,25.400000\du)--(-29.000000\du,26.400000\du)--(-18.490000\du,26.400000\du)--(-18.490000\du,25.400000\du)--cycle;
\definecolor{dialinecolor}{rgb}{0.000000, 0.000000, 0.000000}
\pgfsetstrokecolor{dialinecolor}
\draw (-29.000000\du,25.400000\du)--(-29.000000\du,26.400000\du)--(-18.490000\du,26.400000\du)--(-18.490000\du,25.400000\du)--cycle;
% setfont left to latex
\definecolor{dialinecolor}{rgb}{0.000000, 0.000000, 0.000000}
\pgfsetstrokecolor{dialinecolor}
\node[anchor=west] at (-28.850000\du,26.100000\du){-instance: Navigateur};
\definecolor{dialinecolor}{rgb}{1.000000, 1.000000, 1.000000}
\pgfsetfillcolor{dialinecolor}
\fill (-29.000000\du,26.400000\du)--(-29.000000\du,28.200000\du)--(-18.490000\du,28.200000\du)--(-18.490000\du,26.400000\du)--cycle;
\definecolor{dialinecolor}{rgb}{0.000000, 0.000000, 0.000000}
\pgfsetstrokecolor{dialinecolor}
\draw (-29.000000\du,26.400000\du)--(-29.000000\du,28.200000\du)--(-18.490000\du,28.200000\du)--(-18.490000\du,26.400000\du)--cycle;
% setfont left to latex
\definecolor{dialinecolor}{rgb}{0.000000, 0.000000, 0.000000}
\pgfsetstrokecolor{dialinecolor}
\node[anchor=west] at (-28.850000\du,27.100000\du){+getInstance(): Navigateur};
% setfont left to latex
\definecolor{dialinecolor}{rgb}{0.000000, 0.000000, 0.000000}
\pgfsetstrokecolor{dialinecolor}
\node[anchor=west] at (-28.850000\du,27.900000\du){-<<Constructeur>> ()};
\pgfsetlinewidth{0.100000\du}
\pgfsetdash{}{0pt}
\pgfsetmiterjoin
\pgfsetbuttcap
{
\definecolor{dialinecolor}{rgb}{0.000000, 0.000000, 0.000000}
\pgfsetfillcolor{dialinecolor}
% was here!!!
\definecolor{dialinecolor}{rgb}{0.000000, 0.000000, 0.000000}
\pgfsetstrokecolor{dialinecolor}
\draw (25.737700\du,1.481220\du)--(25.737700\du,9.000000\du)--(19.076400\du,9.000000\du)--(19.076400\du,11.809900\du);
}
\definecolor{dialinecolor}{rgb}{0.000000, 0.000000, 0.000000}
\pgfsetstrokecolor{dialinecolor}
\draw (25.737700\du,2.393023\du)--(25.737700\du,9.000000\du)--(19.076400\du,9.000000\du)--(19.076400\du,11.809900\du);
\pgfsetmiterjoin
\definecolor{dialinecolor}{rgb}{1.000000, 1.000000, 1.000000}
\pgfsetfillcolor{dialinecolor}
\fill (26.137700\du,2.393023\du)--(25.737700\du,1.593023\du)--(25.337700\du,2.393023\du)--cycle;
\pgfsetlinewidth{0.100000\du}
\pgfsetdash{}{0pt}
\pgfsetmiterjoin
\definecolor{dialinecolor}{rgb}{0.000000, 0.000000, 0.000000}
\pgfsetstrokecolor{dialinecolor}
\draw (26.137700\du,2.393023\du)--(25.737700\du,1.593023\du)--(25.337700\du,2.393023\du)--cycle;
% setfont left to latex
\pgfsetlinewidth{0.100000\du}
\pgfsetdash{}{0pt}
\pgfsetmiterjoin
\pgfsetbuttcap
{
\definecolor{dialinecolor}{rgb}{0.000000, 0.000000, 0.000000}
\pgfsetfillcolor{dialinecolor}
% was here!!!
\definecolor{dialinecolor}{rgb}{0.000000, 0.000000, 0.000000}
\pgfsetstrokecolor{dialinecolor}
\draw (25.737700\du,1.481220\du)--(25.737700\du,9.000000\du)--(29.535400\du,9.000000\du)--(29.535400\du,13.798731\du);
}
\definecolor{dialinecolor}{rgb}{0.000000, 0.000000, 0.000000}
\pgfsetstrokecolor{dialinecolor}
\draw (25.737700\du,2.393023\du)--(25.737700\du,9.000000\du)--(29.535400\du,9.000000\du)--(29.535400\du,13.798731\du);
\pgfsetmiterjoin
\definecolor{dialinecolor}{rgb}{1.000000, 1.000000, 1.000000}
\pgfsetfillcolor{dialinecolor}
\fill (26.137700\du,2.393023\du)--(25.737700\du,1.593023\du)--(25.337700\du,2.393023\du)--cycle;
\pgfsetlinewidth{0.100000\du}
\pgfsetdash{}{0pt}
\pgfsetmiterjoin
\definecolor{dialinecolor}{rgb}{0.000000, 0.000000, 0.000000}
\pgfsetstrokecolor{dialinecolor}
\draw (26.137700\du,2.393023\du)--(25.737700\du,1.593023\du)--(25.337700\du,2.393023\du)--cycle;
% setfont left to latex
\pgfsetlinewidth{0.100000\du}
\pgfsetdash{}{0pt}
\definecolor{dialinecolor}{rgb}{1.000000, 1.000000, 1.000000}
\pgfsetfillcolor{dialinecolor}
\fill (-39.000000\du,-9.000000\du)--(-39.000000\du,-7.600000\du)--(-25.795000\du,-7.600000\du)--(-25.795000\du,-9.000000\du)--cycle;
\definecolor{dialinecolor}{rgb}{0.000000, 0.000000, 0.000000}
\pgfsetstrokecolor{dialinecolor}
\draw (-39.000000\du,-9.000000\du)--(-39.000000\du,-7.600000\du)--(-25.795000\du,-7.600000\du)--(-25.795000\du,-9.000000\du)--cycle;
% setfont left to latex
\definecolor{dialinecolor}{rgb}{0.000000, 0.000000, 0.000000}
\pgfsetstrokecolor{dialinecolor}
\node at (-32.397500\du,-8.050000\du){Observable};
\definecolor{dialinecolor}{rgb}{1.000000, 1.000000, 1.000000}
\pgfsetfillcolor{dialinecolor}
\fill (-39.000000\du,-7.600000\du)--(-39.000000\du,-7.200000\du)--(-25.795000\du,-7.200000\du)--(-25.795000\du,-7.600000\du)--cycle;
\definecolor{dialinecolor}{rgb}{0.000000, 0.000000, 0.000000}
\pgfsetstrokecolor{dialinecolor}
\draw (-39.000000\du,-7.600000\du)--(-39.000000\du,-7.200000\du)--(-25.795000\du,-7.200000\du)--(-25.795000\du,-7.600000\du)--cycle;
\definecolor{dialinecolor}{rgb}{1.000000, 1.000000, 1.000000}
\pgfsetfillcolor{dialinecolor}
\fill (-39.000000\du,-7.200000\du)--(-39.000000\du,2.600000\du)--(-25.795000\du,2.600000\du)--(-25.795000\du,-7.200000\du)--cycle;
\definecolor{dialinecolor}{rgb}{0.000000, 0.000000, 0.000000}
\pgfsetstrokecolor{dialinecolor}
\draw (-39.000000\du,-7.200000\du)--(-39.000000\du,2.600000\du)--(-25.795000\du,2.600000\du)--(-25.795000\du,-7.200000\du)--cycle;
% setfont left to latex
\definecolor{dialinecolor}{rgb}{0.000000, 0.000000, 0.000000}
\pgfsetstrokecolor{dialinecolor}
\node[anchor=west] at (-38.850000\du,-6.500000\du){+attach(observer:ChangeObserver)};
% setfont left to latex
\definecolor{dialinecolor}{rgb}{0.000000, 0.000000, 0.000000}
\pgfsetstrokecolor{dialinecolor}
\node[anchor=west] at (-38.850000\du,-5.700000\du){+attach(observer:DeleteObserver)};
% setfont left to latex
\definecolor{dialinecolor}{rgb}{0.000000, 0.000000, 0.000000}
\pgfsetstrokecolor{dialinecolor}
\node[anchor=west] at (-38.850000\du,-4.900000\du){+attach(observer:OpenObserver)};
% setfont left to latex
\definecolor{dialinecolor}{rgb}{0.000000, 0.000000, 0.000000}
\pgfsetstrokecolor{dialinecolor}
\node[anchor=west] at (-38.850000\du,-4.100000\du){+attach(observer:CloseObserver)};
% setfont left to latex
\definecolor{dialinecolor}{rgb}{0.000000, 0.000000, 0.000000}
\pgfsetstrokecolor{dialinecolor}
\node[anchor=west] at (-38.850000\du,-3.300000\du){+dettach(observer:ChangeObserve)};
% setfont left to latex
\definecolor{dialinecolor}{rgb}{0.000000, 0.000000, 0.000000}
\pgfsetstrokecolor{dialinecolor}
\node[anchor=west] at (-38.850000\du,-2.500000\du){+dettach(observer:DeleteObserver)};
% setfont left to latex
\definecolor{dialinecolor}{rgb}{0.000000, 0.000000, 0.000000}
\pgfsetstrokecolor{dialinecolor}
\node[anchor=west] at (-38.850000\du,-1.700000\du){+dettach(observer:OpenObserver)};
% setfont left to latex
\definecolor{dialinecolor}{rgb}{0.000000, 0.000000, 0.000000}
\pgfsetstrokecolor{dialinecolor}
\node[anchor=west] at (-38.850000\du,-0.900000\du){+dettach(observer:CloseObserver)};
% setfont left to latex
\definecolor{dialinecolor}{rgb}{0.000000, 0.000000, 0.000000}
\pgfsetstrokecolor{dialinecolor}
\node[anchor=west] at (-38.850000\du,-0.100000\du){+notifyChange()};
% setfont left to latex
\definecolor{dialinecolor}{rgb}{0.000000, 0.000000, 0.000000}
\pgfsetstrokecolor{dialinecolor}
\node[anchor=west] at (-38.850000\du,0.700000\du){+notifyDelete()};
% setfont left to latex
\definecolor{dialinecolor}{rgb}{0.000000, 0.000000, 0.000000}
\pgfsetstrokecolor{dialinecolor}
\node[anchor=west] at (-38.850000\du,1.500000\du){+notifyOpen()};
% setfont left to latex
\definecolor{dialinecolor}{rgb}{0.000000, 0.000000, 0.000000}
\pgfsetstrokecolor{dialinecolor}
\node[anchor=west] at (-38.850000\du,2.300000\du){+notifyClose()};
\pgfsetlinewidth{0.100000\du}
\pgfsetdash{}{0pt}
\pgfsetmiterjoin
\pgfsetbuttcap
{
\definecolor{dialinecolor}{rgb}{0.000000, 0.000000, 0.000000}
\pgfsetfillcolor{dialinecolor}
% was here!!!
\definecolor{dialinecolor}{rgb}{0.000000, 0.000000, 0.000000}
\pgfsetstrokecolor{dialinecolor}
\draw (16.003800\du,11.886700\du)--(16.003800\du,11.886700\du)--(16.003800\du,-2.618780\du)--(18.507974\du,-2.618780\du);
}
\definecolor{dialinecolor}{rgb}{0.000000, 0.000000, 0.000000}
\pgfsetstrokecolor{dialinecolor}
\draw (16.003800\du,10.628121\du)--(16.003800\du,-2.618780\du)--(18.507974\du,-2.618780\du);
\pgfsetdash{}{0pt}
\pgfsetmiterjoin
\pgfsetbuttcap
\definecolor{dialinecolor}{rgb}{0.000000, 0.000000, 0.000000}
\pgfsetfillcolor{dialinecolor}
\fill (16.003800\du,11.886700\du)--(15.763800\du,11.186700\du)--(16.003800\du,10.486700\du)--(16.243800\du,11.186700\du)--cycle;
\pgfsetlinewidth{0.100000\du}
\pgfsetdash{}{0pt}
\pgfsetmiterjoin
\pgfsetbuttcap
\definecolor{dialinecolor}{rgb}{0.000000, 0.000000, 0.000000}
\pgfsetstrokecolor{dialinecolor}
\draw (16.003800\du,11.886700\du)--(15.763800\du,11.186700\du)--(16.003800\du,10.486700\du)--(16.243800\du,11.186700\du)--cycle;
% setfont left to latex
\definecolor{dialinecolor}{rgb}{0.000000, 0.000000, 0.000000}
\pgfsetstrokecolor{dialinecolor}
\node[anchor=west] at (16.103800\du,4.483960\du){};
\definecolor{dialinecolor}{rgb}{0.000000, 0.000000, 0.000000}
\pgfsetstrokecolor{dialinecolor}
\node[anchor=west] at (16.553800\du,11.686700\du){};
\definecolor{dialinecolor}{rgb}{0.000000, 0.000000, 0.000000}
\pgfsetstrokecolor{dialinecolor}
\node[anchor=east] at (18.307974\du,-2.768780\du){ possède};
\definecolor{dialinecolor}{rgb}{0.000000, 0.000000, 0.000000}
\pgfsetstrokecolor{dialinecolor}
\node[anchor=east] at (18.307974\du,-1.968780\du){*};
\pgfsetlinewidth{0.100000\du}
\pgfsetdash{}{0pt}
\pgfsetmiterjoin
\pgfsetbuttcap
{
\definecolor{dialinecolor}{rgb}{0.000000, 0.000000, 0.000000}
\pgfsetfillcolor{dialinecolor}
% was here!!!
\pgfsetarrowsend{to}
\definecolor{dialinecolor}{rgb}{0.000000, 0.000000, 0.000000}
\pgfsetstrokecolor{dialinecolor}
\draw (-18.439685\du,26.100000\du)--(26.149400\du,26.100000\du)--(26.149400\du,14.600000\du)--(21.013256\du,14.600000\du);
}
% setfont left to latex
\definecolor{dialinecolor}{rgb}{0.000000, 0.000000, 0.000000}
\pgfsetstrokecolor{dialinecolor}
\node[anchor=west] at (26.249400\du,20.200000\du){};
\definecolor{dialinecolor}{rgb}{0.000000, 0.000000, 0.000000}
\pgfsetstrokecolor{dialinecolor}
\node[anchor=west] at (-18.239685\du,25.950000\du){};
\definecolor{dialinecolor}{rgb}{0.000000, 0.000000, 0.000000}
\pgfsetstrokecolor{dialinecolor}
\node[anchor=west] at (22.013256\du,14.450000\du){ active};
\definecolor{dialinecolor}{rgb}{0.000000, 0.000000, 0.000000}
\pgfsetstrokecolor{dialinecolor}
\node[anchor=west] at (22.013256\du,15.250000\du){0..1};
\pgfsetlinewidth{0.100000\du}
\pgfsetdash{}{0pt}
\definecolor{dialinecolor}{rgb}{1.000000, 1.000000, 1.000000}
\pgfsetfillcolor{dialinecolor}
\fill (-10.000000\du,12.000000\du)--(-10.000000\du,13.400000\du)--(0.510000\du,13.400000\du)--(0.510000\du,12.000000\du)--cycle;
\definecolor{dialinecolor}{rgb}{0.000000, 0.000000, 0.000000}
\pgfsetstrokecolor{dialinecolor}
\draw (-10.000000\du,12.000000\du)--(-10.000000\du,13.400000\du)--(0.510000\du,13.400000\du)--(0.510000\du,12.000000\du)--cycle;
% setfont left to latex
\definecolor{dialinecolor}{rgb}{0.000000, 0.000000, 0.000000}
\pgfsetstrokecolor{dialinecolor}
\node at (-4.745000\du,12.950000\du){ActivateObserver};
\definecolor{dialinecolor}{rgb}{1.000000, 1.000000, 1.000000}
\pgfsetfillcolor{dialinecolor}
\fill (-10.000000\du,13.400000\du)--(-10.000000\du,13.800000\du)--(0.510000\du,13.800000\du)--(0.510000\du,13.400000\du)--cycle;
\definecolor{dialinecolor}{rgb}{0.000000, 0.000000, 0.000000}
\pgfsetstrokecolor{dialinecolor}
\draw (-10.000000\du,13.400000\du)--(-10.000000\du,13.800000\du)--(0.510000\du,13.800000\du)--(0.510000\du,13.400000\du)--cycle;
\definecolor{dialinecolor}{rgb}{1.000000, 1.000000, 1.000000}
\pgfsetfillcolor{dialinecolor}
\fill (-10.000000\du,13.800000\du)--(-10.000000\du,14.800000\du)--(0.510000\du,14.800000\du)--(0.510000\du,13.800000\du)--cycle;
\definecolor{dialinecolor}{rgb}{0.000000, 0.000000, 0.000000}
\pgfsetstrokecolor{dialinecolor}
\draw (-10.000000\du,13.800000\du)--(-10.000000\du,14.800000\du)--(0.510000\du,14.800000\du)--(0.510000\du,13.800000\du)--cycle;
% setfont left to latex
\definecolor{dialinecolor}{rgb}{0.000000, 0.000000, 0.000000}
\pgfsetstrokecolor{dialinecolor}
\node[anchor=west] at (-9.850000\du,14.500000\du){+updateActivate(d:Dossier)};
\pgfsetlinewidth{0.100000\du}
\pgfsetdash{}{0pt}
\definecolor{dialinecolor}{rgb}{1.000000, 1.000000, 1.000000}
\pgfsetfillcolor{dialinecolor}
\fill (-21.000000\du,12.000000\du)--(-21.000000\du,13.400000\du)--(-11.260000\du,13.400000\du)--(-11.260000\du,12.000000\du)--cycle;
\definecolor{dialinecolor}{rgb}{0.000000, 0.000000, 0.000000}
\pgfsetstrokecolor{dialinecolor}
\draw (-21.000000\du,12.000000\du)--(-21.000000\du,13.400000\du)--(-11.260000\du,13.400000\du)--(-11.260000\du,12.000000\du)--cycle;
% setfont left to latex
\definecolor{dialinecolor}{rgb}{0.000000, 0.000000, 0.000000}
\pgfsetstrokecolor{dialinecolor}
\node at (-16.130000\du,12.950000\du){DeleteObserver};
\definecolor{dialinecolor}{rgb}{1.000000, 1.000000, 1.000000}
\pgfsetfillcolor{dialinecolor}
\fill (-21.000000\du,13.400000\du)--(-21.000000\du,13.800000\du)--(-11.260000\du,13.800000\du)--(-11.260000\du,13.400000\du)--cycle;
\definecolor{dialinecolor}{rgb}{0.000000, 0.000000, 0.000000}
\pgfsetstrokecolor{dialinecolor}
\draw (-21.000000\du,13.400000\du)--(-21.000000\du,13.800000\du)--(-11.260000\du,13.800000\du)--(-11.260000\du,13.400000\du)--cycle;
\definecolor{dialinecolor}{rgb}{1.000000, 1.000000, 1.000000}
\pgfsetfillcolor{dialinecolor}
\fill (-21.000000\du,13.800000\du)--(-21.000000\du,14.800000\du)--(-11.260000\du,14.800000\du)--(-11.260000\du,13.800000\du)--cycle;
\definecolor{dialinecolor}{rgb}{0.000000, 0.000000, 0.000000}
\pgfsetstrokecolor{dialinecolor}
\draw (-21.000000\du,13.800000\du)--(-21.000000\du,14.800000\du)--(-11.260000\du,14.800000\du)--(-11.260000\du,13.800000\du)--cycle;
% setfont left to latex
\definecolor{dialinecolor}{rgb}{0.000000, 0.000000, 0.000000}
\pgfsetstrokecolor{dialinecolor}
\node[anchor=west] at (-20.850000\du,14.500000\du){+updateDelete(e:Element)};
\pgfsetlinewidth{0.100000\du}
\pgfsetdash{}{0pt}
\definecolor{dialinecolor}{rgb}{1.000000, 1.000000, 1.000000}
\pgfsetfillcolor{dialinecolor}
\fill (-31.000000\du,12.000000\du)--(-31.000000\du,13.400000\du)--(-22.030000\du,13.400000\du)--(-22.030000\du,12.000000\du)--cycle;
\definecolor{dialinecolor}{rgb}{0.000000, 0.000000, 0.000000}
\pgfsetstrokecolor{dialinecolor}
\draw (-31.000000\du,12.000000\du)--(-31.000000\du,13.400000\du)--(-22.030000\du,13.400000\du)--(-22.030000\du,12.000000\du)--cycle;
% setfont left to latex
\definecolor{dialinecolor}{rgb}{0.000000, 0.000000, 0.000000}
\pgfsetstrokecolor{dialinecolor}
\node at (-26.515000\du,12.950000\du){OpenObserver};
\definecolor{dialinecolor}{rgb}{1.000000, 1.000000, 1.000000}
\pgfsetfillcolor{dialinecolor}
\fill (-31.000000\du,13.400000\du)--(-31.000000\du,13.800000\du)--(-22.030000\du,13.800000\du)--(-22.030000\du,13.400000\du)--cycle;
\definecolor{dialinecolor}{rgb}{0.000000, 0.000000, 0.000000}
\pgfsetstrokecolor{dialinecolor}
\draw (-31.000000\du,13.400000\du)--(-31.000000\du,13.800000\du)--(-22.030000\du,13.800000\du)--(-22.030000\du,13.400000\du)--cycle;
\definecolor{dialinecolor}{rgb}{1.000000, 1.000000, 1.000000}
\pgfsetfillcolor{dialinecolor}
\fill (-31.000000\du,13.800000\du)--(-31.000000\du,14.800000\du)--(-22.030000\du,14.800000\du)--(-22.030000\du,13.800000\du)--cycle;
\definecolor{dialinecolor}{rgb}{0.000000, 0.000000, 0.000000}
\pgfsetstrokecolor{dialinecolor}
\draw (-31.000000\du,13.800000\du)--(-31.000000\du,14.800000\du)--(-22.030000\du,14.800000\du)--(-22.030000\du,13.800000\du)--cycle;
% setfont left to latex
\definecolor{dialinecolor}{rgb}{0.000000, 0.000000, 0.000000}
\pgfsetstrokecolor{dialinecolor}
\node[anchor=west] at (-30.850000\du,14.500000\du){+updateOpen(e:Element)};
\pgfsetlinewidth{0.100000\du}
\pgfsetdash{}{0pt}
\definecolor{dialinecolor}{rgb}{1.000000, 1.000000, 1.000000}
\pgfsetfillcolor{dialinecolor}
\fill (-43.000000\du,12.000000\du)--(-43.000000\du,13.400000\du)--(-33.260000\du,13.400000\du)--(-33.260000\du,12.000000\du)--cycle;
\definecolor{dialinecolor}{rgb}{0.000000, 0.000000, 0.000000}
\pgfsetstrokecolor{dialinecolor}
\draw (-43.000000\du,12.000000\du)--(-43.000000\du,13.400000\du)--(-33.260000\du,13.400000\du)--(-33.260000\du,12.000000\du)--cycle;
% setfont left to latex
\definecolor{dialinecolor}{rgb}{0.000000, 0.000000, 0.000000}
\pgfsetstrokecolor{dialinecolor}
\node at (-38.130000\du,12.950000\du){ChangeObserver};
\definecolor{dialinecolor}{rgb}{1.000000, 1.000000, 1.000000}
\pgfsetfillcolor{dialinecolor}
\fill (-43.000000\du,13.400000\du)--(-43.000000\du,13.800000\du)--(-33.260000\du,13.800000\du)--(-33.260000\du,13.400000\du)--cycle;
\definecolor{dialinecolor}{rgb}{0.000000, 0.000000, 0.000000}
\pgfsetstrokecolor{dialinecolor}
\draw (-43.000000\du,13.400000\du)--(-43.000000\du,13.800000\du)--(-33.260000\du,13.800000\du)--(-33.260000\du,13.400000\du)--cycle;
\definecolor{dialinecolor}{rgb}{1.000000, 1.000000, 1.000000}
\pgfsetfillcolor{dialinecolor}
\fill (-43.000000\du,13.800000\du)--(-43.000000\du,14.800000\du)--(-33.260000\du,14.800000\du)--(-33.260000\du,13.800000\du)--cycle;
\definecolor{dialinecolor}{rgb}{0.000000, 0.000000, 0.000000}
\pgfsetstrokecolor{dialinecolor}
\draw (-43.000000\du,13.800000\du)--(-43.000000\du,14.800000\du)--(-33.260000\du,14.800000\du)--(-33.260000\du,13.800000\du)--cycle;
% setfont left to latex
\definecolor{dialinecolor}{rgb}{0.000000, 0.000000, 0.000000}
\pgfsetstrokecolor{dialinecolor}
\node[anchor=west] at (-42.850000\du,14.500000\du){+updateChange(e:Element)};
\pgfsetlinewidth{0.100000\du}
\pgfsetdash{}{0pt}
\definecolor{dialinecolor}{rgb}{1.000000, 1.000000, 1.000000}
\pgfsetfillcolor{dialinecolor}
\fill (-54.175500\du,12.000000\du)--(-54.175500\du,13.400000\du)--(-44.820500\du,13.400000\du)--(-44.820500\du,12.000000\du)--cycle;
\definecolor{dialinecolor}{rgb}{0.000000, 0.000000, 0.000000}
\pgfsetstrokecolor{dialinecolor}
\draw (-54.175500\du,12.000000\du)--(-54.175500\du,13.400000\du)--(-44.820500\du,13.400000\du)--(-44.820500\du,12.000000\du)--cycle;
% setfont left to latex
\definecolor{dialinecolor}{rgb}{0.000000, 0.000000, 0.000000}
\pgfsetstrokecolor{dialinecolor}
\node at (-49.498000\du,12.950000\du){CloseObserver};
\definecolor{dialinecolor}{rgb}{1.000000, 1.000000, 1.000000}
\pgfsetfillcolor{dialinecolor}
\fill (-54.175500\du,13.400000\du)--(-54.175500\du,13.800000\du)--(-44.820500\du,13.800000\du)--(-44.820500\du,13.400000\du)--cycle;
\definecolor{dialinecolor}{rgb}{0.000000, 0.000000, 0.000000}
\pgfsetstrokecolor{dialinecolor}
\draw (-54.175500\du,13.400000\du)--(-54.175500\du,13.800000\du)--(-44.820500\du,13.800000\du)--(-44.820500\du,13.400000\du)--cycle;
\definecolor{dialinecolor}{rgb}{1.000000, 1.000000, 1.000000}
\pgfsetfillcolor{dialinecolor}
\fill (-54.175500\du,13.800000\du)--(-54.175500\du,14.800000\du)--(-44.820500\du,14.800000\du)--(-44.820500\du,13.800000\du)--cycle;
\definecolor{dialinecolor}{rgb}{0.000000, 0.000000, 0.000000}
\pgfsetstrokecolor{dialinecolor}
\draw (-54.175500\du,13.800000\du)--(-54.175500\du,14.800000\du)--(-44.820500\du,14.800000\du)--(-44.820500\du,13.800000\du)--cycle;
% setfont left to latex
\definecolor{dialinecolor}{rgb}{0.000000, 0.000000, 0.000000}
\pgfsetstrokecolor{dialinecolor}
\node[anchor=west] at (-54.025500\du,14.500000\du){+updateClose(e:Element)};
\pgfsetlinewidth{0.100000\du}
\pgfsetdash{{1.000000\du}{1.000000\du}}{0\du}
\pgfsetdash{{0.400000\du}{0.400000\du}}{0\du}
\pgfsetmiterjoin
\pgfsetbuttcap
{
\definecolor{dialinecolor}{rgb}{0.000000, 0.000000, 0.000000}
\pgfsetfillcolor{dialinecolor}
% was here!!!
\definecolor{dialinecolor}{rgb}{0.000000, 0.000000, 0.000000}
\pgfsetstrokecolor{dialinecolor}
\draw (-38.130000\du,14.849121\du)--(-38.130000\du,21.000000\du)--(-23.745000\du,21.000000\du)--(-23.745000\du,23.950305\du);
}
\definecolor{dialinecolor}{rgb}{0.000000, 0.000000, 0.000000}
\pgfsetstrokecolor{dialinecolor}
\draw (-38.130000\du,15.760924\du)--(-38.130000\du,21.000000\du)--(-23.745000\du,21.000000\du)--(-23.745000\du,23.950305\du);
\pgfsetmiterjoin
\definecolor{dialinecolor}{rgb}{1.000000, 1.000000, 1.000000}
\pgfsetfillcolor{dialinecolor}
\fill (-37.730000\du,15.760924\du)--(-38.130000\du,14.960924\du)--(-38.530000\du,15.760924\du)--cycle;
\pgfsetlinewidth{0.100000\du}
\pgfsetdash{}{0pt}
\pgfsetmiterjoin
\definecolor{dialinecolor}{rgb}{0.000000, 0.000000, 0.000000}
\pgfsetstrokecolor{dialinecolor}
\draw (-37.730000\du,15.760924\du)--(-38.130000\du,14.960924\du)--(-38.530000\du,15.760924\du)--cycle;
% setfont left to latex
\pgfsetlinewidth{0.100000\du}
\pgfsetdash{{0.400000\du}{0.400000\du}}{0\du}
\pgfsetdash{{0.400000\du}{0.400000\du}}{0\du}
\pgfsetmiterjoin
\pgfsetbuttcap
{
\definecolor{dialinecolor}{rgb}{0.000000, 0.000000, 0.000000}
\pgfsetfillcolor{dialinecolor}
% was here!!!
\definecolor{dialinecolor}{rgb}{0.000000, 0.000000, 0.000000}
\pgfsetstrokecolor{dialinecolor}
\draw (-16.130000\du,14.849707\du)--(-16.130000\du,19.754200\du)--(-23.745000\du,19.754200\du)--(-23.745000\du,23.952715\du);
}
\definecolor{dialinecolor}{rgb}{0.000000, 0.000000, 0.000000}
\pgfsetstrokecolor{dialinecolor}
\draw (-16.130000\du,15.761510\du)--(-16.130000\du,19.754200\du)--(-23.745000\du,19.754200\du)--(-23.745000\du,23.952715\du);
\pgfsetmiterjoin
\definecolor{dialinecolor}{rgb}{1.000000, 1.000000, 1.000000}
\pgfsetfillcolor{dialinecolor}
\fill (-15.730000\du,15.761510\du)--(-16.130000\du,14.961510\du)--(-16.530000\du,15.761510\du)--cycle;
\pgfsetlinewidth{0.100000\du}
\pgfsetdash{}{0pt}
\pgfsetmiterjoin
\definecolor{dialinecolor}{rgb}{0.000000, 0.000000, 0.000000}
\pgfsetstrokecolor{dialinecolor}
\draw (-15.730000\du,15.761510\du)--(-16.130000\du,14.961510\du)--(-16.530000\du,15.761510\du)--cycle;
% setfont left to latex
\pgfsetlinewidth{0.100000\du}
\pgfsetdash{{0.400000\du}{0.400000\du}}{0\du}
\pgfsetdash{{0.400000\du}{0.400000\du}}{0\du}
\pgfsetmiterjoin
\pgfsetbuttcap
{
\definecolor{dialinecolor}{rgb}{0.000000, 0.000000, 0.000000}
\pgfsetfillcolor{dialinecolor}
% was here!!!
\definecolor{dialinecolor}{rgb}{0.000000, 0.000000, 0.000000}
\pgfsetstrokecolor{dialinecolor}
\draw (-26.515000\du,14.849121\du)--(-26.515000\du,21.000000\du)--(-23.745000\du,21.000000\du)--(-23.745000\du,23.950305\du);
}
\definecolor{dialinecolor}{rgb}{0.000000, 0.000000, 0.000000}
\pgfsetstrokecolor{dialinecolor}
\draw (-26.515000\du,15.760924\du)--(-26.515000\du,21.000000\du)--(-23.745000\du,21.000000\du)--(-23.745000\du,23.950305\du);
\pgfsetmiterjoin
\definecolor{dialinecolor}{rgb}{1.000000, 1.000000, 1.000000}
\pgfsetfillcolor{dialinecolor}
\fill (-26.115000\du,15.760924\du)--(-26.515000\du,14.960924\du)--(-26.915000\du,15.760924\du)--cycle;
\pgfsetlinewidth{0.100000\du}
\pgfsetdash{}{0pt}
\pgfsetmiterjoin
\definecolor{dialinecolor}{rgb}{0.000000, 0.000000, 0.000000}
\pgfsetstrokecolor{dialinecolor}
\draw (-26.115000\du,15.760924\du)--(-26.515000\du,14.960924\du)--(-26.915000\du,15.760924\du)--cycle;
% setfont left to latex
\pgfsetlinewidth{0.100000\du}
\pgfsetdash{{0.400000\du}{0.400000\du}}{0\du}
\pgfsetdash{{0.400000\du}{0.400000\du}}{0\du}
\pgfsetmiterjoin
\pgfsetbuttcap
{
\definecolor{dialinecolor}{rgb}{0.000000, 0.000000, 0.000000}
\pgfsetfillcolor{dialinecolor}
% was here!!!
\definecolor{dialinecolor}{rgb}{0.000000, 0.000000, 0.000000}
\pgfsetstrokecolor{dialinecolor}
\draw (-49.498000\du,14.849121\du)--(-49.498000\du,21.000000\du)--(-23.745000\du,21.000000\du)--(-23.745000\du,23.950305\du);
}
\definecolor{dialinecolor}{rgb}{0.000000, 0.000000, 0.000000}
\pgfsetstrokecolor{dialinecolor}
\draw (-49.498000\du,15.760924\du)--(-49.498000\du,21.000000\du)--(-23.745000\du,21.000000\du)--(-23.745000\du,23.950305\du);
\pgfsetmiterjoin
\definecolor{dialinecolor}{rgb}{1.000000, 1.000000, 1.000000}
\pgfsetfillcolor{dialinecolor}
\fill (-49.098000\du,15.760924\du)--(-49.498000\du,14.960924\du)--(-49.898000\du,15.760924\du)--cycle;
\pgfsetlinewidth{0.100000\du}
\pgfsetdash{}{0pt}
\pgfsetmiterjoin
\definecolor{dialinecolor}{rgb}{0.000000, 0.000000, 0.000000}
\pgfsetstrokecolor{dialinecolor}
\draw (-49.098000\du,15.760924\du)--(-49.498000\du,14.960924\du)--(-49.898000\du,15.760924\du)--cycle;
% setfont left to latex
\pgfsetlinewidth{0.100000\du}
\pgfsetdash{{0.400000\du}{0.400000\du}}{0\du}
\pgfsetdash{{0.400000\du}{0.400000\du}}{0\du}
\pgfsetmiterjoin
\pgfsetbuttcap
{
\definecolor{dialinecolor}{rgb}{0.000000, 0.000000, 0.000000}
\pgfsetfillcolor{dialinecolor}
% was here!!!
\definecolor{dialinecolor}{rgb}{0.000000, 0.000000, 0.000000}
\pgfsetstrokecolor{dialinecolor}
\draw (-4.745000\du,14.849121\du)--(-4.745000\du,21.000000\du)--(-23.745000\du,21.000000\du)--(-23.745000\du,23.950305\du);
}
\definecolor{dialinecolor}{rgb}{0.000000, 0.000000, 0.000000}
\pgfsetstrokecolor{dialinecolor}
\draw (-4.745000\du,15.760924\du)--(-4.745000\du,21.000000\du)--(-23.745000\du,21.000000\du)--(-23.745000\du,23.950305\du);
\pgfsetmiterjoin
\definecolor{dialinecolor}{rgb}{1.000000, 1.000000, 1.000000}
\pgfsetfillcolor{dialinecolor}
\fill (-4.345000\du,15.760924\du)--(-4.745000\du,14.960924\du)--(-5.145000\du,15.760924\du)--cycle;
\pgfsetlinewidth{0.100000\du}
\pgfsetdash{}{0pt}
\pgfsetmiterjoin
\definecolor{dialinecolor}{rgb}{0.000000, 0.000000, 0.000000}
\pgfsetstrokecolor{dialinecolor}
\draw (-4.345000\du,15.760924\du)--(-4.745000\du,14.960924\du)--(-5.145000\du,15.760924\du)--cycle;
% setfont left to latex
\pgfsetlinewidth{0.100000\du}
\pgfsetdash{}{0pt}
\pgfsetmiterjoin
\pgfsetbuttcap
{
\definecolor{dialinecolor}{rgb}{0.000000, 0.000000, 0.000000}
\pgfsetfillcolor{dialinecolor}
% was here!!!
\definecolor{dialinecolor}{rgb}{0.000000, 0.000000, 0.000000}
\pgfsetstrokecolor{dialinecolor}
\draw (-25.795000\du,-5.900000\du)--(-12.000000\du,-5.900000\du)--(-12.000000\du,-6.018780\du)--(18.557700\du,-6.018780\du);
}
\definecolor{dialinecolor}{rgb}{0.000000, 0.000000, 0.000000}
\pgfsetstrokecolor{dialinecolor}
\draw (-24.883197\du,-5.900000\du)--(-12.000000\du,-5.900000\du)--(-12.000000\du,-6.018780\du)--(18.557700\du,-6.018780\du);
\pgfsetmiterjoin
\definecolor{dialinecolor}{rgb}{1.000000, 1.000000, 1.000000}
\pgfsetfillcolor{dialinecolor}
\fill (-24.883197\du,-6.300000\du)--(-25.683197\du,-5.900000\du)--(-24.883197\du,-5.500000\du)--cycle;
\pgfsetlinewidth{0.100000\du}
\pgfsetdash{}{0pt}
\pgfsetmiterjoin
\definecolor{dialinecolor}{rgb}{0.000000, 0.000000, 0.000000}
\pgfsetstrokecolor{dialinecolor}
\draw (-24.883197\du,-6.300000\du)--(-25.683197\du,-5.900000\du)--(-24.883197\du,-5.500000\du)--cycle;
% setfont left to latex
\pgfsetlinewidth{0.100000\du}
\pgfsetdash{}{0pt}
\pgfsetmiterjoin
\pgfsetbuttcap
{
\definecolor{dialinecolor}{rgb}{0.000000, 0.000000, 0.000000}
\pgfsetfillcolor{dialinecolor}
% was here!!!
\pgfsetarrowsend{to}
\definecolor{dialinecolor}{rgb}{0.000000, 0.000000, 0.000000}
\pgfsetstrokecolor{dialinecolor}
\draw (-32.397500\du,2.649512\du)--(-32.397500\du,8.000000\du)--(-49.498000\du,8.000000\du)--(-49.498000\du,12.000000\du);
}
% setfont left to latex
\definecolor{dialinecolor}{rgb}{0.000000, 0.000000, 0.000000}
\pgfsetstrokecolor{dialinecolor}
\node at (-40.947750\du,7.850000\du){};
\definecolor{dialinecolor}{rgb}{0.000000, 0.000000, 0.000000}
\pgfsetstrokecolor{dialinecolor}
\node[anchor=west] at (-32.197500\du,3.249512\du){};
\definecolor{dialinecolor}{rgb}{0.000000, 0.000000, 0.000000}
\pgfsetstrokecolor{dialinecolor}
\node[anchor=west] at (-48.948000\du,11.800000\du){*};
\pgfsetlinewidth{0.100000\du}
\pgfsetdash{}{0pt}
\pgfsetmiterjoin
\pgfsetbuttcap
{
\definecolor{dialinecolor}{rgb}{0.000000, 0.000000, 0.000000}
\pgfsetfillcolor{dialinecolor}
% was here!!!
\pgfsetarrowsend{to}
\definecolor{dialinecolor}{rgb}{0.000000, 0.000000, 0.000000}
\pgfsetstrokecolor{dialinecolor}
\draw (-32.397500\du,2.649512\du)--(-32.397500\du,8.000000\du)--(-16.130000\du,8.000000\du)--(-16.130000\du,12.000000\du);
}
% setfont left to latex
\definecolor{dialinecolor}{rgb}{0.000000, 0.000000, 0.000000}
\pgfsetstrokecolor{dialinecolor}
\node at (-24.263750\du,7.850000\du){};
\definecolor{dialinecolor}{rgb}{0.000000, 0.000000, 0.000000}
\pgfsetstrokecolor{dialinecolor}
\node[anchor=west] at (-32.197500\du,3.249512\du){};
\definecolor{dialinecolor}{rgb}{0.000000, 0.000000, 0.000000}
\pgfsetstrokecolor{dialinecolor}
\node[anchor=west] at (-15.580000\du,11.800000\du){*};
\pgfsetlinewidth{0.100000\du}
\pgfsetdash{}{0pt}
\pgfsetmiterjoin
\pgfsetbuttcap
{
\definecolor{dialinecolor}{rgb}{0.000000, 0.000000, 0.000000}
\pgfsetfillcolor{dialinecolor}
% was here!!!
\pgfsetarrowsend{to}
\definecolor{dialinecolor}{rgb}{0.000000, 0.000000, 0.000000}
\pgfsetstrokecolor{dialinecolor}
\draw (9.000000\du,12.000000\du)--(9.000000\du,9.000000\du)--(-4.745000\du,9.000000\du)--(-4.745000\du,12.000000\du);
}
% setfont left to latex
\definecolor{dialinecolor}{rgb}{0.000000, 0.000000, 0.000000}
\pgfsetstrokecolor{dialinecolor}
\node at (2.127500\du,8.850000\du){};
\definecolor{dialinecolor}{rgb}{0.000000, 0.000000, 0.000000}
\pgfsetstrokecolor{dialinecolor}
\node[anchor=west] at (9.200000\du,11.000000\du){ observe};
\definecolor{dialinecolor}{rgb}{0.000000, 0.000000, 0.000000}
\pgfsetstrokecolor{dialinecolor}
\node[anchor=west] at (9.200000\du,11.800000\du){*};
\definecolor{dialinecolor}{rgb}{0.000000, 0.000000, 0.000000}
\pgfsetstrokecolor{dialinecolor}
\node[anchor=west] at (-4.195000\du,11.800000\du){*};
\pgfsetlinewidth{0.100000\du}
\pgfsetdash{}{0pt}
\pgfsetmiterjoin
\pgfsetbuttcap
{
\definecolor{dialinecolor}{rgb}{0.000000, 0.000000, 0.000000}
\pgfsetfillcolor{dialinecolor}
% was here!!!
\pgfsetarrowsend{to}
\definecolor{dialinecolor}{rgb}{0.000000, 0.000000, 0.000000}
\pgfsetstrokecolor{dialinecolor}
\draw (-32.397500\du,2.649512\du)--(-32.397500\du,8.000000\du)--(-26.515000\du,8.000000\du)--(-26.515000\du,11.952441\du);
}
% setfont left to latex
\definecolor{dialinecolor}{rgb}{0.000000, 0.000000, 0.000000}
\pgfsetstrokecolor{dialinecolor}
\node at (-29.456250\du,7.850000\du){};
\definecolor{dialinecolor}{rgb}{0.000000, 0.000000, 0.000000}
\pgfsetstrokecolor{dialinecolor}
\node[anchor=west] at (-32.197500\du,3.249512\du){};
\definecolor{dialinecolor}{rgb}{0.000000, 0.000000, 0.000000}
\pgfsetstrokecolor{dialinecolor}
\node[anchor=west] at (-25.965000\du,11.752441\du){*};
\pgfsetlinewidth{0.100000\du}
\pgfsetdash{}{0pt}
\pgfsetmiterjoin
\pgfsetbuttcap
{
\definecolor{dialinecolor}{rgb}{0.000000, 0.000000, 0.000000}
\pgfsetfillcolor{dialinecolor}
% was here!!!
\pgfsetarrowsend{to}
\definecolor{dialinecolor}{rgb}{0.000000, 0.000000, 0.000000}
\pgfsetstrokecolor{dialinecolor}
\draw (-32.397500\du,2.649512\du)--(-32.397500\du,8.000000\du)--(-38.130000\du,8.000000\du)--(-38.130000\du,12.000000\du);
}
% setfont left to latex
\definecolor{dialinecolor}{rgb}{0.000000, 0.000000, 0.000000}
\pgfsetstrokecolor{dialinecolor}
\node at (-35.263750\du,7.850000\du){};
\definecolor{dialinecolor}{rgb}{0.000000, 0.000000, 0.000000}
\pgfsetstrokecolor{dialinecolor}
\node[anchor=west] at (-32.197500\du,3.249512\du){1};
\definecolor{dialinecolor}{rgb}{0.000000, 0.000000, 0.000000}
\pgfsetstrokecolor{dialinecolor}
\node[anchor=west] at (-37.580000\du,11.800000\du){*};
\pgfsetlinewidth{0.100000\du}
\pgfsetdash{}{0pt}
\pgfsetmiterjoin
\pgfsetbuttcap
{
\definecolor{dialinecolor}{rgb}{0.000000, 0.000000, 0.000000}
\pgfsetfillcolor{dialinecolor}
% was here!!!
\pgfsetarrowsend{to}
\definecolor{dialinecolor}{rgb}{0.000000, 0.000000, 0.000000}
\pgfsetstrokecolor{dialinecolor}
\draw (-18.440361\du,26.100000\du)--(4.000000\du,26.100000\du)--(4.000000\du,14.300000\du)--(7.373110\du,14.300000\du);
}
% setfont left to latex
\definecolor{dialinecolor}{rgb}{0.000000, 0.000000, 0.000000}
\pgfsetstrokecolor{dialinecolor}
\node[anchor=west] at (4.100000\du,20.050000\du){};
\definecolor{dialinecolor}{rgb}{0.000000, 0.000000, 0.000000}
\pgfsetstrokecolor{dialinecolor}
\node[anchor=west] at (-18.240361\du,25.950000\du){1};
\definecolor{dialinecolor}{rgb}{0.000000, 0.000000, 0.000000}
\pgfsetstrokecolor{dialinecolor}
\node[anchor=east] at (6.373110\du,14.150000\du){ ouvre};
\definecolor{dialinecolor}{rgb}{0.000000, 0.000000, 0.000000}
\pgfsetstrokecolor{dialinecolor}
\node[anchor=east] at (6.373110\du,14.950000\du){*};
\pgfsetlinewidth{0.100000\du}
\pgfsetdash{{0.400000\du}{0.400000\du}}{0\du}
\pgfsetdash{{0.400000\du}{0.400000\du}}{0\du}
\pgfsetmiterjoin
\pgfsetbuttcap
{
\definecolor{dialinecolor}{rgb}{0.000000, 0.000000, 0.000000}
\pgfsetfillcolor{dialinecolor}
% was here!!!
\definecolor{dialinecolor}{rgb}{0.000000, 0.000000, 0.000000}
\pgfsetstrokecolor{dialinecolor}
\draw (-14.000000\du,12.000000\du)--(-14.000000\du,8.000000\du)--(14.168106\du,8.000000\du)--(14.168108\du,11.950171\du);
}
\definecolor{dialinecolor}{rgb}{0.000000, 0.000000, 0.000000}
\pgfsetstrokecolor{dialinecolor}
\draw (-14.000000\du,11.088197\du)--(-14.000000\du,8.000000\du)--(14.168106\du,8.000000\du)--(14.168108\du,11.950171\du);
\pgfsetmiterjoin
\definecolor{dialinecolor}{rgb}{1.000000, 1.000000, 1.000000}
\pgfsetfillcolor{dialinecolor}
\fill (-14.400000\du,11.088197\du)--(-14.000000\du,11.888197\du)--(-13.600000\du,11.088197\du)--cycle;
\pgfsetlinewidth{0.100000\du}
\pgfsetdash{}{0pt}
\pgfsetmiterjoin
\definecolor{dialinecolor}{rgb}{0.000000, 0.000000, 0.000000}
\pgfsetstrokecolor{dialinecolor}
\draw (-14.400000\du,11.088197\du)--(-14.000000\du,11.888197\du)--(-13.600000\du,11.088197\du)--cycle;
% setfont left to latex
\end{tikzpicture}
}
    \caption{Diagramme de classes initial}
  \end{sidewaysfigure}

  Les différents types d'observateurs ont été découpé en plusieurs interfaces au
  lieu d'être combinés en une seule afin de laisser la liberté aux éléments de
  décider à quels types d'observateurs ils désirent s'attacher.

  Par exemple, l'interface \textsf{ActivateObserver} est utile que pour la
  classe \textsf{Dossier}, ce qui évite d'avoir des fichiers activables.

  La classe \textsf{Element} a été séparée de l'interface \textsf{Observable}
  afin de mieux découper le programme en composantes réutilisables.

  \section{Diagramme de classe modifié}
  \begin{sidewaysfigure}
    \centering
    \resizebox{\textwidth}{!}{% Graphic for TeX using PGF
% Title: /home/guillaume/Documents/Université de Montréal/Automne 2014/Génie logiciel/Devoir/devoir-3/diagramme-de-classes-b.dia
% Creator: Dia v0.97.3
% CreationDate: Sat Dec  6 12:51:08 2014
% For: guillaume
% \usepackage{tikz}
% The following commands are not supported in PSTricks at present
% We define them conditionally, so when they are implemented,
% this pgf file will use them.
\ifx\du\undefined
  \newlength{\du}
\fi
\setlength{\du}{15\unitlength}
\begin{tikzpicture}
\pgftransformxscale{1.000000}
\pgftransformyscale{-1.000000}
\definecolor{dialinecolor}{rgb}{0.000000, 0.000000, 0.000000}
\pgfsetstrokecolor{dialinecolor}
\definecolor{dialinecolor}{rgb}{1.000000, 1.000000, 1.000000}
\pgfsetfillcolor{dialinecolor}
\pgfsetlinewidth{0.100000\du}
\pgfsetdash{}{0pt}
\definecolor{dialinecolor}{rgb}{1.000000, 1.000000, 1.000000}
\pgfsetfillcolor{dialinecolor}
\fill (23.000000\du,5.000000\du)--(23.000000\du,6.400000\du)--(32.250000\du,6.400000\du)--(32.250000\du,5.000000\du)--cycle;
\definecolor{dialinecolor}{rgb}{0.000000, 0.000000, 0.000000}
\pgfsetstrokecolor{dialinecolor}
\draw (23.000000\du,5.000000\du)--(23.000000\du,6.400000\du)--(32.250000\du,6.400000\du)--(32.250000\du,5.000000\du)--cycle;
% setfont left to latex
\definecolor{dialinecolor}{rgb}{0.000000, 0.000000, 0.000000}
\pgfsetstrokecolor{dialinecolor}
\node at (27.625000\du,5.950000\du){ElementDecorateur};
\definecolor{dialinecolor}{rgb}{1.000000, 1.000000, 1.000000}
\pgfsetfillcolor{dialinecolor}
\fill (23.000000\du,6.400000\du)--(23.000000\du,6.800000\du)--(32.250000\du,6.800000\du)--(32.250000\du,6.400000\du)--cycle;
\definecolor{dialinecolor}{rgb}{0.000000, 0.000000, 0.000000}
\pgfsetstrokecolor{dialinecolor}
\draw (23.000000\du,6.400000\du)--(23.000000\du,6.800000\du)--(32.250000\du,6.800000\du)--(32.250000\du,6.400000\du)--cycle;
\definecolor{dialinecolor}{rgb}{1.000000, 1.000000, 1.000000}
\pgfsetfillcolor{dialinecolor}
\fill (23.000000\du,6.800000\du)--(23.000000\du,7.200000\du)--(32.250000\du,7.200000\du)--(32.250000\du,6.800000\du)--cycle;
\definecolor{dialinecolor}{rgb}{0.000000, 0.000000, 0.000000}
\pgfsetstrokecolor{dialinecolor}
\draw (23.000000\du,6.800000\du)--(23.000000\du,7.200000\du)--(32.250000\du,7.200000\du)--(32.250000\du,6.800000\du)--cycle;
\pgfsetlinewidth{0.100000\du}
\pgfsetdash{}{0pt}
\pgfsetmiterjoin
\pgfsetbuttcap
{
\definecolor{dialinecolor}{rgb}{0.000000, 0.000000, 0.000000}
\pgfsetfillcolor{dialinecolor}
% was here!!!
\definecolor{dialinecolor}{rgb}{0.000000, 0.000000, 0.000000}
\pgfsetstrokecolor{dialinecolor}
\draw (22.949715\du,6.100000\du)--(18.953893\du,6.100000\du)--(18.953893\du,-1.218780\du)--(15.658071\du,-1.218780\du);
}
\definecolor{dialinecolor}{rgb}{0.000000, 0.000000, 0.000000}
\pgfsetstrokecolor{dialinecolor}
\draw (21.691136\du,6.100000\du)--(18.953893\du,6.100000\du)--(18.953893\du,-1.218780\du)--(15.658071\du,-1.218780\du);
\pgfsetdash{}{0pt}
\pgfsetmiterjoin
\pgfsetbuttcap
\definecolor{dialinecolor}{rgb}{0.000000, 0.000000, 0.000000}
\pgfsetfillcolor{dialinecolor}
\fill (22.949715\du,6.100000\du)--(22.249715\du,6.340000\du)--(21.549715\du,6.100000\du)--(22.249715\du,5.860000\du)--cycle;
\pgfsetlinewidth{0.100000\du}
\pgfsetdash{}{0pt}
\pgfsetmiterjoin
\pgfsetbuttcap
\definecolor{dialinecolor}{rgb}{0.000000, 0.000000, 0.000000}
\pgfsetstrokecolor{dialinecolor}
\draw (22.949715\du,6.100000\du)--(22.249715\du,6.340000\du)--(21.549715\du,6.100000\du)--(22.249715\du,5.860000\du)--cycle;
% setfont left to latex
\definecolor{dialinecolor}{rgb}{0.000000, 0.000000, 0.000000}
\pgfsetstrokecolor{dialinecolor}
\node[anchor=west] at (19.053893\du,2.290610\du){};
\definecolor{dialinecolor}{rgb}{0.000000, 0.000000, 0.000000}
\pgfsetstrokecolor{dialinecolor}
\node[anchor=east] at (21.349715\du,5.950000\du){};
\definecolor{dialinecolor}{rgb}{0.000000, 0.000000, 0.000000}
\pgfsetstrokecolor{dialinecolor}
\node[anchor=west] at (15.858071\du,-1.368780\du){1};
\pgfsetlinewidth{0.100000\du}
\pgfsetdash{}{0pt}
\definecolor{dialinecolor}{rgb}{1.000000, 1.000000, 1.000000}
\pgfsetfillcolor{dialinecolor}
\fill (18.531200\du,17.397200\du)--(18.531200\du,18.797200\du)--(27.886200\du,18.797200\du)--(27.886200\du,17.397200\du)--cycle;
\definecolor{dialinecolor}{rgb}{0.000000, 0.000000, 0.000000}
\pgfsetstrokecolor{dialinecolor}
\draw (18.531200\du,17.397200\du)--(18.531200\du,18.797200\du)--(27.886200\du,18.797200\du)--(27.886200\du,17.397200\du)--cycle;
% setfont left to latex
\definecolor{dialinecolor}{rgb}{0.000000, 0.000000, 0.000000}
\pgfsetstrokecolor{dialinecolor}
\node at (23.208700\du,18.347200\du){ElementIntelligent};
\definecolor{dialinecolor}{rgb}{1.000000, 1.000000, 1.000000}
\pgfsetfillcolor{dialinecolor}
\fill (18.531200\du,18.797200\du)--(18.531200\du,19.197200\du)--(27.886200\du,19.197200\du)--(27.886200\du,18.797200\du)--cycle;
\definecolor{dialinecolor}{rgb}{0.000000, 0.000000, 0.000000}
\pgfsetstrokecolor{dialinecolor}
\draw (18.531200\du,18.797200\du)--(18.531200\du,19.197200\du)--(27.886200\du,19.197200\du)--(27.886200\du,18.797200\du)--cycle;
\definecolor{dialinecolor}{rgb}{1.000000, 1.000000, 1.000000}
\pgfsetfillcolor{dialinecolor}
\fill (18.531200\du,19.197200\du)--(18.531200\du,20.997200\du)--(27.886200\du,20.997200\du)--(27.886200\du,19.197200\du)--cycle;
\definecolor{dialinecolor}{rgb}{0.000000, 0.000000, 0.000000}
\pgfsetstrokecolor{dialinecolor}
\draw (18.531200\du,19.197200\du)--(18.531200\du,20.997200\du)--(27.886200\du,20.997200\du)--(27.886200\du,19.197200\du)--cycle;
% setfont left to latex
\definecolor{dialinecolor}{rgb}{0.000000, 0.000000, 0.000000}
\pgfsetstrokecolor{dialinecolor}
\node[anchor=west] at (18.681200\du,19.897200\du){-autoEvaluer()};
% setfont left to latex
\definecolor{dialinecolor}{rgb}{0.000000, 0.000000, 0.000000}
\pgfsetstrokecolor{dialinecolor}
\node[anchor=west] at (18.681200\du,20.697200\du){-proposerAmelioration()};
\pgfsetlinewidth{0.100000\du}
\pgfsetdash{}{0pt}
\definecolor{dialinecolor}{rgb}{1.000000, 1.000000, 1.000000}
\pgfsetfillcolor{dialinecolor}
\fill (28.549900\du,17.397200\du)--(28.549900\du,18.797200\du)--(36.212400\du,18.797200\du)--(36.212400\du,17.397200\du)--cycle;
\definecolor{dialinecolor}{rgb}{0.000000, 0.000000, 0.000000}
\pgfsetstrokecolor{dialinecolor}
\draw (28.549900\du,17.397200\du)--(28.549900\du,18.797200\du)--(36.212400\du,18.797200\du)--(36.212400\du,17.397200\du)--cycle;
% setfont left to latex
\definecolor{dialinecolor}{rgb}{0.000000, 0.000000, 0.000000}
\pgfsetstrokecolor{dialinecolor}
\node at (32.381150\du,18.347200\du){ElementEvolutif};
\definecolor{dialinecolor}{rgb}{1.000000, 1.000000, 1.000000}
\pgfsetfillcolor{dialinecolor}
\fill (28.549900\du,18.797200\du)--(28.549900\du,19.197200\du)--(36.212400\du,19.197200\du)--(36.212400\du,18.797200\du)--cycle;
\definecolor{dialinecolor}{rgb}{0.000000, 0.000000, 0.000000}
\pgfsetstrokecolor{dialinecolor}
\draw (28.549900\du,18.797200\du)--(28.549900\du,19.197200\du)--(36.212400\du,19.197200\du)--(36.212400\du,18.797200\du)--cycle;
\definecolor{dialinecolor}{rgb}{1.000000, 1.000000, 1.000000}
\pgfsetfillcolor{dialinecolor}
\fill (28.549900\du,19.197200\du)--(28.549900\du,20.197200\du)--(36.212400\du,20.197200\du)--(36.212400\du,19.197200\du)--cycle;
\definecolor{dialinecolor}{rgb}{0.000000, 0.000000, 0.000000}
\pgfsetstrokecolor{dialinecolor}
\draw (28.549900\du,19.197200\du)--(28.549900\du,20.197200\du)--(36.212400\du,20.197200\du)--(36.212400\du,19.197200\du)--cycle;
% setfont left to latex
\definecolor{dialinecolor}{rgb}{0.000000, 0.000000, 0.000000}
\pgfsetstrokecolor{dialinecolor}
\node[anchor=west] at (28.699900\du,19.897200\du){-évoluer()};
\pgfsetlinewidth{0.100000\du}
\pgfsetdash{}{0pt}
\pgfsetmiterjoin
\pgfsetbuttcap
{
\definecolor{dialinecolor}{rgb}{0.000000, 0.000000, 0.000000}
\pgfsetfillcolor{dialinecolor}
% was here!!!
\definecolor{dialinecolor}{rgb}{0.000000, 0.000000, 0.000000}
\pgfsetstrokecolor{dialinecolor}
\draw (27.625000\du,7.250281\du)--(27.625000\du,12.723740\du)--(23.208700\du,12.723740\du)--(23.208700\du,17.397200\du);
}
\definecolor{dialinecolor}{rgb}{0.000000, 0.000000, 0.000000}
\pgfsetstrokecolor{dialinecolor}
\draw (27.625000\du,8.162084\du)--(27.625000\du,12.723740\du)--(23.208700\du,12.723740\du)--(23.208700\du,17.397200\du);
\pgfsetmiterjoin
\definecolor{dialinecolor}{rgb}{1.000000, 1.000000, 1.000000}
\pgfsetfillcolor{dialinecolor}
\fill (28.025000\du,8.162084\du)--(27.625000\du,7.362084\du)--(27.225000\du,8.162084\du)--cycle;
\pgfsetlinewidth{0.100000\du}
\pgfsetdash{}{0pt}
\pgfsetmiterjoin
\definecolor{dialinecolor}{rgb}{0.000000, 0.000000, 0.000000}
\pgfsetstrokecolor{dialinecolor}
\draw (28.025000\du,8.162084\du)--(27.625000\du,7.362084\du)--(27.225000\du,8.162084\du)--cycle;
% setfont left to latex
\pgfsetlinewidth{0.100000\du}
\pgfsetdash{}{0pt}
\pgfsetmiterjoin
\pgfsetbuttcap
{
\definecolor{dialinecolor}{rgb}{0.000000, 0.000000, 0.000000}
\pgfsetfillcolor{dialinecolor}
% was here!!!
\definecolor{dialinecolor}{rgb}{0.000000, 0.000000, 0.000000}
\pgfsetstrokecolor{dialinecolor}
\draw (27.625000\du,7.250281\du)--(27.625000\du,12.698563\du)--(32.381150\du,12.698563\du)--(32.381150\du,17.346846\du);
}
\definecolor{dialinecolor}{rgb}{0.000000, 0.000000, 0.000000}
\pgfsetstrokecolor{dialinecolor}
\draw (27.625000\du,8.162084\du)--(27.625000\du,12.698563\du)--(32.381150\du,12.698563\du)--(32.381150\du,17.346846\du);
\pgfsetmiterjoin
\definecolor{dialinecolor}{rgb}{1.000000, 1.000000, 1.000000}
\pgfsetfillcolor{dialinecolor}
\fill (28.025000\du,8.162084\du)--(27.625000\du,7.362084\du)--(27.225000\du,8.162084\du)--cycle;
\pgfsetlinewidth{0.100000\du}
\pgfsetdash{}{0pt}
\pgfsetmiterjoin
\definecolor{dialinecolor}{rgb}{0.000000, 0.000000, 0.000000}
\pgfsetstrokecolor{dialinecolor}
\draw (28.025000\du,8.162084\du)--(27.625000\du,7.362084\du)--(27.225000\du,8.162084\du)--cycle;
% setfont left to latex
\pgfsetlinewidth{0.100000\du}
\pgfsetdash{}{0pt}
\pgfsetmiterjoin
\pgfsetbuttcap
{
\definecolor{dialinecolor}{rgb}{0.000000, 0.000000, 0.000000}
\pgfsetfillcolor{dialinecolor}
% was here!!!
\definecolor{dialinecolor}{rgb}{0.000000, 0.000000, 0.000000}
\pgfsetstrokecolor{dialinecolor}
\draw (15.607700\du,-3.818780\du)--(24.892100\du,-3.818780\du)--(24.892100\du,5.000000\du)--(27.142500\du,5.000000\du);
}
\definecolor{dialinecolor}{rgb}{0.000000, 0.000000, 0.000000}
\pgfsetstrokecolor{dialinecolor}
\draw (16.519503\du,-3.818780\du)--(24.892100\du,-3.818780\du)--(24.892100\du,5.000000\du)--(27.142500\du,5.000000\du);
\pgfsetmiterjoin
\definecolor{dialinecolor}{rgb}{1.000000, 1.000000, 1.000000}
\pgfsetfillcolor{dialinecolor}
\fill (16.519503\du,-4.218780\du)--(15.719503\du,-3.818780\du)--(16.519503\du,-3.418780\du)--cycle;
\pgfsetlinewidth{0.100000\du}
\pgfsetdash{}{0pt}
\pgfsetmiterjoin
\definecolor{dialinecolor}{rgb}{0.000000, 0.000000, 0.000000}
\pgfsetstrokecolor{dialinecolor}
\draw (16.519503\du,-4.218780\du)--(15.719503\du,-3.818780\du)--(16.519503\du,-3.418780\du)--cycle;
% setfont left to latex
\pgfsetlinewidth{0.100000\du}
\pgfsetdash{}{0pt}
\definecolor{dialinecolor}{rgb}{1.000000, 1.000000, 1.000000}
\pgfsetfillcolor{dialinecolor}
\fill (3.557700\du,-5.718780\du)--(3.557700\du,-4.318780\du)--(15.607700\du,-4.318780\du)--(15.607700\du,-5.718780\du)--cycle;
\definecolor{dialinecolor}{rgb}{0.000000, 0.000000, 0.000000}
\pgfsetstrokecolor{dialinecolor}
\draw (3.557700\du,-5.718780\du)--(3.557700\du,-4.318780\du)--(15.607700\du,-4.318780\du)--(15.607700\du,-5.718780\du)--cycle;
% setfont left to latex
\definecolor{dialinecolor}{rgb}{0.000000, 0.000000, 0.000000}
\pgfsetstrokecolor{dialinecolor}
\node at (9.582700\du,-4.768780\du){Element};
\definecolor{dialinecolor}{rgb}{1.000000, 1.000000, 1.000000}
\pgfsetfillcolor{dialinecolor}
\fill (3.557700\du,-4.318780\du)--(3.557700\du,-0.118780\du)--(15.607700\du,-0.118780\du)--(15.607700\du,-4.318780\du)--cycle;
\definecolor{dialinecolor}{rgb}{0.000000, 0.000000, 0.000000}
\pgfsetstrokecolor{dialinecolor}
\draw (3.557700\du,-4.318780\du)--(3.557700\du,-0.118780\du)--(15.607700\du,-0.118780\du)--(15.607700\du,-4.318780\du)--cycle;
% setfont left to latex
\definecolor{dialinecolor}{rgb}{0.000000, 0.000000, 0.000000}
\pgfsetstrokecolor{dialinecolor}
\node[anchor=west] at (3.707700\du,-3.618780\du){\#Nom};
% setfont left to latex
\definecolor{dialinecolor}{rgb}{0.000000, 0.000000, 0.000000}
\pgfsetstrokecolor{dialinecolor}
\node[anchor=west] at (3.707700\du,-2.818780\du){\#Date de création};
% setfont left to latex
\definecolor{dialinecolor}{rgb}{0.000000, 0.000000, 0.000000}
\pgfsetstrokecolor{dialinecolor}
\node[anchor=west] at (3.707700\du,-2.018780\du){\#Date de dernière modification};
% setfont left to latex
\definecolor{dialinecolor}{rgb}{0.000000, 0.000000, 0.000000}
\pgfsetstrokecolor{dialinecolor}
\node[anchor=west] at (3.707700\du,-1.218780\du){\#Chemin};
% setfont left to latex
\definecolor{dialinecolor}{rgb}{0.000000, 0.000000, 0.000000}
\pgfsetstrokecolor{dialinecolor}
\node[anchor=west] at (3.707700\du,-0.418780\du){\#Ouvert: Boolean = False};
\definecolor{dialinecolor}{rgb}{1.000000, 1.000000, 1.000000}
\pgfsetfillcolor{dialinecolor}
\fill (3.557700\du,-0.118780\du)--(3.557700\du,3.281220\du)--(15.607700\du,3.281220\du)--(15.607700\du,-0.118780\du)--cycle;
\definecolor{dialinecolor}{rgb}{0.000000, 0.000000, 0.000000}
\pgfsetstrokecolor{dialinecolor}
\draw (3.557700\du,-0.118780\du)--(3.557700\du,3.281220\du)--(15.607700\du,3.281220\du)--(15.607700\du,-0.118780\du)--cycle;
% setfont left to latex
\definecolor{dialinecolor}{rgb}{0.000000, 0.000000, 0.000000}
\pgfsetstrokecolor{dialinecolor}
\node[anchor=west] at (3.707700\du,0.581220\du){+taille()};
% setfont left to latex
\definecolor{dialinecolor}{rgb}{0.000000, 0.000000, 0.000000}
\pgfsetstrokecolor{dialinecolor}
\node[anchor=west] at (3.707700\du,1.381220\du){+open()};
% setfont left to latex
\definecolor{dialinecolor}{rgb}{0.000000, 0.000000, 0.000000}
\pgfsetstrokecolor{dialinecolor}
\node[anchor=west] at (3.707700\du,2.181220\du){+close()};
% setfont left to latex
\definecolor{dialinecolor}{rgb}{0.000000, 0.000000, 0.000000}
\pgfsetstrokecolor{dialinecolor}
\node[anchor=west] at (3.707700\du,2.981220\du){+delete()};
\pgfsetlinewidth{0.100000\du}
\pgfsetdash{}{0pt}
\definecolor{dialinecolor}{rgb}{1.000000, 1.000000, 1.000000}
\pgfsetfillcolor{dialinecolor}
\fill (12.745400\du,14.848800\du)--(12.745400\du,16.248800\du)--(16.325400\du,16.248800\du)--(16.325400\du,14.848800\du)--cycle;
\definecolor{dialinecolor}{rgb}{0.000000, 0.000000, 0.000000}
\pgfsetstrokecolor{dialinecolor}
\draw (12.745400\du,14.848800\du)--(12.745400\du,16.248800\du)--(16.325400\du,16.248800\du)--(16.325400\du,14.848800\du)--cycle;
% setfont left to latex
\definecolor{dialinecolor}{rgb}{0.000000, 0.000000, 0.000000}
\pgfsetstrokecolor{dialinecolor}
\node at (14.535400\du,15.798800\du){Fichier};
\definecolor{dialinecolor}{rgb}{1.000000, 1.000000, 1.000000}
\pgfsetfillcolor{dialinecolor}
\fill (12.745400\du,16.248800\du)--(12.745400\du,16.648800\du)--(16.325400\du,16.648800\du)--(16.325400\du,16.248800\du)--cycle;
\definecolor{dialinecolor}{rgb}{0.000000, 0.000000, 0.000000}
\pgfsetstrokecolor{dialinecolor}
\draw (12.745400\du,16.248800\du)--(12.745400\du,16.648800\du)--(16.325400\du,16.648800\du)--(16.325400\du,16.248800\du)--cycle;
\definecolor{dialinecolor}{rgb}{1.000000, 1.000000, 1.000000}
\pgfsetfillcolor{dialinecolor}
\fill (12.745400\du,16.648800\du)--(12.745400\du,17.048800\du)--(16.325400\du,17.048800\du)--(16.325400\du,16.648800\du)--cycle;
\definecolor{dialinecolor}{rgb}{0.000000, 0.000000, 0.000000}
\pgfsetstrokecolor{dialinecolor}
\draw (12.745400\du,16.648800\du)--(12.745400\du,17.048800\du)--(16.325400\du,17.048800\du)--(16.325400\du,16.648800\du)--cycle;
\pgfsetlinewidth{0.100000\du}
\pgfsetdash{}{0pt}
\definecolor{dialinecolor}{rgb}{1.000000, 1.000000, 1.000000}
\pgfsetfillcolor{dialinecolor}
\fill (-7.000000\du,13.000000\du)--(-7.000000\du,14.400000\du)--(6.590000\du,14.400000\du)--(6.590000\du,13.000000\du)--cycle;
\definecolor{dialinecolor}{rgb}{0.000000, 0.000000, 0.000000}
\pgfsetstrokecolor{dialinecolor}
\draw (-7.000000\du,13.000000\du)--(-7.000000\du,14.400000\du)--(6.590000\du,14.400000\du)--(6.590000\du,13.000000\du)--cycle;
% setfont left to latex
\definecolor{dialinecolor}{rgb}{0.000000, 0.000000, 0.000000}
\pgfsetstrokecolor{dialinecolor}
\node at (-0.205000\du,13.950000\du){Dossier};
\definecolor{dialinecolor}{rgb}{1.000000, 1.000000, 1.000000}
\pgfsetfillcolor{dialinecolor}
\fill (-7.000000\du,14.400000\du)--(-7.000000\du,14.800000\du)--(6.590000\du,14.800000\du)--(6.590000\du,14.400000\du)--cycle;
\definecolor{dialinecolor}{rgb}{0.000000, 0.000000, 0.000000}
\pgfsetstrokecolor{dialinecolor}
\draw (-7.000000\du,14.400000\du)--(-7.000000\du,14.800000\du)--(6.590000\du,14.800000\du)--(6.590000\du,14.400000\du)--cycle;
\definecolor{dialinecolor}{rgb}{1.000000, 1.000000, 1.000000}
\pgfsetfillcolor{dialinecolor}
\fill (-7.000000\du,14.800000\du)--(-7.000000\du,18.200000\du)--(6.590000\du,18.200000\du)--(6.590000\du,14.800000\du)--cycle;
\definecolor{dialinecolor}{rgb}{0.000000, 0.000000, 0.000000}
\pgfsetstrokecolor{dialinecolor}
\draw (-7.000000\du,14.800000\du)--(-7.000000\du,18.200000\du)--(6.590000\du,18.200000\du)--(6.590000\du,14.800000\du)--cycle;
% setfont left to latex
\definecolor{dialinecolor}{rgb}{0.000000, 0.000000, 0.000000}
\pgfsetstrokecolor{dialinecolor}
\node[anchor=west] at (-6.850000\du,15.500000\du){+activate()};
% setfont left to latex
\definecolor{dialinecolor}{rgb}{0.000000, 0.000000, 0.000000}
\pgfsetstrokecolor{dialinecolor}
\node[anchor=west] at (-6.850000\du,16.300000\du){+attach(observer:ActivateObserver)};
% setfont left to latex
\definecolor{dialinecolor}{rgb}{0.000000, 0.000000, 0.000000}
\pgfsetstrokecolor{dialinecolor}
\node[anchor=west] at (-6.850000\du,17.100000\du){+detach(observer:ActivateObserver)};
% setfont left to latex
\definecolor{dialinecolor}{rgb}{0.000000, 0.000000, 0.000000}
\pgfsetstrokecolor{dialinecolor}
\node[anchor=west] at (-6.850000\du,17.900000\du){+notifyActivate()};
\pgfsetlinewidth{0.100000\du}
\pgfsetdash{}{0pt}
\definecolor{dialinecolor}{rgb}{1.000000, 1.000000, 1.000000}
\pgfsetfillcolor{dialinecolor}
\fill (-44.000000\du,25.000000\du)--(-44.000000\du,26.400000\du)--(-38.110000\du,26.400000\du)--(-38.110000\du,25.000000\du)--cycle;
\definecolor{dialinecolor}{rgb}{0.000000, 0.000000, 0.000000}
\pgfsetstrokecolor{dialinecolor}
\draw (-44.000000\du,25.000000\du)--(-44.000000\du,26.400000\du)--(-38.110000\du,26.400000\du)--(-38.110000\du,25.000000\du)--cycle;
% setfont left to latex
\definecolor{dialinecolor}{rgb}{0.000000, 0.000000, 0.000000}
\pgfsetstrokecolor{dialinecolor}
\node at (-41.055000\du,25.950000\du){Navigateur};
\definecolor{dialinecolor}{rgb}{1.000000, 1.000000, 1.000000}
\pgfsetfillcolor{dialinecolor}
\fill (-44.000000\du,26.400000\du)--(-44.000000\du,28.200000\du)--(-38.110000\du,28.200000\du)--(-38.110000\du,26.400000\du)--cycle;
\definecolor{dialinecolor}{rgb}{0.000000, 0.000000, 0.000000}
\pgfsetstrokecolor{dialinecolor}
\draw (-44.000000\du,26.400000\du)--(-44.000000\du,28.200000\du)--(-38.110000\du,28.200000\du)--(-38.110000\du,26.400000\du)--cycle;
% setfont left to latex
\definecolor{dialinecolor}{rgb}{0.000000, 0.000000, 0.000000}
\pgfsetstrokecolor{dialinecolor}
\node[anchor=west] at (-43.850000\du,27.100000\du){-instance};
% setfont left to latex
\definecolor{dialinecolor}{rgb}{0.000000, 0.000000, 0.000000}
\pgfsetstrokecolor{dialinecolor}
\node[anchor=west] at (-43.850000\du,27.900000\du){-dossierActif};
\definecolor{dialinecolor}{rgb}{1.000000, 1.000000, 1.000000}
\pgfsetfillcolor{dialinecolor}
\fill (-44.000000\du,28.200000\du)--(-44.000000\du,30.000000\du)--(-38.110000\du,30.000000\du)--(-38.110000\du,28.200000\du)--cycle;
\definecolor{dialinecolor}{rgb}{0.000000, 0.000000, 0.000000}
\pgfsetstrokecolor{dialinecolor}
\draw (-44.000000\du,28.200000\du)--(-44.000000\du,30.000000\du)--(-38.110000\du,30.000000\du)--(-38.110000\du,28.200000\du)--cycle;
% setfont left to latex
\definecolor{dialinecolor}{rgb}{0.000000, 0.000000, 0.000000}
\pgfsetstrokecolor{dialinecolor}
\node[anchor=west] at (-43.850000\du,28.900000\du){+getInstance()};
% setfont left to latex
\definecolor{dialinecolor}{rgb}{0.000000, 0.000000, 0.000000}
\pgfsetstrokecolor{dialinecolor}
\node[anchor=west] at (-43.850000\du,29.700000\du){-Navigateur()};
\pgfsetlinewidth{0.100000\du}
\pgfsetdash{}{0pt}
\pgfsetmiterjoin
\pgfsetbuttcap
{
\definecolor{dialinecolor}{rgb}{0.000000, 0.000000, 0.000000}
\pgfsetfillcolor{dialinecolor}
% was here!!!
\definecolor{dialinecolor}{rgb}{0.000000, 0.000000, 0.000000}
\pgfsetstrokecolor{dialinecolor}
\draw (9.582700\du,3.281220\du)--(9.582700\du,10.000000\du)--(4.076380\du,10.000000\du)--(4.076380\du,12.809900\du);
}
\definecolor{dialinecolor}{rgb}{0.000000, 0.000000, 0.000000}
\pgfsetstrokecolor{dialinecolor}
\draw (9.582700\du,4.193023\du)--(9.582700\du,10.000000\du)--(4.076380\du,10.000000\du)--(4.076380\du,12.809900\du);
\pgfsetmiterjoin
\definecolor{dialinecolor}{rgb}{1.000000, 1.000000, 1.000000}
\pgfsetfillcolor{dialinecolor}
\fill (9.982700\du,4.193023\du)--(9.582700\du,3.393023\du)--(9.182700\du,4.193023\du)--cycle;
\pgfsetlinewidth{0.100000\du}
\pgfsetdash{}{0pt}
\pgfsetmiterjoin
\definecolor{dialinecolor}{rgb}{0.000000, 0.000000, 0.000000}
\pgfsetstrokecolor{dialinecolor}
\draw (9.982700\du,4.193023\du)--(9.582700\du,3.393023\du)--(9.182700\du,4.193023\du)--cycle;
% setfont left to latex
\pgfsetlinewidth{0.100000\du}
\pgfsetdash{}{0pt}
\pgfsetmiterjoin
\pgfsetbuttcap
{
\definecolor{dialinecolor}{rgb}{0.000000, 0.000000, 0.000000}
\pgfsetfillcolor{dialinecolor}
% was here!!!
\definecolor{dialinecolor}{rgb}{0.000000, 0.000000, 0.000000}
\pgfsetstrokecolor{dialinecolor}
\draw (9.582700\du,3.281220\du)--(9.582700\du,10.000000\du)--(14.528513\du,10.000000\du)--(14.534075\du,14.804353\du);
}
\definecolor{dialinecolor}{rgb}{0.000000, 0.000000, 0.000000}
\pgfsetstrokecolor{dialinecolor}
\draw (9.582700\du,4.193023\du)--(9.582700\du,10.000000\du)--(14.528513\du,10.000000\du)--(14.534075\du,14.804353\du);
\pgfsetmiterjoin
\definecolor{dialinecolor}{rgb}{1.000000, 1.000000, 1.000000}
\pgfsetfillcolor{dialinecolor}
\fill (9.982700\du,4.193023\du)--(9.582700\du,3.393023\du)--(9.182700\du,4.193023\du)--cycle;
\pgfsetlinewidth{0.100000\du}
\pgfsetdash{}{0pt}
\pgfsetmiterjoin
\definecolor{dialinecolor}{rgb}{0.000000, 0.000000, 0.000000}
\pgfsetstrokecolor{dialinecolor}
\draw (9.982700\du,4.193023\du)--(9.582700\du,3.393023\du)--(9.182700\du,4.193023\du)--cycle;
% setfont left to latex
\pgfsetlinewidth{0.100000\du}
\pgfsetdash{}{0pt}
\definecolor{dialinecolor}{rgb}{1.000000, 1.000000, 1.000000}
\pgfsetfillcolor{dialinecolor}
\fill (-54.000000\du,-8.000000\du)--(-54.000000\du,-6.600000\du)--(-40.795000\du,-6.600000\du)--(-40.795000\du,-8.000000\du)--cycle;
\definecolor{dialinecolor}{rgb}{0.000000, 0.000000, 0.000000}
\pgfsetstrokecolor{dialinecolor}
\draw (-54.000000\du,-8.000000\du)--(-54.000000\du,-6.600000\du)--(-40.795000\du,-6.600000\du)--(-40.795000\du,-8.000000\du)--cycle;
% setfont left to latex
\definecolor{dialinecolor}{rgb}{0.000000, 0.000000, 0.000000}
\pgfsetstrokecolor{dialinecolor}
\node at (-47.397500\du,-7.050000\du){Observable};
\definecolor{dialinecolor}{rgb}{1.000000, 1.000000, 1.000000}
\pgfsetfillcolor{dialinecolor}
\fill (-54.000000\du,-6.600000\du)--(-54.000000\du,-6.200000\du)--(-40.795000\du,-6.200000\du)--(-40.795000\du,-6.600000\du)--cycle;
\definecolor{dialinecolor}{rgb}{0.000000, 0.000000, 0.000000}
\pgfsetstrokecolor{dialinecolor}
\draw (-54.000000\du,-6.600000\du)--(-54.000000\du,-6.200000\du)--(-40.795000\du,-6.200000\du)--(-40.795000\du,-6.600000\du)--cycle;
\definecolor{dialinecolor}{rgb}{1.000000, 1.000000, 1.000000}
\pgfsetfillcolor{dialinecolor}
\fill (-54.000000\du,-6.200000\du)--(-54.000000\du,3.600000\du)--(-40.795000\du,3.600000\du)--(-40.795000\du,-6.200000\du)--cycle;
\definecolor{dialinecolor}{rgb}{0.000000, 0.000000, 0.000000}
\pgfsetstrokecolor{dialinecolor}
\draw (-54.000000\du,-6.200000\du)--(-54.000000\du,3.600000\du)--(-40.795000\du,3.600000\du)--(-40.795000\du,-6.200000\du)--cycle;
% setfont left to latex
\definecolor{dialinecolor}{rgb}{0.000000, 0.000000, 0.000000}
\pgfsetstrokecolor{dialinecolor}
\node[anchor=west] at (-53.850000\du,-5.500000\du){+attach(observer:ChangeObserver)};
% setfont left to latex
\definecolor{dialinecolor}{rgb}{0.000000, 0.000000, 0.000000}
\pgfsetstrokecolor{dialinecolor}
\node[anchor=west] at (-53.850000\du,-4.700000\du){+attach(observer:DeleteObserver)};
% setfont left to latex
\definecolor{dialinecolor}{rgb}{0.000000, 0.000000, 0.000000}
\pgfsetstrokecolor{dialinecolor}
\node[anchor=west] at (-53.850000\du,-3.900000\du){+attach(observer:OpenObserver)};
% setfont left to latex
\definecolor{dialinecolor}{rgb}{0.000000, 0.000000, 0.000000}
\pgfsetstrokecolor{dialinecolor}
\node[anchor=west] at (-53.850000\du,-3.100000\du){+attach(observer:CloseObserver)};
% setfont left to latex
\definecolor{dialinecolor}{rgb}{0.000000, 0.000000, 0.000000}
\pgfsetstrokecolor{dialinecolor}
\node[anchor=west] at (-53.850000\du,-2.300000\du){+dettach(observer:ChangeObserve)};
% setfont left to latex
\definecolor{dialinecolor}{rgb}{0.000000, 0.000000, 0.000000}
\pgfsetstrokecolor{dialinecolor}
\node[anchor=west] at (-53.850000\du,-1.500000\du){+dettach(observer:DeleteObserver)};
% setfont left to latex
\definecolor{dialinecolor}{rgb}{0.000000, 0.000000, 0.000000}
\pgfsetstrokecolor{dialinecolor}
\node[anchor=west] at (-53.850000\du,-0.700000\du){+dettach(observer:OpenObserver)};
% setfont left to latex
\definecolor{dialinecolor}{rgb}{0.000000, 0.000000, 0.000000}
\pgfsetstrokecolor{dialinecolor}
\node[anchor=west] at (-53.850000\du,0.100000\du){+dettach(observer:CloseObserver)};
% setfont left to latex
\definecolor{dialinecolor}{rgb}{0.000000, 0.000000, 0.000000}
\pgfsetstrokecolor{dialinecolor}
\node[anchor=west] at (-53.850000\du,0.900000\du){+notifyChange()};
% setfont left to latex
\definecolor{dialinecolor}{rgb}{0.000000, 0.000000, 0.000000}
\pgfsetstrokecolor{dialinecolor}
\node[anchor=west] at (-53.850000\du,1.700000\du){+notifyDelete()};
% setfont left to latex
\definecolor{dialinecolor}{rgb}{0.000000, 0.000000, 0.000000}
\pgfsetstrokecolor{dialinecolor}
\node[anchor=west] at (-53.850000\du,2.500000\du){+notifyOpen()};
% setfont left to latex
\definecolor{dialinecolor}{rgb}{0.000000, 0.000000, 0.000000}
\pgfsetstrokecolor{dialinecolor}
\node[anchor=west] at (-53.850000\du,3.300000\du){+notifyClose()};
\pgfsetlinewidth{0.100000\du}
\pgfsetdash{}{0pt}
\pgfsetmiterjoin
\pgfsetbuttcap
{
\definecolor{dialinecolor}{rgb}{0.000000, 0.000000, 0.000000}
\pgfsetfillcolor{dialinecolor}
% was here!!!
\definecolor{dialinecolor}{rgb}{0.000000, 0.000000, 0.000000}
\pgfsetstrokecolor{dialinecolor}
\draw (1.003770\du,12.886700\du)--(1.003770\du,12.886700\du)--(1.003770\du,-1.218780\du)--(3.507441\du,-1.218780\du);
}
\definecolor{dialinecolor}{rgb}{0.000000, 0.000000, 0.000000}
\pgfsetstrokecolor{dialinecolor}
\draw (1.003770\du,11.628121\du)--(1.003770\du,-1.218780\du)--(3.507441\du,-1.218780\du);
\pgfsetdash{}{0pt}
\pgfsetmiterjoin
\pgfsetbuttcap
\definecolor{dialinecolor}{rgb}{0.000000, 0.000000, 0.000000}
\pgfsetfillcolor{dialinecolor}
\fill (1.003770\du,12.886700\du)--(0.763770\du,12.186700\du)--(1.003770\du,11.486700\du)--(1.243770\du,12.186700\du)--cycle;
\pgfsetlinewidth{0.100000\du}
\pgfsetdash{}{0pt}
\pgfsetmiterjoin
\pgfsetbuttcap
\definecolor{dialinecolor}{rgb}{0.000000, 0.000000, 0.000000}
\pgfsetstrokecolor{dialinecolor}
\draw (1.003770\du,12.886700\du)--(0.763770\du,12.186700\du)--(1.003770\du,11.486700\du)--(1.243770\du,12.186700\du)--cycle;
% setfont left to latex
\definecolor{dialinecolor}{rgb}{0.000000, 0.000000, 0.000000}
\pgfsetstrokecolor{dialinecolor}
\node[anchor=west] at (1.103770\du,5.683960\du){};
\definecolor{dialinecolor}{rgb}{0.000000, 0.000000, 0.000000}
\pgfsetstrokecolor{dialinecolor}
\node[anchor=west] at (1.553770\du,12.686700\du){};
\definecolor{dialinecolor}{rgb}{0.000000, 0.000000, 0.000000}
\pgfsetstrokecolor{dialinecolor}
\node[anchor=east] at (3.307441\du,-1.368780\du){ possède};
\definecolor{dialinecolor}{rgb}{0.000000, 0.000000, 0.000000}
\pgfsetstrokecolor{dialinecolor}
\node[anchor=east] at (3.307441\du,-0.568780\du){*};
\pgfsetlinewidth{0.100000\du}
\pgfsetdash{}{0pt}
\pgfsetmiterjoin
\pgfsetbuttcap
{
\definecolor{dialinecolor}{rgb}{0.000000, 0.000000, 0.000000}
\pgfsetfillcolor{dialinecolor}
% was here!!!
\pgfsetarrowsend{to}
\definecolor{dialinecolor}{rgb}{0.000000, 0.000000, 0.000000}
\pgfsetstrokecolor{dialinecolor}
\draw (-38.061064\du,27.500000\du)--(11.775000\du,27.500000\du)--(11.775000\du,15.600000\du)--(6.639774\du,15.600000\du);
}
% setfont left to latex
\definecolor{dialinecolor}{rgb}{0.000000, 0.000000, 0.000000}
\pgfsetstrokecolor{dialinecolor}
\node[anchor=west] at (11.875000\du,21.400000\du){};
\definecolor{dialinecolor}{rgb}{0.000000, 0.000000, 0.000000}
\pgfsetstrokecolor{dialinecolor}
\node[anchor=west] at (-37.861064\du,27.350000\du){};
\definecolor{dialinecolor}{rgb}{0.000000, 0.000000, 0.000000}
\pgfsetstrokecolor{dialinecolor}
\node[anchor=west] at (7.639774\du,15.450000\du){ active};
\definecolor{dialinecolor}{rgb}{0.000000, 0.000000, 0.000000}
\pgfsetstrokecolor{dialinecolor}
\node[anchor=west] at (7.639774\du,16.250000\du){0..1};
\pgfsetlinewidth{0.100000\du}
\pgfsetdash{}{0pt}
\definecolor{dialinecolor}{rgb}{1.000000, 1.000000, 1.000000}
\pgfsetfillcolor{dialinecolor}
\fill (-25.000000\du,13.000000\du)--(-25.000000\du,14.400000\du)--(-14.490000\du,14.400000\du)--(-14.490000\du,13.000000\du)--cycle;
\definecolor{dialinecolor}{rgb}{0.000000, 0.000000, 0.000000}
\pgfsetstrokecolor{dialinecolor}
\draw (-25.000000\du,13.000000\du)--(-25.000000\du,14.400000\du)--(-14.490000\du,14.400000\du)--(-14.490000\du,13.000000\du)--cycle;
% setfont left to latex
\definecolor{dialinecolor}{rgb}{0.000000, 0.000000, 0.000000}
\pgfsetstrokecolor{dialinecolor}
\node at (-19.745000\du,13.950000\du){ActivateObserver};
\definecolor{dialinecolor}{rgb}{1.000000, 1.000000, 1.000000}
\pgfsetfillcolor{dialinecolor}
\fill (-25.000000\du,14.400000\du)--(-25.000000\du,14.800000\du)--(-14.490000\du,14.800000\du)--(-14.490000\du,14.400000\du)--cycle;
\definecolor{dialinecolor}{rgb}{0.000000, 0.000000, 0.000000}
\pgfsetstrokecolor{dialinecolor}
\draw (-25.000000\du,14.400000\du)--(-25.000000\du,14.800000\du)--(-14.490000\du,14.800000\du)--(-14.490000\du,14.400000\du)--cycle;
\definecolor{dialinecolor}{rgb}{1.000000, 1.000000, 1.000000}
\pgfsetfillcolor{dialinecolor}
\fill (-25.000000\du,14.800000\du)--(-25.000000\du,15.800000\du)--(-14.490000\du,15.800000\du)--(-14.490000\du,14.800000\du)--cycle;
\definecolor{dialinecolor}{rgb}{0.000000, 0.000000, 0.000000}
\pgfsetstrokecolor{dialinecolor}
\draw (-25.000000\du,14.800000\du)--(-25.000000\du,15.800000\du)--(-14.490000\du,15.800000\du)--(-14.490000\du,14.800000\du)--cycle;
% setfont left to latex
\definecolor{dialinecolor}{rgb}{0.000000, 0.000000, 0.000000}
\pgfsetstrokecolor{dialinecolor}
\node[anchor=west] at (-24.850000\du,15.500000\du){+updateActivate(d:Dossier)};
\pgfsetlinewidth{0.100000\du}
\pgfsetdash{}{0pt}
\definecolor{dialinecolor}{rgb}{1.000000, 1.000000, 1.000000}
\pgfsetfillcolor{dialinecolor}
\fill (-36.000000\du,13.000000\du)--(-36.000000\du,14.400000\du)--(-26.260000\du,14.400000\du)--(-26.260000\du,13.000000\du)--cycle;
\definecolor{dialinecolor}{rgb}{0.000000, 0.000000, 0.000000}
\pgfsetstrokecolor{dialinecolor}
\draw (-36.000000\du,13.000000\du)--(-36.000000\du,14.400000\du)--(-26.260000\du,14.400000\du)--(-26.260000\du,13.000000\du)--cycle;
% setfont left to latex
\definecolor{dialinecolor}{rgb}{0.000000, 0.000000, 0.000000}
\pgfsetstrokecolor{dialinecolor}
\node at (-31.130000\du,13.950000\du){DeleteObserver};
\definecolor{dialinecolor}{rgb}{1.000000, 1.000000, 1.000000}
\pgfsetfillcolor{dialinecolor}
\fill (-36.000000\du,14.400000\du)--(-36.000000\du,14.800000\du)--(-26.260000\du,14.800000\du)--(-26.260000\du,14.400000\du)--cycle;
\definecolor{dialinecolor}{rgb}{0.000000, 0.000000, 0.000000}
\pgfsetstrokecolor{dialinecolor}
\draw (-36.000000\du,14.400000\du)--(-36.000000\du,14.800000\du)--(-26.260000\du,14.800000\du)--(-26.260000\du,14.400000\du)--cycle;
\definecolor{dialinecolor}{rgb}{1.000000, 1.000000, 1.000000}
\pgfsetfillcolor{dialinecolor}
\fill (-36.000000\du,14.800000\du)--(-36.000000\du,15.800000\du)--(-26.260000\du,15.800000\du)--(-26.260000\du,14.800000\du)--cycle;
\definecolor{dialinecolor}{rgb}{0.000000, 0.000000, 0.000000}
\pgfsetstrokecolor{dialinecolor}
\draw (-36.000000\du,14.800000\du)--(-36.000000\du,15.800000\du)--(-26.260000\du,15.800000\du)--(-26.260000\du,14.800000\du)--cycle;
% setfont left to latex
\definecolor{dialinecolor}{rgb}{0.000000, 0.000000, 0.000000}
\pgfsetstrokecolor{dialinecolor}
\node[anchor=west] at (-35.850000\du,15.500000\du){+updateDelete(e:Element)};
\pgfsetlinewidth{0.100000\du}
\pgfsetdash{}{0pt}
\definecolor{dialinecolor}{rgb}{1.000000, 1.000000, 1.000000}
\pgfsetfillcolor{dialinecolor}
\fill (-46.000000\du,13.000000\du)--(-46.000000\du,14.400000\du)--(-37.030000\du,14.400000\du)--(-37.030000\du,13.000000\du)--cycle;
\definecolor{dialinecolor}{rgb}{0.000000, 0.000000, 0.000000}
\pgfsetstrokecolor{dialinecolor}
\draw (-46.000000\du,13.000000\du)--(-46.000000\du,14.400000\du)--(-37.030000\du,14.400000\du)--(-37.030000\du,13.000000\du)--cycle;
% setfont left to latex
\definecolor{dialinecolor}{rgb}{0.000000, 0.000000, 0.000000}
\pgfsetstrokecolor{dialinecolor}
\node at (-41.515000\du,13.950000\du){OpenObserver};
\definecolor{dialinecolor}{rgb}{1.000000, 1.000000, 1.000000}
\pgfsetfillcolor{dialinecolor}
\fill (-46.000000\du,14.400000\du)--(-46.000000\du,14.800000\du)--(-37.030000\du,14.800000\du)--(-37.030000\du,14.400000\du)--cycle;
\definecolor{dialinecolor}{rgb}{0.000000, 0.000000, 0.000000}
\pgfsetstrokecolor{dialinecolor}
\draw (-46.000000\du,14.400000\du)--(-46.000000\du,14.800000\du)--(-37.030000\du,14.800000\du)--(-37.030000\du,14.400000\du)--cycle;
\definecolor{dialinecolor}{rgb}{1.000000, 1.000000, 1.000000}
\pgfsetfillcolor{dialinecolor}
\fill (-46.000000\du,14.800000\du)--(-46.000000\du,15.800000\du)--(-37.030000\du,15.800000\du)--(-37.030000\du,14.800000\du)--cycle;
\definecolor{dialinecolor}{rgb}{0.000000, 0.000000, 0.000000}
\pgfsetstrokecolor{dialinecolor}
\draw (-46.000000\du,14.800000\du)--(-46.000000\du,15.800000\du)--(-37.030000\du,15.800000\du)--(-37.030000\du,14.800000\du)--cycle;
% setfont left to latex
\definecolor{dialinecolor}{rgb}{0.000000, 0.000000, 0.000000}
\pgfsetstrokecolor{dialinecolor}
\node[anchor=west] at (-45.850000\du,15.500000\du){+updateOpen(e:Element)};
\pgfsetlinewidth{0.100000\du}
\pgfsetdash{}{0pt}
\definecolor{dialinecolor}{rgb}{1.000000, 1.000000, 1.000000}
\pgfsetfillcolor{dialinecolor}
\fill (-58.000000\du,13.000000\du)--(-58.000000\du,14.400000\du)--(-48.260000\du,14.400000\du)--(-48.260000\du,13.000000\du)--cycle;
\definecolor{dialinecolor}{rgb}{0.000000, 0.000000, 0.000000}
\pgfsetstrokecolor{dialinecolor}
\draw (-58.000000\du,13.000000\du)--(-58.000000\du,14.400000\du)--(-48.260000\du,14.400000\du)--(-48.260000\du,13.000000\du)--cycle;
% setfont left to latex
\definecolor{dialinecolor}{rgb}{0.000000, 0.000000, 0.000000}
\pgfsetstrokecolor{dialinecolor}
\node at (-53.130000\du,13.950000\du){ChangeObserver};
\definecolor{dialinecolor}{rgb}{1.000000, 1.000000, 1.000000}
\pgfsetfillcolor{dialinecolor}
\fill (-58.000000\du,14.400000\du)--(-58.000000\du,14.800000\du)--(-48.260000\du,14.800000\du)--(-48.260000\du,14.400000\du)--cycle;
\definecolor{dialinecolor}{rgb}{0.000000, 0.000000, 0.000000}
\pgfsetstrokecolor{dialinecolor}
\draw (-58.000000\du,14.400000\du)--(-58.000000\du,14.800000\du)--(-48.260000\du,14.800000\du)--(-48.260000\du,14.400000\du)--cycle;
\definecolor{dialinecolor}{rgb}{1.000000, 1.000000, 1.000000}
\pgfsetfillcolor{dialinecolor}
\fill (-58.000000\du,14.800000\du)--(-58.000000\du,15.800000\du)--(-48.260000\du,15.800000\du)--(-48.260000\du,14.800000\du)--cycle;
\definecolor{dialinecolor}{rgb}{0.000000, 0.000000, 0.000000}
\pgfsetstrokecolor{dialinecolor}
\draw (-58.000000\du,14.800000\du)--(-58.000000\du,15.800000\du)--(-48.260000\du,15.800000\du)--(-48.260000\du,14.800000\du)--cycle;
% setfont left to latex
\definecolor{dialinecolor}{rgb}{0.000000, 0.000000, 0.000000}
\pgfsetstrokecolor{dialinecolor}
\node[anchor=west] at (-57.850000\du,15.500000\du){+updateChange(e:Element)};
\pgfsetlinewidth{0.100000\du}
\pgfsetdash{}{0pt}
\definecolor{dialinecolor}{rgb}{1.000000, 1.000000, 1.000000}
\pgfsetfillcolor{dialinecolor}
\fill (-69.175500\du,13.000000\du)--(-69.175500\du,14.400000\du)--(-59.820500\du,14.400000\du)--(-59.820500\du,13.000000\du)--cycle;
\definecolor{dialinecolor}{rgb}{0.000000, 0.000000, 0.000000}
\pgfsetstrokecolor{dialinecolor}
\draw (-69.175500\du,13.000000\du)--(-69.175500\du,14.400000\du)--(-59.820500\du,14.400000\du)--(-59.820500\du,13.000000\du)--cycle;
% setfont left to latex
\definecolor{dialinecolor}{rgb}{0.000000, 0.000000, 0.000000}
\pgfsetstrokecolor{dialinecolor}
\node at (-64.498000\du,13.950000\du){CloseObserver};
\definecolor{dialinecolor}{rgb}{1.000000, 1.000000, 1.000000}
\pgfsetfillcolor{dialinecolor}
\fill (-69.175500\du,14.400000\du)--(-69.175500\du,14.800000\du)--(-59.820500\du,14.800000\du)--(-59.820500\du,14.400000\du)--cycle;
\definecolor{dialinecolor}{rgb}{0.000000, 0.000000, 0.000000}
\pgfsetstrokecolor{dialinecolor}
\draw (-69.175500\du,14.400000\du)--(-69.175500\du,14.800000\du)--(-59.820500\du,14.800000\du)--(-59.820500\du,14.400000\du)--cycle;
\definecolor{dialinecolor}{rgb}{1.000000, 1.000000, 1.000000}
\pgfsetfillcolor{dialinecolor}
\fill (-69.175500\du,14.800000\du)--(-69.175500\du,15.800000\du)--(-59.820500\du,15.800000\du)--(-59.820500\du,14.800000\du)--cycle;
\definecolor{dialinecolor}{rgb}{0.000000, 0.000000, 0.000000}
\pgfsetstrokecolor{dialinecolor}
\draw (-69.175500\du,14.800000\du)--(-69.175500\du,15.800000\du)--(-59.820500\du,15.800000\du)--(-59.820500\du,14.800000\du)--cycle;
% setfont left to latex
\definecolor{dialinecolor}{rgb}{0.000000, 0.000000, 0.000000}
\pgfsetstrokecolor{dialinecolor}
\node[anchor=west] at (-69.025500\du,15.500000\du){+updateClose(e:Element)};
\pgfsetlinewidth{0.100000\du}
\pgfsetdash{{1.000000\du}{1.000000\du}}{0\du}
\pgfsetdash{{0.400000\du}{0.400000\du}}{0\du}
\pgfsetmiterjoin
\pgfsetbuttcap
{
\definecolor{dialinecolor}{rgb}{0.000000, 0.000000, 0.000000}
\pgfsetfillcolor{dialinecolor}
% was here!!!
\definecolor{dialinecolor}{rgb}{0.000000, 0.000000, 0.000000}
\pgfsetstrokecolor{dialinecolor}
\draw (-53.130000\du,15.849121\du)--(-53.130000\du,22.000000\du)--(-41.055000\du,22.000000\du)--(-41.055000\du,24.949738\du);
}
\definecolor{dialinecolor}{rgb}{0.000000, 0.000000, 0.000000}
\pgfsetstrokecolor{dialinecolor}
\draw (-53.130000\du,16.760924\du)--(-53.130000\du,22.000000\du)--(-41.055000\du,22.000000\du)--(-41.055000\du,24.949738\du);
\pgfsetmiterjoin
\definecolor{dialinecolor}{rgb}{1.000000, 1.000000, 1.000000}
\pgfsetfillcolor{dialinecolor}
\fill (-52.730000\du,16.760924\du)--(-53.130000\du,15.960924\du)--(-53.530000\du,16.760924\du)--cycle;
\pgfsetlinewidth{0.100000\du}
\pgfsetdash{}{0pt}
\pgfsetmiterjoin
\definecolor{dialinecolor}{rgb}{0.000000, 0.000000, 0.000000}
\pgfsetstrokecolor{dialinecolor}
\draw (-52.730000\du,16.760924\du)--(-53.130000\du,15.960924\du)--(-53.530000\du,16.760924\du)--cycle;
% setfont left to latex
\pgfsetlinewidth{0.100000\du}
\pgfsetdash{{0.400000\du}{0.400000\du}}{0\du}
\pgfsetdash{{0.400000\du}{0.400000\du}}{0\du}
\pgfsetmiterjoin
\pgfsetbuttcap
{
\definecolor{dialinecolor}{rgb}{0.000000, 0.000000, 0.000000}
\pgfsetfillcolor{dialinecolor}
% was here!!!
\definecolor{dialinecolor}{rgb}{0.000000, 0.000000, 0.000000}
\pgfsetstrokecolor{dialinecolor}
\draw (-31.130000\du,15.849707\du)--(-31.130000\du,20.754200\du)--(-41.055000\du,20.754200\du)--(-41.055000\du,24.950150\du);
}
\definecolor{dialinecolor}{rgb}{0.000000, 0.000000, 0.000000}
\pgfsetstrokecolor{dialinecolor}
\draw (-31.130000\du,16.761510\du)--(-31.130000\du,20.754200\du)--(-41.055000\du,20.754200\du)--(-41.055000\du,24.950150\du);
\pgfsetmiterjoin
\definecolor{dialinecolor}{rgb}{1.000000, 1.000000, 1.000000}
\pgfsetfillcolor{dialinecolor}
\fill (-30.730000\du,16.761510\du)--(-31.130000\du,15.961510\du)--(-31.530000\du,16.761510\du)--cycle;
\pgfsetlinewidth{0.100000\du}
\pgfsetdash{}{0pt}
\pgfsetmiterjoin
\definecolor{dialinecolor}{rgb}{0.000000, 0.000000, 0.000000}
\pgfsetstrokecolor{dialinecolor}
\draw (-30.730000\du,16.761510\du)--(-31.130000\du,15.961510\du)--(-31.530000\du,16.761510\du)--cycle;
% setfont left to latex
\pgfsetlinewidth{0.100000\du}
\pgfsetdash{{0.400000\du}{0.400000\du}}{0\du}
\pgfsetdash{{0.400000\du}{0.400000\du}}{0\du}
\pgfsetmiterjoin
\pgfsetbuttcap
{
\definecolor{dialinecolor}{rgb}{0.000000, 0.000000, 0.000000}
\pgfsetfillcolor{dialinecolor}
% was here!!!
\definecolor{dialinecolor}{rgb}{0.000000, 0.000000, 0.000000}
\pgfsetstrokecolor{dialinecolor}
\draw (-41.515000\du,15.849121\du)--(-41.515000\du,22.000000\du)--(-41.055000\du,22.000000\du)--(-41.055000\du,24.949738\du);
}
\definecolor{dialinecolor}{rgb}{0.000000, 0.000000, 0.000000}
\pgfsetstrokecolor{dialinecolor}
\draw (-41.515000\du,16.760924\du)--(-41.515000\du,22.000000\du)--(-41.055000\du,22.000000\du)--(-41.055000\du,24.949738\du);
\pgfsetmiterjoin
\definecolor{dialinecolor}{rgb}{1.000000, 1.000000, 1.000000}
\pgfsetfillcolor{dialinecolor}
\fill (-41.115000\du,16.760924\du)--(-41.515000\du,15.960924\du)--(-41.915000\du,16.760924\du)--cycle;
\pgfsetlinewidth{0.100000\du}
\pgfsetdash{}{0pt}
\pgfsetmiterjoin
\definecolor{dialinecolor}{rgb}{0.000000, 0.000000, 0.000000}
\pgfsetstrokecolor{dialinecolor}
\draw (-41.115000\du,16.760924\du)--(-41.515000\du,15.960924\du)--(-41.915000\du,16.760924\du)--cycle;
% setfont left to latex
\pgfsetlinewidth{0.100000\du}
\pgfsetdash{{0.400000\du}{0.400000\du}}{0\du}
\pgfsetdash{{0.400000\du}{0.400000\du}}{0\du}
\pgfsetmiterjoin
\pgfsetbuttcap
{
\definecolor{dialinecolor}{rgb}{0.000000, 0.000000, 0.000000}
\pgfsetfillcolor{dialinecolor}
% was here!!!
\definecolor{dialinecolor}{rgb}{0.000000, 0.000000, 0.000000}
\pgfsetstrokecolor{dialinecolor}
\draw (-64.498000\du,15.849121\du)--(-64.498000\du,22.000000\du)--(-41.055000\du,22.000000\du)--(-41.055000\du,24.949738\du);
}
\definecolor{dialinecolor}{rgb}{0.000000, 0.000000, 0.000000}
\pgfsetstrokecolor{dialinecolor}
\draw (-64.498000\du,16.760924\du)--(-64.498000\du,22.000000\du)--(-41.055000\du,22.000000\du)--(-41.055000\du,24.949738\du);
\pgfsetmiterjoin
\definecolor{dialinecolor}{rgb}{1.000000, 1.000000, 1.000000}
\pgfsetfillcolor{dialinecolor}
\fill (-64.098000\du,16.760924\du)--(-64.498000\du,15.960924\du)--(-64.898000\du,16.760924\du)--cycle;
\pgfsetlinewidth{0.100000\du}
\pgfsetdash{}{0pt}
\pgfsetmiterjoin
\definecolor{dialinecolor}{rgb}{0.000000, 0.000000, 0.000000}
\pgfsetstrokecolor{dialinecolor}
\draw (-64.098000\du,16.760924\du)--(-64.498000\du,15.960924\du)--(-64.898000\du,16.760924\du)--cycle;
% setfont left to latex
\pgfsetlinewidth{0.100000\du}
\pgfsetdash{{0.400000\du}{0.400000\du}}{0\du}
\pgfsetdash{{0.400000\du}{0.400000\du}}{0\du}
\pgfsetmiterjoin
\pgfsetbuttcap
{
\definecolor{dialinecolor}{rgb}{0.000000, 0.000000, 0.000000}
\pgfsetfillcolor{dialinecolor}
% was here!!!
\definecolor{dialinecolor}{rgb}{0.000000, 0.000000, 0.000000}
\pgfsetstrokecolor{dialinecolor}
\draw (-19.745000\du,15.849121\du)--(-19.745000\du,22.000000\du)--(-41.055000\du,22.000000\du)--(-41.055000\du,24.949738\du);
}
\definecolor{dialinecolor}{rgb}{0.000000, 0.000000, 0.000000}
\pgfsetstrokecolor{dialinecolor}
\draw (-19.745000\du,16.760924\du)--(-19.745000\du,22.000000\du)--(-41.055000\du,22.000000\du)--(-41.055000\du,24.949738\du);
\pgfsetmiterjoin
\definecolor{dialinecolor}{rgb}{1.000000, 1.000000, 1.000000}
\pgfsetfillcolor{dialinecolor}
\fill (-19.345000\du,16.760924\du)--(-19.745000\du,15.960924\du)--(-20.145000\du,16.760924\du)--cycle;
\pgfsetlinewidth{0.100000\du}
\pgfsetdash{}{0pt}
\pgfsetmiterjoin
\definecolor{dialinecolor}{rgb}{0.000000, 0.000000, 0.000000}
\pgfsetstrokecolor{dialinecolor}
\draw (-19.345000\du,16.760924\du)--(-19.745000\du,15.960924\du)--(-20.145000\du,16.760924\du)--cycle;
% setfont left to latex
\pgfsetlinewidth{0.100000\du}
\pgfsetdash{}{0pt}
\pgfsetmiterjoin
\pgfsetbuttcap
{
\definecolor{dialinecolor}{rgb}{0.000000, 0.000000, 0.000000}
\pgfsetfillcolor{dialinecolor}
% was here!!!
\definecolor{dialinecolor}{rgb}{0.000000, 0.000000, 0.000000}
\pgfsetstrokecolor{dialinecolor}
\draw (-40.795000\du,-4.900000\du)--(-27.000000\du,-4.900000\du)--(-27.000000\du,-5.018780\du)--(3.557700\du,-5.018780\du);
}
\definecolor{dialinecolor}{rgb}{0.000000, 0.000000, 0.000000}
\pgfsetstrokecolor{dialinecolor}
\draw (-39.883197\du,-4.900000\du)--(-27.000000\du,-4.900000\du)--(-27.000000\du,-5.018780\du)--(3.557700\du,-5.018780\du);
\pgfsetmiterjoin
\definecolor{dialinecolor}{rgb}{1.000000, 1.000000, 1.000000}
\pgfsetfillcolor{dialinecolor}
\fill (-39.883197\du,-5.300000\du)--(-40.683197\du,-4.900000\du)--(-39.883197\du,-4.500000\du)--cycle;
\pgfsetlinewidth{0.100000\du}
\pgfsetdash{}{0pt}
\pgfsetmiterjoin
\definecolor{dialinecolor}{rgb}{0.000000, 0.000000, 0.000000}
\pgfsetstrokecolor{dialinecolor}
\draw (-39.883197\du,-5.300000\du)--(-40.683197\du,-4.900000\du)--(-39.883197\du,-4.500000\du)--cycle;
% setfont left to latex
\pgfsetlinewidth{0.100000\du}
\pgfsetdash{}{0pt}
\pgfsetmiterjoin
\pgfsetbuttcap
{
\definecolor{dialinecolor}{rgb}{0.000000, 0.000000, 0.000000}
\pgfsetfillcolor{dialinecolor}
% was here!!!
\pgfsetarrowsend{to}
\definecolor{dialinecolor}{rgb}{0.000000, 0.000000, 0.000000}
\pgfsetstrokecolor{dialinecolor}
\draw (-47.397500\du,3.649512\du)--(-47.397500\du,9.000000\du)--(-64.498000\du,9.000000\du)--(-64.498000\du,13.000000\du);
}
% setfont left to latex
\definecolor{dialinecolor}{rgb}{0.000000, 0.000000, 0.000000}
\pgfsetstrokecolor{dialinecolor}
\node at (-55.947750\du,8.850000\du){};
\definecolor{dialinecolor}{rgb}{0.000000, 0.000000, 0.000000}
\pgfsetstrokecolor{dialinecolor}
\node[anchor=west] at (-47.197500\du,4.249512\du){};
\definecolor{dialinecolor}{rgb}{0.000000, 0.000000, 0.000000}
\pgfsetstrokecolor{dialinecolor}
\node[anchor=west] at (-63.948000\du,12.800000\du){*};
\pgfsetlinewidth{0.100000\du}
\pgfsetdash{}{0pt}
\pgfsetmiterjoin
\pgfsetbuttcap
{
\definecolor{dialinecolor}{rgb}{0.000000, 0.000000, 0.000000}
\pgfsetfillcolor{dialinecolor}
% was here!!!
\pgfsetarrowsend{to}
\definecolor{dialinecolor}{rgb}{0.000000, 0.000000, 0.000000}
\pgfsetstrokecolor{dialinecolor}
\draw (-47.397500\du,3.649512\du)--(-47.397500\du,9.000000\du)--(-31.130000\du,9.000000\du)--(-31.130000\du,13.000000\du);
}
% setfont left to latex
\definecolor{dialinecolor}{rgb}{0.000000, 0.000000, 0.000000}
\pgfsetstrokecolor{dialinecolor}
\node at (-39.263750\du,8.850000\du){};
\definecolor{dialinecolor}{rgb}{0.000000, 0.000000, 0.000000}
\pgfsetstrokecolor{dialinecolor}
\node[anchor=west] at (-47.197500\du,4.249512\du){};
\definecolor{dialinecolor}{rgb}{0.000000, 0.000000, 0.000000}
\pgfsetstrokecolor{dialinecolor}
\node[anchor=west] at (-30.580000\du,12.800000\du){*};
\pgfsetlinewidth{0.100000\du}
\pgfsetdash{}{0pt}
\pgfsetmiterjoin
\pgfsetbuttcap
{
\definecolor{dialinecolor}{rgb}{0.000000, 0.000000, 0.000000}
\pgfsetfillcolor{dialinecolor}
% was here!!!
\pgfsetarrowsend{to}
\definecolor{dialinecolor}{rgb}{0.000000, 0.000000, 0.000000}
\pgfsetstrokecolor{dialinecolor}
\draw (-6.000000\du,13.000000\du)--(-6.000000\du,10.000000\du)--(-19.745000\du,10.000000\du)--(-19.745000\du,13.000000\du);
}
% setfont left to latex
\definecolor{dialinecolor}{rgb}{0.000000, 0.000000, 0.000000}
\pgfsetstrokecolor{dialinecolor}
\node at (-12.872500\du,9.850000\du){};
\definecolor{dialinecolor}{rgb}{0.000000, 0.000000, 0.000000}
\pgfsetstrokecolor{dialinecolor}
\node[anchor=west] at (-5.800000\du,12.000000\du){ observe};
\definecolor{dialinecolor}{rgb}{0.000000, 0.000000, 0.000000}
\pgfsetstrokecolor{dialinecolor}
\node[anchor=west] at (-5.800000\du,12.800000\du){*};
\definecolor{dialinecolor}{rgb}{0.000000, 0.000000, 0.000000}
\pgfsetstrokecolor{dialinecolor}
\node[anchor=west] at (-19.195000\du,12.800000\du){*};
\pgfsetlinewidth{0.100000\du}
\pgfsetdash{}{0pt}
\pgfsetmiterjoin
\pgfsetbuttcap
{
\definecolor{dialinecolor}{rgb}{0.000000, 0.000000, 0.000000}
\pgfsetfillcolor{dialinecolor}
% was here!!!
\pgfsetarrowsend{to}
\definecolor{dialinecolor}{rgb}{0.000000, 0.000000, 0.000000}
\pgfsetstrokecolor{dialinecolor}
\draw (-47.397500\du,3.649512\du)--(-47.397500\du,9.000000\du)--(-41.515000\du,9.000000\du)--(-41.515000\du,12.952441\du);
}
% setfont left to latex
\definecolor{dialinecolor}{rgb}{0.000000, 0.000000, 0.000000}
\pgfsetstrokecolor{dialinecolor}
\node at (-44.456250\du,8.850000\du){};
\definecolor{dialinecolor}{rgb}{0.000000, 0.000000, 0.000000}
\pgfsetstrokecolor{dialinecolor}
\node[anchor=west] at (-47.197500\du,4.249512\du){};
\definecolor{dialinecolor}{rgb}{0.000000, 0.000000, 0.000000}
\pgfsetstrokecolor{dialinecolor}
\node[anchor=west] at (-40.965000\du,12.752441\du){*};
\pgfsetlinewidth{0.100000\du}
\pgfsetdash{}{0pt}
\pgfsetmiterjoin
\pgfsetbuttcap
{
\definecolor{dialinecolor}{rgb}{0.000000, 0.000000, 0.000000}
\pgfsetfillcolor{dialinecolor}
% was here!!!
\pgfsetarrowsend{to}
\definecolor{dialinecolor}{rgb}{0.000000, 0.000000, 0.000000}
\pgfsetstrokecolor{dialinecolor}
\draw (-47.397500\du,3.649512\du)--(-47.397500\du,9.000000\du)--(-53.130000\du,9.000000\du)--(-53.130000\du,13.000000\du);
}
% setfont left to latex
\definecolor{dialinecolor}{rgb}{0.000000, 0.000000, 0.000000}
\pgfsetstrokecolor{dialinecolor}
\node at (-50.263750\du,8.850000\du){};
\definecolor{dialinecolor}{rgb}{0.000000, 0.000000, 0.000000}
\pgfsetstrokecolor{dialinecolor}
\node[anchor=west] at (-47.197500\du,4.249512\du){1};
\definecolor{dialinecolor}{rgb}{0.000000, 0.000000, 0.000000}
\pgfsetstrokecolor{dialinecolor}
\node[anchor=west] at (-52.580000\du,12.800000\du){*};
\pgfsetlinewidth{0.100000\du}
\pgfsetdash{}{0pt}
\pgfsetmiterjoin
\pgfsetbuttcap
{
\definecolor{dialinecolor}{rgb}{0.000000, 0.000000, 0.000000}
\pgfsetfillcolor{dialinecolor}
% was here!!!
\pgfsetarrowsend{to}
\definecolor{dialinecolor}{rgb}{0.000000, 0.000000, 0.000000}
\pgfsetstrokecolor{dialinecolor}
\draw (-38.060323\du,27.500000\du)--(-11.000000\du,27.500000\du)--(-11.000000\du,15.300000\du)--(-7.000000\du,15.300000\du);
}
% setfont left to latex
\definecolor{dialinecolor}{rgb}{0.000000, 0.000000, 0.000000}
\pgfsetstrokecolor{dialinecolor}
\node[anchor=west] at (-10.900000\du,21.250000\du){};
\definecolor{dialinecolor}{rgb}{0.000000, 0.000000, 0.000000}
\pgfsetstrokecolor{dialinecolor}
\node[anchor=west] at (-37.860323\du,27.350000\du){1};
\definecolor{dialinecolor}{rgb}{0.000000, 0.000000, 0.000000}
\pgfsetstrokecolor{dialinecolor}
\node[anchor=east] at (-8.000000\du,15.150000\du){ ouvre};
\definecolor{dialinecolor}{rgb}{0.000000, 0.000000, 0.000000}
\pgfsetstrokecolor{dialinecolor}
\node[anchor=east] at (-8.000000\du,15.950000\du){*};
\pgfsetlinewidth{0.100000\du}
\pgfsetdash{{0.400000\du}{0.400000\du}}{0\du}
\pgfsetdash{{0.400000\du}{0.400000\du}}{0\du}
\pgfsetmiterjoin
\pgfsetbuttcap
{
\definecolor{dialinecolor}{rgb}{0.000000, 0.000000, 0.000000}
\pgfsetfillcolor{dialinecolor}
% was here!!!
\definecolor{dialinecolor}{rgb}{0.000000, 0.000000, 0.000000}
\pgfsetstrokecolor{dialinecolor}
\draw (-29.000000\du,13.000000\du)--(-29.000000\du,9.000000\du)--(-0.205000\du,9.000000\du)--(-0.205000\du,12.950171\du);
}
\definecolor{dialinecolor}{rgb}{0.000000, 0.000000, 0.000000}
\pgfsetstrokecolor{dialinecolor}
\draw (-29.000000\du,12.088197\du)--(-29.000000\du,9.000000\du)--(-0.205000\du,9.000000\du)--(-0.205000\du,12.950171\du);
\pgfsetmiterjoin
\definecolor{dialinecolor}{rgb}{1.000000, 1.000000, 1.000000}
\pgfsetfillcolor{dialinecolor}
\fill (-29.400000\du,12.088197\du)--(-29.000000\du,12.888197\du)--(-28.600000\du,12.088197\du)--cycle;
\pgfsetlinewidth{0.100000\du}
\pgfsetdash{}{0pt}
\pgfsetmiterjoin
\definecolor{dialinecolor}{rgb}{0.000000, 0.000000, 0.000000}
\pgfsetstrokecolor{dialinecolor}
\draw (-29.400000\du,12.088197\du)--(-29.000000\du,12.888197\du)--(-28.600000\du,12.088197\du)--cycle;
% setfont left to latex
\end{tikzpicture}
}
    \caption{Diagramme de classes}
  \end{sidewaysfigure}

  Le patron décorateur a été implanté simplement par un agrégat et un héritage
  de la classe \textsf{Element}. Il s'agit du seul changement introduit pour
  répondre à l'énoncé.

  Le changement ne pose absolument aucun impact sur les autres classes déjà
  existantes.

  \section{Diagramme de classe avec \textsf{Client}}
  \begin{sidewaysfigure}
    \centering
    \resizebox{\textwidth}{!}{% Graphic for TeX using PGF
% Title: /home/guillaume/Documents/Université de Montréal/Automne 2014/Génie logiciel/Devoir/devoir-3/diagramme-de-classes-c.dia
% Creator: Dia v0.97.3
% CreationDate: Sun Dec  7 16:07:07 2014
% For: guillaume
% \usepackage{tikz}
% The following commands are not supported in PSTricks at present
% We define them conditionally, so when they are implemented,
% this pgf file will use them.
\ifx\du\undefined
  \newlength{\du}
\fi
\setlength{\du}{15\unitlength}
\begin{tikzpicture}
\pgftransformxscale{1.000000}
\pgftransformyscale{-1.000000}
\definecolor{dialinecolor}{rgb}{0.000000, 0.000000, 0.000000}
\pgfsetstrokecolor{dialinecolor}
\definecolor{dialinecolor}{rgb}{1.000000, 1.000000, 1.000000}
\pgfsetfillcolor{dialinecolor}
\pgfsetlinewidth{0.100000\du}
\pgfsetdash{}{0pt}
\definecolor{dialinecolor}{rgb}{1.000000, 1.000000, 1.000000}
\pgfsetfillcolor{dialinecolor}
\fill (0.149826\du,28.411078\du)--(0.149826\du,29.811078\du)--(7.194826\du,29.811078\du)--(7.194826\du,28.411078\du)--cycle;
\definecolor{dialinecolor}{rgb}{0.000000, 0.000000, 0.000000}
\pgfsetstrokecolor{dialinecolor}
\draw (0.149826\du,28.411078\du)--(0.149826\du,29.811078\du)--(7.194826\du,29.811078\du)--(7.194826\du,28.411078\du)--cycle;
% setfont left to latex
\definecolor{dialinecolor}{rgb}{0.000000, 0.000000, 0.000000}
\pgfsetstrokecolor{dialinecolor}
\node at (3.672326\du,29.361078\du){Client};
\definecolor{dialinecolor}{rgb}{1.000000, 1.000000, 1.000000}
\pgfsetfillcolor{dialinecolor}
\fill (0.149826\du,29.811078\du)--(0.149826\du,31.611078\du)--(7.194826\du,31.611078\du)--(7.194826\du,29.811078\du)--cycle;
\definecolor{dialinecolor}{rgb}{0.000000, 0.000000, 0.000000}
\pgfsetstrokecolor{dialinecolor}
\draw (0.149826\du,29.811078\du)--(0.149826\du,31.611078\du)--(7.194826\du,31.611078\du)--(7.194826\du,29.811078\du)--cycle;
% setfont left to latex
\definecolor{dialinecolor}{rgb}{0.000000, 0.000000, 0.000000}
\pgfsetstrokecolor{dialinecolor}
\node[anchor=west] at (0.299826\du,30.511078\du){-username: String};
% setfont left to latex
\definecolor{dialinecolor}{rgb}{0.000000, 0.000000, 0.000000}
\pgfsetstrokecolor{dialinecolor}
\node[anchor=west] at (0.299826\du,31.311078\du){-group: String};
\definecolor{dialinecolor}{rgb}{1.000000, 1.000000, 1.000000}
\pgfsetfillcolor{dialinecolor}
\fill (0.149826\du,31.611078\du)--(0.149826\du,32.011078\du)--(7.194826\du,32.011078\du)--(7.194826\du,31.611078\du)--cycle;
\definecolor{dialinecolor}{rgb}{0.000000, 0.000000, 0.000000}
\pgfsetstrokecolor{dialinecolor}
\draw (0.149826\du,31.611078\du)--(0.149826\du,32.011078\du)--(7.194826\du,32.011078\du)--(7.194826\du,31.611078\du)--cycle;
\pgfsetlinewidth{0.100000\du}
\pgfsetdash{}{0pt}
\pgfsetmiterjoin
\pgfsetbuttcap
{
\definecolor{dialinecolor}{rgb}{0.000000, 0.000000, 0.000000}
\pgfsetfillcolor{dialinecolor}
% was here!!!
\definecolor{dialinecolor}{rgb}{0.000000, 0.000000, 0.000000}
\pgfsetstrokecolor{dialinecolor}
\draw (7.245262\du,30.211078\du)--(13.447469\du,30.211078\du)--(13.447469\du,26.100000\du)--(18.949676\du,26.100000\du);
}
\definecolor{dialinecolor}{rgb}{0.000000, 0.000000, 0.000000}
\pgfsetstrokecolor{dialinecolor}
\draw (8.503841\du,30.211078\du)--(13.447469\du,30.211078\du)--(13.447469\du,26.100000\du)--(18.949676\du,26.100000\du);
\pgfsetdash{}{0pt}
\pgfsetmiterjoin
\pgfsetbuttcap
\definecolor{dialinecolor}{rgb}{0.000000, 0.000000, 0.000000}
\pgfsetfillcolor{dialinecolor}
\fill (7.245262\du,30.211078\du)--(7.945262\du,29.971078\du)--(8.645262\du,30.211078\du)--(7.945262\du,30.451078\du)--cycle;
\pgfsetlinewidth{0.100000\du}
\pgfsetdash{}{0pt}
\pgfsetmiterjoin
\pgfsetbuttcap
\definecolor{dialinecolor}{rgb}{0.000000, 0.000000, 0.000000}
\pgfsetstrokecolor{dialinecolor}
\draw (7.245262\du,30.211078\du)--(7.945262\du,29.971078\du)--(8.645262\du,30.211078\du)--(7.945262\du,30.451078\du)--cycle;
% setfont left to latex
\definecolor{dialinecolor}{rgb}{0.000000, 0.000000, 0.000000}
\pgfsetstrokecolor{dialinecolor}
\node[anchor=west] at (13.547469\du,28.005539\du){};
\definecolor{dialinecolor}{rgb}{0.000000, 0.000000, 0.000000}
\pgfsetstrokecolor{dialinecolor}
\node[anchor=west] at (8.845262\du,30.061078\du){};
\definecolor{dialinecolor}{rgb}{0.000000, 0.000000, 0.000000}
\pgfsetstrokecolor{dialinecolor}
\node[anchor=east] at (18.749676\du,25.950000\du){ utilise};
\definecolor{dialinecolor}{rgb}{0.000000, 0.000000, 0.000000}
\pgfsetstrokecolor{dialinecolor}
\node[anchor=east] at (18.749676\du,26.750000\du){1};
\pgfsetlinewidth{0.100000\du}
\pgfsetdash{}{0pt}
\definecolor{dialinecolor}{rgb}{1.000000, 1.000000, 1.000000}
\pgfsetfillcolor{dialinecolor}
\fill (86.376200\du,3.964670\du)--(86.376200\du,5.364670\du)--(95.626200\du,5.364670\du)--(95.626200\du,3.964670\du)--cycle;
\definecolor{dialinecolor}{rgb}{0.000000, 0.000000, 0.000000}
\pgfsetstrokecolor{dialinecolor}
\draw (86.376200\du,3.964670\du)--(86.376200\du,5.364670\du)--(95.626200\du,5.364670\du)--(95.626200\du,3.964670\du)--cycle;
% setfont left to latex
\definecolor{dialinecolor}{rgb}{0.000000, 0.000000, 0.000000}
\pgfsetstrokecolor{dialinecolor}
\node at (91.001200\du,4.914670\du){ElementDecorateur};
\definecolor{dialinecolor}{rgb}{1.000000, 1.000000, 1.000000}
\pgfsetfillcolor{dialinecolor}
\fill (86.376200\du,5.364670\du)--(86.376200\du,5.764670\du)--(95.626200\du,5.764670\du)--(95.626200\du,5.364670\du)--cycle;
\definecolor{dialinecolor}{rgb}{0.000000, 0.000000, 0.000000}
\pgfsetstrokecolor{dialinecolor}
\draw (86.376200\du,5.364670\du)--(86.376200\du,5.764670\du)--(95.626200\du,5.764670\du)--(95.626200\du,5.364670\du)--cycle;
\definecolor{dialinecolor}{rgb}{1.000000, 1.000000, 1.000000}
\pgfsetfillcolor{dialinecolor}
\fill (86.376200\du,5.764670\du)--(86.376200\du,6.164670\du)--(95.626200\du,6.164670\du)--(95.626200\du,5.764670\du)--cycle;
\definecolor{dialinecolor}{rgb}{0.000000, 0.000000, 0.000000}
\pgfsetstrokecolor{dialinecolor}
\draw (86.376200\du,5.764670\du)--(86.376200\du,6.164670\du)--(95.626200\du,6.164670\du)--(95.626200\du,5.764670\du)--cycle;
\pgfsetlinewidth{0.100000\du}
\pgfsetdash{}{0pt}
\pgfsetmiterjoin
\pgfsetbuttcap
{
\definecolor{dialinecolor}{rgb}{0.000000, 0.000000, 0.000000}
\pgfsetfillcolor{dialinecolor}
% was here!!!
\definecolor{dialinecolor}{rgb}{0.000000, 0.000000, 0.000000}
\pgfsetstrokecolor{dialinecolor}
\draw (86.325915\du,5.064670\du)--(83.485128\du,5.064670\du)--(83.485128\du,-2.254110\du)--(81.344341\du,-2.254110\du);
}
\definecolor{dialinecolor}{rgb}{0.000000, 0.000000, 0.000000}
\pgfsetstrokecolor{dialinecolor}
\draw (85.067336\du,5.064670\du)--(83.485128\du,5.064670\du)--(83.485128\du,-2.254110\du)--(81.344341\du,-2.254110\du);
\pgfsetdash{}{0pt}
\pgfsetmiterjoin
\pgfsetbuttcap
\definecolor{dialinecolor}{rgb}{0.000000, 0.000000, 0.000000}
\pgfsetfillcolor{dialinecolor}
\fill (86.325915\du,5.064670\du)--(85.625915\du,5.304670\du)--(84.925915\du,5.064670\du)--(85.625915\du,4.824670\du)--cycle;
\pgfsetlinewidth{0.100000\du}
\pgfsetdash{}{0pt}
\pgfsetmiterjoin
\pgfsetbuttcap
\definecolor{dialinecolor}{rgb}{0.000000, 0.000000, 0.000000}
\pgfsetstrokecolor{dialinecolor}
\draw (86.325915\du,5.064670\du)--(85.625915\du,5.304670\du)--(84.925915\du,5.064670\du)--(85.625915\du,4.824670\du)--cycle;
% setfont left to latex
\definecolor{dialinecolor}{rgb}{0.000000, 0.000000, 0.000000}
\pgfsetstrokecolor{dialinecolor}
\node[anchor=west] at (83.585128\du,1.255280\du){};
\definecolor{dialinecolor}{rgb}{0.000000, 0.000000, 0.000000}
\pgfsetstrokecolor{dialinecolor}
\node[anchor=east] at (84.725915\du,4.914670\du){};
\definecolor{dialinecolor}{rgb}{0.000000, 0.000000, 0.000000}
\pgfsetstrokecolor{dialinecolor}
\node[anchor=west] at (81.544341\du,-2.404110\du){1};
\pgfsetlinewidth{0.100000\du}
\pgfsetdash{}{0pt}
\definecolor{dialinecolor}{rgb}{1.000000, 1.000000, 1.000000}
\pgfsetfillcolor{dialinecolor}
\fill (81.907400\du,16.361800\du)--(81.907400\du,17.761800\du)--(91.262400\du,17.761800\du)--(91.262400\du,16.361800\du)--cycle;
\definecolor{dialinecolor}{rgb}{0.000000, 0.000000, 0.000000}
\pgfsetstrokecolor{dialinecolor}
\draw (81.907400\du,16.361800\du)--(81.907400\du,17.761800\du)--(91.262400\du,17.761800\du)--(91.262400\du,16.361800\du)--cycle;
% setfont left to latex
\definecolor{dialinecolor}{rgb}{0.000000, 0.000000, 0.000000}
\pgfsetstrokecolor{dialinecolor}
\node at (86.584900\du,17.311800\du){ElementIntelligent};
\definecolor{dialinecolor}{rgb}{1.000000, 1.000000, 1.000000}
\pgfsetfillcolor{dialinecolor}
\fill (81.907400\du,17.761800\du)--(81.907400\du,18.161800\du)--(91.262400\du,18.161800\du)--(91.262400\du,17.761800\du)--cycle;
\definecolor{dialinecolor}{rgb}{0.000000, 0.000000, 0.000000}
\pgfsetstrokecolor{dialinecolor}
\draw (81.907400\du,17.761800\du)--(81.907400\du,18.161800\du)--(91.262400\du,18.161800\du)--(91.262400\du,17.761800\du)--cycle;
\definecolor{dialinecolor}{rgb}{1.000000, 1.000000, 1.000000}
\pgfsetfillcolor{dialinecolor}
\fill (81.907400\du,18.161800\du)--(81.907400\du,19.961800\du)--(91.262400\du,19.961800\du)--(91.262400\du,18.161800\du)--cycle;
\definecolor{dialinecolor}{rgb}{0.000000, 0.000000, 0.000000}
\pgfsetstrokecolor{dialinecolor}
\draw (81.907400\du,18.161800\du)--(81.907400\du,19.961800\du)--(91.262400\du,19.961800\du)--(91.262400\du,18.161800\du)--cycle;
% setfont left to latex
\definecolor{dialinecolor}{rgb}{0.000000, 0.000000, 0.000000}
\pgfsetstrokecolor{dialinecolor}
\node[anchor=west] at (82.057400\du,18.861800\du){-autoEvaluer()};
% setfont left to latex
\definecolor{dialinecolor}{rgb}{0.000000, 0.000000, 0.000000}
\pgfsetstrokecolor{dialinecolor}
\node[anchor=west] at (82.057400\du,19.661800\du){-proposerAmelioration()};
\pgfsetlinewidth{0.100000\du}
\pgfsetdash{}{0pt}
\definecolor{dialinecolor}{rgb}{1.000000, 1.000000, 1.000000}
\pgfsetfillcolor{dialinecolor}
\fill (91.926100\du,16.361800\du)--(91.926100\du,17.761800\du)--(99.588600\du,17.761800\du)--(99.588600\du,16.361800\du)--cycle;
\definecolor{dialinecolor}{rgb}{0.000000, 0.000000, 0.000000}
\pgfsetstrokecolor{dialinecolor}
\draw (91.926100\du,16.361800\du)--(91.926100\du,17.761800\du)--(99.588600\du,17.761800\du)--(99.588600\du,16.361800\du)--cycle;
% setfont left to latex
\definecolor{dialinecolor}{rgb}{0.000000, 0.000000, 0.000000}
\pgfsetstrokecolor{dialinecolor}
\node at (95.757350\du,17.311800\du){ElementEvolutif};
\definecolor{dialinecolor}{rgb}{1.000000, 1.000000, 1.000000}
\pgfsetfillcolor{dialinecolor}
\fill (91.926100\du,17.761800\du)--(91.926100\du,18.161800\du)--(99.588600\du,18.161800\du)--(99.588600\du,17.761800\du)--cycle;
\definecolor{dialinecolor}{rgb}{0.000000, 0.000000, 0.000000}
\pgfsetstrokecolor{dialinecolor}
\draw (91.926100\du,17.761800\du)--(91.926100\du,18.161800\du)--(99.588600\du,18.161800\du)--(99.588600\du,17.761800\du)--cycle;
\definecolor{dialinecolor}{rgb}{1.000000, 1.000000, 1.000000}
\pgfsetfillcolor{dialinecolor}
\fill (91.926100\du,18.161800\du)--(91.926100\du,19.161800\du)--(99.588600\du,19.161800\du)--(99.588600\du,18.161800\du)--cycle;
\definecolor{dialinecolor}{rgb}{0.000000, 0.000000, 0.000000}
\pgfsetstrokecolor{dialinecolor}
\draw (91.926100\du,18.161800\du)--(91.926100\du,19.161800\du)--(99.588600\du,19.161800\du)--(99.588600\du,18.161800\du)--cycle;
% setfont left to latex
\definecolor{dialinecolor}{rgb}{0.000000, 0.000000, 0.000000}
\pgfsetstrokecolor{dialinecolor}
\node[anchor=west] at (92.076100\du,18.861800\du){-évoluer()};
\pgfsetlinewidth{0.100000\du}
\pgfsetdash{}{0pt}
\pgfsetmiterjoin
\pgfsetbuttcap
{
\definecolor{dialinecolor}{rgb}{0.000000, 0.000000, 0.000000}
\pgfsetfillcolor{dialinecolor}
% was here!!!
\definecolor{dialinecolor}{rgb}{0.000000, 0.000000, 0.000000}
\pgfsetstrokecolor{dialinecolor}
\draw (91.001200\du,6.214951\du)--(91.001200\du,11.688375\du)--(86.584900\du,11.688375\du)--(86.584900\du,16.361800\du);
}
\definecolor{dialinecolor}{rgb}{0.000000, 0.000000, 0.000000}
\pgfsetstrokecolor{dialinecolor}
\draw (91.001200\du,7.126754\du)--(91.001200\du,11.688375\du)--(86.584900\du,11.688375\du)--(86.584900\du,16.361800\du);
\pgfsetmiterjoin
\definecolor{dialinecolor}{rgb}{1.000000, 1.000000, 1.000000}
\pgfsetfillcolor{dialinecolor}
\fill (91.401200\du,7.126754\du)--(91.001200\du,6.326754\du)--(90.601200\du,7.126754\du)--cycle;
\pgfsetlinewidth{0.100000\du}
\pgfsetdash{}{0pt}
\pgfsetmiterjoin
\definecolor{dialinecolor}{rgb}{0.000000, 0.000000, 0.000000}
\pgfsetstrokecolor{dialinecolor}
\draw (91.401200\du,7.126754\du)--(91.001200\du,6.326754\du)--(90.601200\du,7.126754\du)--cycle;
% setfont left to latex
\pgfsetlinewidth{0.100000\du}
\pgfsetdash{}{0pt}
\pgfsetmiterjoin
\pgfsetbuttcap
{
\definecolor{dialinecolor}{rgb}{0.000000, 0.000000, 0.000000}
\pgfsetfillcolor{dialinecolor}
% was here!!!
\definecolor{dialinecolor}{rgb}{0.000000, 0.000000, 0.000000}
\pgfsetstrokecolor{dialinecolor}
\draw (91.001200\du,6.214951\du)--(91.001200\du,11.663198\du)--(95.757350\du,11.663198\du)--(95.757350\du,16.311446\du);
}
\definecolor{dialinecolor}{rgb}{0.000000, 0.000000, 0.000000}
\pgfsetstrokecolor{dialinecolor}
\draw (91.001200\du,7.126754\du)--(91.001200\du,11.663198\du)--(95.757350\du,11.663198\du)--(95.757350\du,16.311446\du);
\pgfsetmiterjoin
\definecolor{dialinecolor}{rgb}{1.000000, 1.000000, 1.000000}
\pgfsetfillcolor{dialinecolor}
\fill (91.401200\du,7.126754\du)--(91.001200\du,6.326754\du)--(90.601200\du,7.126754\du)--cycle;
\pgfsetlinewidth{0.100000\du}
\pgfsetdash{}{0pt}
\pgfsetmiterjoin
\definecolor{dialinecolor}{rgb}{0.000000, 0.000000, 0.000000}
\pgfsetstrokecolor{dialinecolor}
\draw (91.401200\du,7.126754\du)--(91.001200\du,6.326754\du)--(90.601200\du,7.126754\du)--cycle;
% setfont left to latex
\pgfsetlinewidth{0.100000\du}
\pgfsetdash{}{0pt}
\pgfsetmiterjoin
\pgfsetbuttcap
{
\definecolor{dialinecolor}{rgb}{0.000000, 0.000000, 0.000000}
\pgfsetfillcolor{dialinecolor}
% was here!!!
\definecolor{dialinecolor}{rgb}{0.000000, 0.000000, 0.000000}
\pgfsetstrokecolor{dialinecolor}
\draw (81.293900\du,-4.854110\du)--(88.268300\du,-4.854110\du)--(88.268300\du,3.964670\du)--(90.518700\du,3.964670\du);
}
\definecolor{dialinecolor}{rgb}{0.000000, 0.000000, 0.000000}
\pgfsetstrokecolor{dialinecolor}
\draw (82.205703\du,-4.854110\du)--(88.268300\du,-4.854110\du)--(88.268300\du,3.964670\du)--(90.518700\du,3.964670\du);
\pgfsetmiterjoin
\definecolor{dialinecolor}{rgb}{1.000000, 1.000000, 1.000000}
\pgfsetfillcolor{dialinecolor}
\fill (82.205703\du,-5.254110\du)--(81.405703\du,-4.854110\du)--(82.205703\du,-4.454110\du)--cycle;
\pgfsetlinewidth{0.100000\du}
\pgfsetdash{}{0pt}
\pgfsetmiterjoin
\definecolor{dialinecolor}{rgb}{0.000000, 0.000000, 0.000000}
\pgfsetstrokecolor{dialinecolor}
\draw (82.205703\du,-5.254110\du)--(81.405703\du,-4.854110\du)--(82.205703\du,-4.454110\du)--cycle;
% setfont left to latex
\pgfsetlinewidth{0.100000\du}
\pgfsetdash{}{0pt}
\definecolor{dialinecolor}{rgb}{1.000000, 1.000000, 1.000000}
\pgfsetfillcolor{dialinecolor}
\fill (66.933900\du,-6.754110\du)--(66.933900\du,-5.354110\du)--(81.293900\du,-5.354110\du)--(81.293900\du,-6.754110\du)--cycle;
\definecolor{dialinecolor}{rgb}{0.000000, 0.000000, 0.000000}
\pgfsetstrokecolor{dialinecolor}
\draw (66.933900\du,-6.754110\du)--(66.933900\du,-5.354110\du)--(81.293900\du,-5.354110\du)--(81.293900\du,-6.754110\du)--cycle;
% setfont left to latex
\definecolor{dialinecolor}{rgb}{0.000000, 0.000000, 0.000000}
\pgfsetstrokecolor{dialinecolor}
\node at (74.113900\du,-5.804110\du){Element};
\definecolor{dialinecolor}{rgb}{1.000000, 1.000000, 1.000000}
\pgfsetfillcolor{dialinecolor}
\fill (66.933900\du,-5.354110\du)--(66.933900\du,-1.154110\du)--(81.293900\du,-1.154110\du)--(81.293900\du,-5.354110\du)--cycle;
\definecolor{dialinecolor}{rgb}{0.000000, 0.000000, 0.000000}
\pgfsetstrokecolor{dialinecolor}
\draw (66.933900\du,-5.354110\du)--(66.933900\du,-1.154110\du)--(81.293900\du,-1.154110\du)--(81.293900\du,-5.354110\du)--cycle;
% setfont left to latex
\definecolor{dialinecolor}{rgb}{0.000000, 0.000000, 0.000000}
\pgfsetstrokecolor{dialinecolor}
\node[anchor=west] at (67.083900\du,-4.654110\du){\#Nom: String};
% setfont left to latex
\definecolor{dialinecolor}{rgb}{0.000000, 0.000000, 0.000000}
\pgfsetstrokecolor{dialinecolor}
\node[anchor=west] at (67.083900\du,-3.854110\du){\#Date de création: Date};
% setfont left to latex
\definecolor{dialinecolor}{rgb}{0.000000, 0.000000, 0.000000}
\pgfsetstrokecolor{dialinecolor}
\node[anchor=west] at (67.083900\du,-3.054110\du){\#Date de dernière modification: Date};
% setfont left to latex
\definecolor{dialinecolor}{rgb}{0.000000, 0.000000, 0.000000}
\pgfsetstrokecolor{dialinecolor}
\node[anchor=west] at (67.083900\du,-2.254110\du){\#Chemin: String};
% setfont left to latex
\definecolor{dialinecolor}{rgb}{0.000000, 0.000000, 0.000000}
\pgfsetstrokecolor{dialinecolor}
\node[anchor=west] at (67.083900\du,-1.454110\du){\#Ouvert: Boolean = False};
\definecolor{dialinecolor}{rgb}{1.000000, 1.000000, 1.000000}
\pgfsetfillcolor{dialinecolor}
\fill (66.933900\du,-1.154110\du)--(66.933900\du,2.245890\du)--(81.293900\du,2.245890\du)--(81.293900\du,-1.154110\du)--cycle;
\definecolor{dialinecolor}{rgb}{0.000000, 0.000000, 0.000000}
\pgfsetstrokecolor{dialinecolor}
\draw (66.933900\du,-1.154110\du)--(66.933900\du,2.245890\du)--(81.293900\du,2.245890\du)--(81.293900\du,-1.154110\du)--cycle;
% setfont left to latex
\definecolor{dialinecolor}{rgb}{0.000000, 0.000000, 0.000000}
\pgfsetstrokecolor{dialinecolor}
\node[anchor=west] at (67.083900\du,-0.454110\du){+open()};
% setfont left to latex
\definecolor{dialinecolor}{rgb}{0.000000, 0.000000, 0.000000}
\pgfsetstrokecolor{dialinecolor}
\node[anchor=west] at (67.083900\du,0.345890\du){+close()};
% setfont left to latex
\definecolor{dialinecolor}{rgb}{0.000000, 0.000000, 0.000000}
\pgfsetstrokecolor{dialinecolor}
\node[anchor=west] at (67.083900\du,1.145890\du){+delete()};
% setfont left to latex
\definecolor{dialinecolor}{rgb}{0.000000, 0.000000, 0.000000}
\pgfsetstrokecolor{dialinecolor}
\node[anchor=west] at (67.083900\du,1.945890\du){+accept(visitor:ElementVisitor)};
\pgfsetlinewidth{0.100000\du}
\pgfsetdash{}{0pt}
\definecolor{dialinecolor}{rgb}{1.000000, 1.000000, 1.000000}
\pgfsetfillcolor{dialinecolor}
\fill (76.121600\du,13.813500\du)--(76.121600\du,15.213500\du)--(79.701600\du,15.213500\du)--(79.701600\du,13.813500\du)--cycle;
\definecolor{dialinecolor}{rgb}{0.000000, 0.000000, 0.000000}
\pgfsetstrokecolor{dialinecolor}
\draw (76.121600\du,13.813500\du)--(76.121600\du,15.213500\du)--(79.701600\du,15.213500\du)--(79.701600\du,13.813500\du)--cycle;
% setfont left to latex
\definecolor{dialinecolor}{rgb}{0.000000, 0.000000, 0.000000}
\pgfsetstrokecolor{dialinecolor}
\node at (77.911600\du,14.763500\du){Fichier};
\definecolor{dialinecolor}{rgb}{1.000000, 1.000000, 1.000000}
\pgfsetfillcolor{dialinecolor}
\fill (76.121600\du,15.213500\du)--(76.121600\du,16.213500\du)--(79.701600\du,16.213500\du)--(79.701600\du,15.213500\du)--cycle;
\definecolor{dialinecolor}{rgb}{0.000000, 0.000000, 0.000000}
\pgfsetstrokecolor{dialinecolor}
\draw (76.121600\du,15.213500\du)--(76.121600\du,16.213500\du)--(79.701600\du,16.213500\du)--(79.701600\du,15.213500\du)--cycle;
% setfont left to latex
\definecolor{dialinecolor}{rgb}{0.000000, 0.000000, 0.000000}
\pgfsetstrokecolor{dialinecolor}
\node[anchor=west] at (76.271600\du,15.913500\du){-taille};
\definecolor{dialinecolor}{rgb}{1.000000, 1.000000, 1.000000}
\pgfsetfillcolor{dialinecolor}
\fill (76.121600\du,16.213500\du)--(76.121600\du,16.613500\du)--(79.701600\du,16.613500\du)--(79.701600\du,16.213500\du)--cycle;
\definecolor{dialinecolor}{rgb}{0.000000, 0.000000, 0.000000}
\pgfsetstrokecolor{dialinecolor}
\draw (76.121600\du,16.213500\du)--(76.121600\du,16.613500\du)--(79.701600\du,16.613500\du)--(79.701600\du,16.213500\du)--cycle;
\pgfsetlinewidth{0.100000\du}
\pgfsetdash{}{0pt}
\definecolor{dialinecolor}{rgb}{1.000000, 1.000000, 1.000000}
\pgfsetfillcolor{dialinecolor}
\fill (56.376200\du,11.964700\du)--(56.376200\du,13.364700\du)--(69.966200\du,13.364700\du)--(69.966200\du,11.964700\du)--cycle;
\definecolor{dialinecolor}{rgb}{0.000000, 0.000000, 0.000000}
\pgfsetstrokecolor{dialinecolor}
\draw (56.376200\du,11.964700\du)--(56.376200\du,13.364700\du)--(69.966200\du,13.364700\du)--(69.966200\du,11.964700\du)--cycle;
% setfont left to latex
\definecolor{dialinecolor}{rgb}{0.000000, 0.000000, 0.000000}
\pgfsetstrokecolor{dialinecolor}
\node at (63.171200\du,12.914700\du){Dossier};
\definecolor{dialinecolor}{rgb}{1.000000, 1.000000, 1.000000}
\pgfsetfillcolor{dialinecolor}
\fill (56.376200\du,13.364700\du)--(56.376200\du,13.764700\du)--(69.966200\du,13.764700\du)--(69.966200\du,13.364700\du)--cycle;
\definecolor{dialinecolor}{rgb}{0.000000, 0.000000, 0.000000}
\pgfsetstrokecolor{dialinecolor}
\draw (56.376200\du,13.364700\du)--(56.376200\du,13.764700\du)--(69.966200\du,13.764700\du)--(69.966200\du,13.364700\du)--cycle;
\definecolor{dialinecolor}{rgb}{1.000000, 1.000000, 1.000000}
\pgfsetfillcolor{dialinecolor}
\fill (56.376200\du,13.764700\du)--(56.376200\du,17.164700\du)--(69.966200\du,17.164700\du)--(69.966200\du,13.764700\du)--cycle;
\definecolor{dialinecolor}{rgb}{0.000000, 0.000000, 0.000000}
\pgfsetstrokecolor{dialinecolor}
\draw (56.376200\du,13.764700\du)--(56.376200\du,17.164700\du)--(69.966200\du,17.164700\du)--(69.966200\du,13.764700\du)--cycle;
% setfont left to latex
\definecolor{dialinecolor}{rgb}{0.000000, 0.000000, 0.000000}
\pgfsetstrokecolor{dialinecolor}
\node[anchor=west] at (56.526200\du,14.464700\du){+activate()};
% setfont left to latex
\definecolor{dialinecolor}{rgb}{0.000000, 0.000000, 0.000000}
\pgfsetstrokecolor{dialinecolor}
\node[anchor=west] at (56.526200\du,15.264700\du){+attach(observer:ActivateObserver)};
% setfont left to latex
\definecolor{dialinecolor}{rgb}{0.000000, 0.000000, 0.000000}
\pgfsetstrokecolor{dialinecolor}
\node[anchor=west] at (56.526200\du,16.064700\du){+detach(observer:ActivateObserver)};
% setfont left to latex
\definecolor{dialinecolor}{rgb}{0.000000, 0.000000, 0.000000}
\pgfsetstrokecolor{dialinecolor}
\node[anchor=west] at (56.526200\du,16.864700\du){+notifyActivate()};
\pgfsetlinewidth{0.100000\du}
\pgfsetdash{}{0pt}
\definecolor{dialinecolor}{rgb}{1.000000, 1.000000, 1.000000}
\pgfsetfillcolor{dialinecolor}
\fill (19.000000\du,24.000000\du)--(19.000000\du,25.400000\du)--(29.510000\du,25.400000\du)--(29.510000\du,24.000000\du)--cycle;
\definecolor{dialinecolor}{rgb}{0.000000, 0.000000, 0.000000}
\pgfsetstrokecolor{dialinecolor}
\draw (19.000000\du,24.000000\du)--(19.000000\du,25.400000\du)--(29.510000\du,25.400000\du)--(29.510000\du,24.000000\du)--cycle;
% setfont left to latex
\definecolor{dialinecolor}{rgb}{0.000000, 0.000000, 0.000000}
\pgfsetstrokecolor{dialinecolor}
\node at (24.255000\du,24.950000\du){Navigateur};
\definecolor{dialinecolor}{rgb}{1.000000, 1.000000, 1.000000}
\pgfsetfillcolor{dialinecolor}
\fill (19.000000\du,25.400000\du)--(19.000000\du,26.400000\du)--(29.510000\du,26.400000\du)--(29.510000\du,25.400000\du)--cycle;
\definecolor{dialinecolor}{rgb}{0.000000, 0.000000, 0.000000}
\pgfsetstrokecolor{dialinecolor}
\draw (19.000000\du,25.400000\du)--(19.000000\du,26.400000\du)--(29.510000\du,26.400000\du)--(29.510000\du,25.400000\du)--cycle;
% setfont left to latex
\definecolor{dialinecolor}{rgb}{0.000000, 0.000000, 0.000000}
\pgfsetstrokecolor{dialinecolor}
\node[anchor=west] at (19.150000\du,26.100000\du){-instance: Navigateur};
\definecolor{dialinecolor}{rgb}{1.000000, 1.000000, 1.000000}
\pgfsetfillcolor{dialinecolor}
\fill (19.000000\du,26.400000\du)--(19.000000\du,28.200000\du)--(29.510000\du,28.200000\du)--(29.510000\du,26.400000\du)--cycle;
\definecolor{dialinecolor}{rgb}{0.000000, 0.000000, 0.000000}
\pgfsetstrokecolor{dialinecolor}
\draw (19.000000\du,26.400000\du)--(19.000000\du,28.200000\du)--(29.510000\du,28.200000\du)--(29.510000\du,26.400000\du)--cycle;
% setfont left to latex
\definecolor{dialinecolor}{rgb}{0.000000, 0.000000, 0.000000}
\pgfsetstrokecolor{dialinecolor}
\node[anchor=west] at (19.150000\du,27.100000\du){+getInstance(): Navigateur};
% setfont left to latex
\definecolor{dialinecolor}{rgb}{0.000000, 0.000000, 0.000000}
\pgfsetstrokecolor{dialinecolor}
\node[anchor=west] at (19.150000\du,27.900000\du){-<<Constructeur>> ()};
\pgfsetlinewidth{0.100000\du}
\pgfsetdash{}{0pt}
\pgfsetmiterjoin
\pgfsetbuttcap
{
\definecolor{dialinecolor}{rgb}{0.000000, 0.000000, 0.000000}
\pgfsetfillcolor{dialinecolor}
% was here!!!
\definecolor{dialinecolor}{rgb}{0.000000, 0.000000, 0.000000}
\pgfsetstrokecolor{dialinecolor}
\draw (74.113900\du,2.245890\du)--(74.113900\du,8.964670\du)--(67.452600\du,8.964670\du)--(67.452600\du,11.774600\du);
}
\definecolor{dialinecolor}{rgb}{0.000000, 0.000000, 0.000000}
\pgfsetstrokecolor{dialinecolor}
\draw (74.113900\du,3.157693\du)--(74.113900\du,8.964670\du)--(67.452600\du,8.964670\du)--(67.452600\du,11.774600\du);
\pgfsetmiterjoin
\definecolor{dialinecolor}{rgb}{1.000000, 1.000000, 1.000000}
\pgfsetfillcolor{dialinecolor}
\fill (74.513900\du,3.157693\du)--(74.113900\du,2.357693\du)--(73.713900\du,3.157693\du)--cycle;
\pgfsetlinewidth{0.100000\du}
\pgfsetdash{}{0pt}
\pgfsetmiterjoin
\definecolor{dialinecolor}{rgb}{0.000000, 0.000000, 0.000000}
\pgfsetstrokecolor{dialinecolor}
\draw (74.513900\du,3.157693\du)--(74.113900\du,2.357693\du)--(73.713900\du,3.157693\du)--cycle;
% setfont left to latex
\pgfsetlinewidth{0.100000\du}
\pgfsetdash{}{0pt}
\pgfsetmiterjoin
\pgfsetbuttcap
{
\definecolor{dialinecolor}{rgb}{0.000000, 0.000000, 0.000000}
\pgfsetfillcolor{dialinecolor}
% was here!!!
\definecolor{dialinecolor}{rgb}{0.000000, 0.000000, 0.000000}
\pgfsetstrokecolor{dialinecolor}
\draw (74.113900\du,2.245890\du)--(74.113900\du,8.964670\du)--(77.911600\du,8.964670\du)--(77.911600\du,13.763424\du);
}
\definecolor{dialinecolor}{rgb}{0.000000, 0.000000, 0.000000}
\pgfsetstrokecolor{dialinecolor}
\draw (74.113900\du,3.157693\du)--(74.113900\du,8.964670\du)--(77.911600\du,8.964670\du)--(77.911600\du,13.763424\du);
\pgfsetmiterjoin
\definecolor{dialinecolor}{rgb}{1.000000, 1.000000, 1.000000}
\pgfsetfillcolor{dialinecolor}
\fill (74.513900\du,3.157693\du)--(74.113900\du,2.357693\du)--(73.713900\du,3.157693\du)--cycle;
\pgfsetlinewidth{0.100000\du}
\pgfsetdash{}{0pt}
\pgfsetmiterjoin
\definecolor{dialinecolor}{rgb}{0.000000, 0.000000, 0.000000}
\pgfsetstrokecolor{dialinecolor}
\draw (74.513900\du,3.157693\du)--(74.113900\du,2.357693\du)--(73.713900\du,3.157693\du)--cycle;
% setfont left to latex
\pgfsetlinewidth{0.100000\du}
\pgfsetdash{}{0pt}
\definecolor{dialinecolor}{rgb}{1.000000, 1.000000, 1.000000}
\pgfsetfillcolor{dialinecolor}
\fill (9.376230\du,-9.035330\du)--(9.376230\du,-7.635330\du)--(22.581230\du,-7.635330\du)--(22.581230\du,-9.035330\du)--cycle;
\definecolor{dialinecolor}{rgb}{0.000000, 0.000000, 0.000000}
\pgfsetstrokecolor{dialinecolor}
\draw (9.376230\du,-9.035330\du)--(9.376230\du,-7.635330\du)--(22.581230\du,-7.635330\du)--(22.581230\du,-9.035330\du)--cycle;
% setfont left to latex
\definecolor{dialinecolor}{rgb}{0.000000, 0.000000, 0.000000}
\pgfsetstrokecolor{dialinecolor}
\node at (15.978730\du,-8.085330\du){Observable};
\definecolor{dialinecolor}{rgb}{1.000000, 1.000000, 1.000000}
\pgfsetfillcolor{dialinecolor}
\fill (9.376230\du,-7.635330\du)--(9.376230\du,-7.235330\du)--(22.581230\du,-7.235330\du)--(22.581230\du,-7.635330\du)--cycle;
\definecolor{dialinecolor}{rgb}{0.000000, 0.000000, 0.000000}
\pgfsetstrokecolor{dialinecolor}
\draw (9.376230\du,-7.635330\du)--(9.376230\du,-7.235330\du)--(22.581230\du,-7.235330\du)--(22.581230\du,-7.635330\du)--cycle;
\definecolor{dialinecolor}{rgb}{1.000000, 1.000000, 1.000000}
\pgfsetfillcolor{dialinecolor}
\fill (9.376230\du,-7.235330\du)--(9.376230\du,2.564670\du)--(22.581230\du,2.564670\du)--(22.581230\du,-7.235330\du)--cycle;
\definecolor{dialinecolor}{rgb}{0.000000, 0.000000, 0.000000}
\pgfsetstrokecolor{dialinecolor}
\draw (9.376230\du,-7.235330\du)--(9.376230\du,2.564670\du)--(22.581230\du,2.564670\du)--(22.581230\du,-7.235330\du)--cycle;
% setfont left to latex
\definecolor{dialinecolor}{rgb}{0.000000, 0.000000, 0.000000}
\pgfsetstrokecolor{dialinecolor}
\node[anchor=west] at (9.526230\du,-6.535330\du){+attach(observer:ChangeObserver)};
% setfont left to latex
\definecolor{dialinecolor}{rgb}{0.000000, 0.000000, 0.000000}
\pgfsetstrokecolor{dialinecolor}
\node[anchor=west] at (9.526230\du,-5.735330\du){+attach(observer:DeleteObserver)};
% setfont left to latex
\definecolor{dialinecolor}{rgb}{0.000000, 0.000000, 0.000000}
\pgfsetstrokecolor{dialinecolor}
\node[anchor=west] at (9.526230\du,-4.935330\du){+attach(observer:OpenObserver)};
% setfont left to latex
\definecolor{dialinecolor}{rgb}{0.000000, 0.000000, 0.000000}
\pgfsetstrokecolor{dialinecolor}
\node[anchor=west] at (9.526230\du,-4.135330\du){+attach(observer:CloseObserver)};
% setfont left to latex
\definecolor{dialinecolor}{rgb}{0.000000, 0.000000, 0.000000}
\pgfsetstrokecolor{dialinecolor}
\node[anchor=west] at (9.526230\du,-3.335330\du){+dettach(observer:ChangeObserve)};
% setfont left to latex
\definecolor{dialinecolor}{rgb}{0.000000, 0.000000, 0.000000}
\pgfsetstrokecolor{dialinecolor}
\node[anchor=west] at (9.526230\du,-2.535330\du){+dettach(observer:DeleteObserver)};
% setfont left to latex
\definecolor{dialinecolor}{rgb}{0.000000, 0.000000, 0.000000}
\pgfsetstrokecolor{dialinecolor}
\node[anchor=west] at (9.526230\du,-1.735330\du){+dettach(observer:OpenObserver)};
% setfont left to latex
\definecolor{dialinecolor}{rgb}{0.000000, 0.000000, 0.000000}
\pgfsetstrokecolor{dialinecolor}
\node[anchor=west] at (9.526230\du,-0.935330\du){+dettach(observer:CloseObserver)};
% setfont left to latex
\definecolor{dialinecolor}{rgb}{0.000000, 0.000000, 0.000000}
\pgfsetstrokecolor{dialinecolor}
\node[anchor=west] at (9.526230\du,-0.135330\du){+notifyChange()};
% setfont left to latex
\definecolor{dialinecolor}{rgb}{0.000000, 0.000000, 0.000000}
\pgfsetstrokecolor{dialinecolor}
\node[anchor=west] at (9.526230\du,0.664670\du){+notifyDelete()};
% setfont left to latex
\definecolor{dialinecolor}{rgb}{0.000000, 0.000000, 0.000000}
\pgfsetstrokecolor{dialinecolor}
\node[anchor=west] at (9.526230\du,1.464670\du){+notifyOpen()};
% setfont left to latex
\definecolor{dialinecolor}{rgb}{0.000000, 0.000000, 0.000000}
\pgfsetstrokecolor{dialinecolor}
\node[anchor=west] at (9.526230\du,2.264670\du){+notifyClose()};
\pgfsetlinewidth{0.100000\du}
\pgfsetdash{}{0pt}
\pgfsetmiterjoin
\pgfsetbuttcap
{
\definecolor{dialinecolor}{rgb}{0.000000, 0.000000, 0.000000}
\pgfsetfillcolor{dialinecolor}
% was here!!!
\definecolor{dialinecolor}{rgb}{0.000000, 0.000000, 0.000000}
\pgfsetstrokecolor{dialinecolor}
\draw (64.380000\du,11.851400\du)--(64.380000\du,11.851400\du)--(64.380000\du,-2.254110\du)--(66.884174\du,-2.254110\du);
}
\definecolor{dialinecolor}{rgb}{0.000000, 0.000000, 0.000000}
\pgfsetstrokecolor{dialinecolor}
\draw (64.380000\du,10.592821\du)--(64.380000\du,-2.254110\du)--(66.884174\du,-2.254110\du);
\pgfsetdash{}{0pt}
\pgfsetmiterjoin
\pgfsetbuttcap
\definecolor{dialinecolor}{rgb}{0.000000, 0.000000, 0.000000}
\pgfsetfillcolor{dialinecolor}
\fill (64.380000\du,11.851400\du)--(64.140000\du,11.151400\du)--(64.380000\du,10.451400\du)--(64.620000\du,11.151400\du)--cycle;
\pgfsetlinewidth{0.100000\du}
\pgfsetdash{}{0pt}
\pgfsetmiterjoin
\pgfsetbuttcap
\definecolor{dialinecolor}{rgb}{0.000000, 0.000000, 0.000000}
\pgfsetstrokecolor{dialinecolor}
\draw (64.380000\du,11.851400\du)--(64.140000\du,11.151400\du)--(64.380000\du,10.451400\du)--(64.620000\du,11.151400\du)--cycle;
% setfont left to latex
\definecolor{dialinecolor}{rgb}{0.000000, 0.000000, 0.000000}
\pgfsetstrokecolor{dialinecolor}
\node[anchor=west] at (64.480000\du,4.648645\du){};
\definecolor{dialinecolor}{rgb}{0.000000, 0.000000, 0.000000}
\pgfsetstrokecolor{dialinecolor}
\node[anchor=west] at (64.930000\du,11.651400\du){};
\definecolor{dialinecolor}{rgb}{0.000000, 0.000000, 0.000000}
\pgfsetstrokecolor{dialinecolor}
\node[anchor=east] at (66.684174\du,-2.404110\du){ possède};
\definecolor{dialinecolor}{rgb}{0.000000, 0.000000, 0.000000}
\pgfsetstrokecolor{dialinecolor}
\node[anchor=east] at (66.684174\du,-1.604110\du){*};
\pgfsetlinewidth{0.100000\du}
\pgfsetdash{}{0pt}
\pgfsetmiterjoin
\pgfsetbuttcap
{
\definecolor{dialinecolor}{rgb}{0.000000, 0.000000, 0.000000}
\pgfsetfillcolor{dialinecolor}
% was here!!!
\pgfsetarrowsend{to}
\definecolor{dialinecolor}{rgb}{0.000000, 0.000000, 0.000000}
\pgfsetstrokecolor{dialinecolor}
\draw (29.560229\du,26.100000\du)--(74.875000\du,26.100000\du)--(74.875000\du,14.564700\du)--(70.016394\du,14.564700\du);
}
% setfont left to latex
\definecolor{dialinecolor}{rgb}{0.000000, 0.000000, 0.000000}
\pgfsetstrokecolor{dialinecolor}
\node[anchor=west] at (74.975000\du,20.182350\du){};
\definecolor{dialinecolor}{rgb}{0.000000, 0.000000, 0.000000}
\pgfsetstrokecolor{dialinecolor}
\node[anchor=west] at (29.760229\du,25.950000\du){};
\definecolor{dialinecolor}{rgb}{0.000000, 0.000000, 0.000000}
\pgfsetstrokecolor{dialinecolor}
\node[anchor=west] at (71.016394\du,14.414700\du){ active};
\definecolor{dialinecolor}{rgb}{0.000000, 0.000000, 0.000000}
\pgfsetstrokecolor{dialinecolor}
\node[anchor=west] at (71.016394\du,15.214700\du){0..1};
\pgfsetlinewidth{0.100000\du}
\pgfsetdash{}{0pt}
\definecolor{dialinecolor}{rgb}{1.000000, 1.000000, 1.000000}
\pgfsetfillcolor{dialinecolor}
\fill (38.376200\du,11.964700\du)--(38.376200\du,13.364700\du)--(48.886200\du,13.364700\du)--(48.886200\du,11.964700\du)--cycle;
\definecolor{dialinecolor}{rgb}{0.000000, 0.000000, 0.000000}
\pgfsetstrokecolor{dialinecolor}
\draw (38.376200\du,11.964700\du)--(38.376200\du,13.364700\du)--(48.886200\du,13.364700\du)--(48.886200\du,11.964700\du)--cycle;
% setfont left to latex
\definecolor{dialinecolor}{rgb}{0.000000, 0.000000, 0.000000}
\pgfsetstrokecolor{dialinecolor}
\node at (43.631200\du,12.914700\du){ActivateObserver};
\definecolor{dialinecolor}{rgb}{1.000000, 1.000000, 1.000000}
\pgfsetfillcolor{dialinecolor}
\fill (38.376200\du,13.364700\du)--(38.376200\du,13.764700\du)--(48.886200\du,13.764700\du)--(48.886200\du,13.364700\du)--cycle;
\definecolor{dialinecolor}{rgb}{0.000000, 0.000000, 0.000000}
\pgfsetstrokecolor{dialinecolor}
\draw (38.376200\du,13.364700\du)--(38.376200\du,13.764700\du)--(48.886200\du,13.764700\du)--(48.886200\du,13.364700\du)--cycle;
\definecolor{dialinecolor}{rgb}{1.000000, 1.000000, 1.000000}
\pgfsetfillcolor{dialinecolor}
\fill (38.376200\du,13.764700\du)--(38.376200\du,14.764700\du)--(48.886200\du,14.764700\du)--(48.886200\du,13.764700\du)--cycle;
\definecolor{dialinecolor}{rgb}{0.000000, 0.000000, 0.000000}
\pgfsetstrokecolor{dialinecolor}
\draw (38.376200\du,13.764700\du)--(38.376200\du,14.764700\du)--(48.886200\du,14.764700\du)--(48.886200\du,13.764700\du)--cycle;
% setfont left to latex
\definecolor{dialinecolor}{rgb}{0.000000, 0.000000, 0.000000}
\pgfsetstrokecolor{dialinecolor}
\node[anchor=west] at (38.526200\du,14.464700\du){+updateActivate(d:Dossier)};
\pgfsetlinewidth{0.100000\du}
\pgfsetdash{}{0pt}
\definecolor{dialinecolor}{rgb}{1.000000, 1.000000, 1.000000}
\pgfsetfillcolor{dialinecolor}
\fill (27.376200\du,11.964700\du)--(27.376200\du,13.364700\du)--(37.116200\du,13.364700\du)--(37.116200\du,11.964700\du)--cycle;
\definecolor{dialinecolor}{rgb}{0.000000, 0.000000, 0.000000}
\pgfsetstrokecolor{dialinecolor}
\draw (27.376200\du,11.964700\du)--(27.376200\du,13.364700\du)--(37.116200\du,13.364700\du)--(37.116200\du,11.964700\du)--cycle;
% setfont left to latex
\definecolor{dialinecolor}{rgb}{0.000000, 0.000000, 0.000000}
\pgfsetstrokecolor{dialinecolor}
\node at (32.246200\du,12.914700\du){DeleteObserver};
\definecolor{dialinecolor}{rgb}{1.000000, 1.000000, 1.000000}
\pgfsetfillcolor{dialinecolor}
\fill (27.376200\du,13.364700\du)--(27.376200\du,13.764700\du)--(37.116200\du,13.764700\du)--(37.116200\du,13.364700\du)--cycle;
\definecolor{dialinecolor}{rgb}{0.000000, 0.000000, 0.000000}
\pgfsetstrokecolor{dialinecolor}
\draw (27.376200\du,13.364700\du)--(27.376200\du,13.764700\du)--(37.116200\du,13.764700\du)--(37.116200\du,13.364700\du)--cycle;
\definecolor{dialinecolor}{rgb}{1.000000, 1.000000, 1.000000}
\pgfsetfillcolor{dialinecolor}
\fill (27.376200\du,13.764700\du)--(27.376200\du,14.764700\du)--(37.116200\du,14.764700\du)--(37.116200\du,13.764700\du)--cycle;
\definecolor{dialinecolor}{rgb}{0.000000, 0.000000, 0.000000}
\pgfsetstrokecolor{dialinecolor}
\draw (27.376200\du,13.764700\du)--(27.376200\du,14.764700\du)--(37.116200\du,14.764700\du)--(37.116200\du,13.764700\du)--cycle;
% setfont left to latex
\definecolor{dialinecolor}{rgb}{0.000000, 0.000000, 0.000000}
\pgfsetstrokecolor{dialinecolor}
\node[anchor=west] at (27.526200\du,14.464700\du){+updateDelete(e:Element)};
\pgfsetlinewidth{0.100000\du}
\pgfsetdash{}{0pt}
\definecolor{dialinecolor}{rgb}{1.000000, 1.000000, 1.000000}
\pgfsetfillcolor{dialinecolor}
\fill (17.376200\du,11.964700\du)--(17.376200\du,13.364700\du)--(26.346200\du,13.364700\du)--(26.346200\du,11.964700\du)--cycle;
\definecolor{dialinecolor}{rgb}{0.000000, 0.000000, 0.000000}
\pgfsetstrokecolor{dialinecolor}
\draw (17.376200\du,11.964700\du)--(17.376200\du,13.364700\du)--(26.346200\du,13.364700\du)--(26.346200\du,11.964700\du)--cycle;
% setfont left to latex
\definecolor{dialinecolor}{rgb}{0.000000, 0.000000, 0.000000}
\pgfsetstrokecolor{dialinecolor}
\node at (21.861200\du,12.914700\du){OpenObserver};
\definecolor{dialinecolor}{rgb}{1.000000, 1.000000, 1.000000}
\pgfsetfillcolor{dialinecolor}
\fill (17.376200\du,13.364700\du)--(17.376200\du,13.764700\du)--(26.346200\du,13.764700\du)--(26.346200\du,13.364700\du)--cycle;
\definecolor{dialinecolor}{rgb}{0.000000, 0.000000, 0.000000}
\pgfsetstrokecolor{dialinecolor}
\draw (17.376200\du,13.364700\du)--(17.376200\du,13.764700\du)--(26.346200\du,13.764700\du)--(26.346200\du,13.364700\du)--cycle;
\definecolor{dialinecolor}{rgb}{1.000000, 1.000000, 1.000000}
\pgfsetfillcolor{dialinecolor}
\fill (17.376200\du,13.764700\du)--(17.376200\du,14.764700\du)--(26.346200\du,14.764700\du)--(26.346200\du,13.764700\du)--cycle;
\definecolor{dialinecolor}{rgb}{0.000000, 0.000000, 0.000000}
\pgfsetstrokecolor{dialinecolor}
\draw (17.376200\du,13.764700\du)--(17.376200\du,14.764700\du)--(26.346200\du,14.764700\du)--(26.346200\du,13.764700\du)--cycle;
% setfont left to latex
\definecolor{dialinecolor}{rgb}{0.000000, 0.000000, 0.000000}
\pgfsetstrokecolor{dialinecolor}
\node[anchor=west] at (17.526200\du,14.464700\du){+updateOpen(e:Element)};
\pgfsetlinewidth{0.100000\du}
\pgfsetdash{}{0pt}
\definecolor{dialinecolor}{rgb}{1.000000, 1.000000, 1.000000}
\pgfsetfillcolor{dialinecolor}
\fill (5.376230\du,11.964700\du)--(5.376230\du,13.364700\du)--(15.116230\du,13.364700\du)--(15.116230\du,11.964700\du)--cycle;
\definecolor{dialinecolor}{rgb}{0.000000, 0.000000, 0.000000}
\pgfsetstrokecolor{dialinecolor}
\draw (5.376230\du,11.964700\du)--(5.376230\du,13.364700\du)--(15.116230\du,13.364700\du)--(15.116230\du,11.964700\du)--cycle;
% setfont left to latex
\definecolor{dialinecolor}{rgb}{0.000000, 0.000000, 0.000000}
\pgfsetstrokecolor{dialinecolor}
\node at (10.246230\du,12.914700\du){ChangeObserver};
\definecolor{dialinecolor}{rgb}{1.000000, 1.000000, 1.000000}
\pgfsetfillcolor{dialinecolor}
\fill (5.376230\du,13.364700\du)--(5.376230\du,13.764700\du)--(15.116230\du,13.764700\du)--(15.116230\du,13.364700\du)--cycle;
\definecolor{dialinecolor}{rgb}{0.000000, 0.000000, 0.000000}
\pgfsetstrokecolor{dialinecolor}
\draw (5.376230\du,13.364700\du)--(5.376230\du,13.764700\du)--(15.116230\du,13.764700\du)--(15.116230\du,13.364700\du)--cycle;
\definecolor{dialinecolor}{rgb}{1.000000, 1.000000, 1.000000}
\pgfsetfillcolor{dialinecolor}
\fill (5.376230\du,13.764700\du)--(5.376230\du,14.764700\du)--(15.116230\du,14.764700\du)--(15.116230\du,13.764700\du)--cycle;
\definecolor{dialinecolor}{rgb}{0.000000, 0.000000, 0.000000}
\pgfsetstrokecolor{dialinecolor}
\draw (5.376230\du,13.764700\du)--(5.376230\du,14.764700\du)--(15.116230\du,14.764700\du)--(15.116230\du,13.764700\du)--cycle;
% setfont left to latex
\definecolor{dialinecolor}{rgb}{0.000000, 0.000000, 0.000000}
\pgfsetstrokecolor{dialinecolor}
\node[anchor=west] at (5.526230\du,14.464700\du){+updateChange(e:Element)};
\pgfsetlinewidth{0.100000\du}
\pgfsetdash{}{0pt}
\definecolor{dialinecolor}{rgb}{1.000000, 1.000000, 1.000000}
\pgfsetfillcolor{dialinecolor}
\fill (-5.799240\du,11.964700\du)--(-5.799240\du,13.364700\du)--(3.555760\du,13.364700\du)--(3.555760\du,11.964700\du)--cycle;
\definecolor{dialinecolor}{rgb}{0.000000, 0.000000, 0.000000}
\pgfsetstrokecolor{dialinecolor}
\draw (-5.799240\du,11.964700\du)--(-5.799240\du,13.364700\du)--(3.555760\du,13.364700\du)--(3.555760\du,11.964700\du)--cycle;
% setfont left to latex
\definecolor{dialinecolor}{rgb}{0.000000, 0.000000, 0.000000}
\pgfsetstrokecolor{dialinecolor}
\node at (-1.121740\du,12.914700\du){CloseObserver};
\definecolor{dialinecolor}{rgb}{1.000000, 1.000000, 1.000000}
\pgfsetfillcolor{dialinecolor}
\fill (-5.799240\du,13.364700\du)--(-5.799240\du,13.764700\du)--(3.555760\du,13.764700\du)--(3.555760\du,13.364700\du)--cycle;
\definecolor{dialinecolor}{rgb}{0.000000, 0.000000, 0.000000}
\pgfsetstrokecolor{dialinecolor}
\draw (-5.799240\du,13.364700\du)--(-5.799240\du,13.764700\du)--(3.555760\du,13.764700\du)--(3.555760\du,13.364700\du)--cycle;
\definecolor{dialinecolor}{rgb}{1.000000, 1.000000, 1.000000}
\pgfsetfillcolor{dialinecolor}
\fill (-5.799240\du,13.764700\du)--(-5.799240\du,14.764700\du)--(3.555760\du,14.764700\du)--(3.555760\du,13.764700\du)--cycle;
\definecolor{dialinecolor}{rgb}{0.000000, 0.000000, 0.000000}
\pgfsetstrokecolor{dialinecolor}
\draw (-5.799240\du,13.764700\du)--(-5.799240\du,14.764700\du)--(3.555760\du,14.764700\du)--(3.555760\du,13.764700\du)--cycle;
% setfont left to latex
\definecolor{dialinecolor}{rgb}{0.000000, 0.000000, 0.000000}
\pgfsetstrokecolor{dialinecolor}
\node[anchor=west] at (-5.649240\du,14.464700\du){+updateClose(e:Element)};
\pgfsetlinewidth{0.100000\du}
\pgfsetdash{{1.000000\du}{1.000000\du}}{0\du}
\pgfsetdash{{0.400000\du}{0.400000\du}}{0\du}
\pgfsetmiterjoin
\pgfsetbuttcap
{
\definecolor{dialinecolor}{rgb}{0.000000, 0.000000, 0.000000}
\pgfsetfillcolor{dialinecolor}
% was here!!!
\definecolor{dialinecolor}{rgb}{0.000000, 0.000000, 0.000000}
\pgfsetstrokecolor{dialinecolor}
\draw (10.246229\du,14.813821\du)--(10.246224\du,20.964700\du)--(24.255000\du,20.964700\du)--(24.255000\du,23.950471\du);
}
\definecolor{dialinecolor}{rgb}{0.000000, 0.000000, 0.000000}
\pgfsetstrokecolor{dialinecolor}
\draw (10.246228\du,15.725624\du)--(10.246224\du,20.964700\du)--(24.255000\du,20.964700\du)--(24.255000\du,23.950471\du);
\pgfsetmiterjoin
\definecolor{dialinecolor}{rgb}{1.000000, 1.000000, 1.000000}
\pgfsetfillcolor{dialinecolor}
\fill (10.646228\du,15.725625\du)--(10.246229\du,14.925624\du)--(9.846228\du,15.725624\du)--cycle;
\pgfsetlinewidth{0.100000\du}
\pgfsetdash{}{0pt}
\pgfsetmiterjoin
\definecolor{dialinecolor}{rgb}{0.000000, 0.000000, 0.000000}
\pgfsetstrokecolor{dialinecolor}
\draw (10.646228\du,15.725625\du)--(10.246229\du,14.925624\du)--(9.846228\du,15.725624\du)--cycle;
% setfont left to latex
\pgfsetlinewidth{0.100000\du}
\pgfsetdash{{0.400000\du}{0.400000\du}}{0\du}
\pgfsetdash{{0.400000\du}{0.400000\du}}{0\du}
\pgfsetmiterjoin
\pgfsetbuttcap
{
\definecolor{dialinecolor}{rgb}{0.000000, 0.000000, 0.000000}
\pgfsetfillcolor{dialinecolor}
% was here!!!
\definecolor{dialinecolor}{rgb}{0.000000, 0.000000, 0.000000}
\pgfsetstrokecolor{dialinecolor}
\draw (32.246200\du,14.814407\du)--(32.246200\du,19.718900\du)--(24.255000\du,19.718900\du)--(24.255000\du,23.949728\du);
}
\definecolor{dialinecolor}{rgb}{0.000000, 0.000000, 0.000000}
\pgfsetstrokecolor{dialinecolor}
\draw (32.246200\du,15.726210\du)--(32.246200\du,19.718900\du)--(24.255000\du,19.718900\du)--(24.255000\du,23.949728\du);
\pgfsetmiterjoin
\definecolor{dialinecolor}{rgb}{1.000000, 1.000000, 1.000000}
\pgfsetfillcolor{dialinecolor}
\fill (32.646200\du,15.726210\du)--(32.246200\du,14.926210\du)--(31.846200\du,15.726210\du)--cycle;
\pgfsetlinewidth{0.100000\du}
\pgfsetdash{}{0pt}
\pgfsetmiterjoin
\definecolor{dialinecolor}{rgb}{0.000000, 0.000000, 0.000000}
\pgfsetstrokecolor{dialinecolor}
\draw (32.646200\du,15.726210\du)--(32.246200\du,14.926210\du)--(31.846200\du,15.726210\du)--cycle;
% setfont left to latex
\pgfsetlinewidth{0.100000\du}
\pgfsetdash{{0.400000\du}{0.400000\du}}{0\du}
\pgfsetdash{{0.400000\du}{0.400000\du}}{0\du}
\pgfsetmiterjoin
\pgfsetbuttcap
{
\definecolor{dialinecolor}{rgb}{0.000000, 0.000000, 0.000000}
\pgfsetfillcolor{dialinecolor}
% was here!!!
\definecolor{dialinecolor}{rgb}{0.000000, 0.000000, 0.000000}
\pgfsetstrokecolor{dialinecolor}
\draw (21.861200\du,14.813821\du)--(21.861200\du,20.964700\du)--(24.255000\du,20.964700\du)--(24.255000\du,23.950471\du);
}
\definecolor{dialinecolor}{rgb}{0.000000, 0.000000, 0.000000}
\pgfsetstrokecolor{dialinecolor}
\draw (21.861200\du,15.725624\du)--(21.861200\du,20.964700\du)--(24.255000\du,20.964700\du)--(24.255000\du,23.950471\du);
\pgfsetmiterjoin
\definecolor{dialinecolor}{rgb}{1.000000, 1.000000, 1.000000}
\pgfsetfillcolor{dialinecolor}
\fill (22.261200\du,15.725624\du)--(21.861200\du,14.925624\du)--(21.461200\du,15.725624\du)--cycle;
\pgfsetlinewidth{0.100000\du}
\pgfsetdash{}{0pt}
\pgfsetmiterjoin
\definecolor{dialinecolor}{rgb}{0.000000, 0.000000, 0.000000}
\pgfsetstrokecolor{dialinecolor}
\draw (22.261200\du,15.725624\du)--(21.861200\du,14.925624\du)--(21.461200\du,15.725624\du)--cycle;
% setfont left to latex
\pgfsetlinewidth{0.100000\du}
\pgfsetdash{{0.400000\du}{0.400000\du}}{0\du}
\pgfsetdash{{0.400000\du}{0.400000\du}}{0\du}
\pgfsetmiterjoin
\pgfsetbuttcap
{
\definecolor{dialinecolor}{rgb}{0.000000, 0.000000, 0.000000}
\pgfsetfillcolor{dialinecolor}
% was here!!!
\definecolor{dialinecolor}{rgb}{0.000000, 0.000000, 0.000000}
\pgfsetstrokecolor{dialinecolor}
\draw (-1.121740\du,14.813821\du)--(-1.121740\du,20.964700\du)--(24.255000\du,20.964700\du)--(24.255000\du,23.950471\du);
}
\definecolor{dialinecolor}{rgb}{0.000000, 0.000000, 0.000000}
\pgfsetstrokecolor{dialinecolor}
\draw (-1.121740\du,15.725624\du)--(-1.121740\du,20.964700\du)--(24.255000\du,20.964700\du)--(24.255000\du,23.950471\du);
\pgfsetmiterjoin
\definecolor{dialinecolor}{rgb}{1.000000, 1.000000, 1.000000}
\pgfsetfillcolor{dialinecolor}
\fill (-0.721740\du,15.725624\du)--(-1.121740\du,14.925624\du)--(-1.521740\du,15.725624\du)--cycle;
\pgfsetlinewidth{0.100000\du}
\pgfsetdash{}{0pt}
\pgfsetmiterjoin
\definecolor{dialinecolor}{rgb}{0.000000, 0.000000, 0.000000}
\pgfsetstrokecolor{dialinecolor}
\draw (-0.721740\du,15.725624\du)--(-1.121740\du,14.925624\du)--(-1.521740\du,15.725624\du)--cycle;
% setfont left to latex
\pgfsetlinewidth{0.100000\du}
\pgfsetdash{{0.400000\du}{0.400000\du}}{0\du}
\pgfsetdash{{0.400000\du}{0.400000\du}}{0\du}
\pgfsetmiterjoin
\pgfsetbuttcap
{
\definecolor{dialinecolor}{rgb}{0.000000, 0.000000, 0.000000}
\pgfsetfillcolor{dialinecolor}
% was here!!!
\definecolor{dialinecolor}{rgb}{0.000000, 0.000000, 0.000000}
\pgfsetstrokecolor{dialinecolor}
\draw (43.631200\du,14.813821\du)--(43.631200\du,20.964700\du)--(24.255000\du,20.964700\du)--(24.255000\du,23.950471\du);
}
\definecolor{dialinecolor}{rgb}{0.000000, 0.000000, 0.000000}
\pgfsetstrokecolor{dialinecolor}
\draw (43.631200\du,15.725624\du)--(43.631200\du,20.964700\du)--(24.255000\du,20.964700\du)--(24.255000\du,23.950471\du);
\pgfsetmiterjoin
\definecolor{dialinecolor}{rgb}{1.000000, 1.000000, 1.000000}
\pgfsetfillcolor{dialinecolor}
\fill (44.031200\du,15.725624\du)--(43.631200\du,14.925624\du)--(43.231200\du,15.725624\du)--cycle;
\pgfsetlinewidth{0.100000\du}
\pgfsetdash{}{0pt}
\pgfsetmiterjoin
\definecolor{dialinecolor}{rgb}{0.000000, 0.000000, 0.000000}
\pgfsetstrokecolor{dialinecolor}
\draw (44.031200\du,15.725624\du)--(43.631200\du,14.925624\du)--(43.231200\du,15.725624\du)--cycle;
% setfont left to latex
\pgfsetlinewidth{0.100000\du}
\pgfsetdash{}{0pt}
\pgfsetmiterjoin
\pgfsetbuttcap
{
\definecolor{dialinecolor}{rgb}{0.000000, 0.000000, 0.000000}
\pgfsetfillcolor{dialinecolor}
% was here!!!
\definecolor{dialinecolor}{rgb}{0.000000, 0.000000, 0.000000}
\pgfsetstrokecolor{dialinecolor}
\draw (22.581230\du,-5.935330\du)--(36.376200\du,-5.935330\du)--(36.376200\du,-6.054110\du)--(66.933900\du,-6.054110\du);
}
\definecolor{dialinecolor}{rgb}{0.000000, 0.000000, 0.000000}
\pgfsetstrokecolor{dialinecolor}
\draw (23.493033\du,-5.935330\du)--(36.376200\du,-5.935330\du)--(36.376200\du,-6.054110\du)--(66.933900\du,-6.054110\du);
\pgfsetmiterjoin
\definecolor{dialinecolor}{rgb}{1.000000, 1.000000, 1.000000}
\pgfsetfillcolor{dialinecolor}
\fill (23.493033\du,-6.335330\du)--(22.693033\du,-5.935330\du)--(23.493033\du,-5.535330\du)--cycle;
\pgfsetlinewidth{0.100000\du}
\pgfsetdash{}{0pt}
\pgfsetmiterjoin
\definecolor{dialinecolor}{rgb}{0.000000, 0.000000, 0.000000}
\pgfsetstrokecolor{dialinecolor}
\draw (23.493033\du,-6.335330\du)--(22.693033\du,-5.935330\du)--(23.493033\du,-5.535330\du)--cycle;
% setfont left to latex
\pgfsetlinewidth{0.100000\du}
\pgfsetdash{}{0pt}
\pgfsetmiterjoin
\pgfsetbuttcap
{
\definecolor{dialinecolor}{rgb}{0.000000, 0.000000, 0.000000}
\pgfsetfillcolor{dialinecolor}
% was here!!!
\pgfsetarrowsend{to}
\definecolor{dialinecolor}{rgb}{0.000000, 0.000000, 0.000000}
\pgfsetstrokecolor{dialinecolor}
\draw (15.978730\du,2.614182\du)--(15.978730\du,7.964670\du)--(-1.121740\du,7.964670\du)--(-1.121740\du,11.964700\du);
}
% setfont left to latex
\definecolor{dialinecolor}{rgb}{0.000000, 0.000000, 0.000000}
\pgfsetstrokecolor{dialinecolor}
\node at (7.428495\du,7.814670\du){};
\definecolor{dialinecolor}{rgb}{0.000000, 0.000000, 0.000000}
\pgfsetstrokecolor{dialinecolor}
\node[anchor=west] at (16.178730\du,3.214182\du){};
\definecolor{dialinecolor}{rgb}{0.000000, 0.000000, 0.000000}
\pgfsetstrokecolor{dialinecolor}
\node[anchor=west] at (-0.571740\du,11.764700\du){*};
\pgfsetlinewidth{0.100000\du}
\pgfsetdash{}{0pt}
\pgfsetmiterjoin
\pgfsetbuttcap
{
\definecolor{dialinecolor}{rgb}{0.000000, 0.000000, 0.000000}
\pgfsetfillcolor{dialinecolor}
% was here!!!
\pgfsetarrowsend{to}
\definecolor{dialinecolor}{rgb}{0.000000, 0.000000, 0.000000}
\pgfsetstrokecolor{dialinecolor}
\draw (15.978730\du,2.614182\du)--(15.978730\du,7.964670\du)--(32.246200\du,7.964670\du)--(32.246200\du,11.964700\du);
}
% setfont left to latex
\definecolor{dialinecolor}{rgb}{0.000000, 0.000000, 0.000000}
\pgfsetstrokecolor{dialinecolor}
\node at (24.112465\du,7.814670\du){};
\definecolor{dialinecolor}{rgb}{0.000000, 0.000000, 0.000000}
\pgfsetstrokecolor{dialinecolor}
\node[anchor=west] at (16.178730\du,3.214182\du){};
\definecolor{dialinecolor}{rgb}{0.000000, 0.000000, 0.000000}
\pgfsetstrokecolor{dialinecolor}
\node[anchor=west] at (32.796200\du,11.764700\du){*};
\pgfsetlinewidth{0.100000\du}
\pgfsetdash{}{0pt}
\pgfsetmiterjoin
\pgfsetbuttcap
{
\definecolor{dialinecolor}{rgb}{0.000000, 0.000000, 0.000000}
\pgfsetfillcolor{dialinecolor}
% was here!!!
\pgfsetarrowsend{to}
\definecolor{dialinecolor}{rgb}{0.000000, 0.000000, 0.000000}
\pgfsetstrokecolor{dialinecolor}
\draw (57.376200\du,11.964700\du)--(57.376200\du,8.964670\du)--(43.631200\du,8.964670\du)--(43.631200\du,11.964700\du);
}
% setfont left to latex
\definecolor{dialinecolor}{rgb}{0.000000, 0.000000, 0.000000}
\pgfsetstrokecolor{dialinecolor}
\node at (50.503700\du,8.814670\du){};
\definecolor{dialinecolor}{rgb}{0.000000, 0.000000, 0.000000}
\pgfsetstrokecolor{dialinecolor}
\node[anchor=west] at (57.576200\du,10.964700\du){ observe};
\definecolor{dialinecolor}{rgb}{0.000000, 0.000000, 0.000000}
\pgfsetstrokecolor{dialinecolor}
\node[anchor=west] at (57.576200\du,11.764700\du){*};
\definecolor{dialinecolor}{rgb}{0.000000, 0.000000, 0.000000}
\pgfsetstrokecolor{dialinecolor}
\node[anchor=west] at (44.181200\du,11.764700\du){*};
\pgfsetlinewidth{0.100000\du}
\pgfsetdash{}{0pt}
\pgfsetmiterjoin
\pgfsetbuttcap
{
\definecolor{dialinecolor}{rgb}{0.000000, 0.000000, 0.000000}
\pgfsetfillcolor{dialinecolor}
% was here!!!
\pgfsetarrowsend{to}
\definecolor{dialinecolor}{rgb}{0.000000, 0.000000, 0.000000}
\pgfsetstrokecolor{dialinecolor}
\draw (15.978730\du,2.614182\du)--(15.978730\du,7.964670\du)--(21.861200\du,7.964670\du)--(21.861200\du,11.917133\du);
}
% setfont left to latex
\definecolor{dialinecolor}{rgb}{0.000000, 0.000000, 0.000000}
\pgfsetstrokecolor{dialinecolor}
\node at (18.919965\du,7.814670\du){};
\definecolor{dialinecolor}{rgb}{0.000000, 0.000000, 0.000000}
\pgfsetstrokecolor{dialinecolor}
\node[anchor=west] at (16.178730\du,3.214182\du){};
\definecolor{dialinecolor}{rgb}{0.000000, 0.000000, 0.000000}
\pgfsetstrokecolor{dialinecolor}
\node[anchor=west] at (22.411200\du,11.717133\du){*};
\pgfsetlinewidth{0.100000\du}
\pgfsetdash{}{0pt}
\pgfsetmiterjoin
\pgfsetbuttcap
{
\definecolor{dialinecolor}{rgb}{0.000000, 0.000000, 0.000000}
\pgfsetfillcolor{dialinecolor}
% was here!!!
\pgfsetarrowsend{to}
\definecolor{dialinecolor}{rgb}{0.000000, 0.000000, 0.000000}
\pgfsetstrokecolor{dialinecolor}
\draw (15.978730\du,2.614182\du)--(15.978730\du,7.964670\du)--(10.246200\du,7.964670\du)--(10.246200\du,11.964700\du);
}
% setfont left to latex
\definecolor{dialinecolor}{rgb}{0.000000, 0.000000, 0.000000}
\pgfsetstrokecolor{dialinecolor}
\node at (13.112465\du,7.814670\du){};
\definecolor{dialinecolor}{rgb}{0.000000, 0.000000, 0.000000}
\pgfsetstrokecolor{dialinecolor}
\node[anchor=west] at (16.178730\du,3.214182\du){1};
\definecolor{dialinecolor}{rgb}{0.000000, 0.000000, 0.000000}
\pgfsetstrokecolor{dialinecolor}
\node[anchor=west] at (10.796200\du,11.764700\du){*};
\pgfsetlinewidth{0.100000\du}
\pgfsetdash{}{0pt}
\pgfsetmiterjoin
\pgfsetbuttcap
{
\definecolor{dialinecolor}{rgb}{0.000000, 0.000000, 0.000000}
\pgfsetfillcolor{dialinecolor}
% was here!!!
\pgfsetarrowsend{to}
\definecolor{dialinecolor}{rgb}{0.000000, 0.000000, 0.000000}
\pgfsetstrokecolor{dialinecolor}
\draw (29.559478\du,26.100000\du)--(52.376200\du,26.100000\du)--(52.376200\du,14.264700\du)--(56.376200\du,14.264700\du);
}
% setfont left to latex
\definecolor{dialinecolor}{rgb}{0.000000, 0.000000, 0.000000}
\pgfsetstrokecolor{dialinecolor}
\node[anchor=west] at (52.476200\du,20.032350\du){};
\definecolor{dialinecolor}{rgb}{0.000000, 0.000000, 0.000000}
\pgfsetstrokecolor{dialinecolor}
\node[anchor=west] at (29.759478\du,25.950000\du){1};
\definecolor{dialinecolor}{rgb}{0.000000, 0.000000, 0.000000}
\pgfsetstrokecolor{dialinecolor}
\node[anchor=east] at (55.376200\du,14.114700\du){ ouvre};
\definecolor{dialinecolor}{rgb}{0.000000, 0.000000, 0.000000}
\pgfsetstrokecolor{dialinecolor}
\node[anchor=east] at (55.376200\du,14.914700\du){*};
\pgfsetlinewidth{0.100000\du}
\pgfsetdash{{0.400000\du}{0.400000\du}}{0\du}
\pgfsetdash{{0.400000\du}{0.400000\du}}{0\du}
\pgfsetmiterjoin
\pgfsetbuttcap
{
\definecolor{dialinecolor}{rgb}{0.000000, 0.000000, 0.000000}
\pgfsetfillcolor{dialinecolor}
% was here!!!
\definecolor{dialinecolor}{rgb}{0.000000, 0.000000, 0.000000}
\pgfsetstrokecolor{dialinecolor}
\draw (34.376200\du,11.964700\du)--(34.376200\du,7.964670\du)--(63.171200\du,7.964670\du)--(63.171200\du,11.914859\du);
}
\definecolor{dialinecolor}{rgb}{0.000000, 0.000000, 0.000000}
\pgfsetstrokecolor{dialinecolor}
\draw (34.376200\du,11.052897\du)--(34.376200\du,7.964670\du)--(63.171200\du,7.964670\du)--(63.171200\du,11.914859\du);
\pgfsetmiterjoin
\definecolor{dialinecolor}{rgb}{1.000000, 1.000000, 1.000000}
\pgfsetfillcolor{dialinecolor}
\fill (33.976200\du,11.052897\du)--(34.376200\du,11.852897\du)--(34.776200\du,11.052897\du)--cycle;
\pgfsetlinewidth{0.100000\du}
\pgfsetdash{}{0pt}
\pgfsetmiterjoin
\definecolor{dialinecolor}{rgb}{0.000000, 0.000000, 0.000000}
\pgfsetstrokecolor{dialinecolor}
\draw (33.976200\du,11.052897\du)--(34.376200\du,11.852897\du)--(34.776200\du,11.052897\du)--cycle;
% setfont left to latex
\pgfsetlinewidth{0.100000\du}
\pgfsetdash{}{0pt}
\definecolor{dialinecolor}{rgb}{1.000000, 1.000000, 1.000000}
\pgfsetfillcolor{dialinecolor}
\fill (-17.955036\du,12.024436\du)--(-17.955036\du,13.424436\du)--(-8.600036\du,13.424436\du)--(-8.600036\du,12.024436\du)--cycle;
\definecolor{dialinecolor}{rgb}{0.000000, 0.000000, 0.000000}
\pgfsetstrokecolor{dialinecolor}
\draw (-17.955036\du,12.024436\du)--(-17.955036\du,13.424436\du)--(-8.600036\du,13.424436\du)--(-8.600036\du,12.024436\du)--cycle;
% setfont left to latex
\definecolor{dialinecolor}{rgb}{0.000000, 0.000000, 0.000000}
\pgfsetstrokecolor{dialinecolor}
\node at (-13.277536\du,12.974436\du){ElementVisitor};
\definecolor{dialinecolor}{rgb}{1.000000, 1.000000, 1.000000}
\pgfsetfillcolor{dialinecolor}
\fill (-17.955036\du,13.424436\du)--(-17.955036\du,13.824436\du)--(-8.600036\du,13.824436\du)--(-8.600036\du,13.424436\du)--cycle;
\definecolor{dialinecolor}{rgb}{0.000000, 0.000000, 0.000000}
\pgfsetstrokecolor{dialinecolor}
\draw (-17.955036\du,13.424436\du)--(-17.955036\du,13.824436\du)--(-8.600036\du,13.824436\du)--(-8.600036\du,13.424436\du)--cycle;
\definecolor{dialinecolor}{rgb}{1.000000, 1.000000, 1.000000}
\pgfsetfillcolor{dialinecolor}
\fill (-17.955036\du,13.824436\du)--(-17.955036\du,14.824436\du)--(-8.600036\du,14.824436\du)--(-8.600036\du,13.824436\du)--cycle;
\definecolor{dialinecolor}{rgb}{0.000000, 0.000000, 0.000000}
\pgfsetstrokecolor{dialinecolor}
\draw (-17.955036\du,13.824436\du)--(-17.955036\du,14.824436\du)--(-8.600036\du,14.824436\du)--(-8.600036\du,13.824436\du)--cycle;
% setfont left to latex
\definecolor{dialinecolor}{rgb}{0.000000, 0.000000, 0.000000}
\pgfsetstrokecolor{dialinecolor}
\node[anchor=west] at (-17.805036\du,14.524436\du){+visit(element:Element)};
\pgfsetlinewidth{0.100000\du}
\pgfsetdash{}{0pt}
\definecolor{dialinecolor}{rgb}{1.000000, 1.000000, 1.000000}
\pgfsetfillcolor{dialinecolor}
\fill (-16.405220\du,29.128137\du)--(-16.405220\du,30.528137\du)--(-10.035220\du,30.528137\du)--(-10.035220\du,29.128137\du)--cycle;
\definecolor{dialinecolor}{rgb}{0.000000, 0.000000, 0.000000}
\pgfsetstrokecolor{dialinecolor}
\draw (-16.405220\du,29.128137\du)--(-16.405220\du,30.528137\du)--(-10.035220\du,30.528137\du)--(-10.035220\du,29.128137\du)--cycle;
% setfont left to latex
\definecolor{dialinecolor}{rgb}{0.000000, 0.000000, 0.000000}
\pgfsetstrokecolor{dialinecolor}
\node at (-13.220220\du,30.078137\du){TailleVisiteur};
\definecolor{dialinecolor}{rgb}{1.000000, 1.000000, 1.000000}
\pgfsetfillcolor{dialinecolor}
\fill (-16.405220\du,30.528137\du)--(-16.405220\du,30.928137\du)--(-10.035220\du,30.928137\du)--(-10.035220\du,30.528137\du)--cycle;
\definecolor{dialinecolor}{rgb}{0.000000, 0.000000, 0.000000}
\pgfsetstrokecolor{dialinecolor}
\draw (-16.405220\du,30.528137\du)--(-16.405220\du,30.928137\du)--(-10.035220\du,30.928137\du)--(-10.035220\du,30.528137\du)--cycle;
\definecolor{dialinecolor}{rgb}{1.000000, 1.000000, 1.000000}
\pgfsetfillcolor{dialinecolor}
\fill (-16.405220\du,30.928137\du)--(-16.405220\du,31.328137\du)--(-10.035220\du,31.328137\du)--(-10.035220\du,30.928137\du)--cycle;
\definecolor{dialinecolor}{rgb}{0.000000, 0.000000, 0.000000}
\pgfsetstrokecolor{dialinecolor}
\draw (-16.405220\du,30.928137\du)--(-16.405220\du,31.328137\du)--(-10.035220\du,31.328137\du)--(-10.035220\du,30.928137\du)--cycle;
\pgfsetlinewidth{0.100000\du}
\pgfsetdash{{0.400000\du}{0.400000\du}}{0\du}
\pgfsetdash{{0.400000\du}{0.400000\du}}{0\du}
\pgfsetmiterjoin
\pgfsetbuttcap
{
\definecolor{dialinecolor}{rgb}{0.000000, 0.000000, 0.000000}
\pgfsetfillcolor{dialinecolor}
% was here!!!
\definecolor{dialinecolor}{rgb}{0.000000, 0.000000, 0.000000}
\pgfsetstrokecolor{dialinecolor}
\draw (-13.277536\du,14.874790\du)--(-13.277536\du,22.376323\du)--(-13.220220\du,22.376323\du)--(-13.220220\du,29.077856\du);
}
\definecolor{dialinecolor}{rgb}{0.000000, 0.000000, 0.000000}
\pgfsetstrokecolor{dialinecolor}
\draw (-13.277536\du,15.786594\du)--(-13.277536\du,22.376323\du)--(-13.220220\du,22.376323\du)--(-13.220220\du,29.077856\du);
\pgfsetmiterjoin
\definecolor{dialinecolor}{rgb}{1.000000, 1.000000, 1.000000}
\pgfsetfillcolor{dialinecolor}
\fill (-12.877536\du,15.786594\du)--(-13.277536\du,14.986594\du)--(-13.677536\du,15.786594\du)--cycle;
\pgfsetlinewidth{0.100000\du}
\pgfsetdash{}{0pt}
\pgfsetmiterjoin
\definecolor{dialinecolor}{rgb}{0.000000, 0.000000, 0.000000}
\pgfsetstrokecolor{dialinecolor}
\draw (-12.877536\du,15.786594\du)--(-13.277536\du,14.986594\du)--(-13.677536\du,15.786594\du)--cycle;
% setfont left to latex
\pgfsetlinewidth{0.100000\du}
\pgfsetdash{}{0pt}
\pgfsetmiterjoin
\pgfsetbuttcap
{
\definecolor{dialinecolor}{rgb}{0.000000, 0.000000, 0.000000}
\pgfsetfillcolor{dialinecolor}
% was here!!!
\definecolor{dialinecolor}{rgb}{0.000000, 0.000000, 0.000000}
\pgfsetstrokecolor{dialinecolor}
\draw (0.099390\du,30.211078\du)--(-5.292718\du,30.211078\du)--(-5.292718\du,30.228137\du)--(-9.984825\du,30.228137\du);
}
\definecolor{dialinecolor}{rgb}{0.000000, 0.000000, 0.000000}
\pgfsetstrokecolor{dialinecolor}
\draw (-1.159189\du,30.211078\du)--(-5.292718\du,30.211078\du)--(-5.292718\du,30.228137\du)--(-9.984825\du,30.228137\du);
\pgfsetdash{}{0pt}
\pgfsetmiterjoin
\pgfsetbuttcap
\definecolor{dialinecolor}{rgb}{0.000000, 0.000000, 0.000000}
\pgfsetfillcolor{dialinecolor}
\fill (0.099390\du,30.211078\du)--(-0.600610\du,30.451078\du)--(-1.300610\du,30.211078\du)--(-0.600610\du,29.971078\du)--cycle;
\pgfsetlinewidth{0.100000\du}
\pgfsetdash{}{0pt}
\pgfsetmiterjoin
\pgfsetbuttcap
\definecolor{dialinecolor}{rgb}{0.000000, 0.000000, 0.000000}
\pgfsetstrokecolor{dialinecolor}
\draw (0.099390\du,30.211078\du)--(-0.600610\du,30.451078\du)--(-1.300610\du,30.211078\du)--(-0.600610\du,29.971078\du)--cycle;
% setfont left to latex
\end{tikzpicture}
}
    \caption{Diagramme de classes}
  \end{sidewaysfigure}

  L'introduction d'un Client qui peut déterminer la taille d'un dossier se fait
  à l'aide d'un patron visiteur.

  Le visiteur abstrait est spécialisé pour être accepté par un élément abstrait.
  Ce qui nous permet de définir n'importe quel type de visiteur.

  Le visiteur de taille est une réalisation du visiteur abstrait

  \section{Introduction de changements}

  \subsection{Première requête de changement}
  Pour effectuer le changement, nous introduisons la classe \textsf{Raccourci}
  et \textsf{ElementRaccourciable}. Un raccourci possède un agrégat vers une
  instance d'\textsf{ElementRaccourciable}.

  % ajouter le petit diagramme ici
  \begin{figure}
    \centering
    \resizebox{\textwidth}{!}{% Graphic for TeX using PGF
% Title: /home/guillaume/Documents/Université de Montréal/Automne 2014/Génie logiciel/Devoir/devoir-3/diagramme-de-classes-changement-1.dia
% Creator: Dia v0.97.3
% CreationDate: Sat Dec  6 12:51:27 2014
% For: guillaume
% \usepackage{tikz}
% The following commands are not supported in PSTricks at present
% We define them conditionally, so when they are implemented,
% this pgf file will use them.
\ifx\du\undefined
  \newlength{\du}
\fi
\setlength{\du}{15\unitlength}
\begin{tikzpicture}
\pgftransformxscale{1.000000}
\pgftransformyscale{-1.000000}
\definecolor{dialinecolor}{rgb}{0.000000, 0.000000, 0.000000}
\pgfsetstrokecolor{dialinecolor}
\definecolor{dialinecolor}{rgb}{1.000000, 1.000000, 1.000000}
\pgfsetfillcolor{dialinecolor}
\pgfsetlinewidth{0.100000\du}
\pgfsetdash{}{0pt}
\definecolor{dialinecolor}{rgb}{1.000000, 1.000000, 1.000000}
\pgfsetfillcolor{dialinecolor}
\fill (31.100000\du,1.350000\du)--(31.100000\du,2.750000\du)--(34.912500\du,2.750000\du)--(34.912500\du,1.350000\du)--cycle;
\definecolor{dialinecolor}{rgb}{0.000000, 0.000000, 0.000000}
\pgfsetstrokecolor{dialinecolor}
\draw (31.100000\du,1.350000\du)--(31.100000\du,2.750000\du)--(34.912500\du,2.750000\du)--(34.912500\du,1.350000\du)--cycle;
% setfont left to latex
\definecolor{dialinecolor}{rgb}{0.000000, 0.000000, 0.000000}
\pgfsetstrokecolor{dialinecolor}
\node at (33.006250\du,2.300000\du){Element};
\definecolor{dialinecolor}{rgb}{1.000000, 1.000000, 1.000000}
\pgfsetfillcolor{dialinecolor}
\fill (31.100000\du,2.750000\du)--(31.100000\du,3.150000\du)--(34.912500\du,3.150000\du)--(34.912500\du,2.750000\du)--cycle;
\definecolor{dialinecolor}{rgb}{0.000000, 0.000000, 0.000000}
\pgfsetstrokecolor{dialinecolor}
\draw (31.100000\du,2.750000\du)--(31.100000\du,3.150000\du)--(34.912500\du,3.150000\du)--(34.912500\du,2.750000\du)--cycle;
\definecolor{dialinecolor}{rgb}{1.000000, 1.000000, 1.000000}
\pgfsetfillcolor{dialinecolor}
\fill (31.100000\du,3.150000\du)--(31.100000\du,3.550000\du)--(34.912500\du,3.550000\du)--(34.912500\du,3.150000\du)--cycle;
\definecolor{dialinecolor}{rgb}{0.000000, 0.000000, 0.000000}
\pgfsetstrokecolor{dialinecolor}
\draw (31.100000\du,3.150000\du)--(31.100000\du,3.550000\du)--(34.912500\du,3.550000\du)--(34.912500\du,3.150000\du)--cycle;
\pgfsetlinewidth{0.100000\du}
\pgfsetdash{}{0pt}
\definecolor{dialinecolor}{rgb}{1.000000, 1.000000, 1.000000}
\pgfsetfillcolor{dialinecolor}
\fill (16.600000\du,18.800000\du)--(16.600000\du,20.200000\du)--(21.470000\du,20.200000\du)--(21.470000\du,18.800000\du)--cycle;
\definecolor{dialinecolor}{rgb}{0.000000, 0.000000, 0.000000}
\pgfsetstrokecolor{dialinecolor}
\draw (16.600000\du,18.800000\du)--(16.600000\du,20.200000\du)--(21.470000\du,20.200000\du)--(21.470000\du,18.800000\du)--cycle;
% setfont left to latex
\definecolor{dialinecolor}{rgb}{0.000000, 0.000000, 0.000000}
\pgfsetstrokecolor{dialinecolor}
\node at (19.035000\du,19.750000\du){Raccourci};
\definecolor{dialinecolor}{rgb}{1.000000, 1.000000, 1.000000}
\pgfsetfillcolor{dialinecolor}
\fill (16.600000\du,20.200000\du)--(16.600000\du,20.600000\du)--(21.470000\du,20.600000\du)--(21.470000\du,20.200000\du)--cycle;
\definecolor{dialinecolor}{rgb}{0.000000, 0.000000, 0.000000}
\pgfsetstrokecolor{dialinecolor}
\draw (16.600000\du,20.200000\du)--(16.600000\du,20.600000\du)--(21.470000\du,20.600000\du)--(21.470000\du,20.200000\du)--cycle;
\definecolor{dialinecolor}{rgb}{1.000000, 1.000000, 1.000000}
\pgfsetfillcolor{dialinecolor}
\fill (16.600000\du,20.600000\du)--(16.600000\du,21.000000\du)--(21.470000\du,21.000000\du)--(21.470000\du,20.600000\du)--cycle;
\definecolor{dialinecolor}{rgb}{0.000000, 0.000000, 0.000000}
\pgfsetstrokecolor{dialinecolor}
\draw (16.600000\du,20.600000\du)--(16.600000\du,21.000000\du)--(21.470000\du,21.000000\du)--(21.470000\du,20.600000\du)--cycle;
\pgfsetlinewidth{0.100000\du}
\pgfsetdash{}{0pt}
\pgfsetmiterjoin
\pgfsetbuttcap
{
\definecolor{dialinecolor}{rgb}{0.000000, 0.000000, 0.000000}
\pgfsetfillcolor{dialinecolor}
% was here!!!
\definecolor{dialinecolor}{rgb}{0.000000, 0.000000, 0.000000}
\pgfsetstrokecolor{dialinecolor}
\draw (21.519766\du,19.900000\du)--(28.300000\du,19.900000\du)--(28.300000\du,9.450000\du)--(31.400932\du,9.450000\du);
}
\definecolor{dialinecolor}{rgb}{0.000000, 0.000000, 0.000000}
\pgfsetstrokecolor{dialinecolor}
\draw (22.778345\du,19.900000\du)--(28.300000\du,19.900000\du)--(28.300000\du,9.450000\du)--(31.400932\du,9.450000\du);
\pgfsetdash{}{0pt}
\pgfsetmiterjoin
\pgfsetbuttcap
\definecolor{dialinecolor}{rgb}{1.000000, 1.000000, 1.000000}
\pgfsetfillcolor{dialinecolor}
\fill (21.519766\du,19.900000\du)--(22.219766\du,19.660000\du)--(22.919766\du,19.900000\du)--(22.219766\du,20.140000\du)--cycle;
\pgfsetlinewidth{0.100000\du}
\pgfsetdash{}{0pt}
\pgfsetmiterjoin
\pgfsetbuttcap
\definecolor{dialinecolor}{rgb}{0.000000, 0.000000, 0.000000}
\pgfsetstrokecolor{dialinecolor}
\draw (21.519766\du,19.900000\du)--(22.219766\du,19.660000\du)--(22.919766\du,19.900000\du)--(22.219766\du,20.140000\du)--cycle;
% setfont left to latex
\definecolor{dialinecolor}{rgb}{0.000000, 0.000000, 0.000000}
\pgfsetstrokecolor{dialinecolor}
\node[anchor=west] at (28.400000\du,14.525000\du){};
\definecolor{dialinecolor}{rgb}{0.000000, 0.000000, 0.000000}
\pgfsetstrokecolor{dialinecolor}
\node[anchor=west] at (23.119766\du,19.750000\du){};
\definecolor{dialinecolor}{rgb}{0.000000, 0.000000, 0.000000}
\pgfsetstrokecolor{dialinecolor}
\node[anchor=east] at (31.200932\du,9.300000\du){ pointe vers};
\definecolor{dialinecolor}{rgb}{0.000000, 0.000000, 0.000000}
\pgfsetstrokecolor{dialinecolor}
\node[anchor=east] at (31.200932\du,10.100000\du){*};
\pgfsetlinewidth{0.100000\du}
\pgfsetdash{}{0pt}
\pgfsetmiterjoin
\pgfsetbuttcap
{
\definecolor{dialinecolor}{rgb}{0.000000, 0.000000, 0.000000}
\pgfsetfillcolor{dialinecolor}
% was here!!!
\definecolor{dialinecolor}{rgb}{0.000000, 0.000000, 0.000000}
\pgfsetstrokecolor{dialinecolor}
\draw (32.997452\du,3.599658\du)--(32.980231\du,5.850000\du)--(19.033510\du,5.850000\du)--(19.034878\du,18.750034\du);
}
\definecolor{dialinecolor}{rgb}{0.000000, 0.000000, 0.000000}
\pgfsetstrokecolor{dialinecolor}
\draw (32.990474\du,4.511435\du)--(32.980231\du,5.850000\du)--(19.033510\du,5.850000\du)--(19.034878\du,18.750034\du);
\pgfsetmiterjoin
\definecolor{dialinecolor}{rgb}{1.000000, 1.000000, 1.000000}
\pgfsetfillcolor{dialinecolor}
\fill (33.390463\du,4.514496\du)--(32.996596\du,3.711458\du)--(32.590486\du,4.508374\du)--cycle;
\pgfsetlinewidth{0.100000\du}
\pgfsetdash{}{0pt}
\pgfsetmiterjoin
\definecolor{dialinecolor}{rgb}{0.000000, 0.000000, 0.000000}
\pgfsetstrokecolor{dialinecolor}
\draw (33.390463\du,4.514496\du)--(32.996596\du,3.711458\du)--(32.590486\du,4.508374\du)--cycle;
% setfont left to latex
\pgfsetlinewidth{0.100000\du}
\pgfsetdash{}{0pt}
\definecolor{dialinecolor}{rgb}{1.000000, 1.000000, 1.000000}
\pgfsetfillcolor{dialinecolor}
\fill (31.450000\du,8.050000\du)--(31.450000\du,9.450000\du)--(40.855000\du,9.450000\du)--(40.855000\du,8.050000\du)--cycle;
\definecolor{dialinecolor}{rgb}{0.000000, 0.000000, 0.000000}
\pgfsetstrokecolor{dialinecolor}
\draw (31.450000\du,8.050000\du)--(31.450000\du,9.450000\du)--(40.855000\du,9.450000\du)--(40.855000\du,8.050000\du)--cycle;
% setfont left to latex
\definecolor{dialinecolor}{rgb}{0.000000, 0.000000, 0.000000}
\pgfsetstrokecolor{dialinecolor}
\node at (36.152500\du,9.000000\du){ElementRaccourciable};
\definecolor{dialinecolor}{rgb}{1.000000, 1.000000, 1.000000}
\pgfsetfillcolor{dialinecolor}
\fill (31.450000\du,9.450000\du)--(31.450000\du,9.850000\du)--(40.855000\du,9.850000\du)--(40.855000\du,9.450000\du)--cycle;
\definecolor{dialinecolor}{rgb}{0.000000, 0.000000, 0.000000}
\pgfsetstrokecolor{dialinecolor}
\draw (31.450000\du,9.450000\du)--(31.450000\du,9.850000\du)--(40.855000\du,9.850000\du)--(40.855000\du,9.450000\du)--cycle;
\definecolor{dialinecolor}{rgb}{1.000000, 1.000000, 1.000000}
\pgfsetfillcolor{dialinecolor}
\fill (31.450000\du,9.850000\du)--(31.450000\du,10.850000\du)--(40.855000\du,10.850000\du)--(40.855000\du,9.850000\du)--cycle;
\definecolor{dialinecolor}{rgb}{0.000000, 0.000000, 0.000000}
\pgfsetstrokecolor{dialinecolor}
\draw (31.450000\du,9.850000\du)--(31.450000\du,10.850000\du)--(40.855000\du,10.850000\du)--(40.855000\du,9.850000\du)--cycle;
% setfont left to latex
\definecolor{dialinecolor}{rgb}{0.000000, 0.000000, 0.000000}
\pgfsetstrokecolor{dialinecolor}
\node[anchor=west] at (31.600000\du,10.550000\du){+taille()};
\pgfsetlinewidth{0.100000\du}
\pgfsetdash{}{0pt}
\definecolor{dialinecolor}{rgb}{1.000000, 1.000000, 1.000000}
\pgfsetfillcolor{dialinecolor}
\fill (38.550000\du,16.250000\du)--(38.550000\du,17.650000\du)--(42.130000\du,17.650000\du)--(42.130000\du,16.250000\du)--cycle;
\definecolor{dialinecolor}{rgb}{0.000000, 0.000000, 0.000000}
\pgfsetstrokecolor{dialinecolor}
\draw (38.550000\du,16.250000\du)--(38.550000\du,17.650000\du)--(42.130000\du,17.650000\du)--(42.130000\du,16.250000\du)--cycle;
% setfont left to latex
\definecolor{dialinecolor}{rgb}{0.000000, 0.000000, 0.000000}
\pgfsetstrokecolor{dialinecolor}
\node at (40.340000\du,17.200000\du){Fichier};
\definecolor{dialinecolor}{rgb}{1.000000, 1.000000, 1.000000}
\pgfsetfillcolor{dialinecolor}
\fill (38.550000\du,17.650000\du)--(38.550000\du,18.050000\du)--(42.130000\du,18.050000\du)--(42.130000\du,17.650000\du)--cycle;
\definecolor{dialinecolor}{rgb}{0.000000, 0.000000, 0.000000}
\pgfsetstrokecolor{dialinecolor}
\draw (38.550000\du,17.650000\du)--(38.550000\du,18.050000\du)--(42.130000\du,18.050000\du)--(42.130000\du,17.650000\du)--cycle;
\definecolor{dialinecolor}{rgb}{1.000000, 1.000000, 1.000000}
\pgfsetfillcolor{dialinecolor}
\fill (38.550000\du,18.050000\du)--(38.550000\du,18.450000\du)--(42.130000\du,18.450000\du)--(42.130000\du,18.050000\du)--cycle;
\definecolor{dialinecolor}{rgb}{0.000000, 0.000000, 0.000000}
\pgfsetstrokecolor{dialinecolor}
\draw (38.550000\du,18.050000\du)--(38.550000\du,18.450000\du)--(42.130000\du,18.450000\du)--(42.130000\du,18.050000\du)--cycle;
\pgfsetlinewidth{0.100000\du}
\pgfsetdash{}{0pt}
\definecolor{dialinecolor}{rgb}{1.000000, 1.000000, 1.000000}
\pgfsetfillcolor{dialinecolor}
\fill (31.050000\du,16.250000\du)--(31.050000\du,17.650000\du)--(34.927500\du,17.650000\du)--(34.927500\du,16.250000\du)--cycle;
\definecolor{dialinecolor}{rgb}{0.000000, 0.000000, 0.000000}
\pgfsetstrokecolor{dialinecolor}
\draw (31.050000\du,16.250000\du)--(31.050000\du,17.650000\du)--(34.927500\du,17.650000\du)--(34.927500\du,16.250000\du)--cycle;
% setfont left to latex
\definecolor{dialinecolor}{rgb}{0.000000, 0.000000, 0.000000}
\pgfsetstrokecolor{dialinecolor}
\node at (32.988750\du,17.200000\du){Dossier};
\definecolor{dialinecolor}{rgb}{1.000000, 1.000000, 1.000000}
\pgfsetfillcolor{dialinecolor}
\fill (31.050000\du,17.650000\du)--(31.050000\du,18.050000\du)--(34.927500\du,18.050000\du)--(34.927500\du,17.650000\du)--cycle;
\definecolor{dialinecolor}{rgb}{0.000000, 0.000000, 0.000000}
\pgfsetstrokecolor{dialinecolor}
\draw (31.050000\du,17.650000\du)--(31.050000\du,18.050000\du)--(34.927500\du,18.050000\du)--(34.927500\du,17.650000\du)--cycle;
\definecolor{dialinecolor}{rgb}{1.000000, 1.000000, 1.000000}
\pgfsetfillcolor{dialinecolor}
\fill (31.050000\du,18.050000\du)--(31.050000\du,18.450000\du)--(34.927500\du,18.450000\du)--(34.927500\du,18.050000\du)--cycle;
\definecolor{dialinecolor}{rgb}{0.000000, 0.000000, 0.000000}
\pgfsetstrokecolor{dialinecolor}
\draw (31.050000\du,18.050000\du)--(31.050000\du,18.450000\du)--(34.927500\du,18.450000\du)--(34.927500\du,18.050000\du)--cycle;
\pgfsetlinewidth{0.100000\du}
\pgfsetdash{}{0pt}
\pgfsetmiterjoin
\pgfsetbuttcap
{
\definecolor{dialinecolor}{rgb}{0.000000, 0.000000, 0.000000}
\pgfsetfillcolor{dialinecolor}
% was here!!!
\definecolor{dialinecolor}{rgb}{0.000000, 0.000000, 0.000000}
\pgfsetstrokecolor{dialinecolor}
\draw (36.152500\du,10.900354\du)--(36.152500\du,13.950037\du)--(32.988750\du,13.950037\du)--(32.988750\du,16.199719\du);
}
\definecolor{dialinecolor}{rgb}{0.000000, 0.000000, 0.000000}
\pgfsetstrokecolor{dialinecolor}
\draw (36.152500\du,11.812157\du)--(36.152500\du,13.950037\du)--(32.988750\du,13.950037\du)--(32.988750\du,16.199719\du);
\pgfsetmiterjoin
\definecolor{dialinecolor}{rgb}{1.000000, 1.000000, 1.000000}
\pgfsetfillcolor{dialinecolor}
\fill (36.552500\du,11.812157\du)--(36.152500\du,11.012157\du)--(35.752500\du,11.812157\du)--cycle;
\pgfsetlinewidth{0.100000\du}
\pgfsetdash{}{0pt}
\pgfsetmiterjoin
\definecolor{dialinecolor}{rgb}{0.000000, 0.000000, 0.000000}
\pgfsetstrokecolor{dialinecolor}
\draw (36.552500\du,11.812157\du)--(36.152500\du,11.012157\du)--(35.752500\du,11.812157\du)--cycle;
% setfont left to latex
\pgfsetlinewidth{0.100000\du}
\pgfsetdash{}{0pt}
\pgfsetmiterjoin
\pgfsetbuttcap
{
\definecolor{dialinecolor}{rgb}{0.000000, 0.000000, 0.000000}
\pgfsetfillcolor{dialinecolor}
% was here!!!
\definecolor{dialinecolor}{rgb}{0.000000, 0.000000, 0.000000}
\pgfsetstrokecolor{dialinecolor}
\draw (36.152500\du,10.900354\du)--(36.152500\du,13.950037\du)--(40.340000\du,13.950037\du)--(40.340000\du,16.199719\du);
}
\definecolor{dialinecolor}{rgb}{0.000000, 0.000000, 0.000000}
\pgfsetstrokecolor{dialinecolor}
\draw (36.152500\du,11.812157\du)--(36.152500\du,13.950037\du)--(40.340000\du,13.950037\du)--(40.340000\du,16.199719\du);
\pgfsetmiterjoin
\definecolor{dialinecolor}{rgb}{1.000000, 1.000000, 1.000000}
\pgfsetfillcolor{dialinecolor}
\fill (36.552500\du,11.812157\du)--(36.152500\du,11.012157\du)--(35.752500\du,11.812157\du)--cycle;
\pgfsetlinewidth{0.100000\du}
\pgfsetdash{}{0pt}
\pgfsetmiterjoin
\definecolor{dialinecolor}{rgb}{0.000000, 0.000000, 0.000000}
\pgfsetstrokecolor{dialinecolor}
\draw (36.552500\du,11.812157\du)--(36.152500\du,11.012157\du)--(35.752500\du,11.812157\du)--cycle;
% setfont left to latex
\pgfsetlinewidth{0.100000\du}
\pgfsetdash{}{0pt}
\pgfsetmiterjoin
\pgfsetbuttcap
{
\definecolor{dialinecolor}{rgb}{0.000000, 0.000000, 0.000000}
\pgfsetfillcolor{dialinecolor}
% was here!!!
\definecolor{dialinecolor}{rgb}{0.000000, 0.000000, 0.000000}
\pgfsetstrokecolor{dialinecolor}
\draw (32.997035\du,3.600336\du)--(32.979415\du,5.800000\du)--(36.300000\du,5.800000\du)--(36.300000\du,7.950000\du);
}
\definecolor{dialinecolor}{rgb}{0.000000, 0.000000, 0.000000}
\pgfsetstrokecolor{dialinecolor}
\draw (32.989731\du,4.512110\du)--(32.979415\du,5.800000\du)--(36.300000\du,5.800000\du)--(36.300000\du,7.950000\du);
\pgfsetmiterjoin
\definecolor{dialinecolor}{rgb}{1.000000, 1.000000, 1.000000}
\pgfsetfillcolor{dialinecolor}
\fill (33.389718\du,4.515314\du)--(32.996140\du,3.712136\du)--(32.589744\du,4.508906\du)--cycle;
\pgfsetlinewidth{0.100000\du}
\pgfsetdash{}{0pt}
\pgfsetmiterjoin
\definecolor{dialinecolor}{rgb}{0.000000, 0.000000, 0.000000}
\pgfsetstrokecolor{dialinecolor}
\draw (33.389718\du,4.515314\du)--(32.996140\du,3.712136\du)--(32.589744\du,4.508906\du)--cycle;
% setfont left to latex
\end{tikzpicture}
}
    \caption{Diagramme de classes pour les raccourcis}
  \end{figure}

  Il s'agit d'une version modifiée du partron décorateur utilisé pour
  \textsf{ElementDecorator}, car un agrégat le lie à l'élément qu'il pointe.

  Les classes \textsf{Fichier} et \textsf{Dossier} seront impactés, car elles
  devront désormais hériter de la classe \textsf{ElementRaccourciable}.

  Il faut aussi déplacer la fonction \textsf{taille} de la classe
  \textsf{Element} vers la nouvelle classe \textsf{ElementRaccourciable}, car un
  raccourci ne possède pas de taille à proprement parler. On évite notamment un
  problème cycle pour le calcul de la taille d'un dossier dans le cas ou un
  dossier contiendrait un raccourcis qui pointerait vers ce même dossier.

  L'impact n'est pas localisé, car il affecte des parties existantes du
  programme. Notamment les classes \textsf{Fichier} et \textsf{Dossier} qui
  doivent hériter d'un nouvelle classe.

  \subsection{Deuxième requête de changement}
  Pour implanter ce changement, nous introduisons le singleton
  \textsf{GestionnaireRaccourcis} et nous lui faisons implanter l'interface
  \textsf{DeleteObserver}.

  % ajout du deuxième diagramme
  \begin{figure}
    \centering
    \resizebox{\textwidth}{!}{% Graphic for TeX using PGF
% Title: /home/guillaume/Documents/Université de Montréal/Automne 2014/Génie logiciel/Devoir/devoir-3/diagramme-de-classes-changement-2.dia
% Creator: Dia v0.97.3
% CreationDate: Sun Dec  7 16:12:44 2014
% For: guillaume
% \usepackage{tikz}
% The following commands are not supported in PSTricks at present
% We define them conditionally, so when they are implemented,
% this pgf file will use them.
\ifx\du\undefined
  \newlength{\du}
\fi
\setlength{\du}{15\unitlength}
\begin{tikzpicture}
\pgftransformxscale{1.000000}
\pgftransformyscale{-1.000000}
\definecolor{dialinecolor}{rgb}{0.000000, 0.000000, 0.000000}
\pgfsetstrokecolor{dialinecolor}
\definecolor{dialinecolor}{rgb}{1.000000, 1.000000, 1.000000}
\pgfsetfillcolor{dialinecolor}
\pgfsetlinewidth{0.100000\du}
\pgfsetdash{}{0pt}
\definecolor{dialinecolor}{rgb}{1.000000, 1.000000, 1.000000}
\pgfsetfillcolor{dialinecolor}
\fill (13.909500\du,17.090000\du)--(13.909500\du,18.490000\du)--(18.779500\du,18.490000\du)--(18.779500\du,17.090000\du)--cycle;
\definecolor{dialinecolor}{rgb}{0.000000, 0.000000, 0.000000}
\pgfsetstrokecolor{dialinecolor}
\draw (13.909500\du,17.090000\du)--(13.909500\du,18.490000\du)--(18.779500\du,18.490000\du)--(18.779500\du,17.090000\du)--cycle;
% setfont left to latex
\definecolor{dialinecolor}{rgb}{0.000000, 0.000000, 0.000000}
\pgfsetstrokecolor{dialinecolor}
\node at (16.344500\du,18.040000\du){Raccourci};
\definecolor{dialinecolor}{rgb}{1.000000, 1.000000, 1.000000}
\pgfsetfillcolor{dialinecolor}
\fill (13.909500\du,18.490000\du)--(13.909500\du,18.890000\du)--(18.779500\du,18.890000\du)--(18.779500\du,18.490000\du)--cycle;
\definecolor{dialinecolor}{rgb}{0.000000, 0.000000, 0.000000}
\pgfsetstrokecolor{dialinecolor}
\draw (13.909500\du,18.490000\du)--(13.909500\du,18.890000\du)--(18.779500\du,18.890000\du)--(18.779500\du,18.490000\du)--cycle;
\definecolor{dialinecolor}{rgb}{1.000000, 1.000000, 1.000000}
\pgfsetfillcolor{dialinecolor}
\fill (13.909500\du,18.890000\du)--(13.909500\du,19.290000\du)--(18.779500\du,19.290000\du)--(18.779500\du,18.890000\du)--cycle;
\definecolor{dialinecolor}{rgb}{0.000000, 0.000000, 0.000000}
\pgfsetstrokecolor{dialinecolor}
\draw (13.909500\du,18.890000\du)--(13.909500\du,19.290000\du)--(18.779500\du,19.290000\du)--(18.779500\du,18.890000\du)--cycle;
\pgfsetlinewidth{0.100000\du}
\pgfsetdash{}{0pt}
\pgfsetmiterjoin
\pgfsetbuttcap
{
\definecolor{dialinecolor}{rgb}{0.000000, 0.000000, 0.000000}
\pgfsetfillcolor{dialinecolor}
% was here!!!
\definecolor{dialinecolor}{rgb}{0.000000, 0.000000, 0.000000}
\pgfsetstrokecolor{dialinecolor}
\draw (13.860710\du,18.190000\du)--(1.862500\du,18.190000\du)--(1.862500\du,-2.265000\du)--(5.139403\du,-2.265000\du);
}
\definecolor{dialinecolor}{rgb}{0.000000, 0.000000, 0.000000}
\pgfsetstrokecolor{dialinecolor}
\draw (12.602131\du,18.190000\du)--(1.862500\du,18.190000\du)--(1.862500\du,-2.265000\du)--(5.139403\du,-2.265000\du);
\pgfsetdash{}{0pt}
\pgfsetmiterjoin
\pgfsetbuttcap
\definecolor{dialinecolor}{rgb}{1.000000, 1.000000, 1.000000}
\pgfsetfillcolor{dialinecolor}
\fill (13.860710\du,18.190000\du)--(13.160710\du,18.430000\du)--(12.460710\du,18.190000\du)--(13.160710\du,17.950000\du)--cycle;
\pgfsetlinewidth{0.100000\du}
\pgfsetdash{}{0pt}
\pgfsetmiterjoin
\pgfsetbuttcap
\definecolor{dialinecolor}{rgb}{0.000000, 0.000000, 0.000000}
\pgfsetstrokecolor{dialinecolor}
\draw (13.860710\du,18.190000\du)--(13.160710\du,18.430000\du)--(12.460710\du,18.190000\du)--(13.160710\du,17.950000\du)--cycle;
% setfont left to latex
\definecolor{dialinecolor}{rgb}{0.000000, 0.000000, 0.000000}
\pgfsetstrokecolor{dialinecolor}
\node[anchor=west] at (1.962500\du,7.812500\du){};
\definecolor{dialinecolor}{rgb}{0.000000, 0.000000, 0.000000}
\pgfsetfillcolor{dialinecolor}
\fill (2.062500\du,7.812500\du)--(2.062500\du,7.412500\du)--(2.462500\du,7.612500\du)--cycle;
\definecolor{dialinecolor}{rgb}{0.000000, 0.000000, 0.000000}
\pgfsetstrokecolor{dialinecolor}
\node[anchor=east] at (12.260710\du,18.040000\du){};
\definecolor{dialinecolor}{rgb}{0.000000, 0.000000, 0.000000}
\pgfsetstrokecolor{dialinecolor}
\node[anchor=east] at (4.939403\du,-2.415000\du){ pointe vers};
\definecolor{dialinecolor}{rgb}{0.000000, 0.000000, 0.000000}
\pgfsetstrokecolor{dialinecolor}
\node[anchor=east] at (4.939403\du,-1.615000\du){*};
\pgfsetlinewidth{0.100000\du}
\pgfsetdash{}{0pt}
\pgfsetmiterjoin
\pgfsetbuttcap
{
\definecolor{dialinecolor}{rgb}{0.000000, 0.000000, 0.000000}
\pgfsetfillcolor{dialinecolor}
% was here!!!
\definecolor{dialinecolor}{rgb}{0.000000, 0.000000, 0.000000}
\pgfsetstrokecolor{dialinecolor}
\draw (12.365000\du,1.885253\du)--(12.365000\du,9.862486\du)--(16.344500\du,9.862486\du)--(16.344500\du,17.039719\du);
}
\definecolor{dialinecolor}{rgb}{0.000000, 0.000000, 0.000000}
\pgfsetstrokecolor{dialinecolor}
\draw (12.365000\du,2.797057\du)--(12.365000\du,9.862486\du)--(16.344500\du,9.862486\du)--(16.344500\du,17.039719\du);
\pgfsetmiterjoin
\definecolor{dialinecolor}{rgb}{1.000000, 1.000000, 1.000000}
\pgfsetfillcolor{dialinecolor}
\fill (12.765000\du,2.797057\du)--(12.365000\du,1.997057\du)--(11.965000\du,2.797057\du)--cycle;
\pgfsetlinewidth{0.100000\du}
\pgfsetdash{}{0pt}
\pgfsetmiterjoin
\definecolor{dialinecolor}{rgb}{0.000000, 0.000000, 0.000000}
\pgfsetstrokecolor{dialinecolor}
\draw (12.765000\du,2.797057\du)--(12.365000\du,1.997057\du)--(11.965000\du,2.797057\du)--cycle;
% setfont left to latex
\pgfsetlinewidth{0.100000\du}
\pgfsetdash{}{0pt}
\definecolor{dialinecolor}{rgb}{1.000000, 1.000000, 1.000000}
\pgfsetfillcolor{dialinecolor}
\fill (23.859500\du,-1.260000\du)--(23.859500\du,0.140000\du)--(30.659500\du,0.140000\du)--(30.659500\du,-1.260000\du)--cycle;
\definecolor{dialinecolor}{rgb}{0.000000, 0.000000, 0.000000}
\pgfsetstrokecolor{dialinecolor}
\draw (23.859500\du,-1.260000\du)--(23.859500\du,0.140000\du)--(30.659500\du,0.140000\du)--(30.659500\du,-1.260000\du)--cycle;
% setfont left to latex
\definecolor{dialinecolor}{rgb}{0.000000, 0.000000, 0.000000}
\pgfsetstrokecolor{dialinecolor}
\node at (27.259500\du,-0.310000\du){DeleteObserver};
\definecolor{dialinecolor}{rgb}{1.000000, 1.000000, 1.000000}
\pgfsetfillcolor{dialinecolor}
\fill (23.859500\du,0.140000\du)--(23.859500\du,0.540000\du)--(30.659500\du,0.540000\du)--(30.659500\du,0.140000\du)--cycle;
\definecolor{dialinecolor}{rgb}{0.000000, 0.000000, 0.000000}
\pgfsetstrokecolor{dialinecolor}
\draw (23.859500\du,0.140000\du)--(23.859500\du,0.540000\du)--(30.659500\du,0.540000\du)--(30.659500\du,0.140000\du)--cycle;
\definecolor{dialinecolor}{rgb}{1.000000, 1.000000, 1.000000}
\pgfsetfillcolor{dialinecolor}
\fill (23.859500\du,0.540000\du)--(23.859500\du,0.940000\du)--(30.659500\du,0.940000\du)--(30.659500\du,0.540000\du)--cycle;
\definecolor{dialinecolor}{rgb}{0.000000, 0.000000, 0.000000}
\pgfsetstrokecolor{dialinecolor}
\draw (23.859500\du,0.540000\du)--(23.859500\du,0.940000\du)--(30.659500\du,0.940000\du)--(30.659500\du,0.540000\du)--cycle;
\pgfsetlinewidth{0.100000\du}
\pgfsetdash{{1.000000\du}{1.000000\du}}{0\du}
\pgfsetdash{{0.400000\du}{0.400000\du}}{0\du}
\pgfsetmiterjoin
\pgfsetbuttcap
{
\definecolor{dialinecolor}{rgb}{0.000000, 0.000000, 0.000000}
\pgfsetfillcolor{dialinecolor}
% was here!!!
\definecolor{dialinecolor}{rgb}{0.000000, 0.000000, 0.000000}
\pgfsetstrokecolor{dialinecolor}
\draw (27.259500\du,0.990281\du)--(27.259500\du,9.120009\du)--(33.115000\du,9.120009\du)--(33.115000\du,16.449738\du);
}
\definecolor{dialinecolor}{rgb}{0.000000, 0.000000, 0.000000}
\pgfsetstrokecolor{dialinecolor}
\draw (27.259500\du,1.902084\du)--(27.259500\du,9.120009\du)--(33.115000\du,9.120009\du)--(33.115000\du,16.449738\du);
\pgfsetmiterjoin
\definecolor{dialinecolor}{rgb}{1.000000, 1.000000, 1.000000}
\pgfsetfillcolor{dialinecolor}
\fill (27.659500\du,1.902084\du)--(27.259500\du,1.102084\du)--(26.859500\du,1.902084\du)--cycle;
\pgfsetlinewidth{0.100000\du}
\pgfsetdash{}{0pt}
\pgfsetmiterjoin
\definecolor{dialinecolor}{rgb}{0.000000, 0.000000, 0.000000}
\pgfsetstrokecolor{dialinecolor}
\draw (27.659500\du,1.902084\du)--(27.259500\du,1.102084\du)--(26.859500\du,1.902084\du)--cycle;
% setfont left to latex
\pgfsetlinewidth{0.100000\du}
\pgfsetdash{}{0pt}
\definecolor{dialinecolor}{rgb}{1.000000, 1.000000, 1.000000}
\pgfsetfillcolor{dialinecolor}
\fill (25.550000\du,16.500000\du)--(25.550000\du,17.900000\du)--(40.680000\du,17.900000\du)--(40.680000\du,16.500000\du)--cycle;
\definecolor{dialinecolor}{rgb}{0.000000, 0.000000, 0.000000}
\pgfsetstrokecolor{dialinecolor}
\draw (25.550000\du,16.500000\du)--(25.550000\du,17.900000\du)--(40.680000\du,17.900000\du)--(40.680000\du,16.500000\du)--cycle;
% setfont left to latex
\definecolor{dialinecolor}{rgb}{0.000000, 0.000000, 0.000000}
\pgfsetstrokecolor{dialinecolor}
\node at (33.115000\du,17.450000\du){GestionnaireRaccourcis};
\definecolor{dialinecolor}{rgb}{1.000000, 1.000000, 1.000000}
\pgfsetfillcolor{dialinecolor}
\fill (25.550000\du,17.900000\du)--(25.550000\du,18.900000\du)--(40.680000\du,18.900000\du)--(40.680000\du,17.900000\du)--cycle;
\definecolor{dialinecolor}{rgb}{0.000000, 0.000000, 0.000000}
\pgfsetstrokecolor{dialinecolor}
\draw (25.550000\du,17.900000\du)--(25.550000\du,18.900000\du)--(40.680000\du,18.900000\du)--(40.680000\du,17.900000\du)--cycle;
% setfont left to latex
\definecolor{dialinecolor}{rgb}{0.000000, 0.000000, 0.000000}
\pgfsetstrokecolor{dialinecolor}
\node[anchor=west] at (25.700000\du,18.600000\du){-instance: GestionnaireRaccourcis};
\definecolor{dialinecolor}{rgb}{1.000000, 1.000000, 1.000000}
\pgfsetfillcolor{dialinecolor}
\fill (25.550000\du,18.900000\du)--(25.550000\du,20.700000\du)--(40.680000\du,20.700000\du)--(40.680000\du,18.900000\du)--cycle;
\definecolor{dialinecolor}{rgb}{0.000000, 0.000000, 0.000000}
\pgfsetstrokecolor{dialinecolor}
\draw (25.550000\du,18.900000\du)--(25.550000\du,20.700000\du)--(40.680000\du,20.700000\du)--(40.680000\du,18.900000\du)--cycle;
% setfont left to latex
\definecolor{dialinecolor}{rgb}{0.000000, 0.000000, 0.000000}
\pgfsetstrokecolor{dialinecolor}
\node[anchor=west] at (25.700000\du,19.600000\du){+getInstance(): GestionnaireRaccourcis};
% setfont left to latex
\definecolor{dialinecolor}{rgb}{0.000000, 0.000000, 0.000000}
\pgfsetstrokecolor{dialinecolor}
\node[anchor=west] at (25.700000\du,20.400000\du){+<<Constructeur>> ()};
\pgfsetlinewidth{0.100000\du}
\pgfsetdash{}{0pt}
\pgfsetmiterjoin
\pgfsetbuttcap
{
\definecolor{dialinecolor}{rgb}{0.000000, 0.000000, 0.000000}
\pgfsetfillcolor{dialinecolor}
% was here!!!
\definecolor{dialinecolor}{rgb}{0.000000, 0.000000, 0.000000}
\pgfsetstrokecolor{dialinecolor}
\draw (18.829803\du,18.190000\du)--(22.164669\du,18.190000\du)--(22.164669\du,18.600000\du)--(25.499535\du,18.600000\du);
}
% setfont left to latex
\definecolor{dialinecolor}{rgb}{0.000000, 0.000000, 0.000000}
\pgfsetstrokecolor{dialinecolor}
\node[anchor=west] at (22.264669\du,18.245000\du){};
\definecolor{dialinecolor}{rgb}{0.000000, 0.000000, 0.000000}
\pgfsetstrokecolor{dialinecolor}
\node[anchor=west] at (19.029803\du,18.040000\du){*};
\definecolor{dialinecolor}{rgb}{0.000000, 0.000000, 0.000000}
\pgfsetstrokecolor{dialinecolor}
\node[anchor=east] at (25.299535\du,18.450000\du){1};
\pgfsetlinewidth{0.100000\du}
\pgfsetdash{}{0pt}
\definecolor{dialinecolor}{rgb}{1.000000, 1.000000, 1.000000}
\pgfsetfillcolor{dialinecolor}
\fill (5.185000\du,-6.365000\du)--(5.185000\du,-4.965000\du)--(19.545000\du,-4.965000\du)--(19.545000\du,-6.365000\du)--cycle;
\definecolor{dialinecolor}{rgb}{0.000000, 0.000000, 0.000000}
\pgfsetstrokecolor{dialinecolor}
\draw (5.185000\du,-6.365000\du)--(5.185000\du,-4.965000\du)--(19.545000\du,-4.965000\du)--(19.545000\du,-6.365000\du)--cycle;
% setfont left to latex
\definecolor{dialinecolor}{rgb}{0.000000, 0.000000, 0.000000}
\pgfsetstrokecolor{dialinecolor}
\node at (12.365000\du,-5.415000\du){Element};
\definecolor{dialinecolor}{rgb}{1.000000, 1.000000, 1.000000}
\pgfsetfillcolor{dialinecolor}
\fill (5.185000\du,-4.965000\du)--(5.185000\du,-0.765000\du)--(19.545000\du,-0.765000\du)--(19.545000\du,-4.965000\du)--cycle;
\definecolor{dialinecolor}{rgb}{0.000000, 0.000000, 0.000000}
\pgfsetstrokecolor{dialinecolor}
\draw (5.185000\du,-4.965000\du)--(5.185000\du,-0.765000\du)--(19.545000\du,-0.765000\du)--(19.545000\du,-4.965000\du)--cycle;
% setfont left to latex
\definecolor{dialinecolor}{rgb}{0.000000, 0.000000, 0.000000}
\pgfsetstrokecolor{dialinecolor}
\node[anchor=west] at (5.335000\du,-4.265000\du){\#Nom: String};
% setfont left to latex
\definecolor{dialinecolor}{rgb}{0.000000, 0.000000, 0.000000}
\pgfsetstrokecolor{dialinecolor}
\node[anchor=west] at (5.335000\du,-3.465000\du){\#Date de création: Date};
% setfont left to latex
\definecolor{dialinecolor}{rgb}{0.000000, 0.000000, 0.000000}
\pgfsetstrokecolor{dialinecolor}
\node[anchor=west] at (5.335000\du,-2.665000\du){\#Date de dernière modification: Date};
% setfont left to latex
\definecolor{dialinecolor}{rgb}{0.000000, 0.000000, 0.000000}
\pgfsetstrokecolor{dialinecolor}
\node[anchor=west] at (5.335000\du,-1.865000\du){\#Chemin: String};
% setfont left to latex
\definecolor{dialinecolor}{rgb}{0.000000, 0.000000, 0.000000}
\pgfsetstrokecolor{dialinecolor}
\node[anchor=west] at (5.335000\du,-1.065000\du){\#Ouvert: Boolean = False};
\definecolor{dialinecolor}{rgb}{1.000000, 1.000000, 1.000000}
\pgfsetfillcolor{dialinecolor}
\fill (5.185000\du,-0.765000\du)--(5.185000\du,1.835000\du)--(19.545000\du,1.835000\du)--(19.545000\du,-0.765000\du)--cycle;
\definecolor{dialinecolor}{rgb}{0.000000, 0.000000, 0.000000}
\pgfsetstrokecolor{dialinecolor}
\draw (5.185000\du,-0.765000\du)--(5.185000\du,1.835000\du)--(19.545000\du,1.835000\du)--(19.545000\du,-0.765000\du)--cycle;
% setfont left to latex
\definecolor{dialinecolor}{rgb}{0.000000, 0.000000, 0.000000}
\pgfsetstrokecolor{dialinecolor}
\node[anchor=west] at (5.335000\du,-0.065000\du){+open()};
% setfont left to latex
\definecolor{dialinecolor}{rgb}{0.000000, 0.000000, 0.000000}
\pgfsetstrokecolor{dialinecolor}
\node[anchor=west] at (5.335000\du,0.735000\du){+close()};
% setfont left to latex
\definecolor{dialinecolor}{rgb}{0.000000, 0.000000, 0.000000}
\pgfsetstrokecolor{dialinecolor}
\node[anchor=west] at (5.335000\du,1.535000\du){+delete()};
\end{tikzpicture}
}
    \caption{Diagramme de classes pour le gestionnaire de raccourcis}
  \end{figure}

  Lorsqu'un un raccourcis est créé, il attache le gestionnaire en tant que
  DeleteObserver de l'élément qu'il pointe. Lorsque l'élément pointé est
  supprimé, le gestionnaire capture l'événement et est en mesure de supprimer le
  raccourcis approprié.

  Le changement est localisé, car il n'affecte pas le code existant. Seule
  la classe \textsf{Raccourci} est modifiée pour s'enregistrer auprès du
  gestionnaire lorsqu'il est créé.

\end{document}
