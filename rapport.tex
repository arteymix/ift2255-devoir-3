\documentclass{article}

\usepackage[utf8]{inputenc}
\usepackage[T1]{fontenc}
\usepackage[french]{babel}

\usepackage{graphics}
\usepackage{fullpage}
\usepackage{rotating}
\usepackage{tikz}

\title{IFT2255 -- Génie logiciel \\Devoir 3}
\author{Vincent Antaki \\ Guillaume Poirier-Morency \\ Émile Trottier}

\begin{document}

  \maketitle

  \abstract
  L'objectif du travail est d'analyser et conçevoir un système confronté à une
  série de changements.

  Le projet a été séparé en trois paquetages Java. Pour minimiser
  l'implantation, les classes non-modifiées sont réutilisés depuis les
  paquetages précédents.

  Le projet est accompagné d'une série de tests JUnit au lieu de spécifier une
  classe \textsf{Main}. C'est plus simple et cela nous permet de soumettre notre
  implantation à des cas d'utilisation plus réalistes.

  \section{Diagramme de classes initial}
  \begin{sidewaysfigure}
    \centering
    \resizebox{\textwidth}{!}{% Graphic for TeX using PGF
% Title: /home/guillaume/Documents/Université de Montréal/Automne 2014/Génie logiciel/Devoir/devoir-3/diagramme-de-classes.dia
% Creator: Dia v0.97.3
% CreationDate: Thu Dec  4 18:16:33 2014
% For: guillaume
% \usepackage{tikz}
% The following commands are not supported in PSTricks at present
% We define them conditionally, so when they are implemented,
% this pgf file will use them.
\ifx\du\undefined
  \newlength{\du}
\fi
\setlength{\du}{15\unitlength}
\begin{tikzpicture}
\pgftransformxscale{1.000000}
\pgftransformyscale{-1.000000}
\definecolor{dialinecolor}{rgb}{0.000000, 0.000000, 0.000000}
\pgfsetstrokecolor{dialinecolor}
\definecolor{dialinecolor}{rgb}{1.000000, 1.000000, 1.000000}
\pgfsetfillcolor{dialinecolor}
\pgfsetlinewidth{0.100000\du}
\pgfsetdash{}{0pt}
\definecolor{dialinecolor}{rgb}{1.000000, 1.000000, 1.000000}
\pgfsetfillcolor{dialinecolor}
\fill (24.000000\du,13.000000\du)--(24.000000\du,14.400000\du)--(30.660000\du,14.400000\du)--(30.660000\du,13.000000\du)--cycle;
\definecolor{dialinecolor}{rgb}{0.000000, 0.000000, 0.000000}
\pgfsetstrokecolor{dialinecolor}
\draw (24.000000\du,13.000000\du)--(24.000000\du,14.400000\du)--(30.660000\du,14.400000\du)--(30.660000\du,13.000000\du)--cycle;
% setfont left to latex
\definecolor{dialinecolor}{rgb}{0.000000, 0.000000, 0.000000}
\pgfsetstrokecolor{dialinecolor}
\node at (27.330000\du,13.950000\du){Fichier};
\definecolor{dialinecolor}{rgb}{1.000000, 1.000000, 1.000000}
\pgfsetfillcolor{dialinecolor}
\fill (24.000000\du,14.400000\du)--(24.000000\du,15.400000\du)--(30.660000\du,15.400000\du)--(30.660000\du,14.400000\du)--cycle;
\definecolor{dialinecolor}{rgb}{0.000000, 0.000000, 0.000000}
\pgfsetstrokecolor{dialinecolor}
\draw (24.000000\du,14.400000\du)--(24.000000\du,15.400000\du)--(30.660000\du,15.400000\du)--(30.660000\du,14.400000\du)--cycle;
% setfont left to latex
\definecolor{dialinecolor}{rgb}{0.000000, 0.000000, 0.000000}
\pgfsetstrokecolor{dialinecolor}
\node[anchor=west] at (24.150000\du,15.100000\du){-taille: Integer};
\definecolor{dialinecolor}{rgb}{1.000000, 1.000000, 1.000000}
\pgfsetfillcolor{dialinecolor}
\fill (24.000000\du,15.400000\du)--(24.000000\du,15.800000\du)--(30.660000\du,15.800000\du)--(30.660000\du,15.400000\du)--cycle;
\definecolor{dialinecolor}{rgb}{0.000000, 0.000000, 0.000000}
\pgfsetstrokecolor{dialinecolor}
\draw (24.000000\du,15.400000\du)--(24.000000\du,15.800000\du)--(30.660000\du,15.800000\du)--(30.660000\du,15.400000\du)--cycle;
\pgfsetlinewidth{0.100000\du}
\pgfsetdash{}{0pt}
\definecolor{dialinecolor}{rgb}{1.000000, 1.000000, 1.000000}
\pgfsetfillcolor{dialinecolor}
\fill (32.900000\du,17.100000\du)--(32.900000\du,18.500000\du)--(40.330000\du,18.500000\du)--(40.330000\du,17.100000\du)--cycle;
\definecolor{dialinecolor}{rgb}{0.000000, 0.000000, 0.000000}
\pgfsetstrokecolor{dialinecolor}
\draw (32.900000\du,17.100000\du)--(32.900000\du,18.500000\du)--(40.330000\du,18.500000\du)--(40.330000\du,17.100000\du)--cycle;
% setfont left to latex
\definecolor{dialinecolor}{rgb}{0.000000, 0.000000, 0.000000}
\pgfsetstrokecolor{dialinecolor}
\node at (36.615000\du,18.050000\du){Dossier};
\definecolor{dialinecolor}{rgb}{1.000000, 1.000000, 1.000000}
\pgfsetfillcolor{dialinecolor}
\fill (32.900000\du,18.500000\du)--(32.900000\du,18.900000\du)--(40.330000\du,18.900000\du)--(40.330000\du,18.500000\du)--cycle;
\definecolor{dialinecolor}{rgb}{0.000000, 0.000000, 0.000000}
\pgfsetstrokecolor{dialinecolor}
\draw (32.900000\du,18.500000\du)--(32.900000\du,18.900000\du)--(40.330000\du,18.900000\du)--(40.330000\du,18.500000\du)--cycle;
\definecolor{dialinecolor}{rgb}{1.000000, 1.000000, 1.000000}
\pgfsetfillcolor{dialinecolor}
\fill (32.900000\du,18.900000\du)--(32.900000\du,19.900000\du)--(40.330000\du,19.900000\du)--(40.330000\du,18.900000\du)--cycle;
\definecolor{dialinecolor}{rgb}{0.000000, 0.000000, 0.000000}
\pgfsetstrokecolor{dialinecolor}
\draw (32.900000\du,18.900000\du)--(32.900000\du,19.900000\du)--(40.330000\du,19.900000\du)--(40.330000\du,18.900000\du)--cycle;
% setfont left to latex
\definecolor{dialinecolor}{rgb}{0.000000, 0.000000, 0.000000}
\pgfsetstrokecolor{dialinecolor}
\node[anchor=west] at (33.050000\du,19.600000\du){+taille(): Integer};
\pgfsetlinewidth{0.100000\du}
\pgfsetdash{}{0pt}
\definecolor{dialinecolor}{rgb}{1.000000, 1.000000, 1.000000}
\pgfsetfillcolor{dialinecolor}
\fill (26.050000\du,0.950000\du)--(26.050000\du,2.350000\du)--(36.945000\du,2.350000\du)--(36.945000\du,0.950000\du)--cycle;
\definecolor{dialinecolor}{rgb}{0.000000, 0.000000, 0.000000}
\pgfsetstrokecolor{dialinecolor}
\draw (26.050000\du,0.950000\du)--(26.050000\du,2.350000\du)--(36.945000\du,2.350000\du)--(36.945000\du,0.950000\du)--cycle;
% setfont left to latex
\definecolor{dialinecolor}{rgb}{0.000000, 0.000000, 0.000000}
\pgfsetstrokecolor{dialinecolor}
\node at (31.497500\du,1.900000\du){Élément};
\definecolor{dialinecolor}{rgb}{1.000000, 1.000000, 1.000000}
\pgfsetfillcolor{dialinecolor}
\fill (26.050000\du,2.350000\du)--(26.050000\du,5.750000\du)--(36.945000\du,5.750000\du)--(36.945000\du,2.350000\du)--cycle;
\definecolor{dialinecolor}{rgb}{0.000000, 0.000000, 0.000000}
\pgfsetstrokecolor{dialinecolor}
\draw (26.050000\du,2.350000\du)--(26.050000\du,5.750000\du)--(36.945000\du,5.750000\du)--(36.945000\du,2.350000\du)--cycle;
% setfont left to latex
\definecolor{dialinecolor}{rgb}{0.000000, 0.000000, 0.000000}
\pgfsetstrokecolor{dialinecolor}
\node[anchor=west] at (26.200000\du,3.050000\du){+nom};
% setfont left to latex
\definecolor{dialinecolor}{rgb}{0.000000, 0.000000, 0.000000}
\pgfsetstrokecolor{dialinecolor}
\node[anchor=west] at (26.200000\du,3.850000\du){+chemin};
% setfont left to latex
\definecolor{dialinecolor}{rgb}{0.000000, 0.000000, 0.000000}
\pgfsetstrokecolor{dialinecolor}
\node[anchor=west] at (26.200000\du,4.650000\du){-date de création: Date};
% setfont left to latex
\definecolor{dialinecolor}{rgb}{0.000000, 0.000000, 0.000000}
\pgfsetstrokecolor{dialinecolor}
\node[anchor=west] at (26.200000\du,5.450000\du){-date de modification: Date};
\definecolor{dialinecolor}{rgb}{1.000000, 1.000000, 1.000000}
\pgfsetfillcolor{dialinecolor}
\fill (26.050000\du,5.750000\du)--(26.050000\du,9.150000\du)--(36.945000\du,9.150000\du)--(36.945000\du,5.750000\du)--cycle;
\definecolor{dialinecolor}{rgb}{0.000000, 0.000000, 0.000000}
\pgfsetstrokecolor{dialinecolor}
\draw (26.050000\du,5.750000\du)--(26.050000\du,9.150000\du)--(36.945000\du,9.150000\du)--(36.945000\du,5.750000\du)--cycle;
% setfont left to latex
\definecolor{dialinecolor}{rgb}{0.000000, 0.000000, 0.000000}
\pgfsetstrokecolor{dialinecolor}
\node[anchor=west] at (26.200000\du,6.450000\du){+lire(): String};
% setfont left to latex
\definecolor{dialinecolor}{rgb}{0.000000, 0.000000, 0.000000}
\pgfsetstrokecolor{dialinecolor}
\node[anchor=west] at (26.200000\du,7.250000\du){+écrire(contenu:String)};
% setfont left to latex
\definecolor{dialinecolor}{rgb}{0.000000, 0.000000, 0.000000}
\pgfsetstrokecolor{dialinecolor}
\node[anchor=west] at (26.200000\du,8.050000\du){+ouvrir()};
% setfont left to latex
\definecolor{dialinecolor}{rgb}{0.000000, 0.000000, 0.000000}
\pgfsetstrokecolor{dialinecolor}
\node[anchor=west] at (26.200000\du,8.850000\du){+fermer()};
\pgfsetlinewidth{0.100000\du}
\pgfsetdash{}{0pt}
\pgfsetmiterjoin
\pgfsetbuttcap
{
\definecolor{dialinecolor}{rgb}{0.000000, 0.000000, 0.000000}
\pgfsetfillcolor{dialinecolor}
% was here!!!
\definecolor{dialinecolor}{rgb}{0.000000, 0.000000, 0.000000}
\pgfsetstrokecolor{dialinecolor}
\draw (31.497500\du,9.200253\du)--(31.497500\du,11.474950\du)--(27.330000\du,11.474950\du)--(27.330000\du,12.949646\du);
}
\definecolor{dialinecolor}{rgb}{0.000000, 0.000000, 0.000000}
\pgfsetstrokecolor{dialinecolor}
\draw (31.497500\du,10.112057\du)--(31.497500\du,11.474950\du)--(27.330000\du,11.474950\du)--(27.330000\du,12.949646\du);
\pgfsetmiterjoin
\definecolor{dialinecolor}{rgb}{1.000000, 1.000000, 1.000000}
\pgfsetfillcolor{dialinecolor}
\fill (31.897500\du,10.112057\du)--(31.497500\du,9.312057\du)--(31.097500\du,10.112057\du)--cycle;
\pgfsetlinewidth{0.100000\du}
\pgfsetdash{}{0pt}
\pgfsetmiterjoin
\definecolor{dialinecolor}{rgb}{0.000000, 0.000000, 0.000000}
\pgfsetstrokecolor{dialinecolor}
\draw (31.897500\du,10.112057\du)--(31.497500\du,9.312057\du)--(31.097500\du,10.112057\du)--cycle;
% setfont left to latex
\pgfsetlinewidth{0.100000\du}
\pgfsetdash{}{0pt}
\pgfsetmiterjoin
\pgfsetbuttcap
{
\definecolor{dialinecolor}{rgb}{0.000000, 0.000000, 0.000000}
\pgfsetfillcolor{dialinecolor}
% was here!!!
\definecolor{dialinecolor}{rgb}{0.000000, 0.000000, 0.000000}
\pgfsetstrokecolor{dialinecolor}
\draw (31.497500\du,9.200253\du)--(31.497500\du,13.524950\du)--(36.615000\du,13.524950\du)--(36.615000\du,17.049646\du);
}
\definecolor{dialinecolor}{rgb}{0.000000, 0.000000, 0.000000}
\pgfsetstrokecolor{dialinecolor}
\draw (31.497500\du,10.112057\du)--(31.497500\du,13.524950\du)--(36.615000\du,13.524950\du)--(36.615000\du,17.049646\du);
\pgfsetmiterjoin
\definecolor{dialinecolor}{rgb}{1.000000, 1.000000, 1.000000}
\pgfsetfillcolor{dialinecolor}
\fill (31.897500\du,10.112057\du)--(31.497500\du,9.312057\du)--(31.097500\du,10.112057\du)--cycle;
\pgfsetlinewidth{0.100000\du}
\pgfsetdash{}{0pt}
\pgfsetmiterjoin
\definecolor{dialinecolor}{rgb}{0.000000, 0.000000, 0.000000}
\pgfsetstrokecolor{dialinecolor}
\draw (31.897500\du,10.112057\du)--(31.497500\du,9.312057\du)--(31.097500\du,10.112057\du)--cycle;
% setfont left to latex
\pgfsetlinewidth{0.100000\du}
\pgfsetdash{}{0pt}
\definecolor{dialinecolor}{rgb}{1.000000, 1.000000, 1.000000}
\pgfsetfillcolor{dialinecolor}
\fill (49.000000\du,7.000000\du)--(49.000000\du,8.400000\du)--(54.510000\du,8.400000\du)--(54.510000\du,7.000000\du)--cycle;
\definecolor{dialinecolor}{rgb}{0.000000, 0.000000, 0.000000}
\pgfsetstrokecolor{dialinecolor}
\draw (49.000000\du,7.000000\du)--(49.000000\du,8.400000\du)--(54.510000\du,8.400000\du)--(54.510000\du,7.000000\du)--cycle;
% setfont left to latex
\definecolor{dialinecolor}{rgb}{0.000000, 0.000000, 0.000000}
\pgfsetstrokecolor{dialinecolor}
\node at (51.755000\du,7.950000\du){Navigateur};
\definecolor{dialinecolor}{rgb}{1.000000, 1.000000, 1.000000}
\pgfsetfillcolor{dialinecolor}
\fill (49.000000\du,8.400000\du)--(49.000000\du,8.800000\du)--(54.510000\du,8.800000\du)--(54.510000\du,8.400000\du)--cycle;
\definecolor{dialinecolor}{rgb}{0.000000, 0.000000, 0.000000}
\pgfsetstrokecolor{dialinecolor}
\draw (49.000000\du,8.400000\du)--(49.000000\du,8.800000\du)--(54.510000\du,8.800000\du)--(54.510000\du,8.400000\du)--cycle;
\definecolor{dialinecolor}{rgb}{1.000000, 1.000000, 1.000000}
\pgfsetfillcolor{dialinecolor}
\fill (49.000000\du,8.800000\du)--(49.000000\du,9.200000\du)--(54.510000\du,9.200000\du)--(54.510000\du,8.800000\du)--cycle;
\definecolor{dialinecolor}{rgb}{0.000000, 0.000000, 0.000000}
\pgfsetstrokecolor{dialinecolor}
\draw (49.000000\du,8.800000\du)--(49.000000\du,9.200000\du)--(54.510000\du,9.200000\du)--(54.510000\du,8.800000\du)--cycle;
\pgfsetlinewidth{0.100000\du}
\pgfsetdash{}{0pt}
\pgfsetmiterjoin
\pgfsetbuttcap
{
\definecolor{dialinecolor}{rgb}{0.000000, 0.000000, 0.000000}
\pgfsetfillcolor{dialinecolor}
% was here!!!
\pgfsetarrowsend{to}
\definecolor{dialinecolor}{rgb}{0.000000, 0.000000, 0.000000}
\pgfsetstrokecolor{dialinecolor}
\draw (48.950587\du,8.100000\du)--(45.000000\du,8.100000\du)--(45.000000\du,18.500000\du)--(40.380164\du,18.500000\du);
}
% setfont left to latex
\definecolor{dialinecolor}{rgb}{0.000000, 0.000000, 0.000000}
\pgfsetstrokecolor{dialinecolor}
\node[anchor=west] at (45.100000\du,13.150000\du){};
\definecolor{dialinecolor}{rgb}{0.000000, 0.000000, 0.000000}
\pgfsetfillcolor{dialinecolor}
\fill (45.200000\du,13.150000\du)--(45.200000\du,12.750000\du)--(45.600000\du,12.950000\du)--cycle;
\definecolor{dialinecolor}{rgb}{0.000000, 0.000000, 0.000000}
\pgfsetstrokecolor{dialinecolor}
\node[anchor=east] at (48.750587\du,7.950000\du){ 1};
\definecolor{dialinecolor}{rgb}{0.000000, 0.000000, 0.000000}
\pgfsetstrokecolor{dialinecolor}
\node[anchor=east] at (48.750587\du,8.750000\du){notifie};
\definecolor{dialinecolor}{rgb}{0.000000, 0.000000, 0.000000}
\pgfsetstrokecolor{dialinecolor}
\node[anchor=west] at (41.380164\du,18.350000\du){ *};
\definecolor{dialinecolor}{rgb}{0.000000, 0.000000, 0.000000}
\pgfsetstrokecolor{dialinecolor}
\node[anchor=west] at (41.380164\du,19.150000\du){gère};
\pgfsetlinewidth{0.100000\du}
\pgfsetdash{}{0pt}
\pgfsetmiterjoin
\pgfsetbuttcap
{
\definecolor{dialinecolor}{rgb}{0.000000, 0.000000, 0.000000}
\pgfsetfillcolor{dialinecolor}
% was here!!!
\pgfsetarrowsend{to}
\definecolor{dialinecolor}{rgb}{0.000000, 0.000000, 0.000000}
\pgfsetstrokecolor{dialinecolor}
\draw (51.755000\du,9.249994\du)--(51.755000\du,24.000000\du)--(36.615000\du,24.000000\du)--(36.615000\du,19.944824\du);
}
% setfont left to latex
\definecolor{dialinecolor}{rgb}{0.000000, 0.000000, 0.000000}
\pgfsetstrokecolor{dialinecolor}
\node at (44.185000\du,23.850000\du){};
\definecolor{dialinecolor}{rgb}{0.000000, 0.000000, 0.000000}
\pgfsetfillcolor{dialinecolor}
\fill (44.285000\du,23.850000\du)--(44.285000\du,23.450000\du)--(44.685000\du,23.650000\du)--cycle;
\definecolor{dialinecolor}{rgb}{0.000000, 0.000000, 0.000000}
\pgfsetstrokecolor{dialinecolor}
\node[anchor=west] at (51.955000\du,9.849994\du){1};
\definecolor{dialinecolor}{rgb}{0.000000, 0.000000, 0.000000}
\pgfsetstrokecolor{dialinecolor}
\node[anchor=west] at (37.165000\du,20.544824\du){ active};
\definecolor{dialinecolor}{rgb}{0.000000, 0.000000, 0.000000}
\pgfsetstrokecolor{dialinecolor}
\node[anchor=west] at (37.165000\du,21.344824\du){0..1};
\pgfsetlinewidth{0.100000\du}
\pgfsetdash{}{0pt}
\pgfsetmiterjoin
\pgfsetbuttcap
{
\definecolor{dialinecolor}{rgb}{0.000000, 0.000000, 0.000000}
\pgfsetfillcolor{dialinecolor}
% was here!!!
\definecolor{dialinecolor}{rgb}{0.000000, 0.000000, 0.000000}
\pgfsetstrokecolor{dialinecolor}
\draw (32.849921\du,18.500000\du)--(21.000000\du,18.500000\du)--(21.000000\du,5.050000\du)--(25.999832\du,5.050000\du);
}
\definecolor{dialinecolor}{rgb}{0.000000, 0.000000, 0.000000}
\pgfsetstrokecolor{dialinecolor}
\draw (31.591342\du,18.500000\du)--(21.000000\du,18.500000\du)--(21.000000\du,5.050000\du)--(25.999832\du,5.050000\du);
\pgfsetdash{}{0pt}
\pgfsetmiterjoin
\pgfsetbuttcap
\definecolor{dialinecolor}{rgb}{0.000000, 0.000000, 0.000000}
\pgfsetfillcolor{dialinecolor}
\fill (32.849921\du,18.500000\du)--(32.149921\du,18.740000\du)--(31.449921\du,18.500000\du)--(32.149921\du,18.260000\du)--cycle;
\pgfsetlinewidth{0.100000\du}
\pgfsetdash{}{0pt}
\pgfsetmiterjoin
\pgfsetbuttcap
\definecolor{dialinecolor}{rgb}{0.000000, 0.000000, 0.000000}
\pgfsetstrokecolor{dialinecolor}
\draw (32.849921\du,18.500000\du)--(32.149921\du,18.740000\du)--(31.449921\du,18.500000\du)--(32.149921\du,18.260000\du)--cycle;
% setfont left to latex
\definecolor{dialinecolor}{rgb}{0.000000, 0.000000, 0.000000}
\pgfsetstrokecolor{dialinecolor}
\node[anchor=west] at (21.100000\du,11.625000\du){};
\definecolor{dialinecolor}{rgb}{0.000000, 0.000000, 0.000000}
\pgfsetstrokecolor{dialinecolor}
\node[anchor=east] at (31.249921\du,18.350000\du){};
\definecolor{dialinecolor}{rgb}{0.000000, 0.000000, 0.000000}
\pgfsetstrokecolor{dialinecolor}
\node[anchor=east] at (25.799832\du,4.900000\du){ possède};
\definecolor{dialinecolor}{rgb}{0.000000, 0.000000, 0.000000}
\pgfsetstrokecolor{dialinecolor}
\node[anchor=east] at (25.799832\du,5.700000\du){*};
\end{tikzpicture}
}
    \caption{Diagramme de classes initial}
  \end{sidewaysfigure}

  \section{Diagramme de classe modifié}
  \begin{sidewaysfigure}
    \centering
    \resizebox{\textwidth}{!}{% Graphic for TeX using PGF
% Title: /home/guillaume/Documents/Université de Montréal/Automne 2014/Génie logiciel/Devoir/devoir-3/diagramme-de-classes-b.dia
% Creator: Dia v0.97.3
% CreationDate: Sat Dec  6 11:14:32 2014
% For: guillaume
% \usepackage{tikz}
% The following commands are not supported in PSTricks at present
% We define them conditionally, so when they are implemented,
% this pgf file will use them.
\ifx\du\undefined
  \newlength{\du}
\fi
\setlength{\du}{15\unitlength}
\begin{tikzpicture}
\pgftransformxscale{1.000000}
\pgftransformyscale{-1.000000}
\definecolor{dialinecolor}{rgb}{0.000000, 0.000000, 0.000000}
\pgfsetstrokecolor{dialinecolor}
\definecolor{dialinecolor}{rgb}{1.000000, 1.000000, 1.000000}
\pgfsetfillcolor{dialinecolor}
\pgfsetlinewidth{0.100000\du}
\pgfsetdash{}{0pt}
\definecolor{dialinecolor}{rgb}{1.000000, 1.000000, 1.000000}
\pgfsetfillcolor{dialinecolor}
\fill (23.000000\du,5.000000\du)--(23.000000\du,6.400000\du)--(32.250000\du,6.400000\du)--(32.250000\du,5.000000\du)--cycle;
\definecolor{dialinecolor}{rgb}{0.000000, 0.000000, 0.000000}
\pgfsetstrokecolor{dialinecolor}
\draw (23.000000\du,5.000000\du)--(23.000000\du,6.400000\du)--(32.250000\du,6.400000\du)--(32.250000\du,5.000000\du)--cycle;
% setfont left to latex
\definecolor{dialinecolor}{rgb}{0.000000, 0.000000, 0.000000}
\pgfsetstrokecolor{dialinecolor}
\node at (27.625000\du,5.950000\du){ElementDecorateur};
\definecolor{dialinecolor}{rgb}{1.000000, 1.000000, 1.000000}
\pgfsetfillcolor{dialinecolor}
\fill (23.000000\du,6.400000\du)--(23.000000\du,6.800000\du)--(32.250000\du,6.800000\du)--(32.250000\du,6.400000\du)--cycle;
\definecolor{dialinecolor}{rgb}{0.000000, 0.000000, 0.000000}
\pgfsetstrokecolor{dialinecolor}
\draw (23.000000\du,6.400000\du)--(23.000000\du,6.800000\du)--(32.250000\du,6.800000\du)--(32.250000\du,6.400000\du)--cycle;
\definecolor{dialinecolor}{rgb}{1.000000, 1.000000, 1.000000}
\pgfsetfillcolor{dialinecolor}
\fill (23.000000\du,6.800000\du)--(23.000000\du,7.200000\du)--(32.250000\du,7.200000\du)--(32.250000\du,6.800000\du)--cycle;
\definecolor{dialinecolor}{rgb}{0.000000, 0.000000, 0.000000}
\pgfsetstrokecolor{dialinecolor}
\draw (23.000000\du,6.800000\du)--(23.000000\du,7.200000\du)--(32.250000\du,7.200000\du)--(32.250000\du,6.800000\du)--cycle;
\pgfsetlinewidth{0.100000\du}
\pgfsetdash{}{0pt}
\pgfsetmiterjoin
\pgfsetbuttcap
{
\definecolor{dialinecolor}{rgb}{0.000000, 0.000000, 0.000000}
\pgfsetfillcolor{dialinecolor}
% was here!!!
\definecolor{dialinecolor}{rgb}{0.000000, 0.000000, 0.000000}
\pgfsetstrokecolor{dialinecolor}
\draw (22.949715\du,6.100000\du)--(18.953893\du,6.100000\du)--(18.953893\du,-1.218780\du)--(15.658071\du,-1.218780\du);
}
\definecolor{dialinecolor}{rgb}{0.000000, 0.000000, 0.000000}
\pgfsetstrokecolor{dialinecolor}
\draw (21.691136\du,6.100000\du)--(18.953893\du,6.100000\du)--(18.953893\du,-1.218780\du)--(15.658071\du,-1.218780\du);
\pgfsetdash{}{0pt}
\pgfsetmiterjoin
\pgfsetbuttcap
\definecolor{dialinecolor}{rgb}{0.000000, 0.000000, 0.000000}
\pgfsetfillcolor{dialinecolor}
\fill (22.949715\du,6.100000\du)--(22.249715\du,6.340000\du)--(21.549715\du,6.100000\du)--(22.249715\du,5.860000\du)--cycle;
\pgfsetlinewidth{0.100000\du}
\pgfsetdash{}{0pt}
\pgfsetmiterjoin
\pgfsetbuttcap
\definecolor{dialinecolor}{rgb}{0.000000, 0.000000, 0.000000}
\pgfsetstrokecolor{dialinecolor}
\draw (22.949715\du,6.100000\du)--(22.249715\du,6.340000\du)--(21.549715\du,6.100000\du)--(22.249715\du,5.860000\du)--cycle;
% setfont left to latex
\definecolor{dialinecolor}{rgb}{0.000000, 0.000000, 0.000000}
\pgfsetstrokecolor{dialinecolor}
\node[anchor=west] at (19.053893\du,2.290610\du){};
\definecolor{dialinecolor}{rgb}{0.000000, 0.000000, 0.000000}
\pgfsetstrokecolor{dialinecolor}
\node[anchor=east] at (21.349715\du,5.950000\du){};
\definecolor{dialinecolor}{rgb}{0.000000, 0.000000, 0.000000}
\pgfsetstrokecolor{dialinecolor}
\node[anchor=west] at (15.858071\du,-1.368780\du){1};
\pgfsetlinewidth{0.100000\du}
\pgfsetdash{}{0pt}
\definecolor{dialinecolor}{rgb}{1.000000, 1.000000, 1.000000}
\pgfsetfillcolor{dialinecolor}
\fill (18.531200\du,17.397200\du)--(18.531200\du,18.797200\du)--(27.886200\du,18.797200\du)--(27.886200\du,17.397200\du)--cycle;
\definecolor{dialinecolor}{rgb}{0.000000, 0.000000, 0.000000}
\pgfsetstrokecolor{dialinecolor}
\draw (18.531200\du,17.397200\du)--(18.531200\du,18.797200\du)--(27.886200\du,18.797200\du)--(27.886200\du,17.397200\du)--cycle;
% setfont left to latex
\definecolor{dialinecolor}{rgb}{0.000000, 0.000000, 0.000000}
\pgfsetstrokecolor{dialinecolor}
\node at (23.208700\du,18.347200\du){ElementIntelligent};
\definecolor{dialinecolor}{rgb}{1.000000, 1.000000, 1.000000}
\pgfsetfillcolor{dialinecolor}
\fill (18.531200\du,18.797200\du)--(18.531200\du,19.197200\du)--(27.886200\du,19.197200\du)--(27.886200\du,18.797200\du)--cycle;
\definecolor{dialinecolor}{rgb}{0.000000, 0.000000, 0.000000}
\pgfsetstrokecolor{dialinecolor}
\draw (18.531200\du,18.797200\du)--(18.531200\du,19.197200\du)--(27.886200\du,19.197200\du)--(27.886200\du,18.797200\du)--cycle;
\definecolor{dialinecolor}{rgb}{1.000000, 1.000000, 1.000000}
\pgfsetfillcolor{dialinecolor}
\fill (18.531200\du,19.197200\du)--(18.531200\du,20.997200\du)--(27.886200\du,20.997200\du)--(27.886200\du,19.197200\du)--cycle;
\definecolor{dialinecolor}{rgb}{0.000000, 0.000000, 0.000000}
\pgfsetstrokecolor{dialinecolor}
\draw (18.531200\du,19.197200\du)--(18.531200\du,20.997200\du)--(27.886200\du,20.997200\du)--(27.886200\du,19.197200\du)--cycle;
% setfont left to latex
\definecolor{dialinecolor}{rgb}{0.000000, 0.000000, 0.000000}
\pgfsetstrokecolor{dialinecolor}
\node[anchor=west] at (18.681200\du,19.897200\du){-autoEvaluer()};
% setfont left to latex
\definecolor{dialinecolor}{rgb}{0.000000, 0.000000, 0.000000}
\pgfsetstrokecolor{dialinecolor}
\node[anchor=west] at (18.681200\du,20.697200\du){-proposerAmelioration()};
\pgfsetlinewidth{0.100000\du}
\pgfsetdash{}{0pt}
\definecolor{dialinecolor}{rgb}{1.000000, 1.000000, 1.000000}
\pgfsetfillcolor{dialinecolor}
\fill (28.549900\du,17.397200\du)--(28.549900\du,18.797200\du)--(36.212400\du,18.797200\du)--(36.212400\du,17.397200\du)--cycle;
\definecolor{dialinecolor}{rgb}{0.000000, 0.000000, 0.000000}
\pgfsetstrokecolor{dialinecolor}
\draw (28.549900\du,17.397200\du)--(28.549900\du,18.797200\du)--(36.212400\du,18.797200\du)--(36.212400\du,17.397200\du)--cycle;
% setfont left to latex
\definecolor{dialinecolor}{rgb}{0.000000, 0.000000, 0.000000}
\pgfsetstrokecolor{dialinecolor}
\node at (32.381150\du,18.347200\du){ElementEvolutif};
\definecolor{dialinecolor}{rgb}{1.000000, 1.000000, 1.000000}
\pgfsetfillcolor{dialinecolor}
\fill (28.549900\du,18.797200\du)--(28.549900\du,19.197200\du)--(36.212400\du,19.197200\du)--(36.212400\du,18.797200\du)--cycle;
\definecolor{dialinecolor}{rgb}{0.000000, 0.000000, 0.000000}
\pgfsetstrokecolor{dialinecolor}
\draw (28.549900\du,18.797200\du)--(28.549900\du,19.197200\du)--(36.212400\du,19.197200\du)--(36.212400\du,18.797200\du)--cycle;
\definecolor{dialinecolor}{rgb}{1.000000, 1.000000, 1.000000}
\pgfsetfillcolor{dialinecolor}
\fill (28.549900\du,19.197200\du)--(28.549900\du,20.197200\du)--(36.212400\du,20.197200\du)--(36.212400\du,19.197200\du)--cycle;
\definecolor{dialinecolor}{rgb}{0.000000, 0.000000, 0.000000}
\pgfsetstrokecolor{dialinecolor}
\draw (28.549900\du,19.197200\du)--(28.549900\du,20.197200\du)--(36.212400\du,20.197200\du)--(36.212400\du,19.197200\du)--cycle;
% setfont left to latex
\definecolor{dialinecolor}{rgb}{0.000000, 0.000000, 0.000000}
\pgfsetstrokecolor{dialinecolor}
\node[anchor=west] at (28.699900\du,19.897200\du){-évoluer()};
\pgfsetlinewidth{0.100000\du}
\pgfsetdash{}{0pt}
\pgfsetmiterjoin
\pgfsetbuttcap
{
\definecolor{dialinecolor}{rgb}{0.000000, 0.000000, 0.000000}
\pgfsetfillcolor{dialinecolor}
% was here!!!
\definecolor{dialinecolor}{rgb}{0.000000, 0.000000, 0.000000}
\pgfsetstrokecolor{dialinecolor}
\draw (27.625000\du,7.250281\du)--(27.625000\du,12.723740\du)--(23.208700\du,12.723740\du)--(23.208700\du,17.397200\du);
}
\definecolor{dialinecolor}{rgb}{0.000000, 0.000000, 0.000000}
\pgfsetstrokecolor{dialinecolor}
\draw (27.625000\du,8.162084\du)--(27.625000\du,12.723740\du)--(23.208700\du,12.723740\du)--(23.208700\du,17.397200\du);
\pgfsetmiterjoin
\definecolor{dialinecolor}{rgb}{1.000000, 1.000000, 1.000000}
\pgfsetfillcolor{dialinecolor}
\fill (28.025000\du,8.162084\du)--(27.625000\du,7.362084\du)--(27.225000\du,8.162084\du)--cycle;
\pgfsetlinewidth{0.100000\du}
\pgfsetdash{}{0pt}
\pgfsetmiterjoin
\definecolor{dialinecolor}{rgb}{0.000000, 0.000000, 0.000000}
\pgfsetstrokecolor{dialinecolor}
\draw (28.025000\du,8.162084\du)--(27.625000\du,7.362084\du)--(27.225000\du,8.162084\du)--cycle;
% setfont left to latex
\pgfsetlinewidth{0.100000\du}
\pgfsetdash{}{0pt}
\pgfsetmiterjoin
\pgfsetbuttcap
{
\definecolor{dialinecolor}{rgb}{0.000000, 0.000000, 0.000000}
\pgfsetfillcolor{dialinecolor}
% was here!!!
\definecolor{dialinecolor}{rgb}{0.000000, 0.000000, 0.000000}
\pgfsetstrokecolor{dialinecolor}
\draw (27.625000\du,7.250281\du)--(27.625000\du,12.698563\du)--(32.381150\du,12.698563\du)--(32.381150\du,17.346846\du);
}
\definecolor{dialinecolor}{rgb}{0.000000, 0.000000, 0.000000}
\pgfsetstrokecolor{dialinecolor}
\draw (27.625000\du,8.162084\du)--(27.625000\du,12.698563\du)--(32.381150\du,12.698563\du)--(32.381150\du,17.346846\du);
\pgfsetmiterjoin
\definecolor{dialinecolor}{rgb}{1.000000, 1.000000, 1.000000}
\pgfsetfillcolor{dialinecolor}
\fill (28.025000\du,8.162084\du)--(27.625000\du,7.362084\du)--(27.225000\du,8.162084\du)--cycle;
\pgfsetlinewidth{0.100000\du}
\pgfsetdash{}{0pt}
\pgfsetmiterjoin
\definecolor{dialinecolor}{rgb}{0.000000, 0.000000, 0.000000}
\pgfsetstrokecolor{dialinecolor}
\draw (28.025000\du,8.162084\du)--(27.625000\du,7.362084\du)--(27.225000\du,8.162084\du)--cycle;
% setfont left to latex
\pgfsetlinewidth{0.100000\du}
\pgfsetdash{}{0pt}
\pgfsetmiterjoin
\pgfsetbuttcap
{
\definecolor{dialinecolor}{rgb}{0.000000, 0.000000, 0.000000}
\pgfsetfillcolor{dialinecolor}
% was here!!!
\definecolor{dialinecolor}{rgb}{0.000000, 0.000000, 0.000000}
\pgfsetstrokecolor{dialinecolor}
\draw (15.607700\du,-3.818780\du)--(24.892100\du,-3.818780\du)--(24.892100\du,5.000000\du)--(27.142500\du,5.000000\du);
}
\definecolor{dialinecolor}{rgb}{0.000000, 0.000000, 0.000000}
\pgfsetstrokecolor{dialinecolor}
\draw (16.519503\du,-3.818780\du)--(24.892100\du,-3.818780\du)--(24.892100\du,5.000000\du)--(27.142500\du,5.000000\du);
\pgfsetmiterjoin
\definecolor{dialinecolor}{rgb}{1.000000, 1.000000, 1.000000}
\pgfsetfillcolor{dialinecolor}
\fill (16.519503\du,-4.218780\du)--(15.719503\du,-3.818780\du)--(16.519503\du,-3.418780\du)--cycle;
\pgfsetlinewidth{0.100000\du}
\pgfsetdash{}{0pt}
\pgfsetmiterjoin
\definecolor{dialinecolor}{rgb}{0.000000, 0.000000, 0.000000}
\pgfsetstrokecolor{dialinecolor}
\draw (16.519503\du,-4.218780\du)--(15.719503\du,-3.818780\du)--(16.519503\du,-3.418780\du)--cycle;
% setfont left to latex
\pgfsetlinewidth{0.100000\du}
\pgfsetdash{}{0pt}
\definecolor{dialinecolor}{rgb}{1.000000, 1.000000, 1.000000}
\pgfsetfillcolor{dialinecolor}
\fill (3.557700\du,-5.718780\du)--(3.557700\du,-4.318780\du)--(15.607700\du,-4.318780\du)--(15.607700\du,-5.718780\du)--cycle;
\definecolor{dialinecolor}{rgb}{0.000000, 0.000000, 0.000000}
\pgfsetstrokecolor{dialinecolor}
\draw (3.557700\du,-5.718780\du)--(3.557700\du,-4.318780\du)--(15.607700\du,-4.318780\du)--(15.607700\du,-5.718780\du)--cycle;
% setfont left to latex
\definecolor{dialinecolor}{rgb}{0.000000, 0.000000, 0.000000}
\pgfsetstrokecolor{dialinecolor}
\node at (9.582700\du,-4.768780\du){Element};
\definecolor{dialinecolor}{rgb}{1.000000, 1.000000, 1.000000}
\pgfsetfillcolor{dialinecolor}
\fill (3.557700\du,-4.318780\du)--(3.557700\du,-0.118780\du)--(15.607700\du,-0.118780\du)--(15.607700\du,-4.318780\du)--cycle;
\definecolor{dialinecolor}{rgb}{0.000000, 0.000000, 0.000000}
\pgfsetstrokecolor{dialinecolor}
\draw (3.557700\du,-4.318780\du)--(3.557700\du,-0.118780\du)--(15.607700\du,-0.118780\du)--(15.607700\du,-4.318780\du)--cycle;
% setfont left to latex
\definecolor{dialinecolor}{rgb}{0.000000, 0.000000, 0.000000}
\pgfsetstrokecolor{dialinecolor}
\node[anchor=west] at (3.707700\du,-3.618780\du){\#Nom};
% setfont left to latex
\definecolor{dialinecolor}{rgb}{0.000000, 0.000000, 0.000000}
\pgfsetstrokecolor{dialinecolor}
\node[anchor=west] at (3.707700\du,-2.818780\du){\#Date de création};
% setfont left to latex
\definecolor{dialinecolor}{rgb}{0.000000, 0.000000, 0.000000}
\pgfsetstrokecolor{dialinecolor}
\node[anchor=west] at (3.707700\du,-2.018780\du){\#Date de dernière modification};
% setfont left to latex
\definecolor{dialinecolor}{rgb}{0.000000, 0.000000, 0.000000}
\pgfsetstrokecolor{dialinecolor}
\node[anchor=west] at (3.707700\du,-1.218780\du){\#Chemin};
% setfont left to latex
\definecolor{dialinecolor}{rgb}{0.000000, 0.000000, 0.000000}
\pgfsetstrokecolor{dialinecolor}
\node[anchor=west] at (3.707700\du,-0.418780\du){\#Ouvert: Boolean = False};
\definecolor{dialinecolor}{rgb}{1.000000, 1.000000, 1.000000}
\pgfsetfillcolor{dialinecolor}
\fill (3.557700\du,-0.118780\du)--(3.557700\du,3.281220\du)--(15.607700\du,3.281220\du)--(15.607700\du,-0.118780\du)--cycle;
\definecolor{dialinecolor}{rgb}{0.000000, 0.000000, 0.000000}
\pgfsetstrokecolor{dialinecolor}
\draw (3.557700\du,-0.118780\du)--(3.557700\du,3.281220\du)--(15.607700\du,3.281220\du)--(15.607700\du,-0.118780\du)--cycle;
% setfont left to latex
\definecolor{dialinecolor}{rgb}{0.000000, 0.000000, 0.000000}
\pgfsetstrokecolor{dialinecolor}
\node[anchor=west] at (3.707700\du,0.581220\du){+taille()};
% setfont left to latex
\definecolor{dialinecolor}{rgb}{0.000000, 0.000000, 0.000000}
\pgfsetstrokecolor{dialinecolor}
\node[anchor=west] at (3.707700\du,1.381220\du){+open()};
% setfont left to latex
\definecolor{dialinecolor}{rgb}{0.000000, 0.000000, 0.000000}
\pgfsetstrokecolor{dialinecolor}
\node[anchor=west] at (3.707700\du,2.181220\du){+close()};
% setfont left to latex
\definecolor{dialinecolor}{rgb}{0.000000, 0.000000, 0.000000}
\pgfsetstrokecolor{dialinecolor}
\node[anchor=west] at (3.707700\du,2.981220\du){+delete()};
\pgfsetlinewidth{0.100000\du}
\pgfsetdash{}{0pt}
\definecolor{dialinecolor}{rgb}{1.000000, 1.000000, 1.000000}
\pgfsetfillcolor{dialinecolor}
\fill (12.745400\du,14.848800\du)--(12.745400\du,16.248800\du)--(16.325400\du,16.248800\du)--(16.325400\du,14.848800\du)--cycle;
\definecolor{dialinecolor}{rgb}{0.000000, 0.000000, 0.000000}
\pgfsetstrokecolor{dialinecolor}
\draw (12.745400\du,14.848800\du)--(12.745400\du,16.248800\du)--(16.325400\du,16.248800\du)--(16.325400\du,14.848800\du)--cycle;
% setfont left to latex
\definecolor{dialinecolor}{rgb}{0.000000, 0.000000, 0.000000}
\pgfsetstrokecolor{dialinecolor}
\node at (14.535400\du,15.798800\du){Fichier};
\definecolor{dialinecolor}{rgb}{1.000000, 1.000000, 1.000000}
\pgfsetfillcolor{dialinecolor}
\fill (12.745400\du,16.248800\du)--(12.745400\du,16.648800\du)--(16.325400\du,16.648800\du)--(16.325400\du,16.248800\du)--cycle;
\definecolor{dialinecolor}{rgb}{0.000000, 0.000000, 0.000000}
\pgfsetstrokecolor{dialinecolor}
\draw (12.745400\du,16.248800\du)--(12.745400\du,16.648800\du)--(16.325400\du,16.648800\du)--(16.325400\du,16.248800\du)--cycle;
\definecolor{dialinecolor}{rgb}{1.000000, 1.000000, 1.000000}
\pgfsetfillcolor{dialinecolor}
\fill (12.745400\du,16.648800\du)--(12.745400\du,17.048800\du)--(16.325400\du,17.048800\du)--(16.325400\du,16.648800\du)--cycle;
\definecolor{dialinecolor}{rgb}{0.000000, 0.000000, 0.000000}
\pgfsetstrokecolor{dialinecolor}
\draw (12.745400\du,16.648800\du)--(12.745400\du,17.048800\du)--(16.325400\du,17.048800\du)--(16.325400\du,16.648800\du)--cycle;
\pgfsetlinewidth{0.100000\du}
\pgfsetdash{}{0pt}
\definecolor{dialinecolor}{rgb}{1.000000, 1.000000, 1.000000}
\pgfsetfillcolor{dialinecolor}
\fill (-7.000000\du,13.000000\du)--(-7.000000\du,14.400000\du)--(6.590000\du,14.400000\du)--(6.590000\du,13.000000\du)--cycle;
\definecolor{dialinecolor}{rgb}{0.000000, 0.000000, 0.000000}
\pgfsetstrokecolor{dialinecolor}
\draw (-7.000000\du,13.000000\du)--(-7.000000\du,14.400000\du)--(6.590000\du,14.400000\du)--(6.590000\du,13.000000\du)--cycle;
% setfont left to latex
\definecolor{dialinecolor}{rgb}{0.000000, 0.000000, 0.000000}
\pgfsetstrokecolor{dialinecolor}
\node at (-0.205000\du,13.950000\du){Dossier};
\definecolor{dialinecolor}{rgb}{1.000000, 1.000000, 1.000000}
\pgfsetfillcolor{dialinecolor}
\fill (-7.000000\du,14.400000\du)--(-7.000000\du,14.800000\du)--(6.590000\du,14.800000\du)--(6.590000\du,14.400000\du)--cycle;
\definecolor{dialinecolor}{rgb}{0.000000, 0.000000, 0.000000}
\pgfsetstrokecolor{dialinecolor}
\draw (-7.000000\du,14.400000\du)--(-7.000000\du,14.800000\du)--(6.590000\du,14.800000\du)--(6.590000\du,14.400000\du)--cycle;
\definecolor{dialinecolor}{rgb}{1.000000, 1.000000, 1.000000}
\pgfsetfillcolor{dialinecolor}
\fill (-7.000000\du,14.800000\du)--(-7.000000\du,17.400000\du)--(6.590000\du,17.400000\du)--(6.590000\du,14.800000\du)--cycle;
\definecolor{dialinecolor}{rgb}{0.000000, 0.000000, 0.000000}
\pgfsetstrokecolor{dialinecolor}
\draw (-7.000000\du,14.800000\du)--(-7.000000\du,17.400000\du)--(6.590000\du,17.400000\du)--(6.590000\du,14.800000\du)--cycle;
% setfont left to latex
\definecolor{dialinecolor}{rgb}{0.000000, 0.000000, 0.000000}
\pgfsetstrokecolor{dialinecolor}
\node[anchor=west] at (-6.850000\du,15.500000\du){+activate()};
% setfont left to latex
\definecolor{dialinecolor}{rgb}{0.000000, 0.000000, 0.000000}
\pgfsetstrokecolor{dialinecolor}
\node[anchor=west] at (-6.850000\du,16.300000\du){+attach(observer:ActivateObserver)};
% setfont left to latex
\definecolor{dialinecolor}{rgb}{0.000000, 0.000000, 0.000000}
\pgfsetstrokecolor{dialinecolor}
\node[anchor=west] at (-6.850000\du,17.100000\du){+detach(observer:ActivateObserver)};
\pgfsetlinewidth{0.100000\du}
\pgfsetdash{}{0pt}
\definecolor{dialinecolor}{rgb}{1.000000, 1.000000, 1.000000}
\pgfsetfillcolor{dialinecolor}
\fill (-44.000000\du,25.000000\du)--(-44.000000\du,26.400000\du)--(-38.110000\du,26.400000\du)--(-38.110000\du,25.000000\du)--cycle;
\definecolor{dialinecolor}{rgb}{0.000000, 0.000000, 0.000000}
\pgfsetstrokecolor{dialinecolor}
\draw (-44.000000\du,25.000000\du)--(-44.000000\du,26.400000\du)--(-38.110000\du,26.400000\du)--(-38.110000\du,25.000000\du)--cycle;
% setfont left to latex
\definecolor{dialinecolor}{rgb}{0.000000, 0.000000, 0.000000}
\pgfsetstrokecolor{dialinecolor}
\node at (-41.055000\du,25.950000\du){Navigateur};
\definecolor{dialinecolor}{rgb}{1.000000, 1.000000, 1.000000}
\pgfsetfillcolor{dialinecolor}
\fill (-44.000000\du,26.400000\du)--(-44.000000\du,28.200000\du)--(-38.110000\du,28.200000\du)--(-38.110000\du,26.400000\du)--cycle;
\definecolor{dialinecolor}{rgb}{0.000000, 0.000000, 0.000000}
\pgfsetstrokecolor{dialinecolor}
\draw (-44.000000\du,26.400000\du)--(-44.000000\du,28.200000\du)--(-38.110000\du,28.200000\du)--(-38.110000\du,26.400000\du)--cycle;
% setfont left to latex
\definecolor{dialinecolor}{rgb}{0.000000, 0.000000, 0.000000}
\pgfsetstrokecolor{dialinecolor}
\node[anchor=west] at (-43.850000\du,27.100000\du){-instance};
% setfont left to latex
\definecolor{dialinecolor}{rgb}{0.000000, 0.000000, 0.000000}
\pgfsetstrokecolor{dialinecolor}
\node[anchor=west] at (-43.850000\du,27.900000\du){-dossierActif};
\definecolor{dialinecolor}{rgb}{1.000000, 1.000000, 1.000000}
\pgfsetfillcolor{dialinecolor}
\fill (-44.000000\du,28.200000\du)--(-44.000000\du,30.000000\du)--(-38.110000\du,30.000000\du)--(-38.110000\du,28.200000\du)--cycle;
\definecolor{dialinecolor}{rgb}{0.000000, 0.000000, 0.000000}
\pgfsetstrokecolor{dialinecolor}
\draw (-44.000000\du,28.200000\du)--(-44.000000\du,30.000000\du)--(-38.110000\du,30.000000\du)--(-38.110000\du,28.200000\du)--cycle;
% setfont left to latex
\definecolor{dialinecolor}{rgb}{0.000000, 0.000000, 0.000000}
\pgfsetstrokecolor{dialinecolor}
\node[anchor=west] at (-43.850000\du,28.900000\du){+getInstance()};
% setfont left to latex
\definecolor{dialinecolor}{rgb}{0.000000, 0.000000, 0.000000}
\pgfsetstrokecolor{dialinecolor}
\node[anchor=west] at (-43.850000\du,29.700000\du){-Navigateur()};
\pgfsetlinewidth{0.100000\du}
\pgfsetdash{}{0pt}
\pgfsetmiterjoin
\pgfsetbuttcap
{
\definecolor{dialinecolor}{rgb}{0.000000, 0.000000, 0.000000}
\pgfsetfillcolor{dialinecolor}
% was here!!!
\definecolor{dialinecolor}{rgb}{0.000000, 0.000000, 0.000000}
\pgfsetstrokecolor{dialinecolor}
\draw (9.582700\du,3.281220\du)--(9.582700\du,10.000000\du)--(4.076380\du,10.000000\du)--(4.076380\du,12.809900\du);
}
\definecolor{dialinecolor}{rgb}{0.000000, 0.000000, 0.000000}
\pgfsetstrokecolor{dialinecolor}
\draw (9.582700\du,4.193023\du)--(9.582700\du,10.000000\du)--(4.076380\du,10.000000\du)--(4.076380\du,12.809900\du);
\pgfsetmiterjoin
\definecolor{dialinecolor}{rgb}{1.000000, 1.000000, 1.000000}
\pgfsetfillcolor{dialinecolor}
\fill (9.982700\du,4.193023\du)--(9.582700\du,3.393023\du)--(9.182700\du,4.193023\du)--cycle;
\pgfsetlinewidth{0.100000\du}
\pgfsetdash{}{0pt}
\pgfsetmiterjoin
\definecolor{dialinecolor}{rgb}{0.000000, 0.000000, 0.000000}
\pgfsetstrokecolor{dialinecolor}
\draw (9.982700\du,4.193023\du)--(9.582700\du,3.393023\du)--(9.182700\du,4.193023\du)--cycle;
% setfont left to latex
\pgfsetlinewidth{0.100000\du}
\pgfsetdash{}{0pt}
\pgfsetmiterjoin
\pgfsetbuttcap
{
\definecolor{dialinecolor}{rgb}{0.000000, 0.000000, 0.000000}
\pgfsetfillcolor{dialinecolor}
% was here!!!
\definecolor{dialinecolor}{rgb}{0.000000, 0.000000, 0.000000}
\pgfsetstrokecolor{dialinecolor}
\draw (9.582700\du,3.281220\du)--(9.582700\du,10.000000\du)--(14.499559\du,10.000000\du)--(14.528505\du,14.804353\du);
}
\definecolor{dialinecolor}{rgb}{0.000000, 0.000000, 0.000000}
\pgfsetstrokecolor{dialinecolor}
\draw (9.582700\du,4.193023\du)--(9.582700\du,10.000000\du)--(14.499559\du,10.000000\du)--(14.528505\du,14.804353\du);
\pgfsetmiterjoin
\definecolor{dialinecolor}{rgb}{1.000000, 1.000000, 1.000000}
\pgfsetfillcolor{dialinecolor}
\fill (9.982700\du,4.193023\du)--(9.582700\du,3.393023\du)--(9.182700\du,4.193023\du)--cycle;
\pgfsetlinewidth{0.100000\du}
\pgfsetdash{}{0pt}
\pgfsetmiterjoin
\definecolor{dialinecolor}{rgb}{0.000000, 0.000000, 0.000000}
\pgfsetstrokecolor{dialinecolor}
\draw (9.982700\du,4.193023\du)--(9.582700\du,3.393023\du)--(9.182700\du,4.193023\du)--cycle;
% setfont left to latex
\pgfsetlinewidth{0.100000\du}
\pgfsetdash{}{0pt}
\definecolor{dialinecolor}{rgb}{1.000000, 1.000000, 1.000000}
\pgfsetfillcolor{dialinecolor}
\fill (-54.000000\du,-8.000000\du)--(-54.000000\du,-6.600000\du)--(-40.795000\du,-6.600000\du)--(-40.795000\du,-8.000000\du)--cycle;
\definecolor{dialinecolor}{rgb}{0.000000, 0.000000, 0.000000}
\pgfsetstrokecolor{dialinecolor}
\draw (-54.000000\du,-8.000000\du)--(-54.000000\du,-6.600000\du)--(-40.795000\du,-6.600000\du)--(-40.795000\du,-8.000000\du)--cycle;
% setfont left to latex
\definecolor{dialinecolor}{rgb}{0.000000, 0.000000, 0.000000}
\pgfsetstrokecolor{dialinecolor}
\node at (-47.397500\du,-7.050000\du){Observable};
\definecolor{dialinecolor}{rgb}{1.000000, 1.000000, 1.000000}
\pgfsetfillcolor{dialinecolor}
\fill (-54.000000\du,-6.600000\du)--(-54.000000\du,-6.200000\du)--(-40.795000\du,-6.200000\du)--(-40.795000\du,-6.600000\du)--cycle;
\definecolor{dialinecolor}{rgb}{0.000000, 0.000000, 0.000000}
\pgfsetstrokecolor{dialinecolor}
\draw (-54.000000\du,-6.600000\du)--(-54.000000\du,-6.200000\du)--(-40.795000\du,-6.200000\du)--(-40.795000\du,-6.600000\du)--cycle;
\definecolor{dialinecolor}{rgb}{1.000000, 1.000000, 1.000000}
\pgfsetfillcolor{dialinecolor}
\fill (-54.000000\du,-6.200000\du)--(-54.000000\du,3.600000\du)--(-40.795000\du,3.600000\du)--(-40.795000\du,-6.200000\du)--cycle;
\definecolor{dialinecolor}{rgb}{0.000000, 0.000000, 0.000000}
\pgfsetstrokecolor{dialinecolor}
\draw (-54.000000\du,-6.200000\du)--(-54.000000\du,3.600000\du)--(-40.795000\du,3.600000\du)--(-40.795000\du,-6.200000\du)--cycle;
% setfont left to latex
\definecolor{dialinecolor}{rgb}{0.000000, 0.000000, 0.000000}
\pgfsetstrokecolor{dialinecolor}
\node[anchor=west] at (-53.850000\du,-5.500000\du){+attach(observer:ChangeObserver)};
% setfont left to latex
\definecolor{dialinecolor}{rgb}{0.000000, 0.000000, 0.000000}
\pgfsetstrokecolor{dialinecolor}
\node[anchor=west] at (-53.850000\du,-4.700000\du){+attach(observer:DeleteObserver)};
% setfont left to latex
\definecolor{dialinecolor}{rgb}{0.000000, 0.000000, 0.000000}
\pgfsetstrokecolor{dialinecolor}
\node[anchor=west] at (-53.850000\du,-3.900000\du){+attach(observer:OpenObserver)};
% setfont left to latex
\definecolor{dialinecolor}{rgb}{0.000000, 0.000000, 0.000000}
\pgfsetstrokecolor{dialinecolor}
\node[anchor=west] at (-53.850000\du,-3.100000\du){+attach(observer:CloseObserver)};
% setfont left to latex
\definecolor{dialinecolor}{rgb}{0.000000, 0.000000, 0.000000}
\pgfsetstrokecolor{dialinecolor}
\node[anchor=west] at (-53.850000\du,-2.300000\du){+dettach(observer:ChangeObserve)};
% setfont left to latex
\definecolor{dialinecolor}{rgb}{0.000000, 0.000000, 0.000000}
\pgfsetstrokecolor{dialinecolor}
\node[anchor=west] at (-53.850000\du,-1.500000\du){+dettach(observer:DeleteObserver)};
% setfont left to latex
\definecolor{dialinecolor}{rgb}{0.000000, 0.000000, 0.000000}
\pgfsetstrokecolor{dialinecolor}
\node[anchor=west] at (-53.850000\du,-0.700000\du){+dettach(observer:OpenObserver)};
% setfont left to latex
\definecolor{dialinecolor}{rgb}{0.000000, 0.000000, 0.000000}
\pgfsetstrokecolor{dialinecolor}
\node[anchor=west] at (-53.850000\du,0.100000\du){+dettach(observer:CloseObserver)};
% setfont left to latex
\definecolor{dialinecolor}{rgb}{0.000000, 0.000000, 0.000000}
\pgfsetstrokecolor{dialinecolor}
\node[anchor=west] at (-53.850000\du,0.900000\du){+notifyChange()};
% setfont left to latex
\definecolor{dialinecolor}{rgb}{0.000000, 0.000000, 0.000000}
\pgfsetstrokecolor{dialinecolor}
\node[anchor=west] at (-53.850000\du,1.700000\du){+notifyDelete()};
% setfont left to latex
\definecolor{dialinecolor}{rgb}{0.000000, 0.000000, 0.000000}
\pgfsetstrokecolor{dialinecolor}
\node[anchor=west] at (-53.850000\du,2.500000\du){+notifyOpen()};
% setfont left to latex
\definecolor{dialinecolor}{rgb}{0.000000, 0.000000, 0.000000}
\pgfsetstrokecolor{dialinecolor}
\node[anchor=west] at (-53.850000\du,3.300000\du){+notifyClose()};
\pgfsetlinewidth{0.100000\du}
\pgfsetdash{}{0pt}
\pgfsetmiterjoin
\pgfsetbuttcap
{
\definecolor{dialinecolor}{rgb}{0.000000, 0.000000, 0.000000}
\pgfsetfillcolor{dialinecolor}
% was here!!!
\definecolor{dialinecolor}{rgb}{0.000000, 0.000000, 0.000000}
\pgfsetstrokecolor{dialinecolor}
\draw (1.003770\du,12.886700\du)--(1.003770\du,12.886700\du)--(1.003770\du,-1.218780\du)--(3.507441\du,-1.218780\du);
}
\definecolor{dialinecolor}{rgb}{0.000000, 0.000000, 0.000000}
\pgfsetstrokecolor{dialinecolor}
\draw (1.003770\du,11.628121\du)--(1.003770\du,-1.218780\du)--(3.507441\du,-1.218780\du);
\pgfsetdash{}{0pt}
\pgfsetmiterjoin
\pgfsetbuttcap
\definecolor{dialinecolor}{rgb}{0.000000, 0.000000, 0.000000}
\pgfsetfillcolor{dialinecolor}
\fill (1.003770\du,12.886700\du)--(0.763770\du,12.186700\du)--(1.003770\du,11.486700\du)--(1.243770\du,12.186700\du)--cycle;
\pgfsetlinewidth{0.100000\du}
\pgfsetdash{}{0pt}
\pgfsetmiterjoin
\pgfsetbuttcap
\definecolor{dialinecolor}{rgb}{0.000000, 0.000000, 0.000000}
\pgfsetstrokecolor{dialinecolor}
\draw (1.003770\du,12.886700\du)--(0.763770\du,12.186700\du)--(1.003770\du,11.486700\du)--(1.243770\du,12.186700\du)--cycle;
% setfont left to latex
\definecolor{dialinecolor}{rgb}{0.000000, 0.000000, 0.000000}
\pgfsetstrokecolor{dialinecolor}
\node[anchor=west] at (1.103770\du,5.683960\du){};
\definecolor{dialinecolor}{rgb}{0.000000, 0.000000, 0.000000}
\pgfsetstrokecolor{dialinecolor}
\node[anchor=west] at (1.553770\du,12.686700\du){};
\definecolor{dialinecolor}{rgb}{0.000000, 0.000000, 0.000000}
\pgfsetstrokecolor{dialinecolor}
\node[anchor=east] at (3.307441\du,-1.368780\du){ possède};
\definecolor{dialinecolor}{rgb}{0.000000, 0.000000, 0.000000}
\pgfsetstrokecolor{dialinecolor}
\node[anchor=east] at (3.307441\du,-0.568780\du){*};
\pgfsetlinewidth{0.100000\du}
\pgfsetdash{}{0pt}
\pgfsetmiterjoin
\pgfsetbuttcap
{
\definecolor{dialinecolor}{rgb}{0.000000, 0.000000, 0.000000}
\pgfsetfillcolor{dialinecolor}
% was here!!!
\pgfsetarrowsend{to}
\definecolor{dialinecolor}{rgb}{0.000000, 0.000000, 0.000000}
\pgfsetstrokecolor{dialinecolor}
\draw (-38.062358\du,27.500000\du)--(-1.029330\du,27.500000\du)--(-1.029330\du,17.541700\du)--(-1.029330\du,17.541700\du);
}
% setfont left to latex
\definecolor{dialinecolor}{rgb}{0.000000, 0.000000, 0.000000}
\pgfsetstrokecolor{dialinecolor}
\node[anchor=west] at (-0.929330\du,22.370850\du){};
\definecolor{dialinecolor}{rgb}{0.000000, 0.000000, 0.000000}
\pgfsetstrokecolor{dialinecolor}
\node[anchor=west] at (-37.862358\du,27.350000\du){};
\definecolor{dialinecolor}{rgb}{0.000000, 0.000000, 0.000000}
\pgfsetstrokecolor{dialinecolor}
\node[anchor=west] at (-0.479330\du,18.141700\du){ active};
\definecolor{dialinecolor}{rgb}{0.000000, 0.000000, 0.000000}
\pgfsetstrokecolor{dialinecolor}
\node[anchor=west] at (-0.479330\du,18.941700\du){0..1};
\pgfsetlinewidth{0.100000\du}
\pgfsetdash{}{0pt}
\definecolor{dialinecolor}{rgb}{1.000000, 1.000000, 1.000000}
\pgfsetfillcolor{dialinecolor}
\fill (-25.000000\du,13.000000\du)--(-25.000000\du,14.400000\du)--(-14.490000\du,14.400000\du)--(-14.490000\du,13.000000\du)--cycle;
\definecolor{dialinecolor}{rgb}{0.000000, 0.000000, 0.000000}
\pgfsetstrokecolor{dialinecolor}
\draw (-25.000000\du,13.000000\du)--(-25.000000\du,14.400000\du)--(-14.490000\du,14.400000\du)--(-14.490000\du,13.000000\du)--cycle;
% setfont left to latex
\definecolor{dialinecolor}{rgb}{0.000000, 0.000000, 0.000000}
\pgfsetstrokecolor{dialinecolor}
\node at (-19.745000\du,13.950000\du){ActivateObserver};
\definecolor{dialinecolor}{rgb}{1.000000, 1.000000, 1.000000}
\pgfsetfillcolor{dialinecolor}
\fill (-25.000000\du,14.400000\du)--(-25.000000\du,14.800000\du)--(-14.490000\du,14.800000\du)--(-14.490000\du,14.400000\du)--cycle;
\definecolor{dialinecolor}{rgb}{0.000000, 0.000000, 0.000000}
\pgfsetstrokecolor{dialinecolor}
\draw (-25.000000\du,14.400000\du)--(-25.000000\du,14.800000\du)--(-14.490000\du,14.800000\du)--(-14.490000\du,14.400000\du)--cycle;
\definecolor{dialinecolor}{rgb}{1.000000, 1.000000, 1.000000}
\pgfsetfillcolor{dialinecolor}
\fill (-25.000000\du,14.800000\du)--(-25.000000\du,15.800000\du)--(-14.490000\du,15.800000\du)--(-14.490000\du,14.800000\du)--cycle;
\definecolor{dialinecolor}{rgb}{0.000000, 0.000000, 0.000000}
\pgfsetstrokecolor{dialinecolor}
\draw (-25.000000\du,14.800000\du)--(-25.000000\du,15.800000\du)--(-14.490000\du,15.800000\du)--(-14.490000\du,14.800000\du)--cycle;
% setfont left to latex
\definecolor{dialinecolor}{rgb}{0.000000, 0.000000, 0.000000}
\pgfsetstrokecolor{dialinecolor}
\node[anchor=west] at (-24.850000\du,15.500000\du){+updateActivate(d:Dossier)};
\pgfsetlinewidth{0.100000\du}
\pgfsetdash{}{0pt}
\definecolor{dialinecolor}{rgb}{1.000000, 1.000000, 1.000000}
\pgfsetfillcolor{dialinecolor}
\fill (-36.000000\du,13.000000\du)--(-36.000000\du,14.400000\du)--(-26.260000\du,14.400000\du)--(-26.260000\du,13.000000\du)--cycle;
\definecolor{dialinecolor}{rgb}{0.000000, 0.000000, 0.000000}
\pgfsetstrokecolor{dialinecolor}
\draw (-36.000000\du,13.000000\du)--(-36.000000\du,14.400000\du)--(-26.260000\du,14.400000\du)--(-26.260000\du,13.000000\du)--cycle;
% setfont left to latex
\definecolor{dialinecolor}{rgb}{0.000000, 0.000000, 0.000000}
\pgfsetstrokecolor{dialinecolor}
\node at (-31.130000\du,13.950000\du){DeleteObserver};
\definecolor{dialinecolor}{rgb}{1.000000, 1.000000, 1.000000}
\pgfsetfillcolor{dialinecolor}
\fill (-36.000000\du,14.400000\du)--(-36.000000\du,14.800000\du)--(-26.260000\du,14.800000\du)--(-26.260000\du,14.400000\du)--cycle;
\definecolor{dialinecolor}{rgb}{0.000000, 0.000000, 0.000000}
\pgfsetstrokecolor{dialinecolor}
\draw (-36.000000\du,14.400000\du)--(-36.000000\du,14.800000\du)--(-26.260000\du,14.800000\du)--(-26.260000\du,14.400000\du)--cycle;
\definecolor{dialinecolor}{rgb}{1.000000, 1.000000, 1.000000}
\pgfsetfillcolor{dialinecolor}
\fill (-36.000000\du,14.800000\du)--(-36.000000\du,15.800000\du)--(-26.260000\du,15.800000\du)--(-26.260000\du,14.800000\du)--cycle;
\definecolor{dialinecolor}{rgb}{0.000000, 0.000000, 0.000000}
\pgfsetstrokecolor{dialinecolor}
\draw (-36.000000\du,14.800000\du)--(-36.000000\du,15.800000\du)--(-26.260000\du,15.800000\du)--(-26.260000\du,14.800000\du)--cycle;
% setfont left to latex
\definecolor{dialinecolor}{rgb}{0.000000, 0.000000, 0.000000}
\pgfsetstrokecolor{dialinecolor}
\node[anchor=west] at (-35.850000\du,15.500000\du){+updateDelete(e:Element)};
\pgfsetlinewidth{0.100000\du}
\pgfsetdash{}{0pt}
\definecolor{dialinecolor}{rgb}{1.000000, 1.000000, 1.000000}
\pgfsetfillcolor{dialinecolor}
\fill (-46.000000\du,13.000000\du)--(-46.000000\du,14.400000\du)--(-37.030000\du,14.400000\du)--(-37.030000\du,13.000000\du)--cycle;
\definecolor{dialinecolor}{rgb}{0.000000, 0.000000, 0.000000}
\pgfsetstrokecolor{dialinecolor}
\draw (-46.000000\du,13.000000\du)--(-46.000000\du,14.400000\du)--(-37.030000\du,14.400000\du)--(-37.030000\du,13.000000\du)--cycle;
% setfont left to latex
\definecolor{dialinecolor}{rgb}{0.000000, 0.000000, 0.000000}
\pgfsetstrokecolor{dialinecolor}
\node at (-41.515000\du,13.950000\du){OpenObserver};
\definecolor{dialinecolor}{rgb}{1.000000, 1.000000, 1.000000}
\pgfsetfillcolor{dialinecolor}
\fill (-46.000000\du,14.400000\du)--(-46.000000\du,14.800000\du)--(-37.030000\du,14.800000\du)--(-37.030000\du,14.400000\du)--cycle;
\definecolor{dialinecolor}{rgb}{0.000000, 0.000000, 0.000000}
\pgfsetstrokecolor{dialinecolor}
\draw (-46.000000\du,14.400000\du)--(-46.000000\du,14.800000\du)--(-37.030000\du,14.800000\du)--(-37.030000\du,14.400000\du)--cycle;
\definecolor{dialinecolor}{rgb}{1.000000, 1.000000, 1.000000}
\pgfsetfillcolor{dialinecolor}
\fill (-46.000000\du,14.800000\du)--(-46.000000\du,15.800000\du)--(-37.030000\du,15.800000\du)--(-37.030000\du,14.800000\du)--cycle;
\definecolor{dialinecolor}{rgb}{0.000000, 0.000000, 0.000000}
\pgfsetstrokecolor{dialinecolor}
\draw (-46.000000\du,14.800000\du)--(-46.000000\du,15.800000\du)--(-37.030000\du,15.800000\du)--(-37.030000\du,14.800000\du)--cycle;
% setfont left to latex
\definecolor{dialinecolor}{rgb}{0.000000, 0.000000, 0.000000}
\pgfsetstrokecolor{dialinecolor}
\node[anchor=west] at (-45.850000\du,15.500000\du){+updateOpen(e:Element)};
\pgfsetlinewidth{0.100000\du}
\pgfsetdash{}{0pt}
\definecolor{dialinecolor}{rgb}{1.000000, 1.000000, 1.000000}
\pgfsetfillcolor{dialinecolor}
\fill (-58.000000\du,13.000000\du)--(-58.000000\du,14.400000\du)--(-48.260000\du,14.400000\du)--(-48.260000\du,13.000000\du)--cycle;
\definecolor{dialinecolor}{rgb}{0.000000, 0.000000, 0.000000}
\pgfsetstrokecolor{dialinecolor}
\draw (-58.000000\du,13.000000\du)--(-58.000000\du,14.400000\du)--(-48.260000\du,14.400000\du)--(-48.260000\du,13.000000\du)--cycle;
% setfont left to latex
\definecolor{dialinecolor}{rgb}{0.000000, 0.000000, 0.000000}
\pgfsetstrokecolor{dialinecolor}
\node at (-53.130000\du,13.950000\du){ChangeObserver};
\definecolor{dialinecolor}{rgb}{1.000000, 1.000000, 1.000000}
\pgfsetfillcolor{dialinecolor}
\fill (-58.000000\du,14.400000\du)--(-58.000000\du,14.800000\du)--(-48.260000\du,14.800000\du)--(-48.260000\du,14.400000\du)--cycle;
\definecolor{dialinecolor}{rgb}{0.000000, 0.000000, 0.000000}
\pgfsetstrokecolor{dialinecolor}
\draw (-58.000000\du,14.400000\du)--(-58.000000\du,14.800000\du)--(-48.260000\du,14.800000\du)--(-48.260000\du,14.400000\du)--cycle;
\definecolor{dialinecolor}{rgb}{1.000000, 1.000000, 1.000000}
\pgfsetfillcolor{dialinecolor}
\fill (-58.000000\du,14.800000\du)--(-58.000000\du,15.800000\du)--(-48.260000\du,15.800000\du)--(-48.260000\du,14.800000\du)--cycle;
\definecolor{dialinecolor}{rgb}{0.000000, 0.000000, 0.000000}
\pgfsetstrokecolor{dialinecolor}
\draw (-58.000000\du,14.800000\du)--(-58.000000\du,15.800000\du)--(-48.260000\du,15.800000\du)--(-48.260000\du,14.800000\du)--cycle;
% setfont left to latex
\definecolor{dialinecolor}{rgb}{0.000000, 0.000000, 0.000000}
\pgfsetstrokecolor{dialinecolor}
\node[anchor=west] at (-57.850000\du,15.500000\du){+updateChange(e:Element)};
\pgfsetlinewidth{0.100000\du}
\pgfsetdash{}{0pt}
\definecolor{dialinecolor}{rgb}{1.000000, 1.000000, 1.000000}
\pgfsetfillcolor{dialinecolor}
\fill (-69.175500\du,13.000000\du)--(-69.175500\du,14.400000\du)--(-59.820500\du,14.400000\du)--(-59.820500\du,13.000000\du)--cycle;
\definecolor{dialinecolor}{rgb}{0.000000, 0.000000, 0.000000}
\pgfsetstrokecolor{dialinecolor}
\draw (-69.175500\du,13.000000\du)--(-69.175500\du,14.400000\du)--(-59.820500\du,14.400000\du)--(-59.820500\du,13.000000\du)--cycle;
% setfont left to latex
\definecolor{dialinecolor}{rgb}{0.000000, 0.000000, 0.000000}
\pgfsetstrokecolor{dialinecolor}
\node at (-64.498000\du,13.950000\du){CloseObserver};
\definecolor{dialinecolor}{rgb}{1.000000, 1.000000, 1.000000}
\pgfsetfillcolor{dialinecolor}
\fill (-69.175500\du,14.400000\du)--(-69.175500\du,14.800000\du)--(-59.820500\du,14.800000\du)--(-59.820500\du,14.400000\du)--cycle;
\definecolor{dialinecolor}{rgb}{0.000000, 0.000000, 0.000000}
\pgfsetstrokecolor{dialinecolor}
\draw (-69.175500\du,14.400000\du)--(-69.175500\du,14.800000\du)--(-59.820500\du,14.800000\du)--(-59.820500\du,14.400000\du)--cycle;
\definecolor{dialinecolor}{rgb}{1.000000, 1.000000, 1.000000}
\pgfsetfillcolor{dialinecolor}
\fill (-69.175500\du,14.800000\du)--(-69.175500\du,15.800000\du)--(-59.820500\du,15.800000\du)--(-59.820500\du,14.800000\du)--cycle;
\definecolor{dialinecolor}{rgb}{0.000000, 0.000000, 0.000000}
\pgfsetstrokecolor{dialinecolor}
\draw (-69.175500\du,14.800000\du)--(-69.175500\du,15.800000\du)--(-59.820500\du,15.800000\du)--(-59.820500\du,14.800000\du)--cycle;
% setfont left to latex
\definecolor{dialinecolor}{rgb}{0.000000, 0.000000, 0.000000}
\pgfsetstrokecolor{dialinecolor}
\node[anchor=west] at (-69.025500\du,15.500000\du){+updateClose(e:Element)};
\pgfsetlinewidth{0.100000\du}
\pgfsetdash{{1.000000\du}{1.000000\du}}{0\du}
\pgfsetdash{{0.400000\du}{0.400000\du}}{0\du}
\pgfsetmiterjoin
\pgfsetbuttcap
{
\definecolor{dialinecolor}{rgb}{0.000000, 0.000000, 0.000000}
\pgfsetfillcolor{dialinecolor}
% was here!!!
\definecolor{dialinecolor}{rgb}{0.000000, 0.000000, 0.000000}
\pgfsetstrokecolor{dialinecolor}
\draw (-53.130000\du,15.849121\du)--(-53.130000\du,22.000000\du)--(-41.055000\du,22.000000\du)--(-41.055000\du,24.949738\du);
}
\definecolor{dialinecolor}{rgb}{0.000000, 0.000000, 0.000000}
\pgfsetstrokecolor{dialinecolor}
\draw (-53.130000\du,16.760924\du)--(-53.130000\du,22.000000\du)--(-41.055000\du,22.000000\du)--(-41.055000\du,24.949738\du);
\pgfsetmiterjoin
\definecolor{dialinecolor}{rgb}{1.000000, 1.000000, 1.000000}
\pgfsetfillcolor{dialinecolor}
\fill (-52.730000\du,16.760924\du)--(-53.130000\du,15.960924\du)--(-53.530000\du,16.760924\du)--cycle;
\pgfsetlinewidth{0.100000\du}
\pgfsetdash{}{0pt}
\pgfsetmiterjoin
\definecolor{dialinecolor}{rgb}{0.000000, 0.000000, 0.000000}
\pgfsetstrokecolor{dialinecolor}
\draw (-52.730000\du,16.760924\du)--(-53.130000\du,15.960924\du)--(-53.530000\du,16.760924\du)--cycle;
% setfont left to latex
\pgfsetlinewidth{0.100000\du}
\pgfsetdash{{0.400000\du}{0.400000\du}}{0\du}
\pgfsetdash{{0.400000\du}{0.400000\du}}{0\du}
\pgfsetmiterjoin
\pgfsetbuttcap
{
\definecolor{dialinecolor}{rgb}{0.000000, 0.000000, 0.000000}
\pgfsetfillcolor{dialinecolor}
% was here!!!
\definecolor{dialinecolor}{rgb}{0.000000, 0.000000, 0.000000}
\pgfsetstrokecolor{dialinecolor}
\draw (-31.130000\du,15.849707\du)--(-31.130000\du,20.754200\du)--(-41.055000\du,20.754200\du)--(-41.055000\du,24.950150\du);
}
\definecolor{dialinecolor}{rgb}{0.000000, 0.000000, 0.000000}
\pgfsetstrokecolor{dialinecolor}
\draw (-31.130000\du,16.761510\du)--(-31.130000\du,20.754200\du)--(-41.055000\du,20.754200\du)--(-41.055000\du,24.950150\du);
\pgfsetmiterjoin
\definecolor{dialinecolor}{rgb}{1.000000, 1.000000, 1.000000}
\pgfsetfillcolor{dialinecolor}
\fill (-30.730000\du,16.761510\du)--(-31.130000\du,15.961510\du)--(-31.530000\du,16.761510\du)--cycle;
\pgfsetlinewidth{0.100000\du}
\pgfsetdash{}{0pt}
\pgfsetmiterjoin
\definecolor{dialinecolor}{rgb}{0.000000, 0.000000, 0.000000}
\pgfsetstrokecolor{dialinecolor}
\draw (-30.730000\du,16.761510\du)--(-31.130000\du,15.961510\du)--(-31.530000\du,16.761510\du)--cycle;
% setfont left to latex
\pgfsetlinewidth{0.100000\du}
\pgfsetdash{{0.400000\du}{0.400000\du}}{0\du}
\pgfsetdash{{0.400000\du}{0.400000\du}}{0\du}
\pgfsetmiterjoin
\pgfsetbuttcap
{
\definecolor{dialinecolor}{rgb}{0.000000, 0.000000, 0.000000}
\pgfsetfillcolor{dialinecolor}
% was here!!!
\definecolor{dialinecolor}{rgb}{0.000000, 0.000000, 0.000000}
\pgfsetstrokecolor{dialinecolor}
\draw (-41.515000\du,15.849121\du)--(-41.515000\du,22.000000\du)--(-41.055000\du,22.000000\du)--(-41.055000\du,24.949738\du);
}
\definecolor{dialinecolor}{rgb}{0.000000, 0.000000, 0.000000}
\pgfsetstrokecolor{dialinecolor}
\draw (-41.515000\du,16.760924\du)--(-41.515000\du,22.000000\du)--(-41.055000\du,22.000000\du)--(-41.055000\du,24.949738\du);
\pgfsetmiterjoin
\definecolor{dialinecolor}{rgb}{1.000000, 1.000000, 1.000000}
\pgfsetfillcolor{dialinecolor}
\fill (-41.115000\du,16.760924\du)--(-41.515000\du,15.960924\du)--(-41.915000\du,16.760924\du)--cycle;
\pgfsetlinewidth{0.100000\du}
\pgfsetdash{}{0pt}
\pgfsetmiterjoin
\definecolor{dialinecolor}{rgb}{0.000000, 0.000000, 0.000000}
\pgfsetstrokecolor{dialinecolor}
\draw (-41.115000\du,16.760924\du)--(-41.515000\du,15.960924\du)--(-41.915000\du,16.760924\du)--cycle;
% setfont left to latex
\pgfsetlinewidth{0.100000\du}
\pgfsetdash{{0.400000\du}{0.400000\du}}{0\du}
\pgfsetdash{{0.400000\du}{0.400000\du}}{0\du}
\pgfsetmiterjoin
\pgfsetbuttcap
{
\definecolor{dialinecolor}{rgb}{0.000000, 0.000000, 0.000000}
\pgfsetfillcolor{dialinecolor}
% was here!!!
\definecolor{dialinecolor}{rgb}{0.000000, 0.000000, 0.000000}
\pgfsetstrokecolor{dialinecolor}
\draw (-64.498000\du,15.849121\du)--(-64.498000\du,22.000000\du)--(-41.055000\du,22.000000\du)--(-41.055000\du,24.949738\du);
}
\definecolor{dialinecolor}{rgb}{0.000000, 0.000000, 0.000000}
\pgfsetstrokecolor{dialinecolor}
\draw (-64.498000\du,16.760924\du)--(-64.498000\du,22.000000\du)--(-41.055000\du,22.000000\du)--(-41.055000\du,24.949738\du);
\pgfsetmiterjoin
\definecolor{dialinecolor}{rgb}{1.000000, 1.000000, 1.000000}
\pgfsetfillcolor{dialinecolor}
\fill (-64.098000\du,16.760924\du)--(-64.498000\du,15.960924\du)--(-64.898000\du,16.760924\du)--cycle;
\pgfsetlinewidth{0.100000\du}
\pgfsetdash{}{0pt}
\pgfsetmiterjoin
\definecolor{dialinecolor}{rgb}{0.000000, 0.000000, 0.000000}
\pgfsetstrokecolor{dialinecolor}
\draw (-64.098000\du,16.760924\du)--(-64.498000\du,15.960924\du)--(-64.898000\du,16.760924\du)--cycle;
% setfont left to latex
\pgfsetlinewidth{0.100000\du}
\pgfsetdash{{0.400000\du}{0.400000\du}}{0\du}
\pgfsetdash{{0.400000\du}{0.400000\du}}{0\du}
\pgfsetmiterjoin
\pgfsetbuttcap
{
\definecolor{dialinecolor}{rgb}{0.000000, 0.000000, 0.000000}
\pgfsetfillcolor{dialinecolor}
% was here!!!
\definecolor{dialinecolor}{rgb}{0.000000, 0.000000, 0.000000}
\pgfsetstrokecolor{dialinecolor}
\draw (-19.745000\du,15.849121\du)--(-19.745000\du,22.000000\du)--(-41.055000\du,22.000000\du)--(-41.055000\du,24.949738\du);
}
\definecolor{dialinecolor}{rgb}{0.000000, 0.000000, 0.000000}
\pgfsetstrokecolor{dialinecolor}
\draw (-19.745000\du,16.760924\du)--(-19.745000\du,22.000000\du)--(-41.055000\du,22.000000\du)--(-41.055000\du,24.949738\du);
\pgfsetmiterjoin
\definecolor{dialinecolor}{rgb}{1.000000, 1.000000, 1.000000}
\pgfsetfillcolor{dialinecolor}
\fill (-19.345000\du,16.760924\du)--(-19.745000\du,15.960924\du)--(-20.145000\du,16.760924\du)--cycle;
\pgfsetlinewidth{0.100000\du}
\pgfsetdash{}{0pt}
\pgfsetmiterjoin
\definecolor{dialinecolor}{rgb}{0.000000, 0.000000, 0.000000}
\pgfsetstrokecolor{dialinecolor}
\draw (-19.345000\du,16.760924\du)--(-19.745000\du,15.960924\du)--(-20.145000\du,16.760924\du)--cycle;
% setfont left to latex
\pgfsetlinewidth{0.100000\du}
\pgfsetdash{}{0pt}
\pgfsetmiterjoin
\pgfsetbuttcap
{
\definecolor{dialinecolor}{rgb}{0.000000, 0.000000, 0.000000}
\pgfsetfillcolor{dialinecolor}
% was here!!!
\definecolor{dialinecolor}{rgb}{0.000000, 0.000000, 0.000000}
\pgfsetstrokecolor{dialinecolor}
\draw (-40.795000\du,-4.900000\du)--(-27.000000\du,-4.900000\du)--(-27.000000\du,-5.018780\du)--(3.557700\du,-5.018780\du);
}
\definecolor{dialinecolor}{rgb}{0.000000, 0.000000, 0.000000}
\pgfsetstrokecolor{dialinecolor}
\draw (-39.883197\du,-4.900000\du)--(-27.000000\du,-4.900000\du)--(-27.000000\du,-5.018780\du)--(3.557700\du,-5.018780\du);
\pgfsetmiterjoin
\definecolor{dialinecolor}{rgb}{1.000000, 1.000000, 1.000000}
\pgfsetfillcolor{dialinecolor}
\fill (-39.883197\du,-5.300000\du)--(-40.683197\du,-4.900000\du)--(-39.883197\du,-4.500000\du)--cycle;
\pgfsetlinewidth{0.100000\du}
\pgfsetdash{}{0pt}
\pgfsetmiterjoin
\definecolor{dialinecolor}{rgb}{0.000000, 0.000000, 0.000000}
\pgfsetstrokecolor{dialinecolor}
\draw (-39.883197\du,-5.300000\du)--(-40.683197\du,-4.900000\du)--(-39.883197\du,-4.500000\du)--cycle;
% setfont left to latex
\pgfsetlinewidth{0.100000\du}
\pgfsetdash{}{0pt}
\pgfsetmiterjoin
\pgfsetbuttcap
{
\definecolor{dialinecolor}{rgb}{0.000000, 0.000000, 0.000000}
\pgfsetfillcolor{dialinecolor}
% was here!!!
\pgfsetarrowsend{to}
\definecolor{dialinecolor}{rgb}{0.000000, 0.000000, 0.000000}
\pgfsetstrokecolor{dialinecolor}
\draw (-47.397500\du,3.649512\du)--(-47.397500\du,9.000000\du)--(-64.498000\du,9.000000\du)--(-64.498000\du,13.000000\du);
}
% setfont left to latex
\definecolor{dialinecolor}{rgb}{0.000000, 0.000000, 0.000000}
\pgfsetstrokecolor{dialinecolor}
\node at (-55.947750\du,8.850000\du){};
\definecolor{dialinecolor}{rgb}{0.000000, 0.000000, 0.000000}
\pgfsetstrokecolor{dialinecolor}
\node[anchor=west] at (-47.197500\du,4.249512\du){};
\definecolor{dialinecolor}{rgb}{0.000000, 0.000000, 0.000000}
\pgfsetstrokecolor{dialinecolor}
\node[anchor=west] at (-63.948000\du,12.800000\du){*};
\pgfsetlinewidth{0.100000\du}
\pgfsetdash{}{0pt}
\pgfsetmiterjoin
\pgfsetbuttcap
{
\definecolor{dialinecolor}{rgb}{0.000000, 0.000000, 0.000000}
\pgfsetfillcolor{dialinecolor}
% was here!!!
\pgfsetarrowsend{to}
\definecolor{dialinecolor}{rgb}{0.000000, 0.000000, 0.000000}
\pgfsetstrokecolor{dialinecolor}
\draw (-47.397500\du,3.649512\du)--(-47.397500\du,9.000000\du)--(-31.130000\du,9.000000\du)--(-31.130000\du,13.000000\du);
}
% setfont left to latex
\definecolor{dialinecolor}{rgb}{0.000000, 0.000000, 0.000000}
\pgfsetstrokecolor{dialinecolor}
\node at (-39.263750\du,8.850000\du){};
\definecolor{dialinecolor}{rgb}{0.000000, 0.000000, 0.000000}
\pgfsetstrokecolor{dialinecolor}
\node[anchor=west] at (-47.197500\du,4.249512\du){};
\definecolor{dialinecolor}{rgb}{0.000000, 0.000000, 0.000000}
\pgfsetstrokecolor{dialinecolor}
\node[anchor=west] at (-30.580000\du,12.800000\du){*};
\pgfsetlinewidth{0.100000\du}
\pgfsetdash{}{0pt}
\pgfsetmiterjoin
\pgfsetbuttcap
{
\definecolor{dialinecolor}{rgb}{0.000000, 0.000000, 0.000000}
\pgfsetfillcolor{dialinecolor}
% was here!!!
\pgfsetarrowsend{to}
\definecolor{dialinecolor}{rgb}{0.000000, 0.000000, 0.000000}
\pgfsetstrokecolor{dialinecolor}
\draw (-6.000000\du,13.000000\du)--(-6.000000\du,10.000000\du)--(-19.745000\du,10.000000\du)--(-19.745000\du,13.000000\du);
}
% setfont left to latex
\definecolor{dialinecolor}{rgb}{0.000000, 0.000000, 0.000000}
\pgfsetstrokecolor{dialinecolor}
\node at (-12.872500\du,9.850000\du){};
\definecolor{dialinecolor}{rgb}{0.000000, 0.000000, 0.000000}
\pgfsetstrokecolor{dialinecolor}
\node[anchor=west] at (-5.800000\du,12.000000\du){ observe};
\definecolor{dialinecolor}{rgb}{0.000000, 0.000000, 0.000000}
\pgfsetstrokecolor{dialinecolor}
\node[anchor=west] at (-5.800000\du,12.800000\du){*};
\definecolor{dialinecolor}{rgb}{0.000000, 0.000000, 0.000000}
\pgfsetstrokecolor{dialinecolor}
\node[anchor=west] at (-19.195000\du,12.800000\du){*};
\pgfsetlinewidth{0.100000\du}
\pgfsetdash{}{0pt}
\pgfsetmiterjoin
\pgfsetbuttcap
{
\definecolor{dialinecolor}{rgb}{0.000000, 0.000000, 0.000000}
\pgfsetfillcolor{dialinecolor}
% was here!!!
\pgfsetarrowsend{to}
\definecolor{dialinecolor}{rgb}{0.000000, 0.000000, 0.000000}
\pgfsetstrokecolor{dialinecolor}
\draw (-47.397500\du,3.649512\du)--(-47.397500\du,9.000000\du)--(-41.515000\du,9.000000\du)--(-41.515000\du,12.952441\du);
}
% setfont left to latex
\definecolor{dialinecolor}{rgb}{0.000000, 0.000000, 0.000000}
\pgfsetstrokecolor{dialinecolor}
\node at (-44.456250\du,8.850000\du){};
\definecolor{dialinecolor}{rgb}{0.000000, 0.000000, 0.000000}
\pgfsetstrokecolor{dialinecolor}
\node[anchor=west] at (-47.197500\du,4.249512\du){};
\definecolor{dialinecolor}{rgb}{0.000000, 0.000000, 0.000000}
\pgfsetstrokecolor{dialinecolor}
\node[anchor=west] at (-40.965000\du,12.752441\du){*};
\pgfsetlinewidth{0.100000\du}
\pgfsetdash{}{0pt}
\pgfsetmiterjoin
\pgfsetbuttcap
{
\definecolor{dialinecolor}{rgb}{0.000000, 0.000000, 0.000000}
\pgfsetfillcolor{dialinecolor}
% was here!!!
\pgfsetarrowsend{to}
\definecolor{dialinecolor}{rgb}{0.000000, 0.000000, 0.000000}
\pgfsetstrokecolor{dialinecolor}
\draw (-47.397500\du,3.649512\du)--(-47.397500\du,9.000000\du)--(-53.130000\du,9.000000\du)--(-53.130000\du,13.000000\du);
}
% setfont left to latex
\definecolor{dialinecolor}{rgb}{0.000000, 0.000000, 0.000000}
\pgfsetstrokecolor{dialinecolor}
\node at (-50.263750\du,8.850000\du){};
\definecolor{dialinecolor}{rgb}{0.000000, 0.000000, 0.000000}
\pgfsetstrokecolor{dialinecolor}
\node[anchor=west] at (-47.197500\du,4.249512\du){1};
\definecolor{dialinecolor}{rgb}{0.000000, 0.000000, 0.000000}
\pgfsetstrokecolor{dialinecolor}
\node[anchor=west] at (-52.580000\du,12.800000\du){*};
\pgfsetlinewidth{0.100000\du}
\pgfsetdash{}{0pt}
\pgfsetmiterjoin
\pgfsetbuttcap
{
\definecolor{dialinecolor}{rgb}{0.000000, 0.000000, 0.000000}
\pgfsetfillcolor{dialinecolor}
% was here!!!
\pgfsetarrowsend{to}
\definecolor{dialinecolor}{rgb}{0.000000, 0.000000, 0.000000}
\pgfsetstrokecolor{dialinecolor}
\draw (-38.060323\du,27.500000\du)--(-11.000000\du,27.500000\du)--(-11.000000\du,15.300000\du)--(-7.000000\du,15.300000\du);
}
% setfont left to latex
\definecolor{dialinecolor}{rgb}{0.000000, 0.000000, 0.000000}
\pgfsetstrokecolor{dialinecolor}
\node[anchor=west] at (-10.900000\du,21.250000\du){};
\definecolor{dialinecolor}{rgb}{0.000000, 0.000000, 0.000000}
\pgfsetstrokecolor{dialinecolor}
\node[anchor=west] at (-37.860323\du,27.350000\du){1};
\definecolor{dialinecolor}{rgb}{0.000000, 0.000000, 0.000000}
\pgfsetstrokecolor{dialinecolor}
\node[anchor=east] at (-8.000000\du,15.150000\du){ ouvre};
\definecolor{dialinecolor}{rgb}{0.000000, 0.000000, 0.000000}
\pgfsetstrokecolor{dialinecolor}
\node[anchor=east] at (-8.000000\du,15.950000\du){*};
\pgfsetlinewidth{0.100000\du}
\pgfsetdash{{0.400000\du}{0.400000\du}}{0\du}
\pgfsetdash{{0.400000\du}{0.400000\du}}{0\du}
\pgfsetmiterjoin
\pgfsetbuttcap
{
\definecolor{dialinecolor}{rgb}{0.000000, 0.000000, 0.000000}
\pgfsetfillcolor{dialinecolor}
% was here!!!
\definecolor{dialinecolor}{rgb}{0.000000, 0.000000, 0.000000}
\pgfsetstrokecolor{dialinecolor}
\draw (-29.000000\du,13.000000\du)--(-29.000000\du,9.000000\du)--(-0.205000\du,9.000000\du)--(-0.205000\du,12.949927\du);
}
\definecolor{dialinecolor}{rgb}{0.000000, 0.000000, 0.000000}
\pgfsetstrokecolor{dialinecolor}
\draw (-29.000000\du,12.088197\du)--(-29.000000\du,9.000000\du)--(-0.205000\du,9.000000\du)--(-0.205000\du,12.949927\du);
\pgfsetmiterjoin
\definecolor{dialinecolor}{rgb}{1.000000, 1.000000, 1.000000}
\pgfsetfillcolor{dialinecolor}
\fill (-29.400000\du,12.088197\du)--(-29.000000\du,12.888197\du)--(-28.600000\du,12.088197\du)--cycle;
\pgfsetlinewidth{0.100000\du}
\pgfsetdash{}{0pt}
\pgfsetmiterjoin
\definecolor{dialinecolor}{rgb}{0.000000, 0.000000, 0.000000}
\pgfsetstrokecolor{dialinecolor}
\draw (-29.400000\du,12.088197\du)--(-29.000000\du,12.888197\du)--(-28.600000\du,12.088197\du)--cycle;
% setfont left to latex
\end{tikzpicture}
}
    \caption{Diagramme de classes}
  \end{sidewaysfigure}

  \section{Diagramme de classe avec \textsf{Client}}
  \begin{sidewaysfigure}
    \centering
    \resizebox{\textwidth}{!}{% Graphic for TeX using PGF
% Title: /home/guillaume/Documents/Université de Montréal/Automne 2014/Génie logiciel/Devoir/devoir-3/diagramme-de-classes-b.dia
% Creator: Dia v0.97.3
% CreationDate: Sat Dec  6 11:14:32 2014
% For: guillaume
% \usepackage{tikz}
% The following commands are not supported in PSTricks at present
% We define them conditionally, so when they are implemented,
% this pgf file will use them.
\ifx\du\undefined
  \newlength{\du}
\fi
\setlength{\du}{15\unitlength}
\begin{tikzpicture}
\pgftransformxscale{1.000000}
\pgftransformyscale{-1.000000}
\definecolor{dialinecolor}{rgb}{0.000000, 0.000000, 0.000000}
\pgfsetstrokecolor{dialinecolor}
\definecolor{dialinecolor}{rgb}{1.000000, 1.000000, 1.000000}
\pgfsetfillcolor{dialinecolor}
\pgfsetlinewidth{0.100000\du}
\pgfsetdash{}{0pt}
\definecolor{dialinecolor}{rgb}{1.000000, 1.000000, 1.000000}
\pgfsetfillcolor{dialinecolor}
\fill (23.000000\du,5.000000\du)--(23.000000\du,6.400000\du)--(32.250000\du,6.400000\du)--(32.250000\du,5.000000\du)--cycle;
\definecolor{dialinecolor}{rgb}{0.000000, 0.000000, 0.000000}
\pgfsetstrokecolor{dialinecolor}
\draw (23.000000\du,5.000000\du)--(23.000000\du,6.400000\du)--(32.250000\du,6.400000\du)--(32.250000\du,5.000000\du)--cycle;
% setfont left to latex
\definecolor{dialinecolor}{rgb}{0.000000, 0.000000, 0.000000}
\pgfsetstrokecolor{dialinecolor}
\node at (27.625000\du,5.950000\du){ElementDecorateur};
\definecolor{dialinecolor}{rgb}{1.000000, 1.000000, 1.000000}
\pgfsetfillcolor{dialinecolor}
\fill (23.000000\du,6.400000\du)--(23.000000\du,6.800000\du)--(32.250000\du,6.800000\du)--(32.250000\du,6.400000\du)--cycle;
\definecolor{dialinecolor}{rgb}{0.000000, 0.000000, 0.000000}
\pgfsetstrokecolor{dialinecolor}
\draw (23.000000\du,6.400000\du)--(23.000000\du,6.800000\du)--(32.250000\du,6.800000\du)--(32.250000\du,6.400000\du)--cycle;
\definecolor{dialinecolor}{rgb}{1.000000, 1.000000, 1.000000}
\pgfsetfillcolor{dialinecolor}
\fill (23.000000\du,6.800000\du)--(23.000000\du,7.200000\du)--(32.250000\du,7.200000\du)--(32.250000\du,6.800000\du)--cycle;
\definecolor{dialinecolor}{rgb}{0.000000, 0.000000, 0.000000}
\pgfsetstrokecolor{dialinecolor}
\draw (23.000000\du,6.800000\du)--(23.000000\du,7.200000\du)--(32.250000\du,7.200000\du)--(32.250000\du,6.800000\du)--cycle;
\pgfsetlinewidth{0.100000\du}
\pgfsetdash{}{0pt}
\pgfsetmiterjoin
\pgfsetbuttcap
{
\definecolor{dialinecolor}{rgb}{0.000000, 0.000000, 0.000000}
\pgfsetfillcolor{dialinecolor}
% was here!!!
\definecolor{dialinecolor}{rgb}{0.000000, 0.000000, 0.000000}
\pgfsetstrokecolor{dialinecolor}
\draw (22.949715\du,6.100000\du)--(18.953893\du,6.100000\du)--(18.953893\du,-1.218780\du)--(15.658071\du,-1.218780\du);
}
\definecolor{dialinecolor}{rgb}{0.000000, 0.000000, 0.000000}
\pgfsetstrokecolor{dialinecolor}
\draw (21.691136\du,6.100000\du)--(18.953893\du,6.100000\du)--(18.953893\du,-1.218780\du)--(15.658071\du,-1.218780\du);
\pgfsetdash{}{0pt}
\pgfsetmiterjoin
\pgfsetbuttcap
\definecolor{dialinecolor}{rgb}{0.000000, 0.000000, 0.000000}
\pgfsetfillcolor{dialinecolor}
\fill (22.949715\du,6.100000\du)--(22.249715\du,6.340000\du)--(21.549715\du,6.100000\du)--(22.249715\du,5.860000\du)--cycle;
\pgfsetlinewidth{0.100000\du}
\pgfsetdash{}{0pt}
\pgfsetmiterjoin
\pgfsetbuttcap
\definecolor{dialinecolor}{rgb}{0.000000, 0.000000, 0.000000}
\pgfsetstrokecolor{dialinecolor}
\draw (22.949715\du,6.100000\du)--(22.249715\du,6.340000\du)--(21.549715\du,6.100000\du)--(22.249715\du,5.860000\du)--cycle;
% setfont left to latex
\definecolor{dialinecolor}{rgb}{0.000000, 0.000000, 0.000000}
\pgfsetstrokecolor{dialinecolor}
\node[anchor=west] at (19.053893\du,2.290610\du){};
\definecolor{dialinecolor}{rgb}{0.000000, 0.000000, 0.000000}
\pgfsetstrokecolor{dialinecolor}
\node[anchor=east] at (21.349715\du,5.950000\du){};
\definecolor{dialinecolor}{rgb}{0.000000, 0.000000, 0.000000}
\pgfsetstrokecolor{dialinecolor}
\node[anchor=west] at (15.858071\du,-1.368780\du){1};
\pgfsetlinewidth{0.100000\du}
\pgfsetdash{}{0pt}
\definecolor{dialinecolor}{rgb}{1.000000, 1.000000, 1.000000}
\pgfsetfillcolor{dialinecolor}
\fill (18.531200\du,17.397200\du)--(18.531200\du,18.797200\du)--(27.886200\du,18.797200\du)--(27.886200\du,17.397200\du)--cycle;
\definecolor{dialinecolor}{rgb}{0.000000, 0.000000, 0.000000}
\pgfsetstrokecolor{dialinecolor}
\draw (18.531200\du,17.397200\du)--(18.531200\du,18.797200\du)--(27.886200\du,18.797200\du)--(27.886200\du,17.397200\du)--cycle;
% setfont left to latex
\definecolor{dialinecolor}{rgb}{0.000000, 0.000000, 0.000000}
\pgfsetstrokecolor{dialinecolor}
\node at (23.208700\du,18.347200\du){ElementIntelligent};
\definecolor{dialinecolor}{rgb}{1.000000, 1.000000, 1.000000}
\pgfsetfillcolor{dialinecolor}
\fill (18.531200\du,18.797200\du)--(18.531200\du,19.197200\du)--(27.886200\du,19.197200\du)--(27.886200\du,18.797200\du)--cycle;
\definecolor{dialinecolor}{rgb}{0.000000, 0.000000, 0.000000}
\pgfsetstrokecolor{dialinecolor}
\draw (18.531200\du,18.797200\du)--(18.531200\du,19.197200\du)--(27.886200\du,19.197200\du)--(27.886200\du,18.797200\du)--cycle;
\definecolor{dialinecolor}{rgb}{1.000000, 1.000000, 1.000000}
\pgfsetfillcolor{dialinecolor}
\fill (18.531200\du,19.197200\du)--(18.531200\du,20.997200\du)--(27.886200\du,20.997200\du)--(27.886200\du,19.197200\du)--cycle;
\definecolor{dialinecolor}{rgb}{0.000000, 0.000000, 0.000000}
\pgfsetstrokecolor{dialinecolor}
\draw (18.531200\du,19.197200\du)--(18.531200\du,20.997200\du)--(27.886200\du,20.997200\du)--(27.886200\du,19.197200\du)--cycle;
% setfont left to latex
\definecolor{dialinecolor}{rgb}{0.000000, 0.000000, 0.000000}
\pgfsetstrokecolor{dialinecolor}
\node[anchor=west] at (18.681200\du,19.897200\du){-autoEvaluer()};
% setfont left to latex
\definecolor{dialinecolor}{rgb}{0.000000, 0.000000, 0.000000}
\pgfsetstrokecolor{dialinecolor}
\node[anchor=west] at (18.681200\du,20.697200\du){-proposerAmelioration()};
\pgfsetlinewidth{0.100000\du}
\pgfsetdash{}{0pt}
\definecolor{dialinecolor}{rgb}{1.000000, 1.000000, 1.000000}
\pgfsetfillcolor{dialinecolor}
\fill (28.549900\du,17.397200\du)--(28.549900\du,18.797200\du)--(36.212400\du,18.797200\du)--(36.212400\du,17.397200\du)--cycle;
\definecolor{dialinecolor}{rgb}{0.000000, 0.000000, 0.000000}
\pgfsetstrokecolor{dialinecolor}
\draw (28.549900\du,17.397200\du)--(28.549900\du,18.797200\du)--(36.212400\du,18.797200\du)--(36.212400\du,17.397200\du)--cycle;
% setfont left to latex
\definecolor{dialinecolor}{rgb}{0.000000, 0.000000, 0.000000}
\pgfsetstrokecolor{dialinecolor}
\node at (32.381150\du,18.347200\du){ElementEvolutif};
\definecolor{dialinecolor}{rgb}{1.000000, 1.000000, 1.000000}
\pgfsetfillcolor{dialinecolor}
\fill (28.549900\du,18.797200\du)--(28.549900\du,19.197200\du)--(36.212400\du,19.197200\du)--(36.212400\du,18.797200\du)--cycle;
\definecolor{dialinecolor}{rgb}{0.000000, 0.000000, 0.000000}
\pgfsetstrokecolor{dialinecolor}
\draw (28.549900\du,18.797200\du)--(28.549900\du,19.197200\du)--(36.212400\du,19.197200\du)--(36.212400\du,18.797200\du)--cycle;
\definecolor{dialinecolor}{rgb}{1.000000, 1.000000, 1.000000}
\pgfsetfillcolor{dialinecolor}
\fill (28.549900\du,19.197200\du)--(28.549900\du,20.197200\du)--(36.212400\du,20.197200\du)--(36.212400\du,19.197200\du)--cycle;
\definecolor{dialinecolor}{rgb}{0.000000, 0.000000, 0.000000}
\pgfsetstrokecolor{dialinecolor}
\draw (28.549900\du,19.197200\du)--(28.549900\du,20.197200\du)--(36.212400\du,20.197200\du)--(36.212400\du,19.197200\du)--cycle;
% setfont left to latex
\definecolor{dialinecolor}{rgb}{0.000000, 0.000000, 0.000000}
\pgfsetstrokecolor{dialinecolor}
\node[anchor=west] at (28.699900\du,19.897200\du){-évoluer()};
\pgfsetlinewidth{0.100000\du}
\pgfsetdash{}{0pt}
\pgfsetmiterjoin
\pgfsetbuttcap
{
\definecolor{dialinecolor}{rgb}{0.000000, 0.000000, 0.000000}
\pgfsetfillcolor{dialinecolor}
% was here!!!
\definecolor{dialinecolor}{rgb}{0.000000, 0.000000, 0.000000}
\pgfsetstrokecolor{dialinecolor}
\draw (27.625000\du,7.250281\du)--(27.625000\du,12.723740\du)--(23.208700\du,12.723740\du)--(23.208700\du,17.397200\du);
}
\definecolor{dialinecolor}{rgb}{0.000000, 0.000000, 0.000000}
\pgfsetstrokecolor{dialinecolor}
\draw (27.625000\du,8.162084\du)--(27.625000\du,12.723740\du)--(23.208700\du,12.723740\du)--(23.208700\du,17.397200\du);
\pgfsetmiterjoin
\definecolor{dialinecolor}{rgb}{1.000000, 1.000000, 1.000000}
\pgfsetfillcolor{dialinecolor}
\fill (28.025000\du,8.162084\du)--(27.625000\du,7.362084\du)--(27.225000\du,8.162084\du)--cycle;
\pgfsetlinewidth{0.100000\du}
\pgfsetdash{}{0pt}
\pgfsetmiterjoin
\definecolor{dialinecolor}{rgb}{0.000000, 0.000000, 0.000000}
\pgfsetstrokecolor{dialinecolor}
\draw (28.025000\du,8.162084\du)--(27.625000\du,7.362084\du)--(27.225000\du,8.162084\du)--cycle;
% setfont left to latex
\pgfsetlinewidth{0.100000\du}
\pgfsetdash{}{0pt}
\pgfsetmiterjoin
\pgfsetbuttcap
{
\definecolor{dialinecolor}{rgb}{0.000000, 0.000000, 0.000000}
\pgfsetfillcolor{dialinecolor}
% was here!!!
\definecolor{dialinecolor}{rgb}{0.000000, 0.000000, 0.000000}
\pgfsetstrokecolor{dialinecolor}
\draw (27.625000\du,7.250281\du)--(27.625000\du,12.698563\du)--(32.381150\du,12.698563\du)--(32.381150\du,17.346846\du);
}
\definecolor{dialinecolor}{rgb}{0.000000, 0.000000, 0.000000}
\pgfsetstrokecolor{dialinecolor}
\draw (27.625000\du,8.162084\du)--(27.625000\du,12.698563\du)--(32.381150\du,12.698563\du)--(32.381150\du,17.346846\du);
\pgfsetmiterjoin
\definecolor{dialinecolor}{rgb}{1.000000, 1.000000, 1.000000}
\pgfsetfillcolor{dialinecolor}
\fill (28.025000\du,8.162084\du)--(27.625000\du,7.362084\du)--(27.225000\du,8.162084\du)--cycle;
\pgfsetlinewidth{0.100000\du}
\pgfsetdash{}{0pt}
\pgfsetmiterjoin
\definecolor{dialinecolor}{rgb}{0.000000, 0.000000, 0.000000}
\pgfsetstrokecolor{dialinecolor}
\draw (28.025000\du,8.162084\du)--(27.625000\du,7.362084\du)--(27.225000\du,8.162084\du)--cycle;
% setfont left to latex
\pgfsetlinewidth{0.100000\du}
\pgfsetdash{}{0pt}
\pgfsetmiterjoin
\pgfsetbuttcap
{
\definecolor{dialinecolor}{rgb}{0.000000, 0.000000, 0.000000}
\pgfsetfillcolor{dialinecolor}
% was here!!!
\definecolor{dialinecolor}{rgb}{0.000000, 0.000000, 0.000000}
\pgfsetstrokecolor{dialinecolor}
\draw (15.607700\du,-3.818780\du)--(24.892100\du,-3.818780\du)--(24.892100\du,5.000000\du)--(27.142500\du,5.000000\du);
}
\definecolor{dialinecolor}{rgb}{0.000000, 0.000000, 0.000000}
\pgfsetstrokecolor{dialinecolor}
\draw (16.519503\du,-3.818780\du)--(24.892100\du,-3.818780\du)--(24.892100\du,5.000000\du)--(27.142500\du,5.000000\du);
\pgfsetmiterjoin
\definecolor{dialinecolor}{rgb}{1.000000, 1.000000, 1.000000}
\pgfsetfillcolor{dialinecolor}
\fill (16.519503\du,-4.218780\du)--(15.719503\du,-3.818780\du)--(16.519503\du,-3.418780\du)--cycle;
\pgfsetlinewidth{0.100000\du}
\pgfsetdash{}{0pt}
\pgfsetmiterjoin
\definecolor{dialinecolor}{rgb}{0.000000, 0.000000, 0.000000}
\pgfsetstrokecolor{dialinecolor}
\draw (16.519503\du,-4.218780\du)--(15.719503\du,-3.818780\du)--(16.519503\du,-3.418780\du)--cycle;
% setfont left to latex
\pgfsetlinewidth{0.100000\du}
\pgfsetdash{}{0pt}
\definecolor{dialinecolor}{rgb}{1.000000, 1.000000, 1.000000}
\pgfsetfillcolor{dialinecolor}
\fill (3.557700\du,-5.718780\du)--(3.557700\du,-4.318780\du)--(15.607700\du,-4.318780\du)--(15.607700\du,-5.718780\du)--cycle;
\definecolor{dialinecolor}{rgb}{0.000000, 0.000000, 0.000000}
\pgfsetstrokecolor{dialinecolor}
\draw (3.557700\du,-5.718780\du)--(3.557700\du,-4.318780\du)--(15.607700\du,-4.318780\du)--(15.607700\du,-5.718780\du)--cycle;
% setfont left to latex
\definecolor{dialinecolor}{rgb}{0.000000, 0.000000, 0.000000}
\pgfsetstrokecolor{dialinecolor}
\node at (9.582700\du,-4.768780\du){Element};
\definecolor{dialinecolor}{rgb}{1.000000, 1.000000, 1.000000}
\pgfsetfillcolor{dialinecolor}
\fill (3.557700\du,-4.318780\du)--(3.557700\du,-0.118780\du)--(15.607700\du,-0.118780\du)--(15.607700\du,-4.318780\du)--cycle;
\definecolor{dialinecolor}{rgb}{0.000000, 0.000000, 0.000000}
\pgfsetstrokecolor{dialinecolor}
\draw (3.557700\du,-4.318780\du)--(3.557700\du,-0.118780\du)--(15.607700\du,-0.118780\du)--(15.607700\du,-4.318780\du)--cycle;
% setfont left to latex
\definecolor{dialinecolor}{rgb}{0.000000, 0.000000, 0.000000}
\pgfsetstrokecolor{dialinecolor}
\node[anchor=west] at (3.707700\du,-3.618780\du){\#Nom};
% setfont left to latex
\definecolor{dialinecolor}{rgb}{0.000000, 0.000000, 0.000000}
\pgfsetstrokecolor{dialinecolor}
\node[anchor=west] at (3.707700\du,-2.818780\du){\#Date de création};
% setfont left to latex
\definecolor{dialinecolor}{rgb}{0.000000, 0.000000, 0.000000}
\pgfsetstrokecolor{dialinecolor}
\node[anchor=west] at (3.707700\du,-2.018780\du){\#Date de dernière modification};
% setfont left to latex
\definecolor{dialinecolor}{rgb}{0.000000, 0.000000, 0.000000}
\pgfsetstrokecolor{dialinecolor}
\node[anchor=west] at (3.707700\du,-1.218780\du){\#Chemin};
% setfont left to latex
\definecolor{dialinecolor}{rgb}{0.000000, 0.000000, 0.000000}
\pgfsetstrokecolor{dialinecolor}
\node[anchor=west] at (3.707700\du,-0.418780\du){\#Ouvert: Boolean = False};
\definecolor{dialinecolor}{rgb}{1.000000, 1.000000, 1.000000}
\pgfsetfillcolor{dialinecolor}
\fill (3.557700\du,-0.118780\du)--(3.557700\du,3.281220\du)--(15.607700\du,3.281220\du)--(15.607700\du,-0.118780\du)--cycle;
\definecolor{dialinecolor}{rgb}{0.000000, 0.000000, 0.000000}
\pgfsetstrokecolor{dialinecolor}
\draw (3.557700\du,-0.118780\du)--(3.557700\du,3.281220\du)--(15.607700\du,3.281220\du)--(15.607700\du,-0.118780\du)--cycle;
% setfont left to latex
\definecolor{dialinecolor}{rgb}{0.000000, 0.000000, 0.000000}
\pgfsetstrokecolor{dialinecolor}
\node[anchor=west] at (3.707700\du,0.581220\du){+taille()};
% setfont left to latex
\definecolor{dialinecolor}{rgb}{0.000000, 0.000000, 0.000000}
\pgfsetstrokecolor{dialinecolor}
\node[anchor=west] at (3.707700\du,1.381220\du){+open()};
% setfont left to latex
\definecolor{dialinecolor}{rgb}{0.000000, 0.000000, 0.000000}
\pgfsetstrokecolor{dialinecolor}
\node[anchor=west] at (3.707700\du,2.181220\du){+close()};
% setfont left to latex
\definecolor{dialinecolor}{rgb}{0.000000, 0.000000, 0.000000}
\pgfsetstrokecolor{dialinecolor}
\node[anchor=west] at (3.707700\du,2.981220\du){+delete()};
\pgfsetlinewidth{0.100000\du}
\pgfsetdash{}{0pt}
\definecolor{dialinecolor}{rgb}{1.000000, 1.000000, 1.000000}
\pgfsetfillcolor{dialinecolor}
\fill (12.745400\du,14.848800\du)--(12.745400\du,16.248800\du)--(16.325400\du,16.248800\du)--(16.325400\du,14.848800\du)--cycle;
\definecolor{dialinecolor}{rgb}{0.000000, 0.000000, 0.000000}
\pgfsetstrokecolor{dialinecolor}
\draw (12.745400\du,14.848800\du)--(12.745400\du,16.248800\du)--(16.325400\du,16.248800\du)--(16.325400\du,14.848800\du)--cycle;
% setfont left to latex
\definecolor{dialinecolor}{rgb}{0.000000, 0.000000, 0.000000}
\pgfsetstrokecolor{dialinecolor}
\node at (14.535400\du,15.798800\du){Fichier};
\definecolor{dialinecolor}{rgb}{1.000000, 1.000000, 1.000000}
\pgfsetfillcolor{dialinecolor}
\fill (12.745400\du,16.248800\du)--(12.745400\du,16.648800\du)--(16.325400\du,16.648800\du)--(16.325400\du,16.248800\du)--cycle;
\definecolor{dialinecolor}{rgb}{0.000000, 0.000000, 0.000000}
\pgfsetstrokecolor{dialinecolor}
\draw (12.745400\du,16.248800\du)--(12.745400\du,16.648800\du)--(16.325400\du,16.648800\du)--(16.325400\du,16.248800\du)--cycle;
\definecolor{dialinecolor}{rgb}{1.000000, 1.000000, 1.000000}
\pgfsetfillcolor{dialinecolor}
\fill (12.745400\du,16.648800\du)--(12.745400\du,17.048800\du)--(16.325400\du,17.048800\du)--(16.325400\du,16.648800\du)--cycle;
\definecolor{dialinecolor}{rgb}{0.000000, 0.000000, 0.000000}
\pgfsetstrokecolor{dialinecolor}
\draw (12.745400\du,16.648800\du)--(12.745400\du,17.048800\du)--(16.325400\du,17.048800\du)--(16.325400\du,16.648800\du)--cycle;
\pgfsetlinewidth{0.100000\du}
\pgfsetdash{}{0pt}
\definecolor{dialinecolor}{rgb}{1.000000, 1.000000, 1.000000}
\pgfsetfillcolor{dialinecolor}
\fill (-7.000000\du,13.000000\du)--(-7.000000\du,14.400000\du)--(6.590000\du,14.400000\du)--(6.590000\du,13.000000\du)--cycle;
\definecolor{dialinecolor}{rgb}{0.000000, 0.000000, 0.000000}
\pgfsetstrokecolor{dialinecolor}
\draw (-7.000000\du,13.000000\du)--(-7.000000\du,14.400000\du)--(6.590000\du,14.400000\du)--(6.590000\du,13.000000\du)--cycle;
% setfont left to latex
\definecolor{dialinecolor}{rgb}{0.000000, 0.000000, 0.000000}
\pgfsetstrokecolor{dialinecolor}
\node at (-0.205000\du,13.950000\du){Dossier};
\definecolor{dialinecolor}{rgb}{1.000000, 1.000000, 1.000000}
\pgfsetfillcolor{dialinecolor}
\fill (-7.000000\du,14.400000\du)--(-7.000000\du,14.800000\du)--(6.590000\du,14.800000\du)--(6.590000\du,14.400000\du)--cycle;
\definecolor{dialinecolor}{rgb}{0.000000, 0.000000, 0.000000}
\pgfsetstrokecolor{dialinecolor}
\draw (-7.000000\du,14.400000\du)--(-7.000000\du,14.800000\du)--(6.590000\du,14.800000\du)--(6.590000\du,14.400000\du)--cycle;
\definecolor{dialinecolor}{rgb}{1.000000, 1.000000, 1.000000}
\pgfsetfillcolor{dialinecolor}
\fill (-7.000000\du,14.800000\du)--(-7.000000\du,17.400000\du)--(6.590000\du,17.400000\du)--(6.590000\du,14.800000\du)--cycle;
\definecolor{dialinecolor}{rgb}{0.000000, 0.000000, 0.000000}
\pgfsetstrokecolor{dialinecolor}
\draw (-7.000000\du,14.800000\du)--(-7.000000\du,17.400000\du)--(6.590000\du,17.400000\du)--(6.590000\du,14.800000\du)--cycle;
% setfont left to latex
\definecolor{dialinecolor}{rgb}{0.000000, 0.000000, 0.000000}
\pgfsetstrokecolor{dialinecolor}
\node[anchor=west] at (-6.850000\du,15.500000\du){+activate()};
% setfont left to latex
\definecolor{dialinecolor}{rgb}{0.000000, 0.000000, 0.000000}
\pgfsetstrokecolor{dialinecolor}
\node[anchor=west] at (-6.850000\du,16.300000\du){+attach(observer:ActivateObserver)};
% setfont left to latex
\definecolor{dialinecolor}{rgb}{0.000000, 0.000000, 0.000000}
\pgfsetstrokecolor{dialinecolor}
\node[anchor=west] at (-6.850000\du,17.100000\du){+detach(observer:ActivateObserver)};
\pgfsetlinewidth{0.100000\du}
\pgfsetdash{}{0pt}
\definecolor{dialinecolor}{rgb}{1.000000, 1.000000, 1.000000}
\pgfsetfillcolor{dialinecolor}
\fill (-44.000000\du,25.000000\du)--(-44.000000\du,26.400000\du)--(-38.110000\du,26.400000\du)--(-38.110000\du,25.000000\du)--cycle;
\definecolor{dialinecolor}{rgb}{0.000000, 0.000000, 0.000000}
\pgfsetstrokecolor{dialinecolor}
\draw (-44.000000\du,25.000000\du)--(-44.000000\du,26.400000\du)--(-38.110000\du,26.400000\du)--(-38.110000\du,25.000000\du)--cycle;
% setfont left to latex
\definecolor{dialinecolor}{rgb}{0.000000, 0.000000, 0.000000}
\pgfsetstrokecolor{dialinecolor}
\node at (-41.055000\du,25.950000\du){Navigateur};
\definecolor{dialinecolor}{rgb}{1.000000, 1.000000, 1.000000}
\pgfsetfillcolor{dialinecolor}
\fill (-44.000000\du,26.400000\du)--(-44.000000\du,28.200000\du)--(-38.110000\du,28.200000\du)--(-38.110000\du,26.400000\du)--cycle;
\definecolor{dialinecolor}{rgb}{0.000000, 0.000000, 0.000000}
\pgfsetstrokecolor{dialinecolor}
\draw (-44.000000\du,26.400000\du)--(-44.000000\du,28.200000\du)--(-38.110000\du,28.200000\du)--(-38.110000\du,26.400000\du)--cycle;
% setfont left to latex
\definecolor{dialinecolor}{rgb}{0.000000, 0.000000, 0.000000}
\pgfsetstrokecolor{dialinecolor}
\node[anchor=west] at (-43.850000\du,27.100000\du){-instance};
% setfont left to latex
\definecolor{dialinecolor}{rgb}{0.000000, 0.000000, 0.000000}
\pgfsetstrokecolor{dialinecolor}
\node[anchor=west] at (-43.850000\du,27.900000\du){-dossierActif};
\definecolor{dialinecolor}{rgb}{1.000000, 1.000000, 1.000000}
\pgfsetfillcolor{dialinecolor}
\fill (-44.000000\du,28.200000\du)--(-44.000000\du,30.000000\du)--(-38.110000\du,30.000000\du)--(-38.110000\du,28.200000\du)--cycle;
\definecolor{dialinecolor}{rgb}{0.000000, 0.000000, 0.000000}
\pgfsetstrokecolor{dialinecolor}
\draw (-44.000000\du,28.200000\du)--(-44.000000\du,30.000000\du)--(-38.110000\du,30.000000\du)--(-38.110000\du,28.200000\du)--cycle;
% setfont left to latex
\definecolor{dialinecolor}{rgb}{0.000000, 0.000000, 0.000000}
\pgfsetstrokecolor{dialinecolor}
\node[anchor=west] at (-43.850000\du,28.900000\du){+getInstance()};
% setfont left to latex
\definecolor{dialinecolor}{rgb}{0.000000, 0.000000, 0.000000}
\pgfsetstrokecolor{dialinecolor}
\node[anchor=west] at (-43.850000\du,29.700000\du){-Navigateur()};
\pgfsetlinewidth{0.100000\du}
\pgfsetdash{}{0pt}
\pgfsetmiterjoin
\pgfsetbuttcap
{
\definecolor{dialinecolor}{rgb}{0.000000, 0.000000, 0.000000}
\pgfsetfillcolor{dialinecolor}
% was here!!!
\definecolor{dialinecolor}{rgb}{0.000000, 0.000000, 0.000000}
\pgfsetstrokecolor{dialinecolor}
\draw (9.582700\du,3.281220\du)--(9.582700\du,10.000000\du)--(4.076380\du,10.000000\du)--(4.076380\du,12.809900\du);
}
\definecolor{dialinecolor}{rgb}{0.000000, 0.000000, 0.000000}
\pgfsetstrokecolor{dialinecolor}
\draw (9.582700\du,4.193023\du)--(9.582700\du,10.000000\du)--(4.076380\du,10.000000\du)--(4.076380\du,12.809900\du);
\pgfsetmiterjoin
\definecolor{dialinecolor}{rgb}{1.000000, 1.000000, 1.000000}
\pgfsetfillcolor{dialinecolor}
\fill (9.982700\du,4.193023\du)--(9.582700\du,3.393023\du)--(9.182700\du,4.193023\du)--cycle;
\pgfsetlinewidth{0.100000\du}
\pgfsetdash{}{0pt}
\pgfsetmiterjoin
\definecolor{dialinecolor}{rgb}{0.000000, 0.000000, 0.000000}
\pgfsetstrokecolor{dialinecolor}
\draw (9.982700\du,4.193023\du)--(9.582700\du,3.393023\du)--(9.182700\du,4.193023\du)--cycle;
% setfont left to latex
\pgfsetlinewidth{0.100000\du}
\pgfsetdash{}{0pt}
\pgfsetmiterjoin
\pgfsetbuttcap
{
\definecolor{dialinecolor}{rgb}{0.000000, 0.000000, 0.000000}
\pgfsetfillcolor{dialinecolor}
% was here!!!
\definecolor{dialinecolor}{rgb}{0.000000, 0.000000, 0.000000}
\pgfsetstrokecolor{dialinecolor}
\draw (9.582700\du,3.281220\du)--(9.582700\du,10.000000\du)--(14.499559\du,10.000000\du)--(14.528505\du,14.804353\du);
}
\definecolor{dialinecolor}{rgb}{0.000000, 0.000000, 0.000000}
\pgfsetstrokecolor{dialinecolor}
\draw (9.582700\du,4.193023\du)--(9.582700\du,10.000000\du)--(14.499559\du,10.000000\du)--(14.528505\du,14.804353\du);
\pgfsetmiterjoin
\definecolor{dialinecolor}{rgb}{1.000000, 1.000000, 1.000000}
\pgfsetfillcolor{dialinecolor}
\fill (9.982700\du,4.193023\du)--(9.582700\du,3.393023\du)--(9.182700\du,4.193023\du)--cycle;
\pgfsetlinewidth{0.100000\du}
\pgfsetdash{}{0pt}
\pgfsetmiterjoin
\definecolor{dialinecolor}{rgb}{0.000000, 0.000000, 0.000000}
\pgfsetstrokecolor{dialinecolor}
\draw (9.982700\du,4.193023\du)--(9.582700\du,3.393023\du)--(9.182700\du,4.193023\du)--cycle;
% setfont left to latex
\pgfsetlinewidth{0.100000\du}
\pgfsetdash{}{0pt}
\definecolor{dialinecolor}{rgb}{1.000000, 1.000000, 1.000000}
\pgfsetfillcolor{dialinecolor}
\fill (-54.000000\du,-8.000000\du)--(-54.000000\du,-6.600000\du)--(-40.795000\du,-6.600000\du)--(-40.795000\du,-8.000000\du)--cycle;
\definecolor{dialinecolor}{rgb}{0.000000, 0.000000, 0.000000}
\pgfsetstrokecolor{dialinecolor}
\draw (-54.000000\du,-8.000000\du)--(-54.000000\du,-6.600000\du)--(-40.795000\du,-6.600000\du)--(-40.795000\du,-8.000000\du)--cycle;
% setfont left to latex
\definecolor{dialinecolor}{rgb}{0.000000, 0.000000, 0.000000}
\pgfsetstrokecolor{dialinecolor}
\node at (-47.397500\du,-7.050000\du){Observable};
\definecolor{dialinecolor}{rgb}{1.000000, 1.000000, 1.000000}
\pgfsetfillcolor{dialinecolor}
\fill (-54.000000\du,-6.600000\du)--(-54.000000\du,-6.200000\du)--(-40.795000\du,-6.200000\du)--(-40.795000\du,-6.600000\du)--cycle;
\definecolor{dialinecolor}{rgb}{0.000000, 0.000000, 0.000000}
\pgfsetstrokecolor{dialinecolor}
\draw (-54.000000\du,-6.600000\du)--(-54.000000\du,-6.200000\du)--(-40.795000\du,-6.200000\du)--(-40.795000\du,-6.600000\du)--cycle;
\definecolor{dialinecolor}{rgb}{1.000000, 1.000000, 1.000000}
\pgfsetfillcolor{dialinecolor}
\fill (-54.000000\du,-6.200000\du)--(-54.000000\du,3.600000\du)--(-40.795000\du,3.600000\du)--(-40.795000\du,-6.200000\du)--cycle;
\definecolor{dialinecolor}{rgb}{0.000000, 0.000000, 0.000000}
\pgfsetstrokecolor{dialinecolor}
\draw (-54.000000\du,-6.200000\du)--(-54.000000\du,3.600000\du)--(-40.795000\du,3.600000\du)--(-40.795000\du,-6.200000\du)--cycle;
% setfont left to latex
\definecolor{dialinecolor}{rgb}{0.000000, 0.000000, 0.000000}
\pgfsetstrokecolor{dialinecolor}
\node[anchor=west] at (-53.850000\du,-5.500000\du){+attach(observer:ChangeObserver)};
% setfont left to latex
\definecolor{dialinecolor}{rgb}{0.000000, 0.000000, 0.000000}
\pgfsetstrokecolor{dialinecolor}
\node[anchor=west] at (-53.850000\du,-4.700000\du){+attach(observer:DeleteObserver)};
% setfont left to latex
\definecolor{dialinecolor}{rgb}{0.000000, 0.000000, 0.000000}
\pgfsetstrokecolor{dialinecolor}
\node[anchor=west] at (-53.850000\du,-3.900000\du){+attach(observer:OpenObserver)};
% setfont left to latex
\definecolor{dialinecolor}{rgb}{0.000000, 0.000000, 0.000000}
\pgfsetstrokecolor{dialinecolor}
\node[anchor=west] at (-53.850000\du,-3.100000\du){+attach(observer:CloseObserver)};
% setfont left to latex
\definecolor{dialinecolor}{rgb}{0.000000, 0.000000, 0.000000}
\pgfsetstrokecolor{dialinecolor}
\node[anchor=west] at (-53.850000\du,-2.300000\du){+dettach(observer:ChangeObserve)};
% setfont left to latex
\definecolor{dialinecolor}{rgb}{0.000000, 0.000000, 0.000000}
\pgfsetstrokecolor{dialinecolor}
\node[anchor=west] at (-53.850000\du,-1.500000\du){+dettach(observer:DeleteObserver)};
% setfont left to latex
\definecolor{dialinecolor}{rgb}{0.000000, 0.000000, 0.000000}
\pgfsetstrokecolor{dialinecolor}
\node[anchor=west] at (-53.850000\du,-0.700000\du){+dettach(observer:OpenObserver)};
% setfont left to latex
\definecolor{dialinecolor}{rgb}{0.000000, 0.000000, 0.000000}
\pgfsetstrokecolor{dialinecolor}
\node[anchor=west] at (-53.850000\du,0.100000\du){+dettach(observer:CloseObserver)};
% setfont left to latex
\definecolor{dialinecolor}{rgb}{0.000000, 0.000000, 0.000000}
\pgfsetstrokecolor{dialinecolor}
\node[anchor=west] at (-53.850000\du,0.900000\du){+notifyChange()};
% setfont left to latex
\definecolor{dialinecolor}{rgb}{0.000000, 0.000000, 0.000000}
\pgfsetstrokecolor{dialinecolor}
\node[anchor=west] at (-53.850000\du,1.700000\du){+notifyDelete()};
% setfont left to latex
\definecolor{dialinecolor}{rgb}{0.000000, 0.000000, 0.000000}
\pgfsetstrokecolor{dialinecolor}
\node[anchor=west] at (-53.850000\du,2.500000\du){+notifyOpen()};
% setfont left to latex
\definecolor{dialinecolor}{rgb}{0.000000, 0.000000, 0.000000}
\pgfsetstrokecolor{dialinecolor}
\node[anchor=west] at (-53.850000\du,3.300000\du){+notifyClose()};
\pgfsetlinewidth{0.100000\du}
\pgfsetdash{}{0pt}
\pgfsetmiterjoin
\pgfsetbuttcap
{
\definecolor{dialinecolor}{rgb}{0.000000, 0.000000, 0.000000}
\pgfsetfillcolor{dialinecolor}
% was here!!!
\definecolor{dialinecolor}{rgb}{0.000000, 0.000000, 0.000000}
\pgfsetstrokecolor{dialinecolor}
\draw (1.003770\du,12.886700\du)--(1.003770\du,12.886700\du)--(1.003770\du,-1.218780\du)--(3.507441\du,-1.218780\du);
}
\definecolor{dialinecolor}{rgb}{0.000000, 0.000000, 0.000000}
\pgfsetstrokecolor{dialinecolor}
\draw (1.003770\du,11.628121\du)--(1.003770\du,-1.218780\du)--(3.507441\du,-1.218780\du);
\pgfsetdash{}{0pt}
\pgfsetmiterjoin
\pgfsetbuttcap
\definecolor{dialinecolor}{rgb}{0.000000, 0.000000, 0.000000}
\pgfsetfillcolor{dialinecolor}
\fill (1.003770\du,12.886700\du)--(0.763770\du,12.186700\du)--(1.003770\du,11.486700\du)--(1.243770\du,12.186700\du)--cycle;
\pgfsetlinewidth{0.100000\du}
\pgfsetdash{}{0pt}
\pgfsetmiterjoin
\pgfsetbuttcap
\definecolor{dialinecolor}{rgb}{0.000000, 0.000000, 0.000000}
\pgfsetstrokecolor{dialinecolor}
\draw (1.003770\du,12.886700\du)--(0.763770\du,12.186700\du)--(1.003770\du,11.486700\du)--(1.243770\du,12.186700\du)--cycle;
% setfont left to latex
\definecolor{dialinecolor}{rgb}{0.000000, 0.000000, 0.000000}
\pgfsetstrokecolor{dialinecolor}
\node[anchor=west] at (1.103770\du,5.683960\du){};
\definecolor{dialinecolor}{rgb}{0.000000, 0.000000, 0.000000}
\pgfsetstrokecolor{dialinecolor}
\node[anchor=west] at (1.553770\du,12.686700\du){};
\definecolor{dialinecolor}{rgb}{0.000000, 0.000000, 0.000000}
\pgfsetstrokecolor{dialinecolor}
\node[anchor=east] at (3.307441\du,-1.368780\du){ possède};
\definecolor{dialinecolor}{rgb}{0.000000, 0.000000, 0.000000}
\pgfsetstrokecolor{dialinecolor}
\node[anchor=east] at (3.307441\du,-0.568780\du){*};
\pgfsetlinewidth{0.100000\du}
\pgfsetdash{}{0pt}
\pgfsetmiterjoin
\pgfsetbuttcap
{
\definecolor{dialinecolor}{rgb}{0.000000, 0.000000, 0.000000}
\pgfsetfillcolor{dialinecolor}
% was here!!!
\pgfsetarrowsend{to}
\definecolor{dialinecolor}{rgb}{0.000000, 0.000000, 0.000000}
\pgfsetstrokecolor{dialinecolor}
\draw (-38.062358\du,27.500000\du)--(-1.029330\du,27.500000\du)--(-1.029330\du,17.541700\du)--(-1.029330\du,17.541700\du);
}
% setfont left to latex
\definecolor{dialinecolor}{rgb}{0.000000, 0.000000, 0.000000}
\pgfsetstrokecolor{dialinecolor}
\node[anchor=west] at (-0.929330\du,22.370850\du){};
\definecolor{dialinecolor}{rgb}{0.000000, 0.000000, 0.000000}
\pgfsetstrokecolor{dialinecolor}
\node[anchor=west] at (-37.862358\du,27.350000\du){};
\definecolor{dialinecolor}{rgb}{0.000000, 0.000000, 0.000000}
\pgfsetstrokecolor{dialinecolor}
\node[anchor=west] at (-0.479330\du,18.141700\du){ active};
\definecolor{dialinecolor}{rgb}{0.000000, 0.000000, 0.000000}
\pgfsetstrokecolor{dialinecolor}
\node[anchor=west] at (-0.479330\du,18.941700\du){0..1};
\pgfsetlinewidth{0.100000\du}
\pgfsetdash{}{0pt}
\definecolor{dialinecolor}{rgb}{1.000000, 1.000000, 1.000000}
\pgfsetfillcolor{dialinecolor}
\fill (-25.000000\du,13.000000\du)--(-25.000000\du,14.400000\du)--(-14.490000\du,14.400000\du)--(-14.490000\du,13.000000\du)--cycle;
\definecolor{dialinecolor}{rgb}{0.000000, 0.000000, 0.000000}
\pgfsetstrokecolor{dialinecolor}
\draw (-25.000000\du,13.000000\du)--(-25.000000\du,14.400000\du)--(-14.490000\du,14.400000\du)--(-14.490000\du,13.000000\du)--cycle;
% setfont left to latex
\definecolor{dialinecolor}{rgb}{0.000000, 0.000000, 0.000000}
\pgfsetstrokecolor{dialinecolor}
\node at (-19.745000\du,13.950000\du){ActivateObserver};
\definecolor{dialinecolor}{rgb}{1.000000, 1.000000, 1.000000}
\pgfsetfillcolor{dialinecolor}
\fill (-25.000000\du,14.400000\du)--(-25.000000\du,14.800000\du)--(-14.490000\du,14.800000\du)--(-14.490000\du,14.400000\du)--cycle;
\definecolor{dialinecolor}{rgb}{0.000000, 0.000000, 0.000000}
\pgfsetstrokecolor{dialinecolor}
\draw (-25.000000\du,14.400000\du)--(-25.000000\du,14.800000\du)--(-14.490000\du,14.800000\du)--(-14.490000\du,14.400000\du)--cycle;
\definecolor{dialinecolor}{rgb}{1.000000, 1.000000, 1.000000}
\pgfsetfillcolor{dialinecolor}
\fill (-25.000000\du,14.800000\du)--(-25.000000\du,15.800000\du)--(-14.490000\du,15.800000\du)--(-14.490000\du,14.800000\du)--cycle;
\definecolor{dialinecolor}{rgb}{0.000000, 0.000000, 0.000000}
\pgfsetstrokecolor{dialinecolor}
\draw (-25.000000\du,14.800000\du)--(-25.000000\du,15.800000\du)--(-14.490000\du,15.800000\du)--(-14.490000\du,14.800000\du)--cycle;
% setfont left to latex
\definecolor{dialinecolor}{rgb}{0.000000, 0.000000, 0.000000}
\pgfsetstrokecolor{dialinecolor}
\node[anchor=west] at (-24.850000\du,15.500000\du){+updateActivate(d:Dossier)};
\pgfsetlinewidth{0.100000\du}
\pgfsetdash{}{0pt}
\definecolor{dialinecolor}{rgb}{1.000000, 1.000000, 1.000000}
\pgfsetfillcolor{dialinecolor}
\fill (-36.000000\du,13.000000\du)--(-36.000000\du,14.400000\du)--(-26.260000\du,14.400000\du)--(-26.260000\du,13.000000\du)--cycle;
\definecolor{dialinecolor}{rgb}{0.000000, 0.000000, 0.000000}
\pgfsetstrokecolor{dialinecolor}
\draw (-36.000000\du,13.000000\du)--(-36.000000\du,14.400000\du)--(-26.260000\du,14.400000\du)--(-26.260000\du,13.000000\du)--cycle;
% setfont left to latex
\definecolor{dialinecolor}{rgb}{0.000000, 0.000000, 0.000000}
\pgfsetstrokecolor{dialinecolor}
\node at (-31.130000\du,13.950000\du){DeleteObserver};
\definecolor{dialinecolor}{rgb}{1.000000, 1.000000, 1.000000}
\pgfsetfillcolor{dialinecolor}
\fill (-36.000000\du,14.400000\du)--(-36.000000\du,14.800000\du)--(-26.260000\du,14.800000\du)--(-26.260000\du,14.400000\du)--cycle;
\definecolor{dialinecolor}{rgb}{0.000000, 0.000000, 0.000000}
\pgfsetstrokecolor{dialinecolor}
\draw (-36.000000\du,14.400000\du)--(-36.000000\du,14.800000\du)--(-26.260000\du,14.800000\du)--(-26.260000\du,14.400000\du)--cycle;
\definecolor{dialinecolor}{rgb}{1.000000, 1.000000, 1.000000}
\pgfsetfillcolor{dialinecolor}
\fill (-36.000000\du,14.800000\du)--(-36.000000\du,15.800000\du)--(-26.260000\du,15.800000\du)--(-26.260000\du,14.800000\du)--cycle;
\definecolor{dialinecolor}{rgb}{0.000000, 0.000000, 0.000000}
\pgfsetstrokecolor{dialinecolor}
\draw (-36.000000\du,14.800000\du)--(-36.000000\du,15.800000\du)--(-26.260000\du,15.800000\du)--(-26.260000\du,14.800000\du)--cycle;
% setfont left to latex
\definecolor{dialinecolor}{rgb}{0.000000, 0.000000, 0.000000}
\pgfsetstrokecolor{dialinecolor}
\node[anchor=west] at (-35.850000\du,15.500000\du){+updateDelete(e:Element)};
\pgfsetlinewidth{0.100000\du}
\pgfsetdash{}{0pt}
\definecolor{dialinecolor}{rgb}{1.000000, 1.000000, 1.000000}
\pgfsetfillcolor{dialinecolor}
\fill (-46.000000\du,13.000000\du)--(-46.000000\du,14.400000\du)--(-37.030000\du,14.400000\du)--(-37.030000\du,13.000000\du)--cycle;
\definecolor{dialinecolor}{rgb}{0.000000, 0.000000, 0.000000}
\pgfsetstrokecolor{dialinecolor}
\draw (-46.000000\du,13.000000\du)--(-46.000000\du,14.400000\du)--(-37.030000\du,14.400000\du)--(-37.030000\du,13.000000\du)--cycle;
% setfont left to latex
\definecolor{dialinecolor}{rgb}{0.000000, 0.000000, 0.000000}
\pgfsetstrokecolor{dialinecolor}
\node at (-41.515000\du,13.950000\du){OpenObserver};
\definecolor{dialinecolor}{rgb}{1.000000, 1.000000, 1.000000}
\pgfsetfillcolor{dialinecolor}
\fill (-46.000000\du,14.400000\du)--(-46.000000\du,14.800000\du)--(-37.030000\du,14.800000\du)--(-37.030000\du,14.400000\du)--cycle;
\definecolor{dialinecolor}{rgb}{0.000000, 0.000000, 0.000000}
\pgfsetstrokecolor{dialinecolor}
\draw (-46.000000\du,14.400000\du)--(-46.000000\du,14.800000\du)--(-37.030000\du,14.800000\du)--(-37.030000\du,14.400000\du)--cycle;
\definecolor{dialinecolor}{rgb}{1.000000, 1.000000, 1.000000}
\pgfsetfillcolor{dialinecolor}
\fill (-46.000000\du,14.800000\du)--(-46.000000\du,15.800000\du)--(-37.030000\du,15.800000\du)--(-37.030000\du,14.800000\du)--cycle;
\definecolor{dialinecolor}{rgb}{0.000000, 0.000000, 0.000000}
\pgfsetstrokecolor{dialinecolor}
\draw (-46.000000\du,14.800000\du)--(-46.000000\du,15.800000\du)--(-37.030000\du,15.800000\du)--(-37.030000\du,14.800000\du)--cycle;
% setfont left to latex
\definecolor{dialinecolor}{rgb}{0.000000, 0.000000, 0.000000}
\pgfsetstrokecolor{dialinecolor}
\node[anchor=west] at (-45.850000\du,15.500000\du){+updateOpen(e:Element)};
\pgfsetlinewidth{0.100000\du}
\pgfsetdash{}{0pt}
\definecolor{dialinecolor}{rgb}{1.000000, 1.000000, 1.000000}
\pgfsetfillcolor{dialinecolor}
\fill (-58.000000\du,13.000000\du)--(-58.000000\du,14.400000\du)--(-48.260000\du,14.400000\du)--(-48.260000\du,13.000000\du)--cycle;
\definecolor{dialinecolor}{rgb}{0.000000, 0.000000, 0.000000}
\pgfsetstrokecolor{dialinecolor}
\draw (-58.000000\du,13.000000\du)--(-58.000000\du,14.400000\du)--(-48.260000\du,14.400000\du)--(-48.260000\du,13.000000\du)--cycle;
% setfont left to latex
\definecolor{dialinecolor}{rgb}{0.000000, 0.000000, 0.000000}
\pgfsetstrokecolor{dialinecolor}
\node at (-53.130000\du,13.950000\du){ChangeObserver};
\definecolor{dialinecolor}{rgb}{1.000000, 1.000000, 1.000000}
\pgfsetfillcolor{dialinecolor}
\fill (-58.000000\du,14.400000\du)--(-58.000000\du,14.800000\du)--(-48.260000\du,14.800000\du)--(-48.260000\du,14.400000\du)--cycle;
\definecolor{dialinecolor}{rgb}{0.000000, 0.000000, 0.000000}
\pgfsetstrokecolor{dialinecolor}
\draw (-58.000000\du,14.400000\du)--(-58.000000\du,14.800000\du)--(-48.260000\du,14.800000\du)--(-48.260000\du,14.400000\du)--cycle;
\definecolor{dialinecolor}{rgb}{1.000000, 1.000000, 1.000000}
\pgfsetfillcolor{dialinecolor}
\fill (-58.000000\du,14.800000\du)--(-58.000000\du,15.800000\du)--(-48.260000\du,15.800000\du)--(-48.260000\du,14.800000\du)--cycle;
\definecolor{dialinecolor}{rgb}{0.000000, 0.000000, 0.000000}
\pgfsetstrokecolor{dialinecolor}
\draw (-58.000000\du,14.800000\du)--(-58.000000\du,15.800000\du)--(-48.260000\du,15.800000\du)--(-48.260000\du,14.800000\du)--cycle;
% setfont left to latex
\definecolor{dialinecolor}{rgb}{0.000000, 0.000000, 0.000000}
\pgfsetstrokecolor{dialinecolor}
\node[anchor=west] at (-57.850000\du,15.500000\du){+updateChange(e:Element)};
\pgfsetlinewidth{0.100000\du}
\pgfsetdash{}{0pt}
\definecolor{dialinecolor}{rgb}{1.000000, 1.000000, 1.000000}
\pgfsetfillcolor{dialinecolor}
\fill (-69.175500\du,13.000000\du)--(-69.175500\du,14.400000\du)--(-59.820500\du,14.400000\du)--(-59.820500\du,13.000000\du)--cycle;
\definecolor{dialinecolor}{rgb}{0.000000, 0.000000, 0.000000}
\pgfsetstrokecolor{dialinecolor}
\draw (-69.175500\du,13.000000\du)--(-69.175500\du,14.400000\du)--(-59.820500\du,14.400000\du)--(-59.820500\du,13.000000\du)--cycle;
% setfont left to latex
\definecolor{dialinecolor}{rgb}{0.000000, 0.000000, 0.000000}
\pgfsetstrokecolor{dialinecolor}
\node at (-64.498000\du,13.950000\du){CloseObserver};
\definecolor{dialinecolor}{rgb}{1.000000, 1.000000, 1.000000}
\pgfsetfillcolor{dialinecolor}
\fill (-69.175500\du,14.400000\du)--(-69.175500\du,14.800000\du)--(-59.820500\du,14.800000\du)--(-59.820500\du,14.400000\du)--cycle;
\definecolor{dialinecolor}{rgb}{0.000000, 0.000000, 0.000000}
\pgfsetstrokecolor{dialinecolor}
\draw (-69.175500\du,14.400000\du)--(-69.175500\du,14.800000\du)--(-59.820500\du,14.800000\du)--(-59.820500\du,14.400000\du)--cycle;
\definecolor{dialinecolor}{rgb}{1.000000, 1.000000, 1.000000}
\pgfsetfillcolor{dialinecolor}
\fill (-69.175500\du,14.800000\du)--(-69.175500\du,15.800000\du)--(-59.820500\du,15.800000\du)--(-59.820500\du,14.800000\du)--cycle;
\definecolor{dialinecolor}{rgb}{0.000000, 0.000000, 0.000000}
\pgfsetstrokecolor{dialinecolor}
\draw (-69.175500\du,14.800000\du)--(-69.175500\du,15.800000\du)--(-59.820500\du,15.800000\du)--(-59.820500\du,14.800000\du)--cycle;
% setfont left to latex
\definecolor{dialinecolor}{rgb}{0.000000, 0.000000, 0.000000}
\pgfsetstrokecolor{dialinecolor}
\node[anchor=west] at (-69.025500\du,15.500000\du){+updateClose(e:Element)};
\pgfsetlinewidth{0.100000\du}
\pgfsetdash{{1.000000\du}{1.000000\du}}{0\du}
\pgfsetdash{{0.400000\du}{0.400000\du}}{0\du}
\pgfsetmiterjoin
\pgfsetbuttcap
{
\definecolor{dialinecolor}{rgb}{0.000000, 0.000000, 0.000000}
\pgfsetfillcolor{dialinecolor}
% was here!!!
\definecolor{dialinecolor}{rgb}{0.000000, 0.000000, 0.000000}
\pgfsetstrokecolor{dialinecolor}
\draw (-53.130000\du,15.849121\du)--(-53.130000\du,22.000000\du)--(-41.055000\du,22.000000\du)--(-41.055000\du,24.949738\du);
}
\definecolor{dialinecolor}{rgb}{0.000000, 0.000000, 0.000000}
\pgfsetstrokecolor{dialinecolor}
\draw (-53.130000\du,16.760924\du)--(-53.130000\du,22.000000\du)--(-41.055000\du,22.000000\du)--(-41.055000\du,24.949738\du);
\pgfsetmiterjoin
\definecolor{dialinecolor}{rgb}{1.000000, 1.000000, 1.000000}
\pgfsetfillcolor{dialinecolor}
\fill (-52.730000\du,16.760924\du)--(-53.130000\du,15.960924\du)--(-53.530000\du,16.760924\du)--cycle;
\pgfsetlinewidth{0.100000\du}
\pgfsetdash{}{0pt}
\pgfsetmiterjoin
\definecolor{dialinecolor}{rgb}{0.000000, 0.000000, 0.000000}
\pgfsetstrokecolor{dialinecolor}
\draw (-52.730000\du,16.760924\du)--(-53.130000\du,15.960924\du)--(-53.530000\du,16.760924\du)--cycle;
% setfont left to latex
\pgfsetlinewidth{0.100000\du}
\pgfsetdash{{0.400000\du}{0.400000\du}}{0\du}
\pgfsetdash{{0.400000\du}{0.400000\du}}{0\du}
\pgfsetmiterjoin
\pgfsetbuttcap
{
\definecolor{dialinecolor}{rgb}{0.000000, 0.000000, 0.000000}
\pgfsetfillcolor{dialinecolor}
% was here!!!
\definecolor{dialinecolor}{rgb}{0.000000, 0.000000, 0.000000}
\pgfsetstrokecolor{dialinecolor}
\draw (-31.130000\du,15.849707\du)--(-31.130000\du,20.754200\du)--(-41.055000\du,20.754200\du)--(-41.055000\du,24.950150\du);
}
\definecolor{dialinecolor}{rgb}{0.000000, 0.000000, 0.000000}
\pgfsetstrokecolor{dialinecolor}
\draw (-31.130000\du,16.761510\du)--(-31.130000\du,20.754200\du)--(-41.055000\du,20.754200\du)--(-41.055000\du,24.950150\du);
\pgfsetmiterjoin
\definecolor{dialinecolor}{rgb}{1.000000, 1.000000, 1.000000}
\pgfsetfillcolor{dialinecolor}
\fill (-30.730000\du,16.761510\du)--(-31.130000\du,15.961510\du)--(-31.530000\du,16.761510\du)--cycle;
\pgfsetlinewidth{0.100000\du}
\pgfsetdash{}{0pt}
\pgfsetmiterjoin
\definecolor{dialinecolor}{rgb}{0.000000, 0.000000, 0.000000}
\pgfsetstrokecolor{dialinecolor}
\draw (-30.730000\du,16.761510\du)--(-31.130000\du,15.961510\du)--(-31.530000\du,16.761510\du)--cycle;
% setfont left to latex
\pgfsetlinewidth{0.100000\du}
\pgfsetdash{{0.400000\du}{0.400000\du}}{0\du}
\pgfsetdash{{0.400000\du}{0.400000\du}}{0\du}
\pgfsetmiterjoin
\pgfsetbuttcap
{
\definecolor{dialinecolor}{rgb}{0.000000, 0.000000, 0.000000}
\pgfsetfillcolor{dialinecolor}
% was here!!!
\definecolor{dialinecolor}{rgb}{0.000000, 0.000000, 0.000000}
\pgfsetstrokecolor{dialinecolor}
\draw (-41.515000\du,15.849121\du)--(-41.515000\du,22.000000\du)--(-41.055000\du,22.000000\du)--(-41.055000\du,24.949738\du);
}
\definecolor{dialinecolor}{rgb}{0.000000, 0.000000, 0.000000}
\pgfsetstrokecolor{dialinecolor}
\draw (-41.515000\du,16.760924\du)--(-41.515000\du,22.000000\du)--(-41.055000\du,22.000000\du)--(-41.055000\du,24.949738\du);
\pgfsetmiterjoin
\definecolor{dialinecolor}{rgb}{1.000000, 1.000000, 1.000000}
\pgfsetfillcolor{dialinecolor}
\fill (-41.115000\du,16.760924\du)--(-41.515000\du,15.960924\du)--(-41.915000\du,16.760924\du)--cycle;
\pgfsetlinewidth{0.100000\du}
\pgfsetdash{}{0pt}
\pgfsetmiterjoin
\definecolor{dialinecolor}{rgb}{0.000000, 0.000000, 0.000000}
\pgfsetstrokecolor{dialinecolor}
\draw (-41.115000\du,16.760924\du)--(-41.515000\du,15.960924\du)--(-41.915000\du,16.760924\du)--cycle;
% setfont left to latex
\pgfsetlinewidth{0.100000\du}
\pgfsetdash{{0.400000\du}{0.400000\du}}{0\du}
\pgfsetdash{{0.400000\du}{0.400000\du}}{0\du}
\pgfsetmiterjoin
\pgfsetbuttcap
{
\definecolor{dialinecolor}{rgb}{0.000000, 0.000000, 0.000000}
\pgfsetfillcolor{dialinecolor}
% was here!!!
\definecolor{dialinecolor}{rgb}{0.000000, 0.000000, 0.000000}
\pgfsetstrokecolor{dialinecolor}
\draw (-64.498000\du,15.849121\du)--(-64.498000\du,22.000000\du)--(-41.055000\du,22.000000\du)--(-41.055000\du,24.949738\du);
}
\definecolor{dialinecolor}{rgb}{0.000000, 0.000000, 0.000000}
\pgfsetstrokecolor{dialinecolor}
\draw (-64.498000\du,16.760924\du)--(-64.498000\du,22.000000\du)--(-41.055000\du,22.000000\du)--(-41.055000\du,24.949738\du);
\pgfsetmiterjoin
\definecolor{dialinecolor}{rgb}{1.000000, 1.000000, 1.000000}
\pgfsetfillcolor{dialinecolor}
\fill (-64.098000\du,16.760924\du)--(-64.498000\du,15.960924\du)--(-64.898000\du,16.760924\du)--cycle;
\pgfsetlinewidth{0.100000\du}
\pgfsetdash{}{0pt}
\pgfsetmiterjoin
\definecolor{dialinecolor}{rgb}{0.000000, 0.000000, 0.000000}
\pgfsetstrokecolor{dialinecolor}
\draw (-64.098000\du,16.760924\du)--(-64.498000\du,15.960924\du)--(-64.898000\du,16.760924\du)--cycle;
% setfont left to latex
\pgfsetlinewidth{0.100000\du}
\pgfsetdash{{0.400000\du}{0.400000\du}}{0\du}
\pgfsetdash{{0.400000\du}{0.400000\du}}{0\du}
\pgfsetmiterjoin
\pgfsetbuttcap
{
\definecolor{dialinecolor}{rgb}{0.000000, 0.000000, 0.000000}
\pgfsetfillcolor{dialinecolor}
% was here!!!
\definecolor{dialinecolor}{rgb}{0.000000, 0.000000, 0.000000}
\pgfsetstrokecolor{dialinecolor}
\draw (-19.745000\du,15.849121\du)--(-19.745000\du,22.000000\du)--(-41.055000\du,22.000000\du)--(-41.055000\du,24.949738\du);
}
\definecolor{dialinecolor}{rgb}{0.000000, 0.000000, 0.000000}
\pgfsetstrokecolor{dialinecolor}
\draw (-19.745000\du,16.760924\du)--(-19.745000\du,22.000000\du)--(-41.055000\du,22.000000\du)--(-41.055000\du,24.949738\du);
\pgfsetmiterjoin
\definecolor{dialinecolor}{rgb}{1.000000, 1.000000, 1.000000}
\pgfsetfillcolor{dialinecolor}
\fill (-19.345000\du,16.760924\du)--(-19.745000\du,15.960924\du)--(-20.145000\du,16.760924\du)--cycle;
\pgfsetlinewidth{0.100000\du}
\pgfsetdash{}{0pt}
\pgfsetmiterjoin
\definecolor{dialinecolor}{rgb}{0.000000, 0.000000, 0.000000}
\pgfsetstrokecolor{dialinecolor}
\draw (-19.345000\du,16.760924\du)--(-19.745000\du,15.960924\du)--(-20.145000\du,16.760924\du)--cycle;
% setfont left to latex
\pgfsetlinewidth{0.100000\du}
\pgfsetdash{}{0pt}
\pgfsetmiterjoin
\pgfsetbuttcap
{
\definecolor{dialinecolor}{rgb}{0.000000, 0.000000, 0.000000}
\pgfsetfillcolor{dialinecolor}
% was here!!!
\definecolor{dialinecolor}{rgb}{0.000000, 0.000000, 0.000000}
\pgfsetstrokecolor{dialinecolor}
\draw (-40.795000\du,-4.900000\du)--(-27.000000\du,-4.900000\du)--(-27.000000\du,-5.018780\du)--(3.557700\du,-5.018780\du);
}
\definecolor{dialinecolor}{rgb}{0.000000, 0.000000, 0.000000}
\pgfsetstrokecolor{dialinecolor}
\draw (-39.883197\du,-4.900000\du)--(-27.000000\du,-4.900000\du)--(-27.000000\du,-5.018780\du)--(3.557700\du,-5.018780\du);
\pgfsetmiterjoin
\definecolor{dialinecolor}{rgb}{1.000000, 1.000000, 1.000000}
\pgfsetfillcolor{dialinecolor}
\fill (-39.883197\du,-5.300000\du)--(-40.683197\du,-4.900000\du)--(-39.883197\du,-4.500000\du)--cycle;
\pgfsetlinewidth{0.100000\du}
\pgfsetdash{}{0pt}
\pgfsetmiterjoin
\definecolor{dialinecolor}{rgb}{0.000000, 0.000000, 0.000000}
\pgfsetstrokecolor{dialinecolor}
\draw (-39.883197\du,-5.300000\du)--(-40.683197\du,-4.900000\du)--(-39.883197\du,-4.500000\du)--cycle;
% setfont left to latex
\pgfsetlinewidth{0.100000\du}
\pgfsetdash{}{0pt}
\pgfsetmiterjoin
\pgfsetbuttcap
{
\definecolor{dialinecolor}{rgb}{0.000000, 0.000000, 0.000000}
\pgfsetfillcolor{dialinecolor}
% was here!!!
\pgfsetarrowsend{to}
\definecolor{dialinecolor}{rgb}{0.000000, 0.000000, 0.000000}
\pgfsetstrokecolor{dialinecolor}
\draw (-47.397500\du,3.649512\du)--(-47.397500\du,9.000000\du)--(-64.498000\du,9.000000\du)--(-64.498000\du,13.000000\du);
}
% setfont left to latex
\definecolor{dialinecolor}{rgb}{0.000000, 0.000000, 0.000000}
\pgfsetstrokecolor{dialinecolor}
\node at (-55.947750\du,8.850000\du){};
\definecolor{dialinecolor}{rgb}{0.000000, 0.000000, 0.000000}
\pgfsetstrokecolor{dialinecolor}
\node[anchor=west] at (-47.197500\du,4.249512\du){};
\definecolor{dialinecolor}{rgb}{0.000000, 0.000000, 0.000000}
\pgfsetstrokecolor{dialinecolor}
\node[anchor=west] at (-63.948000\du,12.800000\du){*};
\pgfsetlinewidth{0.100000\du}
\pgfsetdash{}{0pt}
\pgfsetmiterjoin
\pgfsetbuttcap
{
\definecolor{dialinecolor}{rgb}{0.000000, 0.000000, 0.000000}
\pgfsetfillcolor{dialinecolor}
% was here!!!
\pgfsetarrowsend{to}
\definecolor{dialinecolor}{rgb}{0.000000, 0.000000, 0.000000}
\pgfsetstrokecolor{dialinecolor}
\draw (-47.397500\du,3.649512\du)--(-47.397500\du,9.000000\du)--(-31.130000\du,9.000000\du)--(-31.130000\du,13.000000\du);
}
% setfont left to latex
\definecolor{dialinecolor}{rgb}{0.000000, 0.000000, 0.000000}
\pgfsetstrokecolor{dialinecolor}
\node at (-39.263750\du,8.850000\du){};
\definecolor{dialinecolor}{rgb}{0.000000, 0.000000, 0.000000}
\pgfsetstrokecolor{dialinecolor}
\node[anchor=west] at (-47.197500\du,4.249512\du){};
\definecolor{dialinecolor}{rgb}{0.000000, 0.000000, 0.000000}
\pgfsetstrokecolor{dialinecolor}
\node[anchor=west] at (-30.580000\du,12.800000\du){*};
\pgfsetlinewidth{0.100000\du}
\pgfsetdash{}{0pt}
\pgfsetmiterjoin
\pgfsetbuttcap
{
\definecolor{dialinecolor}{rgb}{0.000000, 0.000000, 0.000000}
\pgfsetfillcolor{dialinecolor}
% was here!!!
\pgfsetarrowsend{to}
\definecolor{dialinecolor}{rgb}{0.000000, 0.000000, 0.000000}
\pgfsetstrokecolor{dialinecolor}
\draw (-6.000000\du,13.000000\du)--(-6.000000\du,10.000000\du)--(-19.745000\du,10.000000\du)--(-19.745000\du,13.000000\du);
}
% setfont left to latex
\definecolor{dialinecolor}{rgb}{0.000000, 0.000000, 0.000000}
\pgfsetstrokecolor{dialinecolor}
\node at (-12.872500\du,9.850000\du){};
\definecolor{dialinecolor}{rgb}{0.000000, 0.000000, 0.000000}
\pgfsetstrokecolor{dialinecolor}
\node[anchor=west] at (-5.800000\du,12.000000\du){ observe};
\definecolor{dialinecolor}{rgb}{0.000000, 0.000000, 0.000000}
\pgfsetstrokecolor{dialinecolor}
\node[anchor=west] at (-5.800000\du,12.800000\du){*};
\definecolor{dialinecolor}{rgb}{0.000000, 0.000000, 0.000000}
\pgfsetstrokecolor{dialinecolor}
\node[anchor=west] at (-19.195000\du,12.800000\du){*};
\pgfsetlinewidth{0.100000\du}
\pgfsetdash{}{0pt}
\pgfsetmiterjoin
\pgfsetbuttcap
{
\definecolor{dialinecolor}{rgb}{0.000000, 0.000000, 0.000000}
\pgfsetfillcolor{dialinecolor}
% was here!!!
\pgfsetarrowsend{to}
\definecolor{dialinecolor}{rgb}{0.000000, 0.000000, 0.000000}
\pgfsetstrokecolor{dialinecolor}
\draw (-47.397500\du,3.649512\du)--(-47.397500\du,9.000000\du)--(-41.515000\du,9.000000\du)--(-41.515000\du,12.952441\du);
}
% setfont left to latex
\definecolor{dialinecolor}{rgb}{0.000000, 0.000000, 0.000000}
\pgfsetstrokecolor{dialinecolor}
\node at (-44.456250\du,8.850000\du){};
\definecolor{dialinecolor}{rgb}{0.000000, 0.000000, 0.000000}
\pgfsetstrokecolor{dialinecolor}
\node[anchor=west] at (-47.197500\du,4.249512\du){};
\definecolor{dialinecolor}{rgb}{0.000000, 0.000000, 0.000000}
\pgfsetstrokecolor{dialinecolor}
\node[anchor=west] at (-40.965000\du,12.752441\du){*};
\pgfsetlinewidth{0.100000\du}
\pgfsetdash{}{0pt}
\pgfsetmiterjoin
\pgfsetbuttcap
{
\definecolor{dialinecolor}{rgb}{0.000000, 0.000000, 0.000000}
\pgfsetfillcolor{dialinecolor}
% was here!!!
\pgfsetarrowsend{to}
\definecolor{dialinecolor}{rgb}{0.000000, 0.000000, 0.000000}
\pgfsetstrokecolor{dialinecolor}
\draw (-47.397500\du,3.649512\du)--(-47.397500\du,9.000000\du)--(-53.130000\du,9.000000\du)--(-53.130000\du,13.000000\du);
}
% setfont left to latex
\definecolor{dialinecolor}{rgb}{0.000000, 0.000000, 0.000000}
\pgfsetstrokecolor{dialinecolor}
\node at (-50.263750\du,8.850000\du){};
\definecolor{dialinecolor}{rgb}{0.000000, 0.000000, 0.000000}
\pgfsetstrokecolor{dialinecolor}
\node[anchor=west] at (-47.197500\du,4.249512\du){1};
\definecolor{dialinecolor}{rgb}{0.000000, 0.000000, 0.000000}
\pgfsetstrokecolor{dialinecolor}
\node[anchor=west] at (-52.580000\du,12.800000\du){*};
\pgfsetlinewidth{0.100000\du}
\pgfsetdash{}{0pt}
\pgfsetmiterjoin
\pgfsetbuttcap
{
\definecolor{dialinecolor}{rgb}{0.000000, 0.000000, 0.000000}
\pgfsetfillcolor{dialinecolor}
% was here!!!
\pgfsetarrowsend{to}
\definecolor{dialinecolor}{rgb}{0.000000, 0.000000, 0.000000}
\pgfsetstrokecolor{dialinecolor}
\draw (-38.060323\du,27.500000\du)--(-11.000000\du,27.500000\du)--(-11.000000\du,15.300000\du)--(-7.000000\du,15.300000\du);
}
% setfont left to latex
\definecolor{dialinecolor}{rgb}{0.000000, 0.000000, 0.000000}
\pgfsetstrokecolor{dialinecolor}
\node[anchor=west] at (-10.900000\du,21.250000\du){};
\definecolor{dialinecolor}{rgb}{0.000000, 0.000000, 0.000000}
\pgfsetstrokecolor{dialinecolor}
\node[anchor=west] at (-37.860323\du,27.350000\du){1};
\definecolor{dialinecolor}{rgb}{0.000000, 0.000000, 0.000000}
\pgfsetstrokecolor{dialinecolor}
\node[anchor=east] at (-8.000000\du,15.150000\du){ ouvre};
\definecolor{dialinecolor}{rgb}{0.000000, 0.000000, 0.000000}
\pgfsetstrokecolor{dialinecolor}
\node[anchor=east] at (-8.000000\du,15.950000\du){*};
\pgfsetlinewidth{0.100000\du}
\pgfsetdash{{0.400000\du}{0.400000\du}}{0\du}
\pgfsetdash{{0.400000\du}{0.400000\du}}{0\du}
\pgfsetmiterjoin
\pgfsetbuttcap
{
\definecolor{dialinecolor}{rgb}{0.000000, 0.000000, 0.000000}
\pgfsetfillcolor{dialinecolor}
% was here!!!
\definecolor{dialinecolor}{rgb}{0.000000, 0.000000, 0.000000}
\pgfsetstrokecolor{dialinecolor}
\draw (-29.000000\du,13.000000\du)--(-29.000000\du,9.000000\du)--(-0.205000\du,9.000000\du)--(-0.205000\du,12.949927\du);
}
\definecolor{dialinecolor}{rgb}{0.000000, 0.000000, 0.000000}
\pgfsetstrokecolor{dialinecolor}
\draw (-29.000000\du,12.088197\du)--(-29.000000\du,9.000000\du)--(-0.205000\du,9.000000\du)--(-0.205000\du,12.949927\du);
\pgfsetmiterjoin
\definecolor{dialinecolor}{rgb}{1.000000, 1.000000, 1.000000}
\pgfsetfillcolor{dialinecolor}
\fill (-29.400000\du,12.088197\du)--(-29.000000\du,12.888197\du)--(-28.600000\du,12.088197\du)--cycle;
\pgfsetlinewidth{0.100000\du}
\pgfsetdash{}{0pt}
\pgfsetmiterjoin
\definecolor{dialinecolor}{rgb}{0.000000, 0.000000, 0.000000}
\pgfsetstrokecolor{dialinecolor}
\draw (-29.400000\du,12.088197\du)--(-29.000000\du,12.888197\du)--(-28.600000\du,12.088197\du)--cycle;
% setfont left to latex
\end{tikzpicture}
}
    \caption{Diagramme de classes}
  \end{sidewaysfigure}

  \section{Changements}

  Pour les deux sous-sections qui suivent:
  \begin{itemize}
    \item identifier les concepts pertinents
    \item préciser l'impact
    \item préciser si l'impact est localisé
  \end{itemize}

  \subsection{Première requête de changement}

  Pour effectuer le changement, nous introduisons la classe \textsf{Raccourci}
  et \textsf{ElementRaccourciable}. Un raccourci possède un agrégat vers une
  instance d'\textsf{ElementRaccourciable}.

  % ajouter le petit diagramme ici
  \begin{figure}
    \centering
    \resizebox{\textwidth}{!}{% Graphic for TeX using PGF
% Title: /home/guillaume/Documents/Université de Montréal/Automne 2014/Génie logiciel/Devoir/devoir-3/diagramme-de-classes-changement-1.dia
% Creator: Dia v0.97.3
% CreationDate: Sun Dec  7 16:12:28 2014
% For: guillaume
% \usepackage{tikz}
% The following commands are not supported in PSTricks at present
% We define them conditionally, so when they are implemented,
% this pgf file will use them.
\ifx\du\undefined
  \newlength{\du}
\fi
\setlength{\du}{15\unitlength}
\begin{tikzpicture}
\pgftransformxscale{1.000000}
\pgftransformyscale{-1.000000}
\definecolor{dialinecolor}{rgb}{0.000000, 0.000000, 0.000000}
\pgfsetstrokecolor{dialinecolor}
\definecolor{dialinecolor}{rgb}{1.000000, 1.000000, 1.000000}
\pgfsetfillcolor{dialinecolor}
\pgfsetlinewidth{0.100000\du}
\pgfsetdash{}{0pt}
\definecolor{dialinecolor}{rgb}{1.000000, 1.000000, 1.000000}
\pgfsetfillcolor{dialinecolor}
\fill (16.600000\du,18.800000\du)--(16.600000\du,20.200000\du)--(21.470000\du,20.200000\du)--(21.470000\du,18.800000\du)--cycle;
\definecolor{dialinecolor}{rgb}{0.000000, 0.000000, 0.000000}
\pgfsetstrokecolor{dialinecolor}
\draw (16.600000\du,18.800000\du)--(16.600000\du,20.200000\du)--(21.470000\du,20.200000\du)--(21.470000\du,18.800000\du)--cycle;
% setfont left to latex
\definecolor{dialinecolor}{rgb}{0.000000, 0.000000, 0.000000}
\pgfsetstrokecolor{dialinecolor}
\node at (19.035000\du,19.750000\du){Raccourci};
\definecolor{dialinecolor}{rgb}{1.000000, 1.000000, 1.000000}
\pgfsetfillcolor{dialinecolor}
\fill (16.600000\du,20.200000\du)--(16.600000\du,20.600000\du)--(21.470000\du,20.600000\du)--(21.470000\du,20.200000\du)--cycle;
\definecolor{dialinecolor}{rgb}{0.000000, 0.000000, 0.000000}
\pgfsetstrokecolor{dialinecolor}
\draw (16.600000\du,20.200000\du)--(16.600000\du,20.600000\du)--(21.470000\du,20.600000\du)--(21.470000\du,20.200000\du)--cycle;
\definecolor{dialinecolor}{rgb}{1.000000, 1.000000, 1.000000}
\pgfsetfillcolor{dialinecolor}
\fill (16.600000\du,20.600000\du)--(16.600000\du,21.000000\du)--(21.470000\du,21.000000\du)--(21.470000\du,20.600000\du)--cycle;
\definecolor{dialinecolor}{rgb}{0.000000, 0.000000, 0.000000}
\pgfsetstrokecolor{dialinecolor}
\draw (16.600000\du,20.600000\du)--(16.600000\du,21.000000\du)--(21.470000\du,21.000000\du)--(21.470000\du,20.600000\du)--cycle;
\pgfsetlinewidth{0.100000\du}
\pgfsetdash{}{0pt}
\pgfsetmiterjoin
\pgfsetbuttcap
{
\definecolor{dialinecolor}{rgb}{0.000000, 0.000000, 0.000000}
\pgfsetfillcolor{dialinecolor}
% was here!!!
\definecolor{dialinecolor}{rgb}{0.000000, 0.000000, 0.000000}
\pgfsetstrokecolor{dialinecolor}
\draw (21.519766\du,19.900000\du)--(28.300000\du,19.900000\du)--(28.300000\du,9.350000\du)--(31.501449\du,9.350000\du);
}
\definecolor{dialinecolor}{rgb}{0.000000, 0.000000, 0.000000}
\pgfsetstrokecolor{dialinecolor}
\draw (22.778345\du,19.900000\du)--(28.300000\du,19.900000\du)--(28.300000\du,9.350000\du)--(31.501449\du,9.350000\du);
\pgfsetdash{}{0pt}
\pgfsetmiterjoin
\pgfsetbuttcap
\definecolor{dialinecolor}{rgb}{1.000000, 1.000000, 1.000000}
\pgfsetfillcolor{dialinecolor}
\fill (21.519766\du,19.900000\du)--(22.219766\du,19.660000\du)--(22.919766\du,19.900000\du)--(22.219766\du,20.140000\du)--cycle;
\pgfsetlinewidth{0.100000\du}
\pgfsetdash{}{0pt}
\pgfsetmiterjoin
\pgfsetbuttcap
\definecolor{dialinecolor}{rgb}{0.000000, 0.000000, 0.000000}
\pgfsetstrokecolor{dialinecolor}
\draw (21.519766\du,19.900000\du)--(22.219766\du,19.660000\du)--(22.919766\du,19.900000\du)--(22.219766\du,20.140000\du)--cycle;
% setfont left to latex
\definecolor{dialinecolor}{rgb}{0.000000, 0.000000, 0.000000}
\pgfsetstrokecolor{dialinecolor}
\node[anchor=west] at (28.400000\du,14.475000\du){};
\definecolor{dialinecolor}{rgb}{0.000000, 0.000000, 0.000000}
\pgfsetstrokecolor{dialinecolor}
\node[anchor=west] at (23.119766\du,19.750000\du){};
\definecolor{dialinecolor}{rgb}{0.000000, 0.000000, 0.000000}
\pgfsetstrokecolor{dialinecolor}
\node[anchor=east] at (31.301449\du,9.200000\du){ pointe vers};
\definecolor{dialinecolor}{rgb}{0.000000, 0.000000, 0.000000}
\pgfsetstrokecolor{dialinecolor}
\node[anchor=east] at (31.301449\du,10.000000\du){*};
\pgfsetlinewidth{0.100000\du}
\pgfsetdash{}{0pt}
\pgfsetmiterjoin
\pgfsetbuttcap
{
\definecolor{dialinecolor}{rgb}{0.000000, 0.000000, 0.000000}
\pgfsetfillcolor{dialinecolor}
% was here!!!
\definecolor{dialinecolor}{rgb}{0.000000, 0.000000, 0.000000}
\pgfsetstrokecolor{dialinecolor}
\draw (32.565000\du,2.509124\du)--(32.565000\du,5.850000\du)--(19.034877\du,5.850000\du)--(19.034990\du,18.750034\du);
}
\definecolor{dialinecolor}{rgb}{0.000000, 0.000000, 0.000000}
\pgfsetstrokecolor{dialinecolor}
\draw (32.565000\du,3.420927\du)--(32.565000\du,5.850000\du)--(19.034877\du,5.850000\du)--(19.034990\du,18.750034\du);
\pgfsetmiterjoin
\definecolor{dialinecolor}{rgb}{1.000000, 1.000000, 1.000000}
\pgfsetfillcolor{dialinecolor}
\fill (32.965000\du,3.420927\du)--(32.565000\du,2.620927\du)--(32.165000\du,3.420927\du)--cycle;
\pgfsetlinewidth{0.100000\du}
\pgfsetdash{}{0pt}
\pgfsetmiterjoin
\definecolor{dialinecolor}{rgb}{0.000000, 0.000000, 0.000000}
\pgfsetstrokecolor{dialinecolor}
\draw (32.965000\du,3.420927\du)--(32.565000\du,2.620927\du)--(32.165000\du,3.420927\du)--cycle;
% setfont left to latex
\pgfsetlinewidth{0.100000\du}
\pgfsetdash{}{0pt}
\definecolor{dialinecolor}{rgb}{1.000000, 1.000000, 1.000000}
\pgfsetfillcolor{dialinecolor}
\fill (31.550000\du,7.950000\du)--(31.550000\du,9.350000\du)--(48.990000\du,9.350000\du)--(48.990000\du,7.950000\du)--cycle;
\definecolor{dialinecolor}{rgb}{0.000000, 0.000000, 0.000000}
\pgfsetstrokecolor{dialinecolor}
\draw (31.550000\du,7.950000\du)--(31.550000\du,9.350000\du)--(48.990000\du,9.350000\du)--(48.990000\du,7.950000\du)--cycle;
% setfont left to latex
\definecolor{dialinecolor}{rgb}{0.000000, 0.000000, 0.000000}
\pgfsetstrokecolor{dialinecolor}
\node at (40.270000\du,8.900000\du){ElementRaccourciable};
\definecolor{dialinecolor}{rgb}{1.000000, 1.000000, 1.000000}
\pgfsetfillcolor{dialinecolor}
\fill (31.550000\du,9.350000\du)--(31.550000\du,9.750000\du)--(48.990000\du,9.750000\du)--(48.990000\du,9.350000\du)--cycle;
\definecolor{dialinecolor}{rgb}{0.000000, 0.000000, 0.000000}
\pgfsetstrokecolor{dialinecolor}
\draw (31.550000\du,9.350000\du)--(31.550000\du,9.750000\du)--(48.990000\du,9.750000\du)--(48.990000\du,9.350000\du)--cycle;
\definecolor{dialinecolor}{rgb}{1.000000, 1.000000, 1.000000}
\pgfsetfillcolor{dialinecolor}
\fill (31.550000\du,9.750000\du)--(31.550000\du,10.750000\du)--(48.990000\du,10.750000\du)--(48.990000\du,9.750000\du)--cycle;
\definecolor{dialinecolor}{rgb}{0.000000, 0.000000, 0.000000}
\pgfsetstrokecolor{dialinecolor}
\draw (31.550000\du,9.750000\du)--(31.550000\du,10.750000\du)--(48.990000\du,10.750000\du)--(48.990000\du,9.750000\du)--cycle;
% setfont left to latex
\definecolor{dialinecolor}{rgb}{0.000000, 0.000000, 0.000000}
\pgfsetstrokecolor{dialinecolor}
\node[anchor=west] at (31.700000\du,10.450000\du){+accept(visitor:ElementRaccourciableVisitor)};
\pgfsetlinewidth{0.100000\du}
\pgfsetdash{}{0pt}
\definecolor{dialinecolor}{rgb}{1.000000, 1.000000, 1.000000}
\pgfsetfillcolor{dialinecolor}
\fill (44.450000\du,16.200000\du)--(44.450000\du,17.600000\du)--(51.110000\du,17.600000\du)--(51.110000\du,16.200000\du)--cycle;
\definecolor{dialinecolor}{rgb}{0.000000, 0.000000, 0.000000}
\pgfsetstrokecolor{dialinecolor}
\draw (44.450000\du,16.200000\du)--(44.450000\du,17.600000\du)--(51.110000\du,17.600000\du)--(51.110000\du,16.200000\du)--cycle;
% setfont left to latex
\definecolor{dialinecolor}{rgb}{0.000000, 0.000000, 0.000000}
\pgfsetstrokecolor{dialinecolor}
\node at (47.780000\du,17.150000\du){Fichier};
\definecolor{dialinecolor}{rgb}{1.000000, 1.000000, 1.000000}
\pgfsetfillcolor{dialinecolor}
\fill (44.450000\du,17.600000\du)--(44.450000\du,18.600000\du)--(51.110000\du,18.600000\du)--(51.110000\du,17.600000\du)--cycle;
\definecolor{dialinecolor}{rgb}{0.000000, 0.000000, 0.000000}
\pgfsetstrokecolor{dialinecolor}
\draw (44.450000\du,17.600000\du)--(44.450000\du,18.600000\du)--(51.110000\du,18.600000\du)--(51.110000\du,17.600000\du)--cycle;
% setfont left to latex
\definecolor{dialinecolor}{rgb}{0.000000, 0.000000, 0.000000}
\pgfsetstrokecolor{dialinecolor}
\node[anchor=west] at (44.600000\du,18.300000\du){-taille: Integer};
\definecolor{dialinecolor}{rgb}{1.000000, 1.000000, 1.000000}
\pgfsetfillcolor{dialinecolor}
\fill (44.450000\du,18.600000\du)--(44.450000\du,19.000000\du)--(51.110000\du,19.000000\du)--(51.110000\du,18.600000\du)--cycle;
\definecolor{dialinecolor}{rgb}{0.000000, 0.000000, 0.000000}
\pgfsetstrokecolor{dialinecolor}
\draw (44.450000\du,18.600000\du)--(44.450000\du,19.000000\du)--(51.110000\du,19.000000\du)--(51.110000\du,18.600000\du)--cycle;
\pgfsetlinewidth{0.100000\du}
\pgfsetdash{}{0pt}
\definecolor{dialinecolor}{rgb}{1.000000, 1.000000, 1.000000}
\pgfsetfillcolor{dialinecolor}
\fill (31.050000\du,16.250000\du)--(31.050000\du,17.650000\du)--(34.927500\du,17.650000\du)--(34.927500\du,16.250000\du)--cycle;
\definecolor{dialinecolor}{rgb}{0.000000, 0.000000, 0.000000}
\pgfsetstrokecolor{dialinecolor}
\draw (31.050000\du,16.250000\du)--(31.050000\du,17.650000\du)--(34.927500\du,17.650000\du)--(34.927500\du,16.250000\du)--cycle;
% setfont left to latex
\definecolor{dialinecolor}{rgb}{0.000000, 0.000000, 0.000000}
\pgfsetstrokecolor{dialinecolor}
\node at (32.988750\du,17.200000\du){Dossier};
\definecolor{dialinecolor}{rgb}{1.000000, 1.000000, 1.000000}
\pgfsetfillcolor{dialinecolor}
\fill (31.050000\du,17.650000\du)--(31.050000\du,18.050000\du)--(34.927500\du,18.050000\du)--(34.927500\du,17.650000\du)--cycle;
\definecolor{dialinecolor}{rgb}{0.000000, 0.000000, 0.000000}
\pgfsetstrokecolor{dialinecolor}
\draw (31.050000\du,17.650000\du)--(31.050000\du,18.050000\du)--(34.927500\du,18.050000\du)--(34.927500\du,17.650000\du)--cycle;
\definecolor{dialinecolor}{rgb}{1.000000, 1.000000, 1.000000}
\pgfsetfillcolor{dialinecolor}
\fill (31.050000\du,18.050000\du)--(31.050000\du,18.450000\du)--(34.927500\du,18.450000\du)--(34.927500\du,18.050000\du)--cycle;
\definecolor{dialinecolor}{rgb}{0.000000, 0.000000, 0.000000}
\pgfsetstrokecolor{dialinecolor}
\draw (31.050000\du,18.050000\du)--(31.050000\du,18.450000\du)--(34.927500\du,18.450000\du)--(34.927500\du,18.050000\du)--cycle;
\pgfsetlinewidth{0.100000\du}
\pgfsetdash{}{0pt}
\pgfsetmiterjoin
\pgfsetbuttcap
{
\definecolor{dialinecolor}{rgb}{0.000000, 0.000000, 0.000000}
\pgfsetfillcolor{dialinecolor}
% was here!!!
\definecolor{dialinecolor}{rgb}{0.000000, 0.000000, 0.000000}
\pgfsetstrokecolor{dialinecolor}
\draw (40.270000\du,10.800354\du)--(40.270000\du,13.900037\du)--(32.988750\du,13.900037\du)--(32.988750\du,16.199719\du);
}
\definecolor{dialinecolor}{rgb}{0.000000, 0.000000, 0.000000}
\pgfsetstrokecolor{dialinecolor}
\draw (40.270000\du,11.712157\du)--(40.270000\du,13.900037\du)--(32.988750\du,13.900037\du)--(32.988750\du,16.199719\du);
\pgfsetmiterjoin
\definecolor{dialinecolor}{rgb}{1.000000, 1.000000, 1.000000}
\pgfsetfillcolor{dialinecolor}
\fill (40.670000\du,11.712157\du)--(40.270000\du,10.912157\du)--(39.870000\du,11.712157\du)--cycle;
\pgfsetlinewidth{0.100000\du}
\pgfsetdash{}{0pt}
\pgfsetmiterjoin
\definecolor{dialinecolor}{rgb}{0.000000, 0.000000, 0.000000}
\pgfsetstrokecolor{dialinecolor}
\draw (40.670000\du,11.712157\du)--(40.270000\du,10.912157\du)--(39.870000\du,11.712157\du)--cycle;
% setfont left to latex
\pgfsetlinewidth{0.100000\du}
\pgfsetdash{}{0pt}
\pgfsetmiterjoin
\pgfsetbuttcap
{
\definecolor{dialinecolor}{rgb}{0.000000, 0.000000, 0.000000}
\pgfsetfillcolor{dialinecolor}
% was here!!!
\definecolor{dialinecolor}{rgb}{0.000000, 0.000000, 0.000000}
\pgfsetstrokecolor{dialinecolor}
\draw (40.270000\du,10.800354\du)--(40.270000\du,13.875000\du)--(47.780000\du,13.875000\du)--(47.780000\du,16.149646\du);
}
\definecolor{dialinecolor}{rgb}{0.000000, 0.000000, 0.000000}
\pgfsetstrokecolor{dialinecolor}
\draw (40.270000\du,11.712157\du)--(40.270000\du,13.875000\du)--(47.780000\du,13.875000\du)--(47.780000\du,16.149646\du);
\pgfsetmiterjoin
\definecolor{dialinecolor}{rgb}{1.000000, 1.000000, 1.000000}
\pgfsetfillcolor{dialinecolor}
\fill (40.670000\du,11.712157\du)--(40.270000\du,10.912157\du)--(39.870000\du,11.712157\du)--cycle;
\pgfsetlinewidth{0.100000\du}
\pgfsetdash{}{0pt}
\pgfsetmiterjoin
\definecolor{dialinecolor}{rgb}{0.000000, 0.000000, 0.000000}
\pgfsetstrokecolor{dialinecolor}
\draw (40.670000\du,11.712157\du)--(40.270000\du,10.912157\du)--(39.870000\du,11.712157\du)--cycle;
% setfont left to latex
\pgfsetlinewidth{0.100000\du}
\pgfsetdash{}{0pt}
\pgfsetmiterjoin
\pgfsetbuttcap
{
\definecolor{dialinecolor}{rgb}{0.000000, 0.000000, 0.000000}
\pgfsetfillcolor{dialinecolor}
% was here!!!
\definecolor{dialinecolor}{rgb}{0.000000, 0.000000, 0.000000}
\pgfsetstrokecolor{dialinecolor}
\draw (32.565000\du,2.509580\du)--(32.565000\du,5.800000\du)--(36.300000\du,5.800000\du)--(36.300000\du,7.950000\du);
}
\definecolor{dialinecolor}{rgb}{0.000000, 0.000000, 0.000000}
\pgfsetstrokecolor{dialinecolor}
\draw (32.565000\du,3.421383\du)--(32.565000\du,5.800000\du)--(36.300000\du,5.800000\du)--(36.300000\du,7.950000\du);
\pgfsetmiterjoin
\definecolor{dialinecolor}{rgb}{1.000000, 1.000000, 1.000000}
\pgfsetfillcolor{dialinecolor}
\fill (32.965000\du,3.421383\du)--(32.565000\du,2.621383\du)--(32.165000\du,3.421383\du)--cycle;
\pgfsetlinewidth{0.100000\du}
\pgfsetdash{}{0pt}
\pgfsetmiterjoin
\definecolor{dialinecolor}{rgb}{0.000000, 0.000000, 0.000000}
\pgfsetstrokecolor{dialinecolor}
\draw (32.965000\du,3.421383\du)--(32.565000\du,2.621383\du)--(32.165000\du,3.421383\du)--cycle;
% setfont left to latex
\pgfsetlinewidth{0.100000\du}
\pgfsetdash{}{0pt}
\definecolor{dialinecolor}{rgb}{1.000000, 1.000000, 1.000000}
\pgfsetfillcolor{dialinecolor}
\fill (4.000000\du,1.400000\du)--(4.000000\du,2.800000\du)--(18.360000\du,2.800000\du)--(18.360000\du,1.400000\du)--cycle;
\definecolor{dialinecolor}{rgb}{0.000000, 0.000000, 0.000000}
\pgfsetstrokecolor{dialinecolor}
\draw (4.000000\du,1.400000\du)--(4.000000\du,2.800000\du)--(18.360000\du,2.800000\du)--(18.360000\du,1.400000\du)--cycle;
% setfont left to latex
\definecolor{dialinecolor}{rgb}{0.000000, 0.000000, 0.000000}
\pgfsetstrokecolor{dialinecolor}
\node at (11.180000\du,2.350000\du){ElementRaccourciableVisitor};
\definecolor{dialinecolor}{rgb}{1.000000, 1.000000, 1.000000}
\pgfsetfillcolor{dialinecolor}
\fill (4.000000\du,2.800000\du)--(4.000000\du,3.200000\du)--(18.360000\du,3.200000\du)--(18.360000\du,2.800000\du)--cycle;
\definecolor{dialinecolor}{rgb}{0.000000, 0.000000, 0.000000}
\pgfsetstrokecolor{dialinecolor}
\draw (4.000000\du,2.800000\du)--(4.000000\du,3.200000\du)--(18.360000\du,3.200000\du)--(18.360000\du,2.800000\du)--cycle;
\definecolor{dialinecolor}{rgb}{1.000000, 1.000000, 1.000000}
\pgfsetfillcolor{dialinecolor}
\fill (4.000000\du,3.200000\du)--(4.000000\du,4.200000\du)--(18.360000\du,4.200000\du)--(18.360000\du,3.200000\du)--cycle;
\definecolor{dialinecolor}{rgb}{0.000000, 0.000000, 0.000000}
\pgfsetstrokecolor{dialinecolor}
\draw (4.000000\du,3.200000\du)--(4.000000\du,4.200000\du)--(18.360000\du,4.200000\du)--(18.360000\du,3.200000\du)--cycle;
% setfont left to latex
\definecolor{dialinecolor}{rgb}{0.000000, 0.000000, 0.000000}
\pgfsetstrokecolor{dialinecolor}
\node[anchor=west] at (4.150000\du,3.900000\du){+visit(element:ElementRaccourciable)};
\pgfsetlinewidth{0.100000\du}
\pgfsetdash{}{0pt}
\definecolor{dialinecolor}{rgb}{1.000000, 1.000000, 1.000000}
\pgfsetfillcolor{dialinecolor}
\fill (5.400000\du,14.600000\du)--(5.400000\du,16.000000\du)--(11.207500\du,16.000000\du)--(11.207500\du,14.600000\du)--cycle;
\definecolor{dialinecolor}{rgb}{0.000000, 0.000000, 0.000000}
\pgfsetstrokecolor{dialinecolor}
\draw (5.400000\du,14.600000\du)--(5.400000\du,16.000000\du)--(11.207500\du,16.000000\du)--(11.207500\du,14.600000\du)--cycle;
% setfont left to latex
\definecolor{dialinecolor}{rgb}{0.000000, 0.000000, 0.000000}
\pgfsetstrokecolor{dialinecolor}
\node at (8.303750\du,15.550000\du){TailleVisitor};
\definecolor{dialinecolor}{rgb}{1.000000, 1.000000, 1.000000}
\pgfsetfillcolor{dialinecolor}
\fill (5.400000\du,16.000000\du)--(5.400000\du,16.400000\du)--(11.207500\du,16.400000\du)--(11.207500\du,16.000000\du)--cycle;
\definecolor{dialinecolor}{rgb}{0.000000, 0.000000, 0.000000}
\pgfsetstrokecolor{dialinecolor}
\draw (5.400000\du,16.000000\du)--(5.400000\du,16.400000\du)--(11.207500\du,16.400000\du)--(11.207500\du,16.000000\du)--cycle;
\definecolor{dialinecolor}{rgb}{1.000000, 1.000000, 1.000000}
\pgfsetfillcolor{dialinecolor}
\fill (5.400000\du,16.400000\du)--(5.400000\du,16.800000\du)--(11.207500\du,16.800000\du)--(11.207500\du,16.400000\du)--cycle;
\definecolor{dialinecolor}{rgb}{0.000000, 0.000000, 0.000000}
\pgfsetstrokecolor{dialinecolor}
\draw (5.400000\du,16.400000\du)--(5.400000\du,16.800000\du)--(11.207500\du,16.800000\du)--(11.207500\du,16.400000\du)--cycle;
\pgfsetlinewidth{0.100000\du}
\pgfsetdash{{1.000000\du}{1.000000\du}}{0\du}
\pgfsetdash{{0.400000\du}{0.400000\du}}{0\du}
\pgfsetmiterjoin
\pgfsetbuttcap
{
\definecolor{dialinecolor}{rgb}{0.000000, 0.000000, 0.000000}
\pgfsetfillcolor{dialinecolor}
% was here!!!
\definecolor{dialinecolor}{rgb}{0.000000, 0.000000, 0.000000}
\pgfsetstrokecolor{dialinecolor}
\draw (11.180000\du,4.250354\du)--(11.180000\du,9.800037\du)--(8.303750\du,9.800037\du)--(8.303750\du,14.549719\du);
}
\definecolor{dialinecolor}{rgb}{0.000000, 0.000000, 0.000000}
\pgfsetstrokecolor{dialinecolor}
\draw (11.180000\du,5.162157\du)--(11.180000\du,9.800037\du)--(8.303750\du,9.800037\du)--(8.303750\du,14.549719\du);
\pgfsetmiterjoin
\definecolor{dialinecolor}{rgb}{1.000000, 1.000000, 1.000000}
\pgfsetfillcolor{dialinecolor}
\fill (11.580000\du,5.162157\du)--(11.180000\du,4.362157\du)--(10.780000\du,5.162157\du)--cycle;
\pgfsetlinewidth{0.100000\du}
\pgfsetdash{}{0pt}
\pgfsetmiterjoin
\definecolor{dialinecolor}{rgb}{0.000000, 0.000000, 0.000000}
\pgfsetstrokecolor{dialinecolor}
\draw (11.580000\du,5.162157\du)--(11.180000\du,4.362157\du)--(10.780000\du,5.162157\du)--cycle;
% setfont left to latex
\pgfsetlinewidth{0.100000\du}
\pgfsetdash{}{0pt}
\definecolor{dialinecolor}{rgb}{1.000000, 1.000000, 1.000000}
\pgfsetfillcolor{dialinecolor}
\fill (25.385000\du,-5.740000\du)--(25.385000\du,-4.340000\du)--(39.745000\du,-4.340000\du)--(39.745000\du,-5.740000\du)--cycle;
\definecolor{dialinecolor}{rgb}{0.000000, 0.000000, 0.000000}
\pgfsetstrokecolor{dialinecolor}
\draw (25.385000\du,-5.740000\du)--(25.385000\du,-4.340000\du)--(39.745000\du,-4.340000\du)--(39.745000\du,-5.740000\du)--cycle;
% setfont left to latex
\definecolor{dialinecolor}{rgb}{0.000000, 0.000000, 0.000000}
\pgfsetstrokecolor{dialinecolor}
\node at (32.565000\du,-4.790000\du){Element};
\definecolor{dialinecolor}{rgb}{1.000000, 1.000000, 1.000000}
\pgfsetfillcolor{dialinecolor}
\fill (25.385000\du,-4.340000\du)--(25.385000\du,-0.140000\du)--(39.745000\du,-0.140000\du)--(39.745000\du,-4.340000\du)--cycle;
\definecolor{dialinecolor}{rgb}{0.000000, 0.000000, 0.000000}
\pgfsetstrokecolor{dialinecolor}
\draw (25.385000\du,-4.340000\du)--(25.385000\du,-0.140000\du)--(39.745000\du,-0.140000\du)--(39.745000\du,-4.340000\du)--cycle;
% setfont left to latex
\definecolor{dialinecolor}{rgb}{0.000000, 0.000000, 0.000000}
\pgfsetstrokecolor{dialinecolor}
\node[anchor=west] at (25.535000\du,-3.640000\du){\#Nom: String};
% setfont left to latex
\definecolor{dialinecolor}{rgb}{0.000000, 0.000000, 0.000000}
\pgfsetstrokecolor{dialinecolor}
\node[anchor=west] at (25.535000\du,-2.840000\du){\#Date de création: Date};
% setfont left to latex
\definecolor{dialinecolor}{rgb}{0.000000, 0.000000, 0.000000}
\pgfsetstrokecolor{dialinecolor}
\node[anchor=west] at (25.535000\du,-2.040000\du){\#Date de dernière modification: Date};
% setfont left to latex
\definecolor{dialinecolor}{rgb}{0.000000, 0.000000, 0.000000}
\pgfsetstrokecolor{dialinecolor}
\node[anchor=west] at (25.535000\du,-1.240000\du){\#Chemin: String};
% setfont left to latex
\definecolor{dialinecolor}{rgb}{0.000000, 0.000000, 0.000000}
\pgfsetstrokecolor{dialinecolor}
\node[anchor=west] at (25.535000\du,-0.440000\du){\#Ouvert: Boolean = False};
\definecolor{dialinecolor}{rgb}{1.000000, 1.000000, 1.000000}
\pgfsetfillcolor{dialinecolor}
\fill (25.385000\du,-0.140000\du)--(25.385000\du,2.460000\du)--(39.745000\du,2.460000\du)--(39.745000\du,-0.140000\du)--cycle;
\definecolor{dialinecolor}{rgb}{0.000000, 0.000000, 0.000000}
\pgfsetstrokecolor{dialinecolor}
\draw (25.385000\du,-0.140000\du)--(25.385000\du,2.460000\du)--(39.745000\du,2.460000\du)--(39.745000\du,-0.140000\du)--cycle;
% setfont left to latex
\definecolor{dialinecolor}{rgb}{0.000000, 0.000000, 0.000000}
\pgfsetstrokecolor{dialinecolor}
\node[anchor=west] at (25.535000\du,0.560000\du){+open()};
% setfont left to latex
\definecolor{dialinecolor}{rgb}{0.000000, 0.000000, 0.000000}
\pgfsetstrokecolor{dialinecolor}
\node[anchor=west] at (25.535000\du,1.360000\du){+close()};
% setfont left to latex
\definecolor{dialinecolor}{rgb}{0.000000, 0.000000, 0.000000}
\pgfsetstrokecolor{dialinecolor}
\node[anchor=west] at (25.535000\du,2.160000\du){+delete()};
\end{tikzpicture}
}
    \caption{Diagramme de classes pour les raccourcis}
  \end{figure}

  Il s'agit d'une version modifiée du partron décorateur utilisé pour
  \textsf{ElementDecorator}, car un agrégat le lie à l'élément qu'il pointe.

  Les classes \textsf{Fichier} et \textsf{Dossier} seront impactés, car elles
  devront désormais hériter de la classe \textsf{ElementRaccourciable}.

  Il faut aussi déplacer la fonction \textsf{taille} de la classe
  \textsf{Element} vers la nouvelle classe \textsf{ElementRaccourciable}, car un
  raccourci ne possède pas de taille à proprement parler. On évite notamment un
  problème cycle pour le calcul de la taille d'un dossier dans le cas ou un
  dossier contiendrait un raccourcis qui pointerait vers ce même dossier.

  \subsection{Deuxième requête de changement}
  Pour implanter ce changement, nous faisons implanter l'interface
  \textsf{DeleteObserver} à la classe \textsf{Raccourci}, de telle sorte que le
  raccouci est notifié lorsque son élément est détruit.

  % ajout du deuxième diagramme
  \begin{figure}
    \centering
    \resizebox{\textwidth}{!}{% Graphic for TeX using PGF
% Title: /home/guillaume/Documents/Université de Montréal/Automne 2014/Génie logiciel/Devoir/devoir-3/diagramme-de-classes-changement-2.dia
% Creator: Dia v0.97.3
% CreationDate: Sat Dec  6 12:51:34 2014
% For: guillaume
% \usepackage{tikz}
% The following commands are not supported in PSTricks at present
% We define them conditionally, so when they are implemented,
% this pgf file will use them.
\ifx\du\undefined
  \newlength{\du}
\fi
\setlength{\du}{15\unitlength}
\begin{tikzpicture}
\pgftransformxscale{1.000000}
\pgftransformyscale{-1.000000}
\definecolor{dialinecolor}{rgb}{0.000000, 0.000000, 0.000000}
\pgfsetstrokecolor{dialinecolor}
\definecolor{dialinecolor}{rgb}{1.000000, 1.000000, 1.000000}
\pgfsetfillcolor{dialinecolor}
\pgfsetlinewidth{0.100000\du}
\pgfsetdash{}{0pt}
\definecolor{dialinecolor}{rgb}{1.000000, 1.000000, 1.000000}
\pgfsetfillcolor{dialinecolor}
\fill (10.309500\du,2.790000\du)--(10.309500\du,4.190000\du)--(14.122000\du,4.190000\du)--(14.122000\du,2.790000\du)--cycle;
\definecolor{dialinecolor}{rgb}{0.000000, 0.000000, 0.000000}
\pgfsetstrokecolor{dialinecolor}
\draw (10.309500\du,2.790000\du)--(10.309500\du,4.190000\du)--(14.122000\du,4.190000\du)--(14.122000\du,2.790000\du)--cycle;
% setfont left to latex
\definecolor{dialinecolor}{rgb}{0.000000, 0.000000, 0.000000}
\pgfsetstrokecolor{dialinecolor}
\node at (12.215750\du,3.740000\du){Élément};
\definecolor{dialinecolor}{rgb}{1.000000, 1.000000, 1.000000}
\pgfsetfillcolor{dialinecolor}
\fill (10.309500\du,4.190000\du)--(10.309500\du,4.590000\du)--(14.122000\du,4.590000\du)--(14.122000\du,4.190000\du)--cycle;
\definecolor{dialinecolor}{rgb}{0.000000, 0.000000, 0.000000}
\pgfsetstrokecolor{dialinecolor}
\draw (10.309500\du,4.190000\du)--(10.309500\du,4.590000\du)--(14.122000\du,4.590000\du)--(14.122000\du,4.190000\du)--cycle;
\definecolor{dialinecolor}{rgb}{1.000000, 1.000000, 1.000000}
\pgfsetfillcolor{dialinecolor}
\fill (10.309500\du,4.590000\du)--(10.309500\du,4.990000\du)--(14.122000\du,4.990000\du)--(14.122000\du,4.590000\du)--cycle;
\definecolor{dialinecolor}{rgb}{0.000000, 0.000000, 0.000000}
\pgfsetstrokecolor{dialinecolor}
\draw (10.309500\du,4.590000\du)--(10.309500\du,4.990000\du)--(14.122000\du,4.990000\du)--(14.122000\du,4.590000\du)--cycle;
\pgfsetlinewidth{0.100000\du}
\pgfsetdash{}{0pt}
\definecolor{dialinecolor}{rgb}{1.000000, 1.000000, 1.000000}
\pgfsetfillcolor{dialinecolor}
\fill (13.909500\du,17.090000\du)--(13.909500\du,18.490000\du)--(18.779500\du,18.490000\du)--(18.779500\du,17.090000\du)--cycle;
\definecolor{dialinecolor}{rgb}{0.000000, 0.000000, 0.000000}
\pgfsetstrokecolor{dialinecolor}
\draw (13.909500\du,17.090000\du)--(13.909500\du,18.490000\du)--(18.779500\du,18.490000\du)--(18.779500\du,17.090000\du)--cycle;
% setfont left to latex
\definecolor{dialinecolor}{rgb}{0.000000, 0.000000, 0.000000}
\pgfsetstrokecolor{dialinecolor}
\node at (16.344500\du,18.040000\du){Raccourci};
\definecolor{dialinecolor}{rgb}{1.000000, 1.000000, 1.000000}
\pgfsetfillcolor{dialinecolor}
\fill (13.909500\du,18.490000\du)--(13.909500\du,18.890000\du)--(18.779500\du,18.890000\du)--(18.779500\du,18.490000\du)--cycle;
\definecolor{dialinecolor}{rgb}{0.000000, 0.000000, 0.000000}
\pgfsetstrokecolor{dialinecolor}
\draw (13.909500\du,18.490000\du)--(13.909500\du,18.890000\du)--(18.779500\du,18.890000\du)--(18.779500\du,18.490000\du)--cycle;
\definecolor{dialinecolor}{rgb}{1.000000, 1.000000, 1.000000}
\pgfsetfillcolor{dialinecolor}
\fill (13.909500\du,18.890000\du)--(13.909500\du,19.290000\du)--(18.779500\du,19.290000\du)--(18.779500\du,18.890000\du)--cycle;
\definecolor{dialinecolor}{rgb}{0.000000, 0.000000, 0.000000}
\pgfsetstrokecolor{dialinecolor}
\draw (13.909500\du,18.890000\du)--(13.909500\du,19.290000\du)--(18.779500\du,19.290000\du)--(18.779500\du,18.890000\du)--cycle;
\pgfsetlinewidth{0.100000\du}
\pgfsetdash{}{0pt}
\pgfsetmiterjoin
\pgfsetbuttcap
{
\definecolor{dialinecolor}{rgb}{0.000000, 0.000000, 0.000000}
\pgfsetfillcolor{dialinecolor}
% was here!!!
\definecolor{dialinecolor}{rgb}{0.000000, 0.000000, 0.000000}
\pgfsetstrokecolor{dialinecolor}
\draw (13.859657\du,18.190000\du)--(7.309510\du,18.190000\du)--(7.309510\du,3.890000\du)--(10.259722\du,3.890000\du);
}
\definecolor{dialinecolor}{rgb}{0.000000, 0.000000, 0.000000}
\pgfsetstrokecolor{dialinecolor}
\draw (12.601079\du,18.190000\du)--(7.309510\du,18.190000\du)--(7.309510\du,3.890000\du)--(10.259722\du,3.890000\du);
\pgfsetdash{}{0pt}
\pgfsetmiterjoin
\pgfsetbuttcap
\definecolor{dialinecolor}{rgb}{1.000000, 1.000000, 1.000000}
\pgfsetfillcolor{dialinecolor}
\fill (13.859657\du,18.190000\du)--(13.159657\du,18.430000\du)--(12.459657\du,18.190000\du)--(13.159657\du,17.950000\du)--cycle;
\pgfsetlinewidth{0.100000\du}
\pgfsetdash{}{0pt}
\pgfsetmiterjoin
\pgfsetbuttcap
\definecolor{dialinecolor}{rgb}{0.000000, 0.000000, 0.000000}
\pgfsetstrokecolor{dialinecolor}
\draw (13.859657\du,18.190000\du)--(13.159657\du,18.430000\du)--(12.459657\du,18.190000\du)--(13.159657\du,17.950000\du)--cycle;
% setfont left to latex
\definecolor{dialinecolor}{rgb}{0.000000, 0.000000, 0.000000}
\pgfsetstrokecolor{dialinecolor}
\node[anchor=west] at (7.409510\du,10.890000\du){};
\definecolor{dialinecolor}{rgb}{0.000000, 0.000000, 0.000000}
\pgfsetfillcolor{dialinecolor}
\fill (7.509510\du,10.890000\du)--(7.509510\du,10.490000\du)--(7.909510\du,10.690000\du)--cycle;
\definecolor{dialinecolor}{rgb}{0.000000, 0.000000, 0.000000}
\pgfsetstrokecolor{dialinecolor}
\node[anchor=east] at (12.259657\du,18.040000\du){};
\definecolor{dialinecolor}{rgb}{0.000000, 0.000000, 0.000000}
\pgfsetstrokecolor{dialinecolor}
\node[anchor=east] at (10.059722\du,3.740000\du){ pointe vers};
\definecolor{dialinecolor}{rgb}{0.000000, 0.000000, 0.000000}
\pgfsetstrokecolor{dialinecolor}
\node[anchor=east] at (10.059722\du,4.540000\du){*};
\pgfsetlinewidth{0.100000\du}
\pgfsetdash{}{0pt}
\pgfsetmiterjoin
\pgfsetbuttcap
{
\definecolor{dialinecolor}{rgb}{0.000000, 0.000000, 0.000000}
\pgfsetfillcolor{dialinecolor}
% was here!!!
\definecolor{dialinecolor}{rgb}{0.000000, 0.000000, 0.000000}
\pgfsetstrokecolor{dialinecolor}
\draw (12.215750\du,5.040281\du)--(12.215750\du,11.440000\du)--(16.344500\du,11.440000\du)--(16.344500\du,17.039719\du);
}
\definecolor{dialinecolor}{rgb}{0.000000, 0.000000, 0.000000}
\pgfsetstrokecolor{dialinecolor}
\draw (12.215750\du,5.952084\du)--(12.215750\du,11.440000\du)--(16.344500\du,11.440000\du)--(16.344500\du,17.039719\du);
\pgfsetmiterjoin
\definecolor{dialinecolor}{rgb}{1.000000, 1.000000, 1.000000}
\pgfsetfillcolor{dialinecolor}
\fill (12.615750\du,5.952084\du)--(12.215750\du,5.152084\du)--(11.815750\du,5.952084\du)--cycle;
\pgfsetlinewidth{0.100000\du}
\pgfsetdash{}{0pt}
\pgfsetmiterjoin
\definecolor{dialinecolor}{rgb}{0.000000, 0.000000, 0.000000}
\pgfsetstrokecolor{dialinecolor}
\draw (12.615750\du,5.952084\du)--(12.215750\du,5.152084\du)--(11.815750\du,5.952084\du)--cycle;
% setfont left to latex
\pgfsetlinewidth{0.100000\du}
\pgfsetdash{}{0pt}
\definecolor{dialinecolor}{rgb}{1.000000, 1.000000, 1.000000}
\pgfsetfillcolor{dialinecolor}
\fill (19.309500\du,2.790000\du)--(19.309500\du,4.190000\du)--(26.109500\du,4.190000\du)--(26.109500\du,2.790000\du)--cycle;
\definecolor{dialinecolor}{rgb}{0.000000, 0.000000, 0.000000}
\pgfsetstrokecolor{dialinecolor}
\draw (19.309500\du,2.790000\du)--(19.309500\du,4.190000\du)--(26.109500\du,4.190000\du)--(26.109500\du,2.790000\du)--cycle;
% setfont left to latex
\definecolor{dialinecolor}{rgb}{0.000000, 0.000000, 0.000000}
\pgfsetstrokecolor{dialinecolor}
\node at (22.709500\du,3.740000\du){DeleteObserver};
\definecolor{dialinecolor}{rgb}{1.000000, 1.000000, 1.000000}
\pgfsetfillcolor{dialinecolor}
\fill (19.309500\du,4.190000\du)--(19.309500\du,4.590000\du)--(26.109500\du,4.590000\du)--(26.109500\du,4.190000\du)--cycle;
\definecolor{dialinecolor}{rgb}{0.000000, 0.000000, 0.000000}
\pgfsetstrokecolor{dialinecolor}
\draw (19.309500\du,4.190000\du)--(19.309500\du,4.590000\du)--(26.109500\du,4.590000\du)--(26.109500\du,4.190000\du)--cycle;
\definecolor{dialinecolor}{rgb}{1.000000, 1.000000, 1.000000}
\pgfsetfillcolor{dialinecolor}
\fill (19.309500\du,4.590000\du)--(19.309500\du,4.990000\du)--(26.109500\du,4.990000\du)--(26.109500\du,4.590000\du)--cycle;
\definecolor{dialinecolor}{rgb}{0.000000, 0.000000, 0.000000}
\pgfsetstrokecolor{dialinecolor}
\draw (19.309500\du,4.590000\du)--(19.309500\du,4.990000\du)--(26.109500\du,4.990000\du)--(26.109500\du,4.590000\du)--cycle;
\pgfsetlinewidth{0.100000\du}
\pgfsetdash{{1.000000\du}{1.000000\du}}{0\du}
\pgfsetdash{{0.400000\du}{0.400000\du}}{0\du}
\pgfsetmiterjoin
\pgfsetbuttcap
{
\definecolor{dialinecolor}{rgb}{0.000000, 0.000000, 0.000000}
\pgfsetfillcolor{dialinecolor}
% was here!!!
\definecolor{dialinecolor}{rgb}{0.000000, 0.000000, 0.000000}
\pgfsetstrokecolor{dialinecolor}
\draw (22.709500\du,5.040281\du)--(22.709500\du,11.144927\du)--(31.110000\du,11.144927\du)--(31.110000\du,16.449573\du);
}
\definecolor{dialinecolor}{rgb}{0.000000, 0.000000, 0.000000}
\pgfsetstrokecolor{dialinecolor}
\draw (22.709500\du,5.952084\du)--(22.709500\du,11.144927\du)--(31.110000\du,11.144927\du)--(31.110000\du,16.449573\du);
\pgfsetmiterjoin
\definecolor{dialinecolor}{rgb}{1.000000, 1.000000, 1.000000}
\pgfsetfillcolor{dialinecolor}
\fill (23.109500\du,5.952084\du)--(22.709500\du,5.152084\du)--(22.309500\du,5.952084\du)--cycle;
\pgfsetlinewidth{0.100000\du}
\pgfsetdash{}{0pt}
\pgfsetmiterjoin
\definecolor{dialinecolor}{rgb}{0.000000, 0.000000, 0.000000}
\pgfsetstrokecolor{dialinecolor}
\draw (23.109500\du,5.952084\du)--(22.709500\du,5.152084\du)--(22.309500\du,5.952084\du)--cycle;
% setfont left to latex
\pgfsetlinewidth{0.100000\du}
\pgfsetdash{}{0pt}
\definecolor{dialinecolor}{rgb}{1.000000, 1.000000, 1.000000}
\pgfsetfillcolor{dialinecolor}
\fill (25.550000\du,16.500000\du)--(25.550000\du,17.900000\du)--(36.670000\du,17.900000\du)--(36.670000\du,16.500000\du)--cycle;
\definecolor{dialinecolor}{rgb}{0.000000, 0.000000, 0.000000}
\pgfsetstrokecolor{dialinecolor}
\draw (25.550000\du,16.500000\du)--(25.550000\du,17.900000\du)--(36.670000\du,17.900000\du)--(36.670000\du,16.500000\du)--cycle;
% setfont left to latex
\definecolor{dialinecolor}{rgb}{0.000000, 0.000000, 0.000000}
\pgfsetstrokecolor{dialinecolor}
\node at (31.110000\du,17.450000\du){GestionnaireRaccourcis};
\definecolor{dialinecolor}{rgb}{1.000000, 1.000000, 1.000000}
\pgfsetfillcolor{dialinecolor}
\fill (25.550000\du,17.900000\du)--(25.550000\du,18.900000\du)--(36.670000\du,18.900000\du)--(36.670000\du,17.900000\du)--cycle;
\definecolor{dialinecolor}{rgb}{0.000000, 0.000000, 0.000000}
\pgfsetstrokecolor{dialinecolor}
\draw (25.550000\du,17.900000\du)--(25.550000\du,18.900000\du)--(36.670000\du,18.900000\du)--(36.670000\du,17.900000\du)--cycle;
% setfont left to latex
\definecolor{dialinecolor}{rgb}{0.000000, 0.000000, 0.000000}
\pgfsetstrokecolor{dialinecolor}
\node[anchor=west] at (25.700000\du,18.600000\du){-instance};
\definecolor{dialinecolor}{rgb}{1.000000, 1.000000, 1.000000}
\pgfsetfillcolor{dialinecolor}
\fill (25.550000\du,18.900000\du)--(25.550000\du,19.900000\du)--(36.670000\du,19.900000\du)--(36.670000\du,18.900000\du)--cycle;
\definecolor{dialinecolor}{rgb}{0.000000, 0.000000, 0.000000}
\pgfsetstrokecolor{dialinecolor}
\draw (25.550000\du,18.900000\du)--(25.550000\du,19.900000\du)--(36.670000\du,19.900000\du)--(36.670000\du,18.900000\du)--cycle;
% setfont left to latex
\definecolor{dialinecolor}{rgb}{0.000000, 0.000000, 0.000000}
\pgfsetstrokecolor{dialinecolor}
\node[anchor=west] at (25.700000\du,19.600000\du){+getInstance()};
\pgfsetlinewidth{0.100000\du}
\pgfsetdash{}{0pt}
\pgfsetmiterjoin
\pgfsetbuttcap
{
\definecolor{dialinecolor}{rgb}{0.000000, 0.000000, 0.000000}
\pgfsetfillcolor{dialinecolor}
% was here!!!
\definecolor{dialinecolor}{rgb}{0.000000, 0.000000, 0.000000}
\pgfsetstrokecolor{dialinecolor}
\draw (18.829803\du,18.190000\du)--(22.164730\du,18.190000\du)--(22.164730\du,18.200000\du)--(25.499658\du,18.200000\du);
}
% setfont left to latex
\definecolor{dialinecolor}{rgb}{0.000000, 0.000000, 0.000000}
\pgfsetstrokecolor{dialinecolor}
\node[anchor=west] at (22.264730\du,18.045000\du){};
\definecolor{dialinecolor}{rgb}{0.000000, 0.000000, 0.000000}
\pgfsetstrokecolor{dialinecolor}
\node[anchor=west] at (19.029803\du,18.040000\du){*};
\definecolor{dialinecolor}{rgb}{0.000000, 0.000000, 0.000000}
\pgfsetstrokecolor{dialinecolor}
\node[anchor=east] at (25.299658\du,18.050000\du){1};
\end{tikzpicture}
}
    \caption{Diagramme de classes pour le gestionnaire de raccourcis}
  \end{figure}

  Lorsqu'un un raccourcis est créé, il attache le gestionnaire et tant que
  DeleteObserver de l'élément qu'il pointe. Lorsque l'élément pointé est
  supprimé, le gestionnaire capture l'événement et est en mesure de supprimer le
  raccourcis approprié.

  L'implantation a pour ensemble d'impact initial la \textsf{Element} et
  l'interface \textsf{DeleteObserver}.

\end{document}
