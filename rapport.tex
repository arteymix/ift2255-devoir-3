\documentclass{article}

\usepackage[utf8]{inputenc}
\usepackage[T1]{fontenc}
\usepackage[french]{babel}

\usepackage{graphics}
\usepackage{fullpage}
\usepackage{rotating}
\usepackage{tikz}

\begin{document}

  \section{Diagramme de classes initial}
  \begin{sidewaysfigure}
  \centering
  \resizebox{\textwidth}{!}{% Graphic for TeX using PGF
% Title: /home/guillaume/Documents/Université de Montréal/Automne 2014/Génie logiciel/Devoir/devoir-3/diagramme-de-classes.dia
% Creator: Dia v0.97.3
% CreationDate: Thu Dec  4 18:16:33 2014
% For: guillaume
% \usepackage{tikz}
% The following commands are not supported in PSTricks at present
% We define them conditionally, so when they are implemented,
% this pgf file will use them.
\ifx\du\undefined
  \newlength{\du}
\fi
\setlength{\du}{15\unitlength}
\begin{tikzpicture}
\pgftransformxscale{1.000000}
\pgftransformyscale{-1.000000}
\definecolor{dialinecolor}{rgb}{0.000000, 0.000000, 0.000000}
\pgfsetstrokecolor{dialinecolor}
\definecolor{dialinecolor}{rgb}{1.000000, 1.000000, 1.000000}
\pgfsetfillcolor{dialinecolor}
\pgfsetlinewidth{0.100000\du}
\pgfsetdash{}{0pt}
\definecolor{dialinecolor}{rgb}{1.000000, 1.000000, 1.000000}
\pgfsetfillcolor{dialinecolor}
\fill (24.000000\du,13.000000\du)--(24.000000\du,14.400000\du)--(30.660000\du,14.400000\du)--(30.660000\du,13.000000\du)--cycle;
\definecolor{dialinecolor}{rgb}{0.000000, 0.000000, 0.000000}
\pgfsetstrokecolor{dialinecolor}
\draw (24.000000\du,13.000000\du)--(24.000000\du,14.400000\du)--(30.660000\du,14.400000\du)--(30.660000\du,13.000000\du)--cycle;
% setfont left to latex
\definecolor{dialinecolor}{rgb}{0.000000, 0.000000, 0.000000}
\pgfsetstrokecolor{dialinecolor}
\node at (27.330000\du,13.950000\du){Fichier};
\definecolor{dialinecolor}{rgb}{1.000000, 1.000000, 1.000000}
\pgfsetfillcolor{dialinecolor}
\fill (24.000000\du,14.400000\du)--(24.000000\du,15.400000\du)--(30.660000\du,15.400000\du)--(30.660000\du,14.400000\du)--cycle;
\definecolor{dialinecolor}{rgb}{0.000000, 0.000000, 0.000000}
\pgfsetstrokecolor{dialinecolor}
\draw (24.000000\du,14.400000\du)--(24.000000\du,15.400000\du)--(30.660000\du,15.400000\du)--(30.660000\du,14.400000\du)--cycle;
% setfont left to latex
\definecolor{dialinecolor}{rgb}{0.000000, 0.000000, 0.000000}
\pgfsetstrokecolor{dialinecolor}
\node[anchor=west] at (24.150000\du,15.100000\du){-taille: Integer};
\definecolor{dialinecolor}{rgb}{1.000000, 1.000000, 1.000000}
\pgfsetfillcolor{dialinecolor}
\fill (24.000000\du,15.400000\du)--(24.000000\du,15.800000\du)--(30.660000\du,15.800000\du)--(30.660000\du,15.400000\du)--cycle;
\definecolor{dialinecolor}{rgb}{0.000000, 0.000000, 0.000000}
\pgfsetstrokecolor{dialinecolor}
\draw (24.000000\du,15.400000\du)--(24.000000\du,15.800000\du)--(30.660000\du,15.800000\du)--(30.660000\du,15.400000\du)--cycle;
\pgfsetlinewidth{0.100000\du}
\pgfsetdash{}{0pt}
\definecolor{dialinecolor}{rgb}{1.000000, 1.000000, 1.000000}
\pgfsetfillcolor{dialinecolor}
\fill (32.900000\du,17.100000\du)--(32.900000\du,18.500000\du)--(40.330000\du,18.500000\du)--(40.330000\du,17.100000\du)--cycle;
\definecolor{dialinecolor}{rgb}{0.000000, 0.000000, 0.000000}
\pgfsetstrokecolor{dialinecolor}
\draw (32.900000\du,17.100000\du)--(32.900000\du,18.500000\du)--(40.330000\du,18.500000\du)--(40.330000\du,17.100000\du)--cycle;
% setfont left to latex
\definecolor{dialinecolor}{rgb}{0.000000, 0.000000, 0.000000}
\pgfsetstrokecolor{dialinecolor}
\node at (36.615000\du,18.050000\du){Dossier};
\definecolor{dialinecolor}{rgb}{1.000000, 1.000000, 1.000000}
\pgfsetfillcolor{dialinecolor}
\fill (32.900000\du,18.500000\du)--(32.900000\du,18.900000\du)--(40.330000\du,18.900000\du)--(40.330000\du,18.500000\du)--cycle;
\definecolor{dialinecolor}{rgb}{0.000000, 0.000000, 0.000000}
\pgfsetstrokecolor{dialinecolor}
\draw (32.900000\du,18.500000\du)--(32.900000\du,18.900000\du)--(40.330000\du,18.900000\du)--(40.330000\du,18.500000\du)--cycle;
\definecolor{dialinecolor}{rgb}{1.000000, 1.000000, 1.000000}
\pgfsetfillcolor{dialinecolor}
\fill (32.900000\du,18.900000\du)--(32.900000\du,19.900000\du)--(40.330000\du,19.900000\du)--(40.330000\du,18.900000\du)--cycle;
\definecolor{dialinecolor}{rgb}{0.000000, 0.000000, 0.000000}
\pgfsetstrokecolor{dialinecolor}
\draw (32.900000\du,18.900000\du)--(32.900000\du,19.900000\du)--(40.330000\du,19.900000\du)--(40.330000\du,18.900000\du)--cycle;
% setfont left to latex
\definecolor{dialinecolor}{rgb}{0.000000, 0.000000, 0.000000}
\pgfsetstrokecolor{dialinecolor}
\node[anchor=west] at (33.050000\du,19.600000\du){+taille(): Integer};
\pgfsetlinewidth{0.100000\du}
\pgfsetdash{}{0pt}
\definecolor{dialinecolor}{rgb}{1.000000, 1.000000, 1.000000}
\pgfsetfillcolor{dialinecolor}
\fill (26.050000\du,0.950000\du)--(26.050000\du,2.350000\du)--(36.945000\du,2.350000\du)--(36.945000\du,0.950000\du)--cycle;
\definecolor{dialinecolor}{rgb}{0.000000, 0.000000, 0.000000}
\pgfsetstrokecolor{dialinecolor}
\draw (26.050000\du,0.950000\du)--(26.050000\du,2.350000\du)--(36.945000\du,2.350000\du)--(36.945000\du,0.950000\du)--cycle;
% setfont left to latex
\definecolor{dialinecolor}{rgb}{0.000000, 0.000000, 0.000000}
\pgfsetstrokecolor{dialinecolor}
\node at (31.497500\du,1.900000\du){Élément};
\definecolor{dialinecolor}{rgb}{1.000000, 1.000000, 1.000000}
\pgfsetfillcolor{dialinecolor}
\fill (26.050000\du,2.350000\du)--(26.050000\du,5.750000\du)--(36.945000\du,5.750000\du)--(36.945000\du,2.350000\du)--cycle;
\definecolor{dialinecolor}{rgb}{0.000000, 0.000000, 0.000000}
\pgfsetstrokecolor{dialinecolor}
\draw (26.050000\du,2.350000\du)--(26.050000\du,5.750000\du)--(36.945000\du,5.750000\du)--(36.945000\du,2.350000\du)--cycle;
% setfont left to latex
\definecolor{dialinecolor}{rgb}{0.000000, 0.000000, 0.000000}
\pgfsetstrokecolor{dialinecolor}
\node[anchor=west] at (26.200000\du,3.050000\du){+nom};
% setfont left to latex
\definecolor{dialinecolor}{rgb}{0.000000, 0.000000, 0.000000}
\pgfsetstrokecolor{dialinecolor}
\node[anchor=west] at (26.200000\du,3.850000\du){+chemin};
% setfont left to latex
\definecolor{dialinecolor}{rgb}{0.000000, 0.000000, 0.000000}
\pgfsetstrokecolor{dialinecolor}
\node[anchor=west] at (26.200000\du,4.650000\du){-date de création: Date};
% setfont left to latex
\definecolor{dialinecolor}{rgb}{0.000000, 0.000000, 0.000000}
\pgfsetstrokecolor{dialinecolor}
\node[anchor=west] at (26.200000\du,5.450000\du){-date de modification: Date};
\definecolor{dialinecolor}{rgb}{1.000000, 1.000000, 1.000000}
\pgfsetfillcolor{dialinecolor}
\fill (26.050000\du,5.750000\du)--(26.050000\du,9.150000\du)--(36.945000\du,9.150000\du)--(36.945000\du,5.750000\du)--cycle;
\definecolor{dialinecolor}{rgb}{0.000000, 0.000000, 0.000000}
\pgfsetstrokecolor{dialinecolor}
\draw (26.050000\du,5.750000\du)--(26.050000\du,9.150000\du)--(36.945000\du,9.150000\du)--(36.945000\du,5.750000\du)--cycle;
% setfont left to latex
\definecolor{dialinecolor}{rgb}{0.000000, 0.000000, 0.000000}
\pgfsetstrokecolor{dialinecolor}
\node[anchor=west] at (26.200000\du,6.450000\du){+lire(): String};
% setfont left to latex
\definecolor{dialinecolor}{rgb}{0.000000, 0.000000, 0.000000}
\pgfsetstrokecolor{dialinecolor}
\node[anchor=west] at (26.200000\du,7.250000\du){+écrire(contenu:String)};
% setfont left to latex
\definecolor{dialinecolor}{rgb}{0.000000, 0.000000, 0.000000}
\pgfsetstrokecolor{dialinecolor}
\node[anchor=west] at (26.200000\du,8.050000\du){+ouvrir()};
% setfont left to latex
\definecolor{dialinecolor}{rgb}{0.000000, 0.000000, 0.000000}
\pgfsetstrokecolor{dialinecolor}
\node[anchor=west] at (26.200000\du,8.850000\du){+fermer()};
\pgfsetlinewidth{0.100000\du}
\pgfsetdash{}{0pt}
\pgfsetmiterjoin
\pgfsetbuttcap
{
\definecolor{dialinecolor}{rgb}{0.000000, 0.000000, 0.000000}
\pgfsetfillcolor{dialinecolor}
% was here!!!
\definecolor{dialinecolor}{rgb}{0.000000, 0.000000, 0.000000}
\pgfsetstrokecolor{dialinecolor}
\draw (31.497500\du,9.200253\du)--(31.497500\du,11.474950\du)--(27.330000\du,11.474950\du)--(27.330000\du,12.949646\du);
}
\definecolor{dialinecolor}{rgb}{0.000000, 0.000000, 0.000000}
\pgfsetstrokecolor{dialinecolor}
\draw (31.497500\du,10.112057\du)--(31.497500\du,11.474950\du)--(27.330000\du,11.474950\du)--(27.330000\du,12.949646\du);
\pgfsetmiterjoin
\definecolor{dialinecolor}{rgb}{1.000000, 1.000000, 1.000000}
\pgfsetfillcolor{dialinecolor}
\fill (31.897500\du,10.112057\du)--(31.497500\du,9.312057\du)--(31.097500\du,10.112057\du)--cycle;
\pgfsetlinewidth{0.100000\du}
\pgfsetdash{}{0pt}
\pgfsetmiterjoin
\definecolor{dialinecolor}{rgb}{0.000000, 0.000000, 0.000000}
\pgfsetstrokecolor{dialinecolor}
\draw (31.897500\du,10.112057\du)--(31.497500\du,9.312057\du)--(31.097500\du,10.112057\du)--cycle;
% setfont left to latex
\pgfsetlinewidth{0.100000\du}
\pgfsetdash{}{0pt}
\pgfsetmiterjoin
\pgfsetbuttcap
{
\definecolor{dialinecolor}{rgb}{0.000000, 0.000000, 0.000000}
\pgfsetfillcolor{dialinecolor}
% was here!!!
\definecolor{dialinecolor}{rgb}{0.000000, 0.000000, 0.000000}
\pgfsetstrokecolor{dialinecolor}
\draw (31.497500\du,9.200253\du)--(31.497500\du,13.524950\du)--(36.615000\du,13.524950\du)--(36.615000\du,17.049646\du);
}
\definecolor{dialinecolor}{rgb}{0.000000, 0.000000, 0.000000}
\pgfsetstrokecolor{dialinecolor}
\draw (31.497500\du,10.112057\du)--(31.497500\du,13.524950\du)--(36.615000\du,13.524950\du)--(36.615000\du,17.049646\du);
\pgfsetmiterjoin
\definecolor{dialinecolor}{rgb}{1.000000, 1.000000, 1.000000}
\pgfsetfillcolor{dialinecolor}
\fill (31.897500\du,10.112057\du)--(31.497500\du,9.312057\du)--(31.097500\du,10.112057\du)--cycle;
\pgfsetlinewidth{0.100000\du}
\pgfsetdash{}{0pt}
\pgfsetmiterjoin
\definecolor{dialinecolor}{rgb}{0.000000, 0.000000, 0.000000}
\pgfsetstrokecolor{dialinecolor}
\draw (31.897500\du,10.112057\du)--(31.497500\du,9.312057\du)--(31.097500\du,10.112057\du)--cycle;
% setfont left to latex
\pgfsetlinewidth{0.100000\du}
\pgfsetdash{}{0pt}
\definecolor{dialinecolor}{rgb}{1.000000, 1.000000, 1.000000}
\pgfsetfillcolor{dialinecolor}
\fill (49.000000\du,7.000000\du)--(49.000000\du,8.400000\du)--(54.510000\du,8.400000\du)--(54.510000\du,7.000000\du)--cycle;
\definecolor{dialinecolor}{rgb}{0.000000, 0.000000, 0.000000}
\pgfsetstrokecolor{dialinecolor}
\draw (49.000000\du,7.000000\du)--(49.000000\du,8.400000\du)--(54.510000\du,8.400000\du)--(54.510000\du,7.000000\du)--cycle;
% setfont left to latex
\definecolor{dialinecolor}{rgb}{0.000000, 0.000000, 0.000000}
\pgfsetstrokecolor{dialinecolor}
\node at (51.755000\du,7.950000\du){Navigateur};
\definecolor{dialinecolor}{rgb}{1.000000, 1.000000, 1.000000}
\pgfsetfillcolor{dialinecolor}
\fill (49.000000\du,8.400000\du)--(49.000000\du,8.800000\du)--(54.510000\du,8.800000\du)--(54.510000\du,8.400000\du)--cycle;
\definecolor{dialinecolor}{rgb}{0.000000, 0.000000, 0.000000}
\pgfsetstrokecolor{dialinecolor}
\draw (49.000000\du,8.400000\du)--(49.000000\du,8.800000\du)--(54.510000\du,8.800000\du)--(54.510000\du,8.400000\du)--cycle;
\definecolor{dialinecolor}{rgb}{1.000000, 1.000000, 1.000000}
\pgfsetfillcolor{dialinecolor}
\fill (49.000000\du,8.800000\du)--(49.000000\du,9.200000\du)--(54.510000\du,9.200000\du)--(54.510000\du,8.800000\du)--cycle;
\definecolor{dialinecolor}{rgb}{0.000000, 0.000000, 0.000000}
\pgfsetstrokecolor{dialinecolor}
\draw (49.000000\du,8.800000\du)--(49.000000\du,9.200000\du)--(54.510000\du,9.200000\du)--(54.510000\du,8.800000\du)--cycle;
\pgfsetlinewidth{0.100000\du}
\pgfsetdash{}{0pt}
\pgfsetmiterjoin
\pgfsetbuttcap
{
\definecolor{dialinecolor}{rgb}{0.000000, 0.000000, 0.000000}
\pgfsetfillcolor{dialinecolor}
% was here!!!
\pgfsetarrowsend{to}
\definecolor{dialinecolor}{rgb}{0.000000, 0.000000, 0.000000}
\pgfsetstrokecolor{dialinecolor}
\draw (48.950587\du,8.100000\du)--(45.000000\du,8.100000\du)--(45.000000\du,18.500000\du)--(40.380164\du,18.500000\du);
}
% setfont left to latex
\definecolor{dialinecolor}{rgb}{0.000000, 0.000000, 0.000000}
\pgfsetstrokecolor{dialinecolor}
\node[anchor=west] at (45.100000\du,13.150000\du){};
\definecolor{dialinecolor}{rgb}{0.000000, 0.000000, 0.000000}
\pgfsetfillcolor{dialinecolor}
\fill (45.200000\du,13.150000\du)--(45.200000\du,12.750000\du)--(45.600000\du,12.950000\du)--cycle;
\definecolor{dialinecolor}{rgb}{0.000000, 0.000000, 0.000000}
\pgfsetstrokecolor{dialinecolor}
\node[anchor=east] at (48.750587\du,7.950000\du){ 1};
\definecolor{dialinecolor}{rgb}{0.000000, 0.000000, 0.000000}
\pgfsetstrokecolor{dialinecolor}
\node[anchor=east] at (48.750587\du,8.750000\du){notifie};
\definecolor{dialinecolor}{rgb}{0.000000, 0.000000, 0.000000}
\pgfsetstrokecolor{dialinecolor}
\node[anchor=west] at (41.380164\du,18.350000\du){ *};
\definecolor{dialinecolor}{rgb}{0.000000, 0.000000, 0.000000}
\pgfsetstrokecolor{dialinecolor}
\node[anchor=west] at (41.380164\du,19.150000\du){gère};
\pgfsetlinewidth{0.100000\du}
\pgfsetdash{}{0pt}
\pgfsetmiterjoin
\pgfsetbuttcap
{
\definecolor{dialinecolor}{rgb}{0.000000, 0.000000, 0.000000}
\pgfsetfillcolor{dialinecolor}
% was here!!!
\pgfsetarrowsend{to}
\definecolor{dialinecolor}{rgb}{0.000000, 0.000000, 0.000000}
\pgfsetstrokecolor{dialinecolor}
\draw (51.755000\du,9.249994\du)--(51.755000\du,24.000000\du)--(36.615000\du,24.000000\du)--(36.615000\du,19.944824\du);
}
% setfont left to latex
\definecolor{dialinecolor}{rgb}{0.000000, 0.000000, 0.000000}
\pgfsetstrokecolor{dialinecolor}
\node at (44.185000\du,23.850000\du){};
\definecolor{dialinecolor}{rgb}{0.000000, 0.000000, 0.000000}
\pgfsetfillcolor{dialinecolor}
\fill (44.285000\du,23.850000\du)--(44.285000\du,23.450000\du)--(44.685000\du,23.650000\du)--cycle;
\definecolor{dialinecolor}{rgb}{0.000000, 0.000000, 0.000000}
\pgfsetstrokecolor{dialinecolor}
\node[anchor=west] at (51.955000\du,9.849994\du){1};
\definecolor{dialinecolor}{rgb}{0.000000, 0.000000, 0.000000}
\pgfsetstrokecolor{dialinecolor}
\node[anchor=west] at (37.165000\du,20.544824\du){ active};
\definecolor{dialinecolor}{rgb}{0.000000, 0.000000, 0.000000}
\pgfsetstrokecolor{dialinecolor}
\node[anchor=west] at (37.165000\du,21.344824\du){0..1};
\pgfsetlinewidth{0.100000\du}
\pgfsetdash{}{0pt}
\pgfsetmiterjoin
\pgfsetbuttcap
{
\definecolor{dialinecolor}{rgb}{0.000000, 0.000000, 0.000000}
\pgfsetfillcolor{dialinecolor}
% was here!!!
\definecolor{dialinecolor}{rgb}{0.000000, 0.000000, 0.000000}
\pgfsetstrokecolor{dialinecolor}
\draw (32.849921\du,18.500000\du)--(21.000000\du,18.500000\du)--(21.000000\du,5.050000\du)--(25.999832\du,5.050000\du);
}
\definecolor{dialinecolor}{rgb}{0.000000, 0.000000, 0.000000}
\pgfsetstrokecolor{dialinecolor}
\draw (31.591342\du,18.500000\du)--(21.000000\du,18.500000\du)--(21.000000\du,5.050000\du)--(25.999832\du,5.050000\du);
\pgfsetdash{}{0pt}
\pgfsetmiterjoin
\pgfsetbuttcap
\definecolor{dialinecolor}{rgb}{0.000000, 0.000000, 0.000000}
\pgfsetfillcolor{dialinecolor}
\fill (32.849921\du,18.500000\du)--(32.149921\du,18.740000\du)--(31.449921\du,18.500000\du)--(32.149921\du,18.260000\du)--cycle;
\pgfsetlinewidth{0.100000\du}
\pgfsetdash{}{0pt}
\pgfsetmiterjoin
\pgfsetbuttcap
\definecolor{dialinecolor}{rgb}{0.000000, 0.000000, 0.000000}
\pgfsetstrokecolor{dialinecolor}
\draw (32.849921\du,18.500000\du)--(32.149921\du,18.740000\du)--(31.449921\du,18.500000\du)--(32.149921\du,18.260000\du)--cycle;
% setfont left to latex
\definecolor{dialinecolor}{rgb}{0.000000, 0.000000, 0.000000}
\pgfsetstrokecolor{dialinecolor}
\node[anchor=west] at (21.100000\du,11.625000\du){};
\definecolor{dialinecolor}{rgb}{0.000000, 0.000000, 0.000000}
\pgfsetstrokecolor{dialinecolor}
\node[anchor=east] at (31.249921\du,18.350000\du){};
\definecolor{dialinecolor}{rgb}{0.000000, 0.000000, 0.000000}
\pgfsetstrokecolor{dialinecolor}
\node[anchor=east] at (25.799832\du,4.900000\du){ possède};
\definecolor{dialinecolor}{rgb}{0.000000, 0.000000, 0.000000}
\pgfsetstrokecolor{dialinecolor}
\node[anchor=east] at (25.799832\du,5.700000\du){*};
\end{tikzpicture}
}
  \caption{Diagramme de classes initial}
  \end{sidewaysfigure}

  \section{Diagramme de classe modifié}

  \section{Client}

  \section{Changements}

  \subsection{Première requête de changement}
  Pour effectuer le changement, nous introduisons la classe \textsf{Raccourci}
  l'agrégat vers la classe \textsf{Element}. Ainsi, une instance de
  \textsf{Raccourci} sera liée à l'élément qu'elle pointe.

  % ajouter le petit diagramme ici
  \begin{figure}
  \caption{Ajout de classe Raccourci au diagramme de classes}
  \end{figure}

  Il s'agit d'une version modifiée du partron décorateur utilisé pour
  \textsf{ElementDecorator}, car un agrégat le lie à l'élément qu'il pointe.

  L'implantation possède un ensemble d'impact réduit qui comporte uniquement la
  classe \textsf{Element}.

  \subsection{Deuxième requête de changement}
  Pour implanter ce changement, nous faisons implanter l'interface
  \textsf{DeleteObserver} à la classe \textsf{Raccourci}, de telle sorte que le
  raccouci est notifié lorsque son élément est détruit.

  % ajout du deuxième diagramme
  \begin{figure}
  \caption{Implantation de l'interface DeleteObserver pour la classe Raccourci}
  \end{figure}

  L'implantation a pour ensemble d'impact initial la \textsf{Element} et
  l'interface \textsf{DeleteObserver}.

\end{document}
